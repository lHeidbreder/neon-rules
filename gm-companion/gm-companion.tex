\documentclass[12pt,a4paper,openany]{book}
%Custom Environments
\newenvironment{exampleblock}[1][1]
{
	\par
	\vspace{-5mm}
	\hfill
	\begin{minipage}
		{\dimexpr\columnwidth-#1cm}
	\begin{mdframed}[
		backgroundcolor=Gray!65,
		rightline=false,
		topline=false
		]
}{
	\end{mdframed}
	\end{minipage}
	\par
}

%%shortened itemize
\newenvironment{sitemize}[1][10]
{
	\begin{itemize}
	\vspace{-#1mm}
	\setlength\itemsep{-#1mm}
}{
	\end{itemize}
}

%Custom Commands
\newcommand{\ul}[1]{\underline{\smash{#1}}}
\newcommand{\breakline}{\vspace{.5cm} \hrule width \columnwidth \relax}
\newcommand{\derivedvalue}[3]{
	\begin{samepage}
	\subsubsection{#1 \textsubscript{\textlangle#2\textrangle}}
	\hfill
	\begin{minipage}{\dimexpr\columnwidth-1cm}
		#3
	\end{minipage}
	\end{samepage}
	\par
}
\newcommand{\specialrule}[2]{
	\begin{minipage}{\columnwidth}
		\textbf{\ul{#1:}}\\
		#2
	\end{minipage}
    \par
}
\newcommand{\supply}[4]{
	\begin{minipage}{\columnwidth}
		\textbf{\ul{#1}}\\
		\textit{Price: #3}\\
		\textit{Weight: #4}\\
		#2
	\end{minipage}
    \par
}
\newcommand{\service}[4]{
	\begin{minipage}{\columnwidth}
		\textbf{\ul{#1}}\\
		#2\\
		\textit{Cost:} #3 per #4
	\end{minipage}
	\par
}
\newcommand{\statuseffect}[2]{
	\begin{minipage}{\columnwidth}
		\textbf{\ul{#1:}}\\
		#2\\
	\end{minipage}
	\vspace{5mm}
}
\newcommand{\skill}[4][Basic]{
	\begin{minipage}{\columnwidth}
		\textbf{\ul{#2}}\\
		Difficulty: #1\\
		Common Characteristic: #3\\
%		Description:\\
		\textit{#4}\\
	\end{minipage}\par
}
\newcommand{\melee}[2]{\skill{#1}{Melee Base}{#2}}
\newcommand{\ranged}[2]{\skill{#1}{Ranged Base}{#2}}
\newcommand{\ability}[4]{
	\begin{minipage}{\columnwidth}
		\textbf{\ul{#1}} (#2 XP)\\
		\textit{Prerequisites}: #3\\
		\textit{Effect}:\\
		#4\\
	\end{minipage}
}
\newcommand{\maneuver}[3]{
	\begin{minipage}{\columnwidth}
		\textbf{\ul{MVR: #1}} (#2 XP)\\
		\textit{Prerequisites: #3\\}
		Effect:\\
		Allows the use of the maneuver \emph{#1} at normal penalties.\\
	\end{minipage}
}
\newcommand{\boon}[4][]{
	\begin{minipage}{\columnwidth}
		\label{boon::#2}
		\textbf{\ul{\smash{#2}}} \mbox{(#4 GP\ifthenelse{\isempty{#1}}{}{ }#1)}\\
		#3
	\end{minipage}
	\par
}
\newcommand{\bane}[4]{
	\boon[#4]{#1}{#2}{#3}
}
\newcommand{\rangedweapon}[9]{
	\vspace{2mm}
	\begin{minipage}{\columnwidth}
		\textbf{\ul{#1}}\\
		\textit{#2}\\
		\textbf{Weight}: #9\\
		\textbf{Price}: #8\\
		#3\\
		\textbf{Mag}: #5\\
		\textbf{Reload}: \mbox{#6 actions}\\
		\textbf{Range}: #4\\
		\ifthenelse{\isempty{#7}}{}{\textbf{Special Rules}: #7}
	\end{minipage}
	\par
}
\newcommand{\weaponmod}[5]{
	\begin{minipage}{\columnwidth}
		\textbf{\ul{#1}}\\
		\textit{#3}; \textit{#4}; \textit{#5}\\
		#2
	\end{minipage}
	\par
}
\newcommand{\meleecomponent}[6]{
	\begin{minipage}{\columnwidth}
		\textbf{\ul{#1:}}\\
		\textit{#2}\\
		\textbf{Weight}: #4; \textbf{Price}: #5\ifthenelse{\isempty{#6}}{}{; \textbf{Requirement}: #6}\\
		\textbf{Effect}: #3
	\end{minipage}
	\par
}
\newcommand{\ammo}[6]{
	\begin{minipage}{\columnwidth}
		\textbf{\ul{#1:}} \textit{#2}\\
		\textbf{Price}: #3; \textbf{Unit of sale}: #4\\
		\textbf{Weight/Bulk}: #5 \textit{\ifthenelse{\isempty{#6}}{each}{per #6}}
	\end{minipage}
	\par
}
\newcommand{\armor}[9]{
	\begin{minipage}{\columnwidth}
		\textbf{\ul{#1}} (covers: #6)\\
		\begin{tabular}{|r|r|r|r|}
			\hline
			Head & Chest & Arms & Legs\\
			\hline
			#2 & #3 & #4 & #5\\
			\hline
		\end{tabular}\par
		\vspace{2mm}
		\textit{Price:} #8; \textit{Weight:} #9\\
		#7
	\end{minipage}
	\par
}
\newcommand{\armormod}[6]{
	\begin{minipage}{\columnwidth}
		\textbf{\ul{#1}}\\
		#2\\
		\textit{Price:} #3; \textit{Weight:} #6\\
		\textit{Mod points:} #4\ifthenelse{\isequivalentto{N}{#5}}{}{; Requires power source}
	\end{minipage}
	\par
}
\newcommand{\pes}[6]{
	\begin{minipage}{\columnwidth}
		\textbf{\ul{#1}}\\
		Armor: #2; Threshold: #3\\
		\textit{Price:} #4; \textit{Weight:} #5
		\ifthenelse{\isempty{#6}}{}{\\} %conditional line break
		#6
	\end{minipage}
	\par
}
\newcommand{\implant}[7]{
	\begin{minipage}{\columnwidth}
		\textbf{#1}\\
		\textit{#2}\\
		\textit{Effect}: #3\\
		\textit{Price:} #6; \textit{Load:} #5\ifthenelse{\isequivalentto{-}{#4}}{}{; \textit{Slot:} #4}\\
		\textit{Available Mods:} #7
	\end{minipage}
	\par
}
\newcommand{\augmod}[3]{
	\begin{minipage}{\columnwidth}
		\textbf{#1}\\
		#2\\
		\textit{Cost:} #3
	\end{minipage}
	\par
}
\newcommand{\mod}[2]{\item \textbf{#1}: #2}
\newcommand{\psicomponent}[4]{\textbf{#1} & #2 & #3 & #4 \\}

%filler images
\newcommand{\filltopageendgraphics}[2][]{%
	\par
	\zsaveposy{top-\thepage}% Mark (baseline of) top of image
	\vfill
	\zsaveposy{bottom-\thepage}% Mark (baseline of) bottom of image
	\smash{\includegraphics[height=\dimexpr\zposy{top-\thepage}sp-\zposy{bottom-\thepage}sp\relax,#1]{#2}}%
	\par
}

%Base Building
\newcommand{\baselocation}[6]{
	\begin{minipage}{\columnwidth}
		\textbf{#1}\\
		Size: #3; Concealment: #4; Defense: #5\\
		Conditions: #6\\
		\textit{Cost:} #2
	\end{minipage}
	\par
}
\newcommand{\baseasset}[6]{
	\begin{minipage}{\columnwidth}
		\textbf{#1}\\
		\textit{#2}\\
		Size: #3\\
		Concealment: #4\\
		Defense: #5\\
		\textit{Cost:} #6
	\end{minipage}
	\par
}

%Narrative
\newcommand{\nrule}[3]{
	\begin{minipage}{\columnwidth}
		\section{#1}
		\textit{#2}\\
		\vspace{8mm}
		\begin{exampleblock}
			#3
		\end{exampleblock}
	\end{minipage}
	\par
}

%GM
\newcommand{\missiontype}[3]{\item \textbf{#1}: \textit{#2} #3}

\usepackage[utf8]{inputenc}
\usepackage[english]{babel}
\usepackage{textcomp}
\usepackage{xifthen}
\usepackage{tabularx}
\usepackage{tabto}
\usepackage{multicol}

\usepackage[bookmarks=true,colorlinks=true,linkcolor=cyan]{hyperref}

%Alignment
\usepackage[skip=10mm]{parskip}
\raggedbottom

% Title Image
\usepackage{wallpaper}
\def\coverimgpath{../art/\@title/cover}
\usepackage{ltablex}

\def\subtitle{GM Companion}

\begin{document}
	{\heading
\ThisCenterWallPaper{1}{\coverimgpath}
\maketitle}
{\hypersetup{hidelinks} \tableofcontents}


	\chapter{Specialties}
%	What differenciates Neon from other PnP games? How do you deal with that as a GM?
%	Lethality
%	Modularity

	\chapter{Mission Creation}
	%	Models: Technoir, Halo Mythic
\section{Introduction}
This chapter is dedicated to making mission creation as easy as possible. 

\section{Goal}
The term "mission" describes a task within a campaign, some type of intermediate step to reach some end goal. Without an end goal in mind, there are no stakes and there is no investment. \par
Every mission has some goal. If that goal is reached, made obsolete or failed irredeemably, the mission ends.\par
Goals depend on the characters undertaking the mission. If they are personally involved in creating the mission goal, they will be more engaged.

\section{Mission Type}
Some types of missions are extremely common. Beware that some mission types may be more or less compatible with certain goals but anything can be made to work.
\begin{enumerate}
	\setlength\itemsep{-5mm}
	\missiontype{Assassination}{Take out one or multiple VIPs.}{They don't necessarily need to be killed but they cannot cross the group again.}
	\missiontype{Demolition}{Take out a key location.}{Factories and supply routes are common targets but so could be safe houses and data centers.}
	\missiontype{Raze}{Take out a whole location.}{Move in, take out hostile forces, destroy the place. Never been simpler.}
	\missiontype{Raid}{Steal supplies or evidence in a lightning strike.}{The concept of a raid is pretty obvious. Just don't overstay your welcome, the enemy will receive reinforcements.}
	\missiontype{Secure}{Rescue a person.}{Retrieve some ally from captivity by force.}
	\missiontype{Protection}{Defend a location or escort a person.}{Make sure that no harm comes to the target.}
	\missiontype{Diversion}{Distract a group from another event.}{A second group is performing another mission - be that a raid or an assassination - and they require some space and time.}
	\missiontype{Espionage}{Gather information.}{Information can be gathered from various sources, including people and records. Wherever they come from, we need them.}
	\missiontype{Locate}{Find a person or an object.}{An item was lost or a person has gone into hiding. We need to know where to.}
	\missiontype{Retrieve}{Collect an item.}{Lost, stolen or simply of interest; move in, gather the item, get it to its new owner.}
\end{enumerate}

\section{Opposition}
The faction working against the party should be obvious at the time of mission planning. However a single faction may employ very different forces, varying in level of expertise and equipment quality. A rough selection looks like the following.
\begin{enumerate}
	\setlength\itemsep{-5mm}
	\missiontype{Goons}{Uncoordinated, badly equipped but cheap.}{Anyone could belong to this category depending on circumstances: laborers, paid or threatened goons, soldiers on leave on multiple drugs. They should have very low stats, be unprepared and relatively high in number.}
	\missiontype{Hit Squad}{Targeted but still cheap.}{We're not talking about professional hitmen here. These are still just goons but are given a target and some time to prepare. They understand basic tactics such as cover and crossfire.}
	\missiontype{Mercenaries}{Professional guns for hire.}{Enemies like this understand tactics and are well-versed in a variety of combat situations. They are prepared and work well together.}
	\missiontype{Officials}{Great defenders.}{Officials are paid to keep law and order. They are usually great in defense scenarios or against barricaded suspects. Usually they have intelligence and are prepared for their aggressors.}
	\missiontype{Headhunters}{Lethal stealth attacks.}{Headhunters are incredibly lethal and usually work only in very small teams. They know their target very well and their target doesn't know them.}
\end{enumerate}

\section{Twist}
No plan survives first contact with the enemy, known for hundreds of years. Every mission should have some sort of twist, possibly multiple. A selection of possible twists can be found below.
\begin{enumerate}
	\setlength\itemsep{-5mm}
%	Mission information
	\missiontype{Timer}{The mission is on the clock.}{Likely the most obvious twist at first. Some condition puts a time constraint on the mission. This may be that an assassination target is leaving the country or a homeowner is returning early.}
	\missiontype{Wrong intel}{They gave us wrong information!}{Some part of mission planning falls flat because the information the group had was faulty. Maybe a compound has way more cameras than expected or it is defended by mercenaries rather than unpaid interns.}
%	Neutral
	\missiontype{Natural disaster}{Let the ground shake.}{During the mission a natural disaster strikes. This may be an earthquake or a typhoon but it could also be a pack of rabid radiophile bears or - if you don't like your players - a Cerberus outbreak.}
	\missiontype{Non-combatants}{Women and children.}{A neutral party - which is not seeking conflict - is present. Should all hell break loose, they are unprepared, panicked, extremely distracting and immoral to shoot.}
	\missiontype{Neutral combatants}{Domestic terrorists.}{A group of neutrals is present but they want in on the action or follow their own goals. These may be third-party mercenaries or literal terrorists. They crave violence and everyone is a valid target.}
%	Player side
	\missiontype{Enemy traitor}{The 13th among them.}{An enemy takes the player's side. This may be after some persuasion or she wanted to after having heard of the group's exploits and was just waiting for the opportunity.}
	\missiontype{Unexpected allies}{What are they doing here?}{One or more allies have joined in uninvited. They may be extremely helpful but could turn into a massive hindrance as well.}
%	Enemy side
	\missiontype{Reinforcements}{A million more well on the way.}{More enemies are on their way or very close by and ready to deploy. Being caught in the open is extremely dangerous now.}
	\missiontype{Additional enemies}{This is getting out of hand.}{A third party is present that is also hostile towards the player group. This may end in improved relationships between both enemy factions after their cooperation, giving both party access to more intel.}
	\missiontype{Relationship}{Of course I know him.}{An enemy is known to at least one player character. He could be an old friend, a brother or a lover.}
\end{enumerate}

%\section{Reward} % TODO: money and XP


	\chapter{Bestiary}
	\section{Hit Locations}
	Not all creatures have to look like humans necessarily; the list at the end of this chapter contains humanoids, quadrupeds, worms and hexapeds. Of course hit locations and location hp differ.
	\subsection{Humanoid}
\begin{minipage}{\columnwidth}
	\begin{tabularx}{\columnwidth}{|X|c|r|}
		\hline
		Locations & HP & Random \\ \hline
		Head      & 15 &   1-10 \\ \hline
		Right Arm & 15 &  11-25 \\ \hline
		Left Arm  & 15 &  26-40 \\ \hline
		Body      & 25 &  41-70 \\ \hline
		Right Leg & 15 &  71-85 \\ \hline
		Left Leg  & 15 & 86-100 \\ \hline
	\end{tabularx}
\end{minipage}

\subsection{Quadruped}
\begin{minipage}{\columnwidth}
	\begin{tabularx}{\columnwidth}{|X|c|r|}
		\hline
		Locations       & HP & Random \\ \hline
		Head            & 15 &   1-10 \\ \hline
		Body            & 25 &  11-60 \\ \hline
		Right Front Leg & 15 &  61-75 \\ \hline
		Left Front Leg  & 15 &  76-90 \\ \hline
		Right Hind Leg  & 15 &  91-95 \\ \hline
		Left Hind Leg   & 15 & 96-100 \\ \hline
	\end{tabularx}
\end{minipage}

\subsection{Winged Biped}
\begin{minipage}{\columnwidth}
	\begin{tabularx}{\columnwidth}{|X|c|r|}
		\hline
		Locations  & HP & Random \\ \hline
		Head       & 12 &   1-10 \\ \hline
		Right Wing & 12 &  11-35 \\ \hline
		Left Wing  & 12 &  36-60 \\ \hline
		Body       & 20 &  61-80 \\ \hline
		Right Leg  & 10 &  81-90 \\ \hline
		Left Leg   & 10 & 91-100 \\ \hline
	\end{tabularx}
\end{minipage}

\subsection{Winged Quadruped}
\begin{minipage}{\columnwidth}
	\begin{tabularx}{\columnwidth}{|X|c|r|}
		\hline
		Locations       & HP & Random \\ \hline
		Head            & 12 &    1-5 \\ \hline
		Right Wing      & 12 &   6-20 \\ \hline
		Left Wing       & 12 &  21-35 \\ \hline
		Body            & 20 &  36-60 \\ \hline
		Right Front Leg & 10 &  61-70 \\ \hline
		Left Front Leg  & 10 &  71-80 \\ \hline
		Right Hind Leg  & 10 &  81-90 \\ \hline
		Left Hind Leg   & 10 & 91-100 \\ \hline
	\end{tabularx}
\end{minipage}

\subsection{Headless Hexaped}
\begin{minipage}{\columnwidth}
	\begin{tabularx}{\columnwidth}{|X|c|r|}
		\hline
		Locations        & HP & Random \\ \hline
		Body             & 28 &   1-40 \\ \hline
		Right Front Leg  & 15 &  41-50 \\ \hline
		Left Front Leg   & 15 &  51-60 \\ \hline
		Right Middle Leg & 15 &  61-70 \\ \hline
		Left Middle Leg  & 15 &  71-80 \\ \hline
		Right Hind Leg   & 15 &  81-90 \\ \hline
		Left Hind Leg    & 15 & 91-100 \\ \hline
	\end{tabularx}
\end{minipage}

\subsection{Worm}
\begin{minipage}{\columnwidth}
	\begin{tabularx}{\columnwidth}{|X|c|r|}
		\hline
		Locations & HP & Random \\ \hline
		Head      & 20 &   1-20 \\ \hline
		Body      & 40 & 21-100 \\ \hline
	\end{tabularx}
\end{minipage}

\subsection{Turret}
\begin{minipage}{\columnwidth}
	\begin{tabularx}{\columnwidth}{|X|c|r|}
		\hline
		Locations        & HP & Random \\ \hline
		Targeting System & 10 &    1-3 \\ \hline
		Core             & 40 &   4-85 \\ \hline
		Weapon           & -  & 86-100 \\ \hline
	\end{tabularx}
\end{minipage}


	\section{Size}
	Larger creatures are more resilient on average. Mechanically this means that HP is multiplied by a factor depending on a creature's size category.\\
	Example: While an \emph{average} humanoid has 25 HP on his body, a \emph{huge} humanoid would have 200\% of that, i.e. 2 times, i.e. 50.
	\begin{tabularx}{\textwidth}{|l|r|}
		\hline
		Minuscule &  50\% \\ \hline
		Puny      &  66\% \\ \hline
		Tiny      &  75\% \\ \hline
		Small     &  88\% \\ \hline
		Average   & 100\% \\ \hline
		Tall      & 150\% \\ \hline
		Huge      & 200\% \\ \hline
		Hulking   & 250\% \\ \hline
		Enormous  & 300\% \\ \hline
	\end{tabularx}

	\section{Tags}
	Creatures may carry tags like \emph{ANIMAL}. Any tag means some collection of traits that these creatures share, whether they are generally similar or not. If you want to create your own creatures, start by adding a tag to an existing creature and reskinning it. Example:\par
	\begin{exampleblock}
		One might say that a bear [an existing creature] by itself is not all that interesting. A \emph{DRUGGED} bear [added a tag] sounds hilarious and much more mechanically engaging but might not exactly be taken seriously by many players. Going on to call it Subject B4R, a recent bio-weapon built by Mishima Zaibatsu for brutal, morale breaking close quarters engagements [reskin] might make for a very interesting plot point and tense encounter.
	\end{exampleblock}
	
	\vspace{5mm}
	{\setlength{\extrarowheight}{2pt}
\begin{tabularx}{\textwidth}{|l|X|}
	\hline
	Name & Effect
	\\ \hline
	ANIMAL & The creature behaves instinctually for most actions. It is assumed to be experienced in skills fitting their nature.
	\\ \hline
	DEADLY MELEE & Unarmed melee attacks performed by this creature deal D10 damage with 4 AP.
	\\ \hline
	DRUGGED & The creature fights under the effects of chemical enhancements.
	\\ \hline
	IMPERSONATION & The creature can appear like another being, be that through voice, text or other means.
	\\ \hline
	LEGION & The creature does not bleed and feels no pain. It does not need to breathe. It feels no fear nor effects of pinning. It shares its senses with other LEGION characters and grants +5 to other LEGION creatures performing combined actions with them.
	\\ \hline
	NODE & All other LEGION creatures within 100m have their intelligence increased by 20.
	\\ \hline
	SQUAD & The creature gains +5 courage when the group's leader is with them.
	\\ \hline
	VOLATILE & When this creature dies, it explodes.
	\\ \hline
	CERBERUS & The creature is infected with the fungus "Cerberus" and feels no pain and no fear. Additionally it can only be killed if the infected part (usually but not always the head) is completely destroyed.
\end{tabularx}\

	\section{Creatures}
	\def\bprefix{components/bestiary-}
\subsection{Civilian}
\luaimport{\bprefix Civilian.csvin}{beast.tpl}{components/civbestiary}

\subsection{Mob}
\luaimport{\bprefix Gangster.csvin}{beast.tpl}{components/mobbestiary}

\subsection{LEGION}
\luaimport{\bprefix Legion.csvin}{beast.tpl}{components/legionbestiary}

\subsection{Militant}
\luaimport{\bprefix Official.csvin}{beast.tpl}{components/milbestiary}

\subsection{Wildlife}
\luaimport{\bprefix Wildlife.csvin}{beast.tpl}{components/animalbestiary}

\subsection{Cerberus}
\luaimport{\bprefix Cerberus.csvin}{beast.tpl}{components/cerberusbestiary}


	\section{Hordes}
	Sometimes heroes stand against insurmountable swarms and hordes. Such hordes are created with the following few simple rules.
	
	\subsection*{Size}
	With numbers increasing, so increases the size of the horde, i.e. the space they take up. For every factor of 4 the size increases by one category.
	\begin{exampleblock}
		\itshape A single bat is puny, 4 bats are tiny, 16 (=4*4) are small and so on. When a horde decreases in size as it loses members is up to the GM.
	\end{exampleblock}
	
	\subsection*{Resilience}
	Swarms and hordes have vastly different vulnerabilities than single targets. For this reason we follow the rules below:
	\begin{itemize}
		\setlength\itemsep{-8mm} \vspace{-8mm}
		\item The horde's armor and toughness rating is the average armor and toughness of all members.
		\item If at least one point of damage is dealt to the horde, one member is incapacitated. There are however a few notable exceptions:
			\begin{itemize}
				\setlength\itemsep{-8mm} \vspace{-8mm}
				\item Weapons with the \emph{Blast} rule take out an amount of members equal to the Blast radius.
				\item \emph{Spray} weapons take out an amount of members equal to a third of their range in meters.
			\end{itemize}
	\end{itemize}
	
	\subsection*{Actions}
	In combat a horde acts as a single entity but gains 2 bonus actions every time its size increases. These bonus actions may be offensive, defensive or utility-driven. \\
	Note, however, that hordes never have to reload. They represent "a bunch of creatures", not professional 17th century firing lines.


	\chapter{Nemesis}
	Very special, personal enemies might be classified as \emph{Nemesis}. They should be few and far between and have a personal connection to one player character or even the whole party.\\
	To make them special mechanically as well they should gain 20-25 GP in Boons, 10-15 GP in Banes and one or two Nemesis Abilities. They may be chosen from the list below or made up.
	\section*{Nemesis Abilities}
	The following section contains a list of possible Nemesis abilities. It is by no means comprehensive.
	{\setlength{\extrarowheight}{5pt}
	\begin{tabularx}{\textwidth}{l|X}
		Ability & Effect\\
		\hline
		\textbf{Blooded} & The Nemesis is incredibly jaded and resilient. He ignores any pain and fear.\\
		\hline
		\textbf{Charismatic} & The Nemesis is charismatic and kind to his underlings. Anyone not hostile towards him will support him in a fight to the best of their abilities.\\
		\hline
		\textbf{Conniving} & The Nemesis is prepared for everything. Once per day when failing a test, he may decide to pass with D5 DoS instead, as he planned for exactly this occasion. If his opponent does something absolutely bananas, the GM may rule even this ability to be outclassed.\\
		\hline
		\textbf{Cruel} & The Nemesis is a heartless individual. Once per encounter he may make an Intimidate test, causing fear in its opponents. Each time a character sees him after the first, the test gains cumulative a bonus of +10.\\
		\hline
		\textbf{Fanatical} & The Nemesis is obsessed and nothing will stop him. He can ignore any and all modifiers that directly distract him from his goal, be that pain, exhaustion, sleep deprivation or someone talking at him while he's driving. While he is blinded by that goal however, his Instinct is reduced by 20.\\
		\hline
		\textbf{Foresightful} & The Nemesis can almost see the future. His instinct is increased by 10 and deceptive maneuvers like \emph{Faints} don't work against him, nor will he ever lose initiative from dirty tricks.\\
		\hline
		\textbf{Honor-bound} & The Nemesis will never stab someone in the back, in turn he has become a great duelist. While in a one-on-one fight - be that melee or shootout - he gains a +15 to combat tests. Otherwise he will gain a -10 to combat tests.\\
		\hline
		\textbf{Masked} & The Nemesis' facial features are very unremarkable and he knows to use this as an advantage. When the Nemesis dies and the body is not closely investigated, it may turn out later that the corpse is just a double.\\
		\hline
		\textbf{Reputable} & The Nemesis is generally well regarded; this means that characters who speak out against him at the wrong place or the wrong time will quite possibly suffer dire consequences.
	\end{tabularx}}
	\section*{Keep In Mind}
	Horde rules may work for every type of creature - technically. However, your miles may vary if you use these rules for smart or extremely large creatures. Hordes are much flimsier and - in a way - less dangerous than single creatures.

	\chapter{Patrons \& Contacts}
	On the opposite side of the Nemesis lay the contacts and patrons - the powerful allies. Usually their individual stats are not that important, it is what they represent: influence, knowledge and goods.
	\section*{Personality}
	We cannot create a character without a personality, so we will start here. There are two basic options to create a personality.\\
	Firstly we can answer the same 20 questions as the player did for their characters. This process takes a long time but is also incredibly precise and will make for a very fleshed out character. This may be a good choice if the player characters are sent on their way time and time again by the same person.
	Alternatively we can go through the following simple steps. It is much faster but possible contradictory.
	\begin{enumerate}
		\setlength\itemsep{-8mm} \vspace{-8mm}
		\item \textbf{Specialties}: What can the character do extremely well and what does she like to do? Anything can work here but don't overdo it.
		\item \textbf{Iniquity}: Under which vice does she (moreso her surroundings) suffer? An easy source are the seven deadly sins, alternatively find something or \emph{someone} she is afraid of.
		\item \textbf{Quirk}: Some quirk makes a character come alive. A speech disorder, an obsession, a physical dependency on chewing gum. It doesn't have to be actually impactful, it just has to be human.
		\item \textbf{Goal}: What goals does the patron pursue? She will not offer assets that conflict with her goal or be reluctant when the deal seems too good.
	\end{enumerate}
	\section*{Assets}
	The assets they offer is often the actual first step in creating a patron as it is the reason to create such a patron. Examples include:
	\begin{enumerate}
		\setlength\itemsep{-8mm} \vspace{-8mm}
		\item Transportation
		\item Black Market Access
		\item Medical Assistance
		\item Intel
		\item Firepower
	\end{enumerate}
	\section*{Costs}
	Tying directly into the assets offered: what does it cost the player characters? Again, many things can work here, e.g. money, protection, favors.
	\begin{exampleblock}
		\itshape A mob boss selling forbidden goods will expect large amounts of money for their services.\par
		A hacker known to one of the player characters will not necessarily want to get paid but may make dangerous enemies that he needs protection from.
	\end{exampleblock}
\end{document}