\documentclass[12pt,a4paper,openany]{book}
%Custom Environments
\newenvironment{exampleblock}[1][1]
{
	\par
	\vspace{-5mm}
	\hfill
	\begin{minipage}
		{\dimexpr\columnwidth-#1cm}
	\begin{mdframed}[
		backgroundcolor=Gray!65,
		rightline=false,
		topline=false
		]
}{
	\end{mdframed}
	\end{minipage}
	\par
}

%%shortened itemize
\newenvironment{sitemize}[1][10]
{
	\begin{itemize}
	\vspace{-#1mm}
	\setlength\itemsep{-#1mm}
}{
	\end{itemize}
}

%Custom Commands
\newcommand{\ul}[1]{\underline{\smash{#1}}}
\newcommand{\breakline}{\vspace{.5cm} \hrule width \columnwidth \relax}
\newcommand{\derivedvalue}[3]{
	\begin{samepage}
	\subsubsection{#1 \textsubscript{\textlangle#2\textrangle}}
	\hfill
	\begin{minipage}{\dimexpr\columnwidth-1cm}
		#3
	\end{minipage}
	\end{samepage}
	\par
}
\newcommand{\specialrule}[2]{
	\begin{minipage}{\columnwidth}
		\textbf{\ul{#1:}}\\
		#2
	\end{minipage}
    \par
}
\newcommand{\supply}[4]{
	\begin{minipage}{\columnwidth}
		\textbf{\ul{#1}}\\
		\textit{Price: #3}\\
		\textit{Weight: #4}\\
		#2
	\end{minipage}
    \par
}
\newcommand{\service}[4]{
	\begin{minipage}{\columnwidth}
		\textbf{\ul{#1}}\\
		#2\\
		\textit{Cost:} #3 per #4
	\end{minipage}
	\par
}
\newcommand{\statuseffect}[2]{
	\begin{minipage}{\columnwidth}
		\textbf{\ul{#1:}}\\
		#2\\
	\end{minipage}
	\vspace{5mm}
}
\newcommand{\skill}[4][Basic]{
	\begin{minipage}{\columnwidth}
		\textbf{\ul{#2}}\\
		Difficulty: #1\\
		Common Characteristic: #3\\
%		Description:\\
		\textit{#4}\\
	\end{minipage}\par
}
\newcommand{\melee}[2]{\skill{#1}{Melee Base}{#2}}
\newcommand{\ranged}[2]{\skill{#1}{Ranged Base}{#2}}
\newcommand{\ability}[4]{
	\begin{minipage}{\columnwidth}
		\textbf{\ul{#1}} (#2 XP)\\
		\textit{Prerequisites}: #3\\
		\textit{Effect}:\\
		#4\\
	\end{minipage}
}
\newcommand{\maneuver}[3]{
	\begin{minipage}{\columnwidth}
		\textbf{\ul{MVR: #1}} (#2 XP)\\
		\textit{Prerequisites: #3\\}
		Effect:\\
		Allows the use of the maneuver \emph{#1} at normal penalties.\\
	\end{minipage}
}
\newcommand{\boon}[4][]{
	\begin{minipage}{\columnwidth}
		\label{boon::#2}
		\textbf{\ul{\smash{#2}}} \mbox{(#4 GP\ifthenelse{\isempty{#1}}{}{ }#1)}\\
		#3
	\end{minipage}
	\par
}
\newcommand{\bane}[4]{
	\boon[#4]{#1}{#2}{#3}
}
\newcommand{\rangedweapon}[9]{
	\vspace{2mm}
	\begin{minipage}{\columnwidth}
		\textbf{\ul{#1}}\\
		\textit{#2}\\
		\textbf{Weight}: #9\\
		\textbf{Price}: #8\\
		#3\\
		\textbf{Mag}: #5\\
		\textbf{Reload}: \mbox{#6 actions}\\
		\textbf{Range}: #4\\
		\ifthenelse{\isempty{#7}}{}{\textbf{Special Rules}: #7}
	\end{minipage}
	\par
}
\newcommand{\weaponmod}[5]{
	\begin{minipage}{\columnwidth}
		\textbf{\ul{#1}}\\
		\textit{#3}; \textit{#4}; \textit{#5}\\
		#2
	\end{minipage}
	\par
}
\newcommand{\meleecomponent}[6]{
	\begin{minipage}{\columnwidth}
		\textbf{\ul{#1:}}\\
		\textit{#2}\\
		\textbf{Weight}: #4; \textbf{Price}: #5\ifthenelse{\isempty{#6}}{}{; \textbf{Requirement}: #6}\\
		\textbf{Effect}: #3
	\end{minipage}
	\par
}
\newcommand{\ammo}[6]{
	\begin{minipage}{\columnwidth}
		\textbf{\ul{#1:}} \textit{#2}\\
		\textbf{Price}: #3; \textbf{Unit of sale}: #4\\
		\textbf{Weight/Bulk}: #5 \textit{\ifthenelse{\isempty{#6}}{each}{per #6}}
	\end{minipage}
	\par
}
\newcommand{\armor}[9]{
	\begin{minipage}{\columnwidth}
		\textbf{\ul{#1}} (covers: #6)\\
		\begin{tabular}{|r|r|r|r|}
			\hline
			Head & Chest & Arms & Legs\\
			\hline
			#2 & #3 & #4 & #5\\
			\hline
		\end{tabular}\par
		\vspace{2mm}
		\textit{Price:} #8; \textit{Weight:} #9\\
		#7
	\end{minipage}
	\par
}
\newcommand{\armormod}[6]{
	\begin{minipage}{\columnwidth}
		\textbf{\ul{#1}}\\
		#2\\
		\textit{Price:} #3; \textit{Weight:} #6\\
		\textit{Mod points:} #4\ifthenelse{\isequivalentto{N}{#5}}{}{; Requires power source}
	\end{minipage}
	\par
}
\newcommand{\pes}[6]{
	\begin{minipage}{\columnwidth}
		\textbf{\ul{#1}}\\
		Armor: #2; Threshold: #3\\
		\textit{Price:} #4; \textit{Weight:} #5
		\ifthenelse{\isempty{#6}}{}{\\} %conditional line break
		#6
	\end{minipage}
	\par
}
\newcommand{\implant}[7]{
	\begin{minipage}{\columnwidth}
		\textbf{#1}\\
		\textit{#2}\\
		\textit{Effect}: #3\\
		\textit{Price:} #6; \textit{Load:} #5\ifthenelse{\isequivalentto{-}{#4}}{}{; \textit{Slot:} #4}\\
		\textit{Available Mods:} #7
	\end{minipage}
	\par
}
\newcommand{\augmod}[3]{
	\begin{minipage}{\columnwidth}
		\textbf{#1}\\
		#2\\
		\textit{Cost:} #3
	\end{minipage}
	\par
}
\newcommand{\mod}[2]{\item \textbf{#1}: #2}
\newcommand{\psicomponent}[4]{\textbf{#1} & #2 & #3 & #4 \\}

%filler images
\newcommand{\filltopageendgraphics}[2][]{%
	\par
	\zsaveposy{top-\thepage}% Mark (baseline of) top of image
	\vfill
	\zsaveposy{bottom-\thepage}% Mark (baseline of) bottom of image
	\smash{\includegraphics[height=\dimexpr\zposy{top-\thepage}sp-\zposy{bottom-\thepage}sp\relax,#1]{#2}}%
	\par
}

%Base Building
\newcommand{\baselocation}[6]{
	\begin{minipage}{\columnwidth}
		\textbf{#1}\\
		Size: #3; Concealment: #4; Defense: #5\\
		Conditions: #6\\
		\textit{Cost:} #2
	\end{minipage}
	\par
}
\newcommand{\baseasset}[6]{
	\begin{minipage}{\columnwidth}
		\textbf{#1}\\
		\textit{#2}\\
		Size: #3\\
		Concealment: #4\\
		Defense: #5\\
		\textit{Cost:} #6
	\end{minipage}
	\par
}

%Narrative
\newcommand{\nrule}[3]{
	\begin{minipage}{\columnwidth}
		\section{#1}
		\textit{#2}\\
		\vspace{8mm}
		\begin{exampleblock}
			#3
		\end{exampleblock}
	\end{minipage}
	\par
}

%GM
\newcommand{\missiontype}[3]{\item \textbf{#1}: \textit{#2} #3}

\usepackage[utf8]{inputenc}
\usepackage[english]{babel}
\usepackage{textcomp}
\usepackage{xifthen}
\usepackage{tabularx}
\usepackage{tabto}
\usepackage{multicol}

\usepackage[bookmarks=true,colorlinks=true,linkcolor=cyan]{hyperref}

%Alignment
\usepackage[skip=10mm]{parskip}
\raggedbottom

% Title Image
\usepackage{wallpaper}
\def\coverimgpath{../art/\@title/cover}

\usepackage{ltablex}
%Title
\title{Neon Pen and Paper - GM Companion}
\author{Fr0sty}
\date{\today}

\begin{document}
	\maketitle
	\tableofcontents
	
	\chapter{Specialties}
%	What differenciates Neon from other PnP games? How do you deal with that as a GM?
%	Lethality
%	Modularity
	
	\chapter{Mission Creation}
%	Models: Technoir, Halo Mythic
	
	\chapter{Bestiary}
	\section{Hit Locations}
	Not all creatures have to look like humans necessarily; the list at the end of this chapter contains humanoids, quadrupeds, worms and hexapeds. Of course hit locations and location hp differ.
	\subsection{Humanoid}
\begin{minipage}{\columnwidth}
	\begin{tabularx}{\columnwidth}{|X|c|r|}
		\hline
		Locations & HP & Random \\ \hline
		Head      & 15 &   1-10 \\ \hline
		Right Arm & 15 &  11-25 \\ \hline
		Left Arm  & 15 &  26-40 \\ \hline
		Body      & 25 &  41-70 \\ \hline
		Right Leg & 15 &  71-85 \\ \hline
		Left Leg  & 15 & 86-100 \\ \hline
	\end{tabularx}
\end{minipage}

\subsection{Quadruped}
\begin{minipage}{\columnwidth}
	\begin{tabularx}{\columnwidth}{|X|c|r|}
		\hline
		Locations       & HP & Random \\ \hline
		Head            & 15 &   1-10 \\ \hline
		Body            & 25 &  11-60 \\ \hline
		Right Front Leg & 15 &  61-75 \\ \hline
		Left Front Leg  & 15 &  76-90 \\ \hline
		Right Hind Leg  & 15 &  91-95 \\ \hline
		Left Hind Leg   & 15 & 96-100 \\ \hline
	\end{tabularx}
\end{minipage}

\subsection{Winged Biped}
\begin{minipage}{\columnwidth}
	\begin{tabularx}{\columnwidth}{|X|c|r|}
		\hline
		Locations  & HP & Random \\ \hline
		Head       & 12 &   1-10 \\ \hline
		Right Wing & 12 &  11-35 \\ \hline
		Left Wing  & 12 &  36-60 \\ \hline
		Body       & 20 &  61-80 \\ \hline
		Right Leg  & 10 &  81-90 \\ \hline
		Left Leg   & 10 & 91-100 \\ \hline
	\end{tabularx}
\end{minipage}

\subsection{Winged Quadruped}
\begin{minipage}{\columnwidth}
	\begin{tabularx}{\columnwidth}{|X|c|r|}
		\hline
		Locations       & HP & Random \\ \hline
		Head            & 12 &    1-5 \\ \hline
		Right Wing      & 12 &   6-20 \\ \hline
		Left Wing       & 12 &  21-35 \\ \hline
		Body            & 20 &  36-60 \\ \hline
		Right Front Leg & 10 &  61-70 \\ \hline
		Left Front Leg  & 10 &  71-80 \\ \hline
		Right Hind Leg  & 10 &  81-90 \\ \hline
		Left Hind Leg   & 10 & 91-100 \\ \hline
	\end{tabularx}
\end{minipage}

\subsection{Headless Hexaped}
\begin{minipage}{\columnwidth}
	\begin{tabularx}{\columnwidth}{|X|c|r|}
		\hline
		Locations        & HP & Random \\ \hline
		Body             & 28 &   1-40 \\ \hline
		Right Front Leg  & 15 &  41-50 \\ \hline
		Left Front Leg   & 15 &  51-60 \\ \hline
		Right Middle Leg & 15 &  61-70 \\ \hline
		Left Middle Leg  & 15 &  71-80 \\ \hline
		Right Hind Leg   & 15 &  81-90 \\ \hline
		Left Hind Leg    & 15 & 91-100 \\ \hline
	\end{tabularx}
\end{minipage}

\subsection{Worm}
\begin{minipage}{\columnwidth}
	\begin{tabularx}{\columnwidth}{|X|c|r|}
		\hline
		Locations & HP & Random \\ \hline
		Head      & 20 &   1-20 \\ \hline
		Body      & 40 & 21-100 \\ \hline
	\end{tabularx}
\end{minipage}

\subsection{Turret}
\begin{minipage}{\columnwidth}
	\begin{tabularx}{\columnwidth}{|X|c|r|}
		\hline
		Locations        & HP & Random \\ \hline
		Targeting System & 10 &    1-3 \\ \hline
		Core             & 40 &   4-85 \\ \hline
		Weapon           & -  & 86-100 \\ \hline
	\end{tabularx}
\end{minipage}


	\section{Size}
	Larger creatures are more resilient on average. Mechanically this means that HP is multiplied by a factor depending on a creature's size category.\\
	Example: While an \emph{average} humanoid has 25 HP on his body, a \emph{huge} humanoid would have 200\% of that, i.e. 2 times, i.e. 50.
	\begin{tabularx}{\textwidth}{|l|r|}
		\hline
		Minuscule & 50\%\\
		\hline
		Puny & 66\%\\
		\hline
		Tiny & 75\%\\
		\hline
		Small & 88\%\\
		\hline
		Average & 100\%\\
		\hline
		Tall & 150\%\\
		\hline
		Huge & 200\%\\
		\hline
		Hulking & 250\%\\
		\hline
		Enormous & 300\%\\
		\hline
	\end{tabularx}
	
	\chapter{Nemesis}
	Very special, personal enemies might be classified as \emph{Nemesis}. They should be few and far between and have a personal connection to one player character or even the whole party.\\
	To make them special mechanically as well they should gain 20-25 GP in Boons, 10-15 GP in Banes and one or two Nemesis Abilities. They may be chosen from the list below or made up.
	\section*{Nemesis Abilities}
	The following section contains a list of possible Nemesis abilities. It is by no means comprehensive.
	{\setlength{\extrarowheight}{5pt}
	\begin{tabularx}{\textwidth}{l|X}
		Ability & Effect\\
		\hline
		\textbf{Blooded} & The Nemesis is incredibly jaded and resilient. He ignores any pain and fear.\\
		\hline
		\textbf{Caring} & The Nemesis is charismatic and kind to his underlings. Anyone not hostile towards him will support him in a fight to the best of their abilities.\\
		\hline
		\textbf{Conniving} & The Nemesis is prepared for everything. Once per day when failing a test, he may decide to pass with D5 DoS instead, as he planned for exactly this occasion. If his opponent does something absolutely bananas, the GM may rule even this ability to be outclassed.\\
		\hline
		\textbf{Cruel} & The Nemesis is a heartless individual. Once per encounter he may make an Intimidate test, causing fear in its opponents. Each time a character sees him after the first, the test gains cumulative a bonus of +10.\\
		\hline
		\textbf{Fanatical} & The Nemesis is obsessed and nothing will stop him. He can ignore any and all modifiers that directly distract him from his goal, be that pain, exhaustion, sleep deprivation or someone talking at him while he's driving. While he is blinded by that goal however, his Instinct is reduced by 20.\\
		\hline
		\textbf{Foresightful} & The Nemesis can almost see the future. His instinct is increased by 10 and deceptive maneuvers like \emph{Faints} don't work against him, nor will he ever lose initiative from dirty tricks.\\
		\hline
		\textbf{Honor-bound} & The Nemesis will never stab someone in the back, in turn he has become a great duelist. While in a one-on-one fight - be that melee or shootout - he gains a +15 to combat tests. Otherwise he will gain a -10 to combat tests.\\
		\hline
		\textbf{Masked} & The Nemesis' facial features are very unremarkable and he knows to use this as an advantage. When the Nemesis dies and the body is not closely investigated, it may turn out later that the corpse is just a double.\\
		\hline
		\textbf{Reputable} & The Nemesis is generally well regarded; this means that characters who speak out against him at the wrong place or the wrong time will quite possibly suffer dire consequences.\\
		\hline
		
	\end{tabularx}}
\end{document}