%	Models: Technoir, Halo Mythic
\section{Introduction}
This chapter is dedicated to making mission creation as easy as possible. 

\section{Goal}
The term "mission" describes a task within a campaign, some type of intermediate step to reach some end goal. Without an end goal in mind, there are no stakes and there is no investment. \par
Every mission has some goal. If that goal is reached, made obsolete or failed irredeemably, the mission ends.\par
Goals depend on the characters undertaking the mission. If they are personally involved in creating the mission goal, they will be more engaged.

\section{Mission Type}
Some types of missions are extremely common. Beware that some mission types may be more or less compatible with certain goals but anything can be made to work.
\begin{enumerate}
	\setlength\itemsep{-5mm}
	\missiontype{Assassination}{Take out one or multiple VIPs.}{They don't necessarily need to be killed but they cannot cross the group again.}
	\missiontype{Demolition}{Take out a key location.}{Factories and supply routes are common targets but so could be safe houses and data centers.}
	\missiontype{Raze}{Take out a whole location.}{Move in, take out hostile forces, destroy the place. Never been simpler.}
	\missiontype{Raid}{Steal supplies or evidence in a lightning strike.}{The concept of a raid is pretty obvious. Just don't overstay your welcome, the enemy will receive reinforcements.}
	\missiontype{Secure}{Rescue a person.}{Retrieve some ally from captivity by force.}
	\missiontype{Protection}{Defend a location or escort a person.}{Make sure that no harm comes to the target.}
	\missiontype{Diversion}{Distract a group from another event.}{A second group is performing another mission - be that a raid or an assassination - and they require some space and time.}
	\missiontype{Espionage}{Gather information.}{Information can be gathered from various sources, including people and records. Wherever they come from, we need them.}
	\missiontype{Locate}{Find a person or an object.}{An item was lost or a person has gone into hiding. We need to know where to.}
	\missiontype{Retrieve}{Collect an item.}{Lost, stolen or simply of interest; move in, gather the item, get it to its new owner.}
\end{enumerate}

\section{Opposition}
The faction working against the party should be obvious at the time of mission planning. However a single faction may employ very different forces, varying in level of expertise and equipment quality. A rough selection looks like the following.
\begin{enumerate}
	\setlength\itemsep{-5mm}
	\missiontype{Goons}{Uncoordinated, badly equipped but cheap.}{Anyone could belong to this category depending on circumstances: laborers, paid or threatened goons, soldiers on leave on multiple drugs. They should have very low stats, be unprepared and relatively high in number.}
	\missiontype{Hit Squad}{Targeted but still cheap.}{We're not talking about professional hitmen here. These are still just goons but are given a target and some time to prepare. They understand basic tactics such as cover and crossfire.}
	\missiontype{Mercenaries}{Professional guns for hire.}{Enemies like this understand tactics and are well-versed in a variety of combat situations. They are prepared and work well together.}
	\missiontype{Officials}{Great defenders.}{Officials are paid to keep law and order. They are usually great in defense scenarios or against barricaded suspects. Usually they have intelligence and are prepared for their aggressors.}
	\missiontype{Headhunters}{Lethal stealth attacks.}{Headhunters are incredibly lethal and usually work only in very small teams. They know their target very well and their target doesn't know them.}
\end{enumerate}

\section{Twist}
No plan survives first contact with the enemy, known for hundreds of years. Every mission should have some sort of twist, possibly multiple. A selection of possible twists can be found below.
\begin{enumerate}
	\setlength\itemsep{-5mm}
%	Mission information
	\missiontype{Timer}{The mission is on the clock.}{Likely the most obvious twist at first. Some condition puts a time constraint on the mission. This may be that an assassination target is leaving the country or a homeowner is returning early.}
	\missiontype{Wrong intel}{They gave us wrong information!}{Some part of mission planning falls flat because the information the group had was faulty. Maybe a compound has way more cameras than expected or it is defended by mercenaries rather than unpaid interns.}
%	Neutral
	\missiontype{Natural disaster}{Let the ground shake.}{During the mission a natural disaster strikes. This may be an earthquake or a typhoon but it could also be a pack of rabid radiophile bears or - if you don't like your players - a Cerberus outbreak.}
	\missiontype{Non-combatants}{Women and children.}{A neutral party - which is not seeking conflict - is present. Should all hell break loose, they are unprepared, panicked, extremely distracting and immoral to shoot.}
	\missiontype{Neutral combatants}{Domestic terrorists.}{A group of neutrals is present but they want in on the action or follow their own goals. These may be third-party mercenaries or literal terrorists. They crave violence and everyone is a valid target.}
%	Player side
	\missiontype{Enemy traitor}{The 13th among them.}{An enemy takes the player's side. This may be after some persuasion or she wanted to after having heard of the group's exploits and was just waiting for the opportunity.}
	\missiontype{Unexpected allies}{What are they doing here?}{One or more allies have joined in uninvited. They may be extremely helpful but could turn into a massive hindrance as well.}
%	Enemy side
	\missiontype{Reinforcements}{A million more well on the way.}{More enemies are on their way or very close by and ready to deploy. Being caught in the open is extremely dangerous now.}
	\missiontype{Additional enemies}{This is getting out of hand.}{A third party is present that is also hostile towards the player group. This may end in improved relationships between both enemy factions after their cooperation, giving both party access to more intel.}
	\missiontype{Relationship}{Of course I know him.}{An enemy is known to at least one player character. He could be an old friend, a brother or a lover.}
\end{enumerate}

%\section{Reward} % TODO: money and XP
