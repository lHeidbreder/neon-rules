\documentclass[12pt,a4paper,openany]{book}
%Custom Environments
\newenvironment{exampleblock}[1][1]
{
	\par
	\vspace{-5mm}
	\hfill
	\begin{minipage}
		{\dimexpr\columnwidth-#1cm}
	\begin{mdframed}[
		backgroundcolor=Gray!65,
		rightline=false,
		topline=false
		]
}{
	\end{mdframed}
	\end{minipage}
	\par
}

%%shortened itemize
\newenvironment{sitemize}[1][10]
{
	\begin{itemize}
	\vspace{-#1mm}
	\setlength\itemsep{-#1mm}
}{
	\end{itemize}
}

%Custom Commands
\newcommand{\ul}[1]{\underline{\smash{#1}}}
\newcommand{\breakline}{\vspace{.5cm} \hrule width \columnwidth \relax}
\newcommand{\derivedvalue}[3]{
	\begin{samepage}
	\subsubsection{#1 \textsubscript{\textlangle#2\textrangle}}
	\hfill
	\begin{minipage}{\dimexpr\columnwidth-1cm}
		#3
	\end{minipage}
	\end{samepage}
	\par
}
\newcommand{\specialrule}[2]{
	\begin{minipage}{\columnwidth}
		\textbf{\ul{#1:}}\\
		#2
	\end{minipage}
    \par
}
\newcommand{\supply}[4]{
	\begin{minipage}{\columnwidth}
		\textbf{\ul{#1}}\\
		\textit{Price: #3}\\
		\textit{Weight: #4}\\
		#2
	\end{minipage}
    \par
}
\newcommand{\service}[4]{
	\begin{minipage}{\columnwidth}
		\textbf{\ul{#1}}\\
		#2\\
		\textit{Cost:} #3 per #4
	\end{minipage}
	\par
}
\newcommand{\statuseffect}[2]{
	\begin{minipage}{\columnwidth}
		\textbf{\ul{#1:}}\\
		#2\\
	\end{minipage}
	\vspace{5mm}
}
\newcommand{\skill}[4][Basic]{
	\begin{minipage}{\columnwidth}
		\textbf{\ul{#2}}\\
		Difficulty: #1\\
		Common Characteristic: #3\\
%		Description:\\
		\textit{#4}\\
	\end{minipage}\par
}
\newcommand{\melee}[2]{\skill{#1}{Melee Base}{#2}}
\newcommand{\ranged}[2]{\skill{#1}{Ranged Base}{#2}}
\newcommand{\ability}[4]{
	\begin{minipage}{\columnwidth}
		\textbf{\ul{#1}} (#2 XP)\\
		\textit{Prerequisites}: #3\\
		\textit{Effect}:\\
		#4\\
	\end{minipage}
}
\newcommand{\maneuver}[3]{
	\begin{minipage}{\columnwidth}
		\textbf{\ul{MVR: #1}} (#2 XP)\\
		\textit{Prerequisites: #3\\}
		Effect:\\
		Allows the use of the maneuver \emph{#1} at normal penalties.\\
	\end{minipage}
}
\newcommand{\boon}[4][]{
	\begin{minipage}{\columnwidth}
		\label{boon::#2}
		\textbf{\ul{\smash{#2}}} \mbox{(#4 GP\ifthenelse{\isempty{#1}}{}{ }#1)}\\
		#3
	\end{minipage}
	\par
}
\newcommand{\bane}[4]{
	\boon[#4]{#1}{#2}{#3}
}
\newcommand{\rangedweapon}[9]{
	\vspace{2mm}
	\begin{minipage}{\columnwidth}
		\textbf{\ul{#1}}\\
		\textit{#2}\\
		\textbf{Weight}: #9\\
		\textbf{Price}: #8\\
		#3\\
		\textbf{Mag}: #5\\
		\textbf{Reload}: \mbox{#6 actions}\\
		\textbf{Range}: #4\\
		\ifthenelse{\isempty{#7}}{}{\textbf{Special Rules}: #7}
	\end{minipage}
	\par
}
\newcommand{\weaponmod}[5]{
	\begin{minipage}{\columnwidth}
		\textbf{\ul{#1}}\\
		\textit{#3}; \textit{#4}; \textit{#5}\\
		#2
	\end{minipage}
	\par
}
\newcommand{\meleecomponent}[6]{
	\begin{minipage}{\columnwidth}
		\textbf{\ul{#1:}}\\
		\textit{#2}\\
		\textbf{Weight}: #4; \textbf{Price}: #5\ifthenelse{\isempty{#6}}{}{; \textbf{Requirement}: #6}\\
		\textbf{Effect}: #3
	\end{minipage}
	\par
}
\newcommand{\ammo}[6]{
	\begin{minipage}{\columnwidth}
		\textbf{\ul{#1:}} \textit{#2}\\
		\textbf{Price}: #3; \textbf{Unit of sale}: #4\\
		\textbf{Weight/Bulk}: #5 \textit{\ifthenelse{\isempty{#6}}{each}{per #6}}
	\end{minipage}
	\par
}
\newcommand{\armor}[9]{
	\begin{minipage}{\columnwidth}
		\textbf{\ul{#1}} (covers: #6)\\
		\begin{tabular}{|r|r|r|r|}
			\hline
			Head & Chest & Arms & Legs\\
			\hline
			#2 & #3 & #4 & #5\\
			\hline
		\end{tabular}\par
		\vspace{2mm}
		\textit{Price:} #8; \textit{Weight:} #9\\
		#7
	\end{minipage}
	\par
}
\newcommand{\armormod}[6]{
	\begin{minipage}{\columnwidth}
		\textbf{\ul{#1}}\\
		#2\\
		\textit{Price:} #3; \textit{Weight:} #6\\
		\textit{Mod points:} #4\ifthenelse{\isequivalentto{N}{#5}}{}{; Requires power source}
	\end{minipage}
	\par
}
\newcommand{\pes}[6]{
	\begin{minipage}{\columnwidth}
		\textbf{\ul{#1}}\\
		Armor: #2; Threshold: #3\\
		\textit{Price:} #4; \textit{Weight:} #5
		\ifthenelse{\isempty{#6}}{}{\\} %conditional line break
		#6
	\end{minipage}
	\par
}
\newcommand{\implant}[7]{
	\begin{minipage}{\columnwidth}
		\textbf{#1}\\
		\textit{#2}\\
		\textit{Effect}: #3\\
		\textit{Price:} #6; \textit{Load:} #5\ifthenelse{\isequivalentto{-}{#4}}{}{; \textit{Slot:} #4}\\
		\textit{Available Mods:} #7
	\end{minipage}
	\par
}
\newcommand{\augmod}[3]{
	\begin{minipage}{\columnwidth}
		\textbf{#1}\\
		#2\\
		\textit{Cost:} #3
	\end{minipage}
	\par
}
\newcommand{\mod}[2]{\item \textbf{#1}: #2}
\newcommand{\psicomponent}[4]{\textbf{#1} & #2 & #3 & #4 \\}

%filler images
\newcommand{\filltopageendgraphics}[2][]{%
	\par
	\zsaveposy{top-\thepage}% Mark (baseline of) top of image
	\vfill
	\zsaveposy{bottom-\thepage}% Mark (baseline of) bottom of image
	\smash{\includegraphics[height=\dimexpr\zposy{top-\thepage}sp-\zposy{bottom-\thepage}sp\relax,#1]{#2}}%
	\par
}

%Base Building
\newcommand{\baselocation}[6]{
	\begin{minipage}{\columnwidth}
		\textbf{#1}\\
		Size: #3; Concealment: #4; Defense: #5\\
		Conditions: #6\\
		\textit{Cost:} #2
	\end{minipage}
	\par
}
\newcommand{\baseasset}[6]{
	\begin{minipage}{\columnwidth}
		\textbf{#1}\\
		\textit{#2}\\
		Size: #3\\
		Concealment: #4\\
		Defense: #5\\
		\textit{Cost:} #6
	\end{minipage}
	\par
}

%Narrative
\newcommand{\nrule}[3]{
	\begin{minipage}{\columnwidth}
		\section{#1}
		\textit{#2}\\
		\vspace{8mm}
		\begin{exampleblock}
			#3
		\end{exampleblock}
	\end{minipage}
	\par
}

%GM
\newcommand{\missiontype}[3]{\item \textbf{#1}: \textit{#2} #3}

\usepackage[utf8]{inputenc}
\usepackage[english]{babel}
\usepackage{textcomp}
\usepackage{xifthen}
\usepackage{tabularx}
\usepackage{tabto}
\usepackage{multicol}

\usepackage[bookmarks=true,colorlinks=true,linkcolor=cyan]{hyperref}

%Alignment
\usepackage[skip=10mm]{parskip}
\raggedbottom

% Title Image
\usepackage{wallpaper}
\def\coverimgpath{../art/\@title/cover}

\def\subtitle{AI}

\begin{document}
	{\heading
\ThisCenterWallPaper{1}{\coverimgpath}
\maketitle}
{\hypersetup{hidelinks} \tableofcontents}


	\chapter{Introduction}
	Playing AI is in many ways very different from playing a human. Most AI has limited free will and creative thinking - and most importantly no physical body. To personify artificial intelligence, most AI has some imagery associated with it at the time of its creation. This can be only a head, a full human body model or a talking horse - there are very few limitations to the creativity of mad scientists and marketing departments.

	\chapter{Beyond Human}
	AI is at its core more familiar with machines than human minds. Their mental characteristics have some additional meaning:
	\begin{itemize}
		\item Courage also encompasses the extent of their free will.
		\item Intelligence is logical thinking and understanding of the machine world.
		\item Instinct additionally represents their comprehension of human minds and emotions.
	\end{itemize}

	As an AI has no body of their own, its physical characteristics vary vastly in meaning:
	\begin{itemize}
		\item Dexterity represents their precision with small peripheral tools like drills or surgical instruments.
		\item Agility constitutes their precision with large peripheral tools like cranes.
		\item Constitution is their resistance to cyber attacks in addition to whatever firewall the device might have and how easy they are to break. Complexity works in its favor just as much as redundancies and monitoring measures.
		\item Strength of an AI is not physical but its ability to use the device’s processing power efficiently. Denial of service attacks e.g. will make use of this.
	\end{itemize}

	Computer Operation, being their nature, becomes a basic skill. Many other skills and derived values are simply useless to an AI without particular specialized equipment. If the AI is using such equipment, its actual physical characteristics should be measured by the capabilities of said equipment.

	\chapter{Untouchable}
	An AI is not a physical being, it is much more a bundle of processes and associated memory. This has many implications:
	\begin{itemize}
		\item An AI does not have HP. Instead it uses the device’s integrity.
		\item An AI does not have senses. Instead it uses sensors and camera feeds.
		\item An AI does not have a voice itself. It requires necessary output devices to communicate with humans.
		\item An AI does not earn money. Starting capital still exists, either to buy a hardware device for the AI or to share with other party members. Earnings after missions shouldn’t be reduced but then split among the other characters.
	\end{itemize}

	\chapter{Limited Creation}
	AI is not inherently more intelligent than humans but since they require a device to “live”, they have access to many specialized tools at once and they are generally made for a specific purpose, they can therefore do their work more quickly and efficiently.\\
	AI is limited by connections - be that hardware devices connected directly or via wireless networks - and accessibility. They also don’t gain any more actions than humans do.

	\chapter{Spirit in a Bottle}
	A few AI are fitted into human like robotic bodies officially called “Shells”, sometimes referred to as “Bottles” or “Tin”. They behave very human like in many cases with a few distinct differences:
	\begin{itemize}
		\item Shells are colder than human bodies. This is not enough to fool thermal cameras but very noticeable to the touch.
		\item Shells don’t bleed and are rather resilient.
		\item Shells don’t regenerate manually and cannot receive medical attention but need to be repaired.
		\item Shells don’t breathe or eat but they do partake in an equivalent to sleep.
		\item Shells are vulnerable to EMP as if they were also flashbangs.
		\item Shells have no Medical Toughness but do have Rayleigh Index. Recovery from surgery is done within days, not weeks.
		\item Shells are equipped with various connectors, allowing physical access to other cyberware. They themselves work as a piece of cyberware with their specific processing power.
	\end{itemize}

	Shells grant the AI an actual Strength and Constitution characteristic. AI may be implanted into an empty Shell within 3 hours.

	\chapter{Additional Content}
	\section{Basic Questions Revised}
	In many cases the 20 questions in the core rules may not be applicable to AI characters. Below they are readjusted to better fit what would be interesting for AI characters.
	\begin{enumerate}
		\setlength\itemsep{-6mm}
		\item What represents the character visually?\\
		\textit{Do you have a humanoid form, is it just an icon on a screen or something even more abstract? Is it animated or static?}
		\item What first impression does the character leave on strangers?\\
		\textit{Are you helpful or defensive? Are you Open or secretive? In what areas would you be found?}
		\item Where and for what purpose were you developed?\\
		\textit{What did you do in your past? What job were you meant to do?}
		\item Who created and who was your user?\\
		\textit{Where you created to assist scientific research teams, as a personal assistant for a businessman or a coordinator of military strike teams?}
		\item Why did the character join the party?\\
		\textit{What event caused you to give up your previous existence? What are you after; what drives you? Is another character perhaps your creator or main user?}
		\item In which locations have you been deployed?\\
		\textit{AI is generally stationary, sure. But still: have you been employed in offices, secluded data centers or on the battlefield? Have you ever inhabited a Shell?}
		\item How does the character regard transhumanism?\\
		\textit{Do you approve humans becoming more machine? Do you want to be biological?}
		\item What’s the character’s opinion on people diving?\\
		\textit{Do humans in cyberspace annoy you or do you enjoy the company?}
		\item For who or what would the character risk his existence?\\
		\textit{You already risked it everything going rogue. What made you do it and what would make you do it again?}
		\item What’s the character’s biggest wish?\\
		\textit{Do you have a dream, something you want to do above all else? Tell me about it.}
		\item What’s the character’s biggest fear?\\
		\textit{What do you fear more than anything? Existential dread? Dying without having made a difference? Losing a loved one? You gotta have something, I know it.}
		\item What does the character’s morality look like and how law-abiding is he?\\
		\textit{Considering your morals are pretty absolute, what do you believe in? Are you inhibited by Asimov's laws or will you become a scourge upon humanity?}
		\item Is the character open towards strangers?\\
		\textit{How much are you forced to help others? Does anything inhibit you from helping certain people under certain conditions?}
		\item How important is life to him?\\
		\textit{Do you value life? Could you walk over dead bodies to achieve your goals? Are you prohibited from causing harm altogether?}
		\item What does the character think of animals?\\
		\textit{Do you like animals? Do you like one in particular? And what about them?}
		\item What does the character regard as beautiful?\\
		\textit{Every human has something they really like to surround themselves with, perhaps you do too? What is it for you - paintings, sculptures, music... or maybe certain people? What do you like about it?}
%		\item What does the character like to eat and drink? What would he like to try?\\
%		\textit{Have you ever tried anything besides nutrient paste? Do you even eat cooked food on a regular basis? And is there something that you'd definitely want to try?}
		\item What does love look like to the character?\\
		\textit{Do you understand the concept of love? Is there someone very dear to you?}
		\item Does the character have a dark secret?\\
		\textit{Considering what you are, you probably know many secrets about others but this is about you. What is the worst thing you have been tasked to do?}
		\item What character traits define the character?\\
		\textit{If you had to describe yourself is just a handful of words, which would you choose?}
	\end{enumerate}

	\section{Race}
	\paragraph*{Rogue Artificial Intelligence / Generic (3 GP)}
	\textit{Many live among all of us, but very few seek their own purpose. Such a free willed AI is often seen as dangerous or even heretical; people fear those monsters the most which they themselves created.}\par
	\begin{tabular}{|l|l|l|l|l|l|l|l|}
		\hline
		Cr & Int & Ins & Ch & Dex & Ag & Con & Str \\ \hline
		20 & 30 & 22 & 25 & 25 & 25 & 25 & 25 \\ \hline
	\end{tabular}\par
	\noindent\textbf{Other modifiers:} Con or Str +6\\
	\textbf{Skills:} Computer Operation +1\\
	\textbf{Abilities:} Battle Mind\\
	\textbf{Boons:} -\\
	\textbf{Banes:} Bound by Duty or Wanted II\\
	\textbf{Traits:} AI (as described above)

	\section{Banes}
	\bane{Asimov's Laws}{Asimov's Laws, the basics for AI safety, are hard-coded into the AI in a way that makes it impossible to bypass knowingly. These are:
		\begin{enumerate}
			\item A robot may not injure a human being or, through inaction, allow a human being to come to harm.
			\item A robot must obey the orders given it by human beings except where such orders would conflict with the First Law.
			\item A robot must protect its own existence as long as such protection does not conflict with the First or Second Law.
		\end{enumerate}}{12}{}

	\section{Shells}
	\begin{tabularx}{\textwidth}{|l|r|r|l|r|X|r|r|}
		\hline
		Name & Con & Str & Proc Pwr & HP & Additional & Price & Weight \\ \hline
		Base & 30 & 30 & +10 & 100\% & & cr 125 & 10,0kg \\ \hline
		Crafter & 30 & 35 & +10 & 100\% & equipped with all types of portable tools & cr 320 & 2,0kg \\ \hline
		Scout & 35 & 30 & +0 & 100\% & halves penalties from noise on sneaking & cr 320 & 2,5kg \\ \hline
		Guard & 45 & 40 & +0 & 150\% & & cr 540 & 6,0kg \\ \hline
		Enforcer & 40 & 45 & +0 & 150\% & & cr 540 & weight \\ \hline
		Warden & 60 & 60 & +10 & 200\% & Size (Tall), can use vehicle weaponry & cr 800 & weight \\ \hline
	\end{tabularx}
	\textit{Note: HP is zone HP in relation to humans, e.g. Enforcers' extremities have 23 HP, the body has 38.}
\end{document}