\chapter{GM Briefing}
\section{Synopsis}
In the sprawling metropolis of \emph{Opportunity},
	the players are hired by a mysterious employer to retrieve a stolen AI prototype chip.
Once the plot doesn't add up,
	the party finds themselves in the middle of a conspiracy.
And it's their turn to choose a side.
\par
{
	\itshape
	An introductory mission
	in 1 to 3 sessions
	for 3 to 5 players
	passing through the three pillars of roleplaying.
}
\section{Player Characters}
It is built for the core rule prefab characters,
but any group of player characters may work.
Just make sure that the party is decently diverse in skills
and all characters are willing to work together.
\section{On the Usage}
Acts I to III outline a simple mission sequence.
Depending on your needs you can do whatever you like with the information
	- you are the GM after all.
\\%
The planned time frame leaves very little time for intricacies.
If you want to use the mission for more than a test drive,
	you will have to weave it into the rest of your world.
\\%
These Acts also refer to the content in the \hyperref[ch:index]{Index}.
It may be advisable to skim through it first.

\chapter{Player Briefing}
\begin{exampleblock}
	The following briefing may be given to the players as a teaser
		or read out loud to set the stage
		without risk of revealing information.
\end{exampleblock}
%TODO: longer briefing or filler image
You just returned from a mission.
Pay was bad, you're exhausted,
yet barely a day passes before you are contacted again.
\\%
The request is trivially easy:
Find a rogue scientist, recover a chip.
What could go wrong?

