\chapter{Act III: Confrontation}
\label{ch:act3}
In the end the players have the choice:
\begin{sitemize}
	\item Steal the chip and sell it to Dusk.
		(see \ref{sec:steal})
	\item Steal the chip and make a break.
		They may wanna use it for themselves,
		sell it to someone else,
		or do something more creative.
		(see \ref{sec:flee})
	\item Confront the fixer for omitting info.
		(see \ref{sec:revenge})
\end{sitemize}

\section{Snatch the chip}
\label{sec:steal}
The party runs a break-and-enter into Hayes' facility.
The facility is heavily guarded and monitored
	- but no security is impervious.
\subsection*{The Facility}
Refer to \ref{fig:facility} (p. \pageref{fig:facility}) for a map.\\
Rooms 1 and 3 are waiting or break rooms.
They include tables, chairs and a water dispenser.
\\%
Rooms 2 and 4 have been fitted as guard break rooms.
In addition to the other break rooms they feature ammunition supplies.
There is also usually a guard or two present.
\\%
Rooms 6, 7 and 8 are meeting rooms.
They each have two large, curved tables set up in form of a horseshoe.
The front features a whiteboard and a lectern,
	while there is a camera in the back to film a speaker.
\\%
Room 5 used to be a meeting room as well,
	but now houses a massive server rack.
%
\par
%
Human combatants and non-combatants (see below) are spread through the facility
	and may be moving constantly.
\\%
In addition to humans,
	automated defenses are in place.
Letters \emph{a} and \emph{m} represent security cameras.
They will sound a minor alarm upon detecting anything off.
\\%
Letters \emph{c} and \emph{i} represent Automated Point Defense Turrets.
They are hidden within the ceiling for as long as no major alarm has gone off.
\\%
Letter \emph{e} is a laser-based motion detector.
When tripped without ID,
	will cause a major alarm.
\subsection*{Enemy Forces}
Helix employs an amount of security guards
	roughly equal to the amount of player characters
	using the \emph{Checkpoint Guard} stat block.
They also employ a \emph{Checkpoint Officer}.
\\%
In addition to these security guards,
	Dr Hayes employs a team of scientists.
They are non-combatants,
	but will ring an alarm
	when things seem off.
This team is between 6 to 10 people in size.
\\%
All of these humans may initiate a major alarm
	when in distress.
\subsection*{The Target}
For Helix, the chip and Hayes are of similar importance.
If the location is threatened,
	the security guards will attempt to extract
	both Hayes and the chip
	to a safer, more protected location that will be impossible for the party to penetrate.
The Nexus chip is embedded into room 5's server rack.
Hayes may be moving.

\subsection{Book it}
\label{sec:flee}
In case the party decides to take the chip for themselves,
	they need to leave the city immediately.
Once the group is spotted on public surveillance,
	Helix' mercs or Dusk's goons
	will engage a chase.
Simple chase rules follow below.
\par
Every turn
	the escape driver and the chasing drivers
	make \emph{Drive} tests.
They may choose to do evasive driving,
	inferring a -20 to both the Drive test
	and all attacks against the vehicle and its passengers
	(in addition to \emph{Attacking while moving}).
\\%
The party starts at 30m distance (3 DoS) to their adversaries.
If the party reaches 120m (12 DoS) to \emph{all} followers,
	they escape.
If the party stays at or below 0 DoS for two turns in a row,
	they are rammed off the street.

\section{Get revenge}
\label{sec:revenge}
The party confronts Dusk 
	be it
	in loyalty to the establishment,
	in hopes of being paid by Helix,
	or for personal revenge.
\subsection*{Environment}
%TODO: combat map
The hideout is propped up inside the ruins of an old industrial complex.
It's secluded and hidden.
\\%
The inside serves as all in one living quarters.
It features several old couches
	as well as a few sleeping spots for Dusk's guards.

\subsection*{Enemy Forces}
Dusk is directly protected by 
	half as many 
	\emph{Body Guards}
	as there are player characters.
	\\%
There are also
	\emph{Addicts}
	equal to the amount of player characters
	hanging around, who are paid to protect her and the hideout
	but their loyalty is questionable and they will flee when faced with overpowering force.
	\\%
Dusk herself is a \emph{Gang Boss}.
\\%
You can find the stat blocks in the GM companion.
\begin{exampleblock}
	Grinding through this encounter is dangerous at best.
	Players are expected to prepare
		and lightning strike
		or otherwise break the Addicts' morale,
		as to not lose to superior numbers.
\end{exampleblock}

\subsection*{Dusk's Escape Plan}
Dusk is not intent on fighting.
The moment she realizes that the party did not come to bring the chip to her,
	she will look to use one of her escape plans.
She has prepared multiple plans, which may include:
\begin{sitemize}
	\item Hidden pathways like opening walls or trap doors.
	\item Gas traps to knock out the assailants,
		while she has adequate lung implants.
	\item A VTOL is parked on the roof, ready to leave at any moment.
\end{sitemize}
