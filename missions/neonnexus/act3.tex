\chapter{Act III: Confrontation}
\label{ch:act3}
In the end the players have the choice:
\begin{sitemize}
	\item Steal the chip and sell it to Dusk.
		(see \ref{sec:steal})
	\item Steal the chip and make a break.
		They may wanna use it for themselves,
		sell it to someone else,
		or do something more creative.
		(see \ref{sec:flee})
	\item Confront the fixer for omitting info.
		(see \ref{sec:revenge})
\end{sitemize}

\section{Snatch the chip}
\label{sec:steal}
The party runs a break-and-enter into Hayes' facility.
The facility is heavily guarded and monitored
	- but no security is impervious.
\subsection*{The Facility}
%TODO: map
%TODO: camera, motion detector, laser positions
\subsection*{Enemy Forces}
Helix employs an amount of security guards
	equal to the amount of player characters
	using the \emph{Checkpoint Guard} stat block.
They also employ a \emph{Checkpoint Officer}.
\\%
In addition to these security guards,
	Dr Hayes employs a team of scientists.
They are non-combatants,
	but will ring an alarm
	when things seem off.
This team is between 6 to 10 people in size.
\subsection*{The Target}
%TODO: target location
For Helix, the chip and Hayes are of similar importance.
If the location is threatened,
	the security guards will attempt to extract
	both Hayes and the chip
	to a safer, more protected location that will be impossible for the party to penetrate.
If escape is not a realistic option,
	they barricade in a back room.

\subsection{Book it}
\label{sec:flee}
In case the party decides to take the chip for themselves,
	they need to leave the city immediately.
Once the group is spotted on public surveillance,
	Helix' mercs or Dusk's goons
	will engage a chase.
Simple chase rules follow below.
\par
Every turn
	the escape driver and the chasing drivers
	make \emph{Drive} tests.
They may choose to do evasive driving,
	inferring a -20 to both the Drive test
	and all attacks against the vehicle and its passengers
	(in addition to \emph{Attacking while moving}).
\\%
The party starts at 30m distance (3 DoS) to their adversaries.
If the party reaches 120m (12 DoS) to \emph{all} followers,
	they escape.
If the party stays at or below 0 DoS for two turns in a row,
	they are rammed off the street.
\\%
The 

\section{Get revenge}
\label{sec:revenge}
The party confronts Dusk 
	be it
	in loyalty to the establishment,
	in hopes of being paid by Helix,
	or for personal revenge.
\subsection*{Environment}
%TODO: combat map

\subsection*{Enemy Forces}
Dusk is directly protected by 
	half as many 
	\emph{Body Guards}
	as there are player characters.
	\\%
There are also
	\emph{Addicts}
	equal to the amount of player characters
	hanging around, who are paid to protect her and the hideout
	but their loyalty is questionable and they will flee when faced with overpowering force.
	\\%
Dusk herself is a \emph{Gang Boss}.
\\%
You can find the stat blocks in the GM companion.
\begin{exampleblock}
	Grinding through this encounter is dangerous at best.
	Players are expected to prepare
		and lightning strike
		or otherwise break the Addicts' morale,
		as to not lose to superior numbers.
\end{exampleblock}

\subsection*{Dusk's Escape Plan}
Dusk is not intent on fighting.
The moment she realizes that the party did not come to bring the chip to her,
	she will look to use one of her escape plans.
She has prepared multiple plans, which may include:
\begin{sitemize}
	\item Hidden pathways like opening walls or trap doors.
	\item Gas traps to knock out the assailants,
		while she has adequate lung implants.
	\item A VTOL is parked on the roof, ready to leave at any moment.
\end{sitemize}
