\section{Melee actions}
\subsection*{Standard attack}
A single armed or unarmed melee attack. Attacking a specific location is modified by its size. This is compatible with any maneuver and does not require an ability. 
\subsection*{Sure strike (+20 modifier, also uses up reaction)}
A single melee attack with higher chances to hit. This maneuver cannot be combined with any other and does not require an ability.
\subsection*{Attack of opportunity (-10 modifier, free action)}
When an enemy passes through the character’s control area, the character may make a single attack as a free action with a -10 modifier in addition to any other maneuver.
\subsection*{Disengage (free action)}
When disengaging, no attacks are allowed to be made that round but no attacks of opportunity are invoked.
\subsection*{Feint (-X modifier)}
A skillful attack that is harder to defend against. The defense against this maneuver is at a penalty equal to the voluntary penalty.
\subsection*{Called jab (-20 -opponent’s armor)}
A strike against an armor’s weak points. The attacker picks the location without additional penalty. The attack ignores half the target’s armor.
\subsection*{Disarm (-30 -X modifier)}
An attack against the opponent’s grip to get control of his weapon. The defender makes a dexterity check at a penalty equal to the voluntary penalty of the maneuver. On a fail the defender loses her weapon, dropping on the ground. Failing by at least 3 degrees gives the attacker control over the weapon.
\subsection*{Flurry (-10 modifier)}
Two attacks in quick succession using both hands. Defending against this attack requires two reactions. On a hit the attacker deals damage once with her primary weapon and once with the weapon in her offhand. If she is not carrying a weapon in her offhand, she deals unarmed damage instead. 
\subsection*{Piercing thrust (-40 -half of opponent’s armor, also uses up reaction)}
An all-out stab attack. The attacker picks the location without additional penalty. The attack ignores all armor and even the injury threshold. Cannot be performed without the corresponding ability.
\subsection*{Take-down (-20 -X modifier)}
A sweeping and pushing attack to take the opponent off his legs. If the attack hits, it deals no damage. Instead the defender makes an agility check at a penalty equal to the voluntary penalty on the maneuver. If that fails, he falls on the ground. If it fails by 3 or more degrees, he also drops his weapon.
\subsection*{Shift (-10 -X modifier)}
Superior footwork or raw force moves an opponent. If the attack hits, the opponent makes a strength check at a penalty equal to the voluntary penalty on the maneuver or be moved by up to a meter per degree of failure in a direction chosen by the attacker.
\subsection*{Powerstrike (-X modifier)}
A strong blow that increases the hit’s damage. For every 5 points of voluntary penalty the damage is increased by one.
\subsection*{Cleave (-15 modifier per target, also uses up the reaction)}
A half circle of steel pushing multiple enemies back. This maneuver can target up to three enemies. A success indicates hits on all targets. A target hit will - in addition to normal effects - be pushed back by the attacker’s StrB in meters -1 per size category the target is bigger. This maneuver is always also a Knockdown.
\subsection*{Stunning blow (-20 -X modifier)}
A blunt attack to stun a target. If the hit is successful and the attack would deal damage, then deal half of the effective damage and the target makes a Constitution check at a penalty equal to the voluntary penalty on the maneuver. By GM discretion the target may also fall prone.
\subsection*{Knockdown (-20 -X modifier)}
A forceful attack to knock the target off their feet. If the attack hits, the target makes a Strength check at a penalty equal to the voluntary penalty on the maneuver. On a fail the target falls prone.
\subsection*{Charge (-20 modifier, also uses up reaction)}
An attack utilizing the momentum of movement. The attack gains a bonus to damage equal to the attacker’s current speed, usually his sprint speed. He needs a minimum of 4 meters of runup.
\subsection*{Crushing blow (-40 -X modifier, also uses up the reaction)}
A powerful, all-out attack that deals greatly increased damage. If the attack is a success, the damage is first increased by one per full 5 points of voluntary penalty and then damage and AP are doubled. Cannot be performed without the corresponding ability.

\section{Melee reactions}
\subsection*{Dodge (-10; -0 against unarmed)}
Evasion without weapons touching. Evasion has to be declared before the attack is rolled. On a success the attack is evaded. This does not require an ability.
\subsection*{Parry}
A basic defense against a melee attack. Parrying has to be declared before the attack is rolled. On a success the attack is evaded.
\subsection*{Bind (-X modifier)}
Enter a bind to lead one’s opponent’s weapon and gain an advantage. The next action the character takes has a bonus equal to the voluntary penalty.
\subsection*{Disarm from reaction (-30 -X modifier)}
A very risky defense to rid the opponent of their weapon during her attack. If the defense is successful, the attacker makes a dexterity check at a penalty equal to the voluntary penalty of the maneuver. On a fail the attacker loses her weapon, dropping on the ground. Failing by at least 3 degrees gives the defender control over the weapon.
\subsection*{Intercept (-20 modifier, uses attack value)}
A reckless attack into the opponent's attack. If only one attack is successful, that attack deals damage as normal. If both succeed, the one that succeeded better (see Dice checks and stats-Opposed Checks) deals full damage, while the other one only deals half damage.\\
Intercepting a Charge is at an additional penalty equal to 5 times the character’s CrB below 8 but deals additional damage equal to the target’s movement speed.
\subsection*{Reversal (min -30 modifier)}
A skillful counterattack that exploits an overswing. This maneuver can only be used against a Powerstrike series maneuver and grants a penalty equal to the attacking powerstrike penalty, at least -30. If the defense is successful, the maneuver also counts as an immediate powerstrike against the original attacker with a voluntary penalty equal to this maneuver’s penalty. The attack may be parried as normal and may also be subject to Reversal. Cannot be performed without the corresponding ability.

\section{Grappling}
Grappling may be somewhat more complicated than striking due to the amount of options one has and how different these options are to striking: control over someone else and dislocation of opponents or their limbs.
\subsection*{Hold}
A hold is the basis of all grappling. By itself it only stops the opponent from leaving but it is necessary for most other maneuvers to initiate.\\
To initiate a hold, the character makes a grappling check which may be opposed by the target as a reaction. If the character wins, a hold is engaged.\\
To break free the target makes a grappling check which may be opposed by the character as a reaction. If the target wins, it breaks free.
\subsection*{Throw}
The character attempts to lift the target off the ground and forcefully put it back.\\
The character makes a grappling check which may be opposed by the target as a reaction. If the character wins, the target takes D10+StrB damage with the trauma quality and falls prone (therefore also losing D5 initiative).
\subsection*{Lock}
The character attempts to prevent his opponent from moving freely. Arm locks, leg locks and head locks are common but bear hugs also fall into this category.\\
To initiate a lock, the character makes a grappling attack roll which may be opposed by the target as a reaction.\\
While the lock is active, the character may use an action to interfere with any action taken by the target: the character takes a grappling check as an action and the target’s action is at a -10 penalty per degree of success.\\
To break free the target may take a -10 grappling check, which may be opposed by the character as a reaction. If the target wins, it breaks free, back into the initial hold.
\subsection*{Lever hold}
While holding an opponent in a lock of some kind, the character may choose to initiate a lever hold as an action. This forfeits the option to control any action in favor of dealing damage to a location every turn.\\
To initiate a lever hold, the character has to have his target in a lock already and takes a grappling check which may be opposed by the target as a reaction. If the character is successful, a location is determined and a lever hold is initiated.\\
While the lever hold is active, the character deals 1D5-2+StrB damage to the location, ignoring armor.\\
To break free the target may make a -20 grappling check, which may be opposed by the character as a reaction. If the target wins, it breaks free.
\subsection*{Choke hold}
Another common grappling target besides joints are the airways.\\
To initiate a chokehold the character makes a grappling check which may be opposed by the target as a reaction. If the character wins, the chokehold is initiated.\\
The chokehold works like a lever hold but instead of dealing damage, it causes suffocation.\\
To break free the target may make a -20 grappling check, which may be opposed by the character as a reaction. If the target wins, it breaks free.
\subsection*{Move}
Moving while grappling is hard, even more so if only one party intends to move.\\
The character makes a grappling check which may be opposed by the target as a reaction. If the character wins, the target is moved up to the difference of DoS in meters or half the amount if on the ground.

\section{Melee dancing}
When attacking or defending in melee combat, the combatants are rarely stationary. After an attack that does not include intentional movement the character and his opponent move a meter into a random direction within the same control areas. Roll a D10:

\begin{tabular}{ll}
	1-2: & towards the opponent\\
	3:   & forward right\\
	4:   & right\\
	5:   & backwards right\\
	6-7: & away from the opponent\\
	8:   & backwards left\\
	9:   & left\\
	10:  & forward left
\end{tabular}
