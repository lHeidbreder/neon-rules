\section{Initiative and combat speed}
Before combat can start, every character needs to have an initiative value. Every creature has a base initiative value and adds 1D5 to it.\\
Every combat round is 3 seconds long and consists of a \textit{declaration phase} and an \textit{action phase}. Having higher initiative means declaring later, when that character knows what slower characters will do this round, and resolving their actions before those of slower characters.\\
Initiative is variable. When a character is disorientated by being blinded, stunned or falling to the ground, he loses initiative. How much is up to the GM but should be between a D5 for taking a heavy hit to the head and a D10 for smacking his head on the floor.
\section{Control area}
Every combatant has a \emph{control area}, which is 120° in front of the character and within melee range. 
\section{Actions}
Every character has 2 actions per combat round, in addition they are allowed to move. Remember that all actions are happening simultaneously, so some actions might be more difficult while moving.\\
There are two types of combat actions: active actions and reactions. By default one of the character’s actions is active, the other is reactive. Active actions include attacks, reactions encompass parries and dodges. Actions are interchangeable: when turning an action into a reaction or vice versa, the turned action will be at a -20 penalty. Turning actions has to be declared.
\section{Maneuver basics}
A maneuver describes a combat action a character can take. These actions might look differently or executed in a different way, yet mechanically act the exact same way: a skillful attack to the most vulnerable parts of the human body and a particularly powerful blow will both increase the impact’s damage at a cost to accuracy and therefore are both mechanically \textit{power strikes}.\\
Unless specified otherwise any character can use any maneuver even without being properly trained (i.e. knowing the ability) but all penalties are twice as high to achieve the same results.
\section{Voluntary penalties}
Most maneuvers take great risk and the bigger the risk, the higher the reward. Voluntary penalties are taken by the acting character and have some sort of effect on maneuvers. The maneuvers that profit from voluntary penalties are designated with a \emph{-X} modifier.\\
Voluntary penalties can’t be taken to reduce the target value below 20. If multiple maneuvers profit from voluntary penalties, the benefits are split freely among them.
\section{Shooting size \& multiple targets}
When attacking a group of targets or one large target in ranged combat, hitting becomes easier. Depending on the overall target size a bonus to hit given on the Size-table is inferred. If the target is a group, the individual target is determined randomly.\\
Attacking a single target in a large, moving group is impossible and the target will have to either be lured out or hit by chance.
\section{One-handed weapons and dual-wielding}
When wielding two weapons, the character may attack using both weapons in one action. Using small arms is at a -30, using large weapons is at -60. Additionally, attacking different targets is at a -10 penalty.\\
Attacking twice is not possible in turned actions.
\section{Combined actions}
\label{combinedaction}
To take combined actions, the involved members need some way to communicate - preferable privately - and go on the lowest initiative. The team may distribute all Degrees of Success among each other freely.
\section{Damage and armor}
When a character takes a hit that is not defended against, first worn armor and cover are reduced by the attack's armor penetration to a minimum of 0. The leftover armor and the target’s injury threshold then reduce the attack’s damage to a minimum of 0. Any leftover damage is then applied to the randomly determined hit location and the target now takes penalties in accordance.
\section{Cover and concealment}
Any object blocking line of sight is considered \textbf{Concealment}. Attacking a target behind such concealment comes with a -30 penalty as if blindfiring.
\textbf{Cover} on the other hand provides additional armor points depending on material and thickness. The table below gives examples of how much armor points a centimeter of some material may provide.\\
Cover degrades. For every 5 points of damage (not armor penetration) a section of cover blocks, the armor it provides is reduced by 1. If a section of cover functions as both cover and concealment, the concealing effect stops when roughly 25\% of armor remains.
