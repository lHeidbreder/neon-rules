\section{Initiative and combat speed}
Before combat can start, every character needs to have an initiative value. Every creature takes its \emph{base initiative} value and adds 1D5 as well as potential other modifiers to it.\\
Every combat round is about 3 seconds long and consists of a \textit{declaration phase} and an \textit{action phase}. Having higher initiative means declaring later, when that character knows what slower characters will do this round, and resolving their actions before those of slower characters.
\subsection{Declaration Phase}
In \emph{ascending} order all characters exclaim their plans for the combat round. This includes movement, converting actions and reactions, attacks and their targets, drawing and readying equipment - pretty much everything besides reactions and exact maneuvers.
\begin{exampleblock}
	\itshape
	Three contenders are involved in a fight.\\
	Jane has 8 initiative after hitting her head multiple times, making her the slowest, therefore first to declare. She intends to pick up a weapon from the floor and whack Paul over the head.\\
	Anne, with an initiative of 12 is next in line. Since she fears being attacked and severely injured before she can attack, she will block twice. As to not cause attacks of opportunity, she does not move.\\
	Paul is the fastest with an initiative of 16 and therefore declares last. Knowing he's at a disadvantage, he disengages and moves away, keeping one reaction for the attack he knows is coming.
\end{exampleblock}
\subsection{Action Phase}
In \emph{descending} order all characters resolve their actions. Depending on the actions that resolve earlier, some declared actions may become invalid due to ranges and line of sight - this is the main advantage of being faster in combat.
\begin{exampleblock}
	\itshape
	First Paul, the fastest combatant, resolves his actions. He takes a disengaging stance as to not suffer attacks of opportunity and runs.\\
	Anne, having prepared to block or evade, essentially skips her turn.\\
	Lastly, Jane moves after Paul and attacks. She's attacking in movement unexpectedly and therefore has to deal with a -20 in addition to the modifiers she already has.
\end{exampleblock}
\subsection{Variable Initiative}
\label{subsec:variableini}
Initiative is variable. When a character is disorientated by being blinded, stunned or falling to the ground, he loses initiative. How much is up to the GM but should be between a D5 for taking a wounding hit to the head and a D10 for dropping to the floor.
\\%
Unlike a reduced Base Initiative, this may be regained by taking the \emph{Gather Senses} action.
\section{Control area}
Every combatant has a \emph{control area}, which is 120° in front of the character and within melee range. \\
When fighting multiple enemies that are not all within the character's control area, he takes -20 to all tests in that combat encounter. This is in addition to being outnumbered.
\section{Actions}
Every character has 2 actions per combat round, \emph{in addition} to being allowed to move. Remember that all actions are happening simultaneously and some actions might be more difficult to do while moving.\\
There are two types of combat actions: active actions and reactions. By default one of the character’s actions is active, the other is reactive. Reactions are parries and dodges, everything else are active actions. Actions are interchangeable: when turning an action into a reaction or vice versa, any test in the turned action will be at a -20 penalty. Turning actions has to be \emph{declared}.
\section{Free actions}
Free actions can be used at any time and don't use up an action. Free actions cannot be used to interrupt another character that has higher initiative. Any character is generally limited to one of any free action, e.g. a character cannot drop to the ground, stand up using an action and drop back down again. Remember that a round of combat is about 3 seconds long - when in doubt, apply common sense.
\section{Maneuver basics}
A maneuver describes a combat action a character can take. These actions might look differently or executed in a different way, yet mechanically act the exact same way: a skillful attack to the most vulnerable parts of the human body and a particularly powerful blow will both increase the impact’s damage at a cost to accuracy and therefore are both mechanically \textit{power strikes}.\\
Unless specified otherwise any character can use any maneuver even without being properly trained (i.e. knowing the ability) but all penalties are twice as high to achieve the same results.
\section{Voluntary penalties}
Most maneuvers take great risk and the bigger the risk, the higher the reward. Voluntary penalties are taken by the acting character and have some sort of effect on maneuvers. The maneuvers that profit from voluntary penalties are designated with a \emph{-X} modifier.\\
Voluntary penalties can’t be taken to reduce the target value below 20. If multiple maneuvers profiting from voluntary penalties are combined - e.g. a \emph{power strike} and a \emph{faint} - the benefits are split freely among them.
%TODO: example?
\section{Shooting size \& multiple targets}
When attacking a group of targets or one large target in \emph{ranged} combat, hitting becomes easier. Depending on the overall target size a bonus to hit given on the Size-table is inferred. If the target is a group, the individual target is determined randomly if necessary.\\
Attacking a single target in a large, moving group is impossible and the target will have to either be lured out or hit by chance or explosives.
\section{Dual-wielding}
When wielding two weapons, the character may attack using both weapons in one action. Using one-handed weapons is at a -30, using two-handed weapons is at -60. Additionally, attacking different targets is at a -10 penalty.\\
Attacking or parrying twice is not possible in turned actions, i.e. dual-wielders cannot take the same action three times per turn.
\section{Combined actions}
\label{combinedaction}
To take combined actions, the involved members need some way to communicate - preferable privately - and all take their go on the lowest involved initiative. They don't have to test against the same characteristic or skill, but they should in some way support each other.\\
The team may then distribute all Degrees of Success among each other freely, allowing them to cancel out Degrees of Failure and making failed tests succeed.
\begin{exampleblock}
	\itshape
	Three people participate in a combined action. Two of them succeed, with 3 and 4 DoS respectively. The last one fails with 3 DoF.\\
	They can now decide to e.g. have everyone succeed (one at 2 DoS, the others at 1), or have one succeed at 6, one at 1 and just have the last one fail as is.
\end{exampleblock}
\section{Damage and armor}
When a character takes a hit that is not defended against, first worn armor and cover are reduced by the attack's armor penetration, down to a minimum of 0. \\
The leftover armor and the target’s injury threshold then reduce the attack’s damage, again down to a minimum of 0. \\
Any leftover damage is then applied to the randomly determined hit location and the target now takes penalties in accordance.
\begin{exampleblock}
	\itshape
	An attack deals 2D10+5 damage (rolling 15 in total) and has 6 AP.\\
	The defender wears "armor" with an armor value of 4 and sits behind a thin wall with armor value of 8, for a total of 12. His Injury Threshold is 3.\\
	The effective armor is reduced to 6. Now both the armor and the Injury Threshold is deducted from the damage, so the hit deals (15-6-3=) 6 damage in total.
\end{exampleblock}
\section{Cover and concealment}
Any object blocking line of sight is considered \emph{Concealment}. Attacking a target behind such concealment comes with a -30 penalty as if blindfiring.
\emph{Cover} on the other hand provides additional armor points depending on material and thickness.
%TODO: The table below gives examples of how much armor points a centimeter of some material may provide.
\\
Cover degrades. For every 5 points of damage a section of cover blocks, the armor it provides is reduced by 1. If a section of cover functions as both cover and concealment, the concealing effect stops when roughly 25\% of armor remains. When in doubt, apply common sense and let the GM make an executive decision.
