\newpage
\section{General combat actions}
\subsection*{Stand up / Hop up / Drop}
Unless restrained a prone character can spend one action to start kneeling, a kneeling character can spend an action to \emph{stand up}. A prone character may choose to make an agility check instead to \emph{hop up}, i.e. stand up fully in one action. On a success he stands up, on a fail he stays prone. The test is modified based on balance modifiers, like slants, liquid or rubble.\\
Dropping lower always is a free action, no matter if to the knees or prone.
\subsection*{Jump to Cover}
When declaring actions for the combat round a character may choose to jump into cover. At that moment he dashes a distance up to his movement characteristic and counts as being in cover effective immediately. This counts as moving and cannot be combined with other movement.
\subsection*{Gather senses}
The character gets a grip of the battlefield again after losing her overview. She re-rolls initiative.
\subsection*{Tactics}
The character quickly shouts out some instructions to assure coordination. She makes a Command test and may distribute her DoS between her allies to grant them additional initiative. This modifier is maintained when they perform Gather Senses actions.\\
This may never be a combined action.
\subsection*{Draw / Ready}
Draw a weapon or ready an equipment piece.
\subsection*{Defensive stance (free action)}
A combat stance to increase one’s defensive capabilities. Grants a -30 to offensive actions but +10 to defensive actions. Unlike other defensive actions this must be declared during the declaration phase and may be done without an ability.
\subsection*{Assistance / Cooperation}
In combat assistance should be handled as a combined action (p. \pageref{combinedaction}).
%\subsection*{Spotting}
%Spotting is a special type of assistance. The shooter and the spotter take a combined action with the spotter using Perception. 10 times his DoS will be added to the shooter’s attack roll instead of 5 times.
\subsection*{Suppress pain}
Penalties from being injured are in large parts due to pain. Trained individuals may attempt to suppress pain and carry on for a while.\\
As an action the character may make a Restraint test. This test is at a penalty equal to the penalty the character wants to suppress. If the test succeeds, the character may ignore said penalties for the combat or scene or until he takes further damage.
\begin{exampleblock}
	\itshape
	In a firefight Nathan was injured in his right arm and left leg, having taken 7 points of damage each. To get the hell outta dodge he tries to push through the pain: He makes a \emph{Restraint} test at a penalty of 20 for his leg and 20 for his arm for 40 in total.
\end{exampleblock}