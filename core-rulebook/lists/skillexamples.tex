\subsection*{Combat}
\vspace{5mm}
\begin{multicols}{2}
	\ul{Situation:}
	\begin{exampleblock}
		\textit{Storm hits Eric’s face, the cold rain almost cutting up his face. The faint moonlight is barely visible through the densely clouded night sky. His old, trusty scope had long since broken down and the barrel was damaged, yet he had to take this chance for it might be the last he’d ever get.}
	\end{exampleblock}
	\columnbreak
	\ul{Mechanical Solution:}
	\begin{exampleblock}
		Both strong winds as well as heavy rain infer a -10 each and low light conditions bring another -20 penalty for a -40 from environmental effects. Taking aim would grant him a bonus of up to +20, yet due to his broken scope and assuming he does not have a functioning backup for this distance, only a +10 is granted. Assuming again that he is shooting over a long distance as to not be noticed, this infers another -20 penalty that is more than made up for by the fact that his target is now unaware, granting a +30 to hit.\\
		In total the shot is at a -20 penalty and the target may not dodge.
	\end{exampleblock}
	\columnbreak
	\ul{Situation:}
	\begin{exampleblock}
		\textit{Within the haze, the noise, the people - there he is, trying to flee the scene. Jenny has her target in her sight, a chance to sneak up - and a chance to strike. Just before she embeds the blade in his spine, he notices, making her job just that much harder.}
	\end{exampleblock}
	\columnbreak
	\ul{Mechanical Solution:}
	\begin{exampleblock}
		The attack within a hazy, cramped environment is marginally more difficult for a knife, resulting in a -10. She snuck up but was noticed last minute, so the GM calls the target surprised for a +20 bonus; resulting in a +10 bonus overall.\\
		The unnamed defender is impaired by the same environmental effects. Additionally he is unarmed but defends against a knife, which adds a penalty of -10. In total the defense roll gains a -20.
	\end{exampleblock}
\end{multicols}

\pagebreak
\begin{multicols}{2}
	\subsection*{Exploration}
	\ul{Situation:}
	\begin{exampleblock}
		\textit{"Huh, footprints?" The snow isn't deep but enough to see the outlines. "Is this our man?"}
	\end{exampleblock}
	\ul{Mechanical Solution:}
	\begin{exampleblock}
		Since the footprints are clear enough to be seen, we don't require a Perception check to identify them. To answer the question however we will require a Survival test based on Instinct to cover tracking.\\
		Modifiers are less clear here than we'd like them to be. Let's say the outline of the footprints is enough to determine the direction and to follow them, granting a +20. To make the situation a little more interesting, let's say it's still snowing and there's not just the one strain of footprints. Just like in combat the snow reduces visibility, reducing the current bonus by 10. Additionally the overlapping footprints make it slightly harder to follow, granting a further -5 for a total of +5.
	\end{exampleblock}
	\columnbreak
	\ul{Situation:}
	\begin{exampleblock}
		\textit{Once again Yokai was on the run again, his pursuer glued to his heels. Hoping he's more agile, that construction fence may make the difference.\\
			While yes, he is more agile, the massive pursuer bashes straight through.}
	\end{exampleblock}
	\ul{Mechanical Solution:}
	\begin{exampleblock}
		Yokai tries to gain an advantage by vaulting over a fence. Since this is done fast, Climbing is out of the question, so we turn to Acrobatics. His pursuer tries just pushing on. Since directing strength is covered by Athletics, let's use that but considering the obstacle in his way, he gains a -10 penalty.\\
		Both roll off against each other and the winner gains a considerable advantage: either the pursuer catches up or Yokai loses him. If both fail, that does not matter for the result, only for the way they get there; they may both stumble except one gets up again faster.
	\end{exampleblock}
\end{multicols}

\pagebreak
\begin{multicols}{2}
	\subsection*{Social}
	\ul{Situation:}
	\begin{exampleblock}
		\textit{Shrapnel flies across the room. A gunshot on the other side. Mere moments later more debris, a second bang.
			"These negotiations were supposed to be way less-"\\
			Dwayne's joke is interrupted by his screaming, as a bullet pierces his armor and embeds itself in his abdomen.\\
			"11mm hollow point, haven't felt that in a while."\\
			Blood drips to the floor, no bandages in sight - and no bullets left. Another shot barely misses his head, leaving a massive hole as it breaks through a weaker part of the table he is hiding behind. He lobs his gun into the middle of the room as his last way out of this mess. Larry approaches.\\
			"Look at you, like a beaten puppy, lying on the floor in a pool of your own blood and tears. You should have just given me the case."\\
			Dwayne stares him down, pulls the pin on a grenade, grasping it with the last ounce of his strength.
			"Piss off now or neither of us will make it."}
	\end{exampleblock}
	\ul{Mechanical Solution:}
	\begin{exampleblock}
		Quite obviously this situation calls for an Intimidation (courage) roll which Dwayne is \emph{experienced} in, netting him a +20 bonus. Due to his severe injuries, he will assist himself with his \emph{trained} Restraint (Constitution) - representing him powering through his pain. Under the effects of strong painkiller - granting him +10 - he scores one degree of success, netting him another +15 to the test. His courage of 33 and modifiers add up to a target value of 78.\\
		He is opposed by Larry's Restraint (courage). He is not in a particularly good spot - not under any enhancing influence and not very steeled to the horrors of war. Therefore his \emph{known} Restraint and his courage of 38 is all he can fall back on, making this opposition a very one-sided affair.
	\end{exampleblock}
	\columnbreak
	\ul{Situation:}
	\begin{exampleblock}
		\textit{"Get your sorry asses up!" Ness' voice was booming, grabbing attention of the rundown mercenaries he once called comrades.\\
			"You look miserable. Oh, how the mighty have fallen. If we can do it, why can't you?! Do you want to live forever!"}
	\end{exampleblock}
	\columnbreak
	\ul{Mechanical Solution:}
	\begin{exampleblock}
		Trying to motivate the mercs to pick up arms again and fight for their cause should be easier than introducing them to something new; so let's assume a +20 modifier. Additionally the mercs already know him; while not quite being friends, a friendly relationship grants another +20 for a total +40. The amount of people following after his call would then depend on the DoS - possibly around 4 and 2 more per DoS, in case numbers are necessary.
	\end{exampleblock}
\end{multicols}