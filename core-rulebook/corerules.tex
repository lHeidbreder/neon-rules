\documentclass[12pt,a4paper,openany]{book}
\pagestyle{plain}

\usepackage[utf8]{inputenc}
\usepackage[english]{babel}
\usepackage{textcomp}
\usepackage{xifthen}
\usepackage{tabularx}
\usepackage{tabto}
\usepackage{multicol}

\usepackage[bookmarks=true,colorlinks=true,linkcolor=cyan]{hyperref}

%Alignment
\usepackage[skip=10mm]{parskip}
\raggedbottom

% Title Image
\usepackage{wallpaper}
\def\coverimgpath{../art/\@title/cover}

% Title definition
\def\subtitle{Core}

%New commands & environments
%Custom Environments
\newenvironment{exampleblock}[1][1]
{
	\par
	\vspace{-5mm}
	\hfill
	\begin{minipage}
		{\dimexpr\columnwidth-#1cm}
	\begin{mdframed}[
		backgroundcolor=Gray!65,
		rightline=false,
		topline=false
		]
}{
	\end{mdframed}
	\end{minipage}
	\par
}

%%shortened itemize
\newenvironment{sitemize}[1][10]
{
	\begin{itemize}
	\vspace{-#1mm}
	\setlength\itemsep{-#1mm}
}{
	\end{itemize}
}

%Custom Commands
\newcommand{\ul}[1]{\underline{\smash{#1}}}
\newcommand{\breakline}{\vspace{.5cm} \hrule width \columnwidth \relax}
\newcommand{\derivedvalue}[3]{
	\begin{samepage}
	\subsubsection{#1 \textsubscript{\textlangle#2\textrangle}}
	\hfill
	\begin{minipage}{\dimexpr\columnwidth-1cm}
		#3
	\end{minipage}
	\end{samepage}
	\par
}
\newcommand{\specialrule}[2]{
	\begin{minipage}{\columnwidth}
		\textbf{\ul{#1:}}\\
		#2
	\end{minipage}
    \par
}
\newcommand{\supply}[4]{
	\begin{minipage}{\columnwidth}
		\textbf{\ul{#1}}\\
		\textit{Price: #3}\\
		\textit{Weight: #4}\\
		#2
	\end{minipage}
    \par
}
\newcommand{\service}[4]{
	\begin{minipage}{\columnwidth}
		\textbf{\ul{#1}}\\
		#2\\
		\textit{Cost:} #3 per #4
	\end{minipage}
	\par
}
\newcommand{\statuseffect}[2]{
	\begin{minipage}{\columnwidth}
		\textbf{\ul{#1:}}\\
		#2\\
	\end{minipage}
	\vspace{5mm}
}
\newcommand{\skill}[4][Basic]{
	\begin{minipage}{\columnwidth}
		\textbf{\ul{#2}}\\
		Difficulty: #1\\
		Common Characteristic: #3\\
%		Description:\\
		\textit{#4}\\
	\end{minipage}\par
}
\newcommand{\melee}[2]{\skill{#1}{Melee Base}{#2}}
\newcommand{\ranged}[2]{\skill{#1}{Ranged Base}{#2}}
\newcommand{\ability}[4]{
	\begin{minipage}{\columnwidth}
		\textbf{\ul{#1}} (#2 XP)\\
		\textit{Prerequisites}: #3\\
		\textit{Effect}:\\
		#4\\
	\end{minipage}
}
\newcommand{\maneuver}[3]{
	\begin{minipage}{\columnwidth}
		\textbf{\ul{MVR: #1}} (#2 XP)\\
		\textit{Prerequisites: #3\\}
		Effect:\\
		Allows the use of the maneuver \emph{#1} at normal penalties.\\
	\end{minipage}
}
\newcommand{\boon}[4][]{
	\begin{minipage}{\columnwidth}
		\label{boon::#2}
		\textbf{\ul{\smash{#2}}} \mbox{(#4 GP\ifthenelse{\isempty{#1}}{}{ }#1)}\\
		#3
	\end{minipage}
	\par
}
\newcommand{\bane}[4]{
	\boon[#4]{#1}{#2}{#3}
}
\newcommand{\rangedweapon}[9]{
	\vspace{2mm}
	\begin{minipage}{\columnwidth}
		\textbf{\ul{#1}}\\
		\textit{#2}\\
		\textbf{Weight}: #9\\
		\textbf{Price}: #8\\
		#3\\
		\textbf{Mag}: #5\\
		\textbf{Reload}: \mbox{#6 actions}\\
		\textbf{Range}: #4\\
		\ifthenelse{\isempty{#7}}{}{\textbf{Special Rules}: #7}
	\end{minipage}
	\par
}
\newcommand{\weaponmod}[5]{
	\begin{minipage}{\columnwidth}
		\textbf{\ul{#1}}\\
		\textit{#3}; \textit{#4}; \textit{#5}\\
		#2
	\end{minipage}
	\par
}
\newcommand{\meleecomponent}[6]{
	\begin{minipage}{\columnwidth}
		\textbf{\ul{#1:}}\\
		\textit{#2}\\
		\textbf{Weight}: #4; \textbf{Price}: #5\ifthenelse{\isempty{#6}}{}{; \textbf{Requirement}: #6}\\
		\textbf{Effect}: #3
	\end{minipage}
	\par
}
\newcommand{\ammo}[6]{
	\begin{minipage}{\columnwidth}
		\textbf{\ul{#1:}} \textit{#2}\\
		\textbf{Price}: #3; \textbf{Unit of sale}: #4\\
		\textbf{Weight/Bulk}: #5 \textit{\ifthenelse{\isempty{#6}}{each}{per #6}}
	\end{minipage}
	\par
}
\newcommand{\armor}[9]{
	\begin{minipage}{\columnwidth}
		\textbf{\ul{#1}} (covers: #6)\\
		\begin{tabular}{|r|r|r|r|}
			\hline
			Head & Chest & Arms & Legs\\
			\hline
			#2 & #3 & #4 & #5\\
			\hline
		\end{tabular}\par
		\vspace{2mm}
		\textit{Price:} #8; \textit{Weight:} #9\\
		#7
	\end{minipage}
	\par
}
\newcommand{\armormod}[6]{
	\begin{minipage}{\columnwidth}
		\textbf{\ul{#1}}\\
		#2\\
		\textit{Price:} #3; \textit{Weight:} #6\\
		\textit{Mod points:} #4\ifthenelse{\isequivalentto{N}{#5}}{}{; Requires power source}
	\end{minipage}
	\par
}
\newcommand{\pes}[6]{
	\begin{minipage}{\columnwidth}
		\textbf{\ul{#1}}\\
		Armor: #2; Threshold: #3\\
		\textit{Price:} #4; \textit{Weight:} #5
		\ifthenelse{\isempty{#6}}{}{\\} %conditional line break
		#6
	\end{minipage}
	\par
}
\newcommand{\implant}[7]{
	\begin{minipage}{\columnwidth}
		\textbf{#1}\\
		\textit{#2}\\
		\textit{Effect}: #3\\
		\textit{Price:} #6; \textit{Load:} #5\ifthenelse{\isequivalentto{-}{#4}}{}{; \textit{Slot:} #4}\\
		\textit{Available Mods:} #7
	\end{minipage}
	\par
}
\newcommand{\augmod}[3]{
	\begin{minipage}{\columnwidth}
		\textbf{#1}\\
		#2\\
		\textit{Cost:} #3
	\end{minipage}
	\par
}
\newcommand{\mod}[2]{\item \textbf{#1}: #2}
\newcommand{\psicomponent}[4]{\textbf{#1} & #2 & #3 & #4 \\}

%filler images
\newcommand{\filltopageendgraphics}[2][]{%
	\par
	\zsaveposy{top-\thepage}% Mark (baseline of) top of image
	\vfill
	\zsaveposy{bottom-\thepage}% Mark (baseline of) bottom of image
	\smash{\includegraphics[height=\dimexpr\zposy{top-\thepage}sp-\zposy{bottom-\thepage}sp\relax,#1]{#2}}%
	\par
}

%Base Building
\newcommand{\baselocation}[6]{
	\begin{minipage}{\columnwidth}
		\textbf{#1}\\
		Size: #3; Concealment: #4; Defense: #5\\
		Conditions: #6\\
		\textit{Cost:} #2
	\end{minipage}
	\par
}
\newcommand{\baseasset}[6]{
	\begin{minipage}{\columnwidth}
		\textbf{#1}\\
		\textit{#2}\\
		Size: #3\\
		Concealment: #4\\
		Defense: #5\\
		\textit{Cost:} #6
	\end{minipage}
	\par
}

%Narrative
\newcommand{\nrule}[3]{
	\begin{minipage}{\columnwidth}
		\section{#1}
		\textit{#2}\\
		\vspace{8mm}
		\begin{exampleblock}
			#3
		\end{exampleblock}
	\end{minipage}
	\par
}

%GM
\newcommand{\missiontype}[3]{\item \textbf{#1}: \textit{#2} #3}


\begin{document}
	{\heading
\ThisCenterWallPaper{1}{\coverimgpath}
\maketitle}
{\hypersetup{hidelinks} \tableofcontents}


	\chapter{The Foundation}
	\section{Basics}
	This booklet will not provide an introduction into tabletop roleplaying in general. If you have never before played a tabletop roleplaying game, there are many great resources out there. It is however not required to have played a game yourself, much less in a D100 system. All game mechanics will be provided on the following pages.
	\section{Scenario}
	\begin{exampleblock}
		\vspace{8mm}\itshape
		The year 2113 is coming to an end. After many, many price cuts last year people forgot how to handle their money responsibly, dividing society into \emph{very} different classes. \\
		Money is power, more than ever. The only thing stopping riots and insurgencies against weakened governments are private security providers - be that gangs and desperate people trying to survive or humanoid robots, unstoppable, unbreakable, of unknown origin. \\
		The future really has everything.
	\end{exampleblock}

	\chapter{Dice Checks and Stats}
	\section{Dice}
	This system uses D100s - "100-sided dice" - to make its decisions. To achieve a Success you have to reach or undershoot a Target Value, which is composed of a \emph{Base Value} (usually a characteristic or skill, more on those later) and Situational Modifiers. Example:

	\begin{exampleblock}
		\textit{Alice is doing reconnaissance. Trying to spot something far away she is asked to make an Investigation check. Her Investigation of 55 (Perception of 45 + 10 for being trained in Investigation) is modified by -15 for low light and -10 for slightly misty weather. Her target value therefore becomes 30, so rolling that or below would have her succeed.}
	\end{exampleblock}

	The required dice will be a D10 and a percentile dice, which essentially is also a D10 with an additional 0 (00, 10, 20 and so on). If a D100 roll is requested, both dice are rolled simultaneously. Both dice are added to each other and that will be the result, so a 4 and a 70 will become 74. The one exception to this is rolling all zeros - this will become 100.
	\section{Re-rolls}
	Some rules will allow re-rolling dice. In this case you roll the dice again and the new result will count as if the first roll never took place. A D100 in this case is a single dice and has to be re-rolled as a whole - not just the 1s or the 10s.
	Also note that any dice can only be re-rolled once, even if multiple rules would allow re-rolls.
	\section{Degrees of success/failure}
	Success can mean a lot of things. Grazing a target and hitting dead center are both successes but they clearly differ in how successful they are. Enter Degrees of Success (or DoS for short). You gain one degree of success for just general success and every full 10 points there are between the eyes on the dice and the target value gives you one additional degree of success. Depending on how many are achieved, the GM describes the result accordingly.
	Another example:

	\begin{exampleblock}\textit{
	Alice tries to throw a can of foam at a fire to put it out. Her ranged base value is 28. She is well-versed in throwing weapons, giving her +20 and because of the large target - about half a room - she is granted another +30, making her target value 78.\\
	In the first scenario she rolls a 76, which does result in a success but without any additional degrees of success beyond the first, making this a bare hit. Only the outermost part of the fire is quenched and it continues to spread to the other side but her way might be passable now.\\
	In the second scenario she rolls a 34, making this 5 degrees of success. She hits dead center and the foam spreads over almost the entire flames, essentially having it die out.\\
	In the third scenario she rolls an 87, barely failing but with only a single degree of failure. The can might hit something and return, hit the wrong parts of the fire or fail to pop.\\
	And finally in the fourth scenario she rolls a 99, scoring 3 degrees of failure. The can could slip, pop too early or spread the foam in the wrong direction. Details are up to the GM's discretion.
		}
	\end{exampleblock}
	\section{Opposed checks}
	\label{sec:opposedchecks}
	We went over success, we went over types of success, now the last thing to go over is direct opposition. Rather straightforward the DoS are compared; if they are the same, then the result was so close that the natural skill of the opponents is decisive.
	\par
	\pagebreak %quick fix
	Compare the two target values with the higher one getting a very small success over the other. If even these two are the same, usually nothing happens. Three more examples depicting the different situations:
	\begin{exampleblock}\textit{
	Alice is stopped from entering a building by Bob, a cautious bouncer. Since she does not want to cause a ruckus that could alert her target inside, she tries to smooth-talk her way through him. To determine the result her game master asks for a Persuasion check, opposed by Bob’s Interrogation check.\\
	Bob, being a naturally perceptive and cautious guy, has 55 Instinct and being trained in Interrogation grants him a further +10 bonus for a target value of \emph{65}. Alice is only slightly less naturally charming, having a Charisma of 45, but since she’s also experienced in persuasion granting her a +20, her target value is also \emph{65}.\\
	In scenario one Alice rolls a 34, Bob rolls a 42, meaning Alice gains 4 degrees of success and Bob gains 3. Therefore Alice has achieved a very slight victory that might just be enough for her to achieve her desired result.\\
	In scenario two Alice rolls a 26 and Bob rolls a 24. Alice gains 4 degrees of success, Bob gains 5. Therefore Alice suffers slight loss here. Whether she may not enter or she enters but is noticed, is up to the GM’s ruling and narration.\\
	In scenario three Alice rolls a 78 and Bob rolls an 82. Therefore Alice scores 2 degrees of failure, so does Bob. The fact both failed their checks is irrelevant for their opposed checks. Since they both passed equally good (or bad), Bob - having the stronger natural inclination - wins the opposition only very slightly. This might just make him more suspicious or completely aware but not hostile, depending on GM narration.\\
		}
	\end{exampleblock}

	\section{Assistance}
	Outside of combat, especially during extended actions, one character is designated as the primary actor and other characters may become assisting actors. Assisting actors make their specific tests first, granting the primary actor 5 * their degrees of success as a bonus or 5 * their degrees of failure as a penalty to her test. Assisting tests are not necessarily required to use the same characteristic or skill, for example:\par
	\begin{exampleblock}
		\itshape
		John attempts to help Kelly jump into a window. Since the window is narrow, Kelly will be making a check against her agility based Acrobatics skill, while John - using controlled, precise strength - uses his strength based Athletics skill.
	\end{exampleblock}

	\section{Characteristics}
	We now know how to perform dice checks but how do we determine our target values? Most basic to the target value is the character’s Characteristic. These are the most basic features that define a creature. Usually one of these values will be the basis for most dice checks:
	\begin{itemize}
		\setlength\itemsep{-8mm}
		\item \textbf{Courage (Cr)} describes the creature’s strength of will and the creature’s ability to overcome emotional impulses like obviously fear but also e.g. anger.
		\item \textbf{Intelligence (Int)} describes the creature’s capability to remember old as well as to understand and learn new things. It also encompasses drawing conclusions from things the creature knows.
		\item \textbf{Instinct (Ins)} describes the creature’s reaction speed in situations that don’t allow for time to think. Also encompasses general perception.
		\item \textbf{Charisma (Ch)} describes the creature’s presence. The more charisma the more self-confident and persuasive the creature appears.
		\item \textbf{Dexterity (Dex)} describes the creature’s finesse with very fine tools such as mundane lockpicks or a blowtorch.
		\item \textbf{Agility (Ag)} describes the creature’s body control and flexibility.
		\item \textbf{Constitution (Con)} describes the creature’s bodily sturdiness, endurance and pain tolerance.
		\item \textbf{Strength (Str)} describes the creature’s bodily, physical strength and ability to use it effectively.
	\end{itemize}
	Each characteristic can be increased 5 times by 5 points each by spending XP. Said advancements cost 200, 400, 800, 1200 and 1600XP respectively, making for 4200 in total. These have to be bought in order for any one given characteristic.
	
	\begin{minipage}{\columnwidth}
	\section{Characteristic bonuses}
	Sometimes a mechanic will require a characteristic bonus. This is the characteristic divided by 10 and rounded down - a characteristic of 31 would mean a characteristic bonus of 3 while a characteristic of 49 would result in a bonus of 4. A theoretical characteristic of 125 would result in a characteristic bonus of 12.
	\end{minipage}
	
	\section{Derived values}
	A few additional values are directly derived from the set of characteristics. \\
	They include the attributes that define \emph{limits for augmentation}:
	\begin{itemize}
		\setlength\itemsep{-8mm}
		\item \textbf{Rayleigh Index \textlangle(Cr+Int+Con)/20\textrangle:}
		Cybernetic augmentation puts a massive strain on the human body, mainly the nervous system.
		This number indicates the body’s limit for said load.
		It must never be surpassed or the character is at immediate, high risk of vital organ failure, brain and nerve damage and more.
		\item \textbf{Medical Toughness \textlangle(Ag+Con+Str)/20\textrangle:}
		This simplified measurement represents all factors that allow or disallow further bio-engineered parts to be accepted by the character’s body.
		This must also never be surpassed, though the exact consequences are dependent on the body part that is being installed 
		- ranging from organ activation failure, over inhibited circulation, up to spontaneous fatal cardiac arrest.
	\end{itemize}

	They also encompass \emph{combat-related values}:
	\begin{itemize}
		\setlength\itemsep{-8mm}
		\item \textbf{Base Initiative \textlangle(Cr+Ins+Ag)/10\textrangle:} The base value for the character’s initiative. While that may be modified by circumstances and abilities (not to mention a dice roll), “base initiative” refers to exactly this value.
		\item \textbf{Base Melee Combat \textlangle(Cr+Ins+Ag+Str)/3\textrangle:} The value representing the character’s basic skill in melee combat, armed or unarmed.
%		This serves as the foundation for proper combat training.
		\item \textbf{Base Ranged Combat \textlangle(Int+Ins+Dex+Str)/3\textrangle:} The value representing the character’s basic skill in ranged combat, be that thrown weapons or firearms.
%		This serves as the foundation for proper combat training.
	\end{itemize}

	Unlike the values above, the following are mainly used for combat yet should be recalculated on the fly, whenever temporary modifiers are applied:
	\begin{itemize}
		\setlength\itemsep{-8mm}
		\item \textbf{Injury Threshold \textlangle ConB\textrangle:} The amount of damage the character can take before he is actually injured to the point of pain, bleeding and weakening. All damage taken is reduced by this and it is not ignored by an attack’s armor penetration.
		\item \textbf{Melee Damage Bonus \textlangle(Ag+Str)/10\textrangle:} Not only technique but also raw ability can increase a person’s damage dealt with a vicious slash or crushing blow in melee combat.
	\end{itemize}

	Lastly there are values that are not necessarily restricted solely to combat:
	\begin{itemize}
		\setlength\itemsep{-8mm}
		\item \textbf{Movement Speed \textlangle(Ag+Ag+Str)/40 + 5\textrangle:} A character’s total sprint speed, explained in more detail in the chapter Movement.
		\item \textbf{Standing Leap \textlangle(Ag+Str)/40\textrangle:} The length a character can pass while jumping forward, explained in more detail in the chapter Movement.
		\item \textbf{Standing Jump \textlangle Standing Leap/4\textrangle:} The height a character can jump up, explained in more detail in the chapter Movement.
	\end{itemize}
	\section{Skills}
	Likely being the most common modifier on dice checks, skills represent training in a field. This training ranges through:
	\begin{itemize}
		\setlength\itemsep{-8mm}
		\item \textbf{unknown:} The character has never had any contact with this very advanced subject. 
		This infers at least a -40 penalty to according skill checks or might also make checks completely impossible outright.
		\item \textbf{untrained:} The character has had very little contact with this complex field or barely touched this simple field.
		This infers a -20 penalty to according skill checks.
		\item \textbf{known:} The character has a grasp of this field’s basics. No modifier is applied.
		\item \textbf{trained:} The character had either some formal training or a natural knack for this field. This grants a +10 bonus to associated skill checks.
		\item \textbf{experienced:} The character had extensive training in this field. This grants a +20 bonus to associated skill checks.
		\item \textbf{mastered:} The character learned everything in this field. This grants a +30 bonus to associated skill checks. If specified or determined by the GM, this may have other prerequisites.
	\end{itemize}
	Basic skills start untrained, advanced skills start unknown. The first increase costs 200 XP, the second one (taking a basic skill from known to trained for example) will cost 400, the third 600 and so on. That means increasing a basic skill from untrained to mastered will cost 2000 XP and taking an advanced skill from unknown to mastered will cost 3000 XP overall.
	\section{Educations}
	\label{eds-explanation}
	In addition to directly applicable fields characters may be trained in, they also have more general, broader knowledge.
	While this might not replace a skill, it might assist them in many different fields, adding their modifier to any test the GM deems applicable.
	Educations are extremely diverse, so this is just a small list of examples:\\ \\
	\begin{tabular*}{\textwidth}{|l|@{\extracolsep{\fill}}r|}
		\hline
		Education               & Example Skills \\
		\hline
		Armoring                & Security, Survival, Technology \\
		Ground Vehicle Dynamics & Drive, Technology \\
		Psychology              & Appeal, Commerce, Deceive, Interrogation \\
		\hline
	\end{tabular*}
	\par \vspace{-9mm}
	An education has two potential levels, granting +5 modifiers each and costing 200 and 300 XP respectively.
	\section{Situational Modifiers}
	Some things become easier or harder in certain situations. Potential situational modifiers are vast and always up to GM discretion. Some common circumstances include lighting and the weather for sight based checks or mood, relationship and general disposition for social checks. Tables from page \pageref{situationalmodifiers} onward contain more potential circumstances and their respective modifiers. Obviously those lists are not exhaustive.
	
	\section{Automatic Successes and Failures}
	Independently of modifiers and target values rolling a 1-3 always counts as an automatic success.
	Likewise rolling a 98-100 always counts as an automatic failure.
	\section{Critical Results}
	Rolling doubles on a test indicates a critical result. Determine whether the test succeeded or failed.
	\textit{In addition 1s and 100s are always critical successes or failures respectively.}\\
	In case of a critical success the Degrees of Success are doubled and ties are always won.
	In combat the attack deals damage even if the opponent evades or deals double damage if he does not.\\
	Conversely in case of a critical failure the Degrees of Failure are doubled and ties are always lost.
	In combat the opponent's action is improved - from failure to success, from success to critical success.\\
	Critical results may always have additional effects as per the narrative and GM discretion.
	
	\section{GM Advice: When to roll}
	You don't always need to demand a roll.\\
	In cases where nothing is at stake and characters have enough time, do not have your players roll. They succeed.\\
	In cases where the target value surpasses 100 or goes below 0, rolls may be unnecessary and the result predetermined.\\
	Note that if you demand a roll, players will demand the result to matter.
	
	\section{Skill check examples}
	To give a better overview of how different modifiers add up, here are two somewhat extraordinary examples for each of the classic pillars of tabletop roleplaying. In general these amounts of modifiers might come up in play but situations like these are rare and sometimes hard to keep track of.
	It should be the table’s collective decision how many different modifiers and which of them in particular are being included. Also this section may only offer mechanical solutions; players might come up with way more creative responses that shake up the situation in a way that makes rolling dice unnecessary.
	
	\pagebreak
	\subsection*{Combat}
\vspace{5mm}
\begin{multicols}{2}
	\ul{Situation:}
	\begin{exampleblock}
		\textit{Storm hits Eric’s face, the cold rain almost cutting up his face. The faint moonlight is barely visible through the densely clouded night sky. His old, trusty scope had long since broken down and the barrel was damaged, yet he had to take this chance for it might be the last he’d ever get.}
	\end{exampleblock}
	\columnbreak
	\ul{Mechanical Solution:}
	\begin{exampleblock}
		Both strong winds as well as heavy rain infer a -10 each and low light conditions bring another -20 penalty for a -40 from environmental effects. Taking aim would grant him a bonus of up to +20, yet due to his broken scope and assuming he does not have a functioning backup for this distance, only a +10 is granted. Assuming again that he is shooting over a long distance as to not be noticed, this infers another -20 penalty that is more than made up for by the fact that his target is now unaware, granting a +30 to hit.\\
		In total the shot is at a -20 penalty and the target may not dodge.
	\end{exampleblock}
	\columnbreak
	\ul{Situation:}
	\begin{exampleblock}
		\textit{Within the haze, the noise, the people - there he is, trying to flee the scene. Jenny has her target in her sight, a chance to sneak up - and a chance to strike. Just before she embeds the blade in his spine, he notices, making her job just that much harder.}
	\end{exampleblock}
	\columnbreak
	\ul{Mechanical Solution:}
	\begin{exampleblock}
		The attack within a hazy, cramped environment is marginally more difficult for a knife, resulting in a -10. She snuck up but was noticed last minute, so the GM calls the target surprised for a +20 bonus; resulting in a +10 bonus overall.\\
		The unnamed defender is impaired by the same environmental effects. Additionally he is unarmed but defends against a knife, which adds a penalty of -10. In total the defense roll gains a -20.
	\end{exampleblock}
\end{multicols}

\pagebreak
\begin{multicols}{2}
	\subsection*{Exploration}
	\ul{Situation:}
	\begin{exampleblock}
		\textit{"Huh, footprints?" The snow isn't deep but enough to see the outlines. "Is this our man?"}
	\end{exampleblock}
	\ul{Mechanical Solution:}
	\begin{exampleblock}
		Since the footprints are clear enough to be seen, we don't require a Perception check to identify them. To answer the question however we will require a Survival test based on Instinct to cover tracking.\\
		Modifiers are less clear here than we'd like them to be. Let's say the outline of the footprints is enough to determine the direction and to follow them, granting a +20. To make the situation a little more interesting, let's say it's still snowing and there's not just the one strain of footprints. Just like in combat the snow reduces visibility, reducing the current bonus by 10. Additionally the overlapping footprints make it slightly harder to follow, granting a further -5 for a total of +5.
	\end{exampleblock}
	\columnbreak
	\ul{Situation:}
	\begin{exampleblock}
		\textit{Once again Yokai was on the run again, his pursuer glued to his heels. Hoping he's more agile, that construction fence may make the difference.\\
			While yes, he is more agile, the massive pursuer bashes straight through.}
	\end{exampleblock}
	\ul{Mechanical Solution:}
	\begin{exampleblock}
		Yokai tries to gain an advantage by vaulting over a fence. Since this is done fast, Climbing is out of the question, so we turn to Acrobatics. His pursuer tries just pushing on. Since directing strength is covered by Athletics, let's use that but considering the obstacle in his way, he gains a -10 penalty.\\
		Both roll off against each other and the winner gains a considerable advantage: either the pursuer catches up or Yokai loses him. If both fail, that does not matter for the result, only for the way they get there; they may both stumble except one gets up again faster.
	\end{exampleblock}
\end{multicols}

\pagebreak
\begin{multicols}{2}
	\subsection*{Social}
	\ul{Situation:}
	\begin{exampleblock}
		\textit{Shrapnel flies across the room. A gunshot on the other side. Mere moments later more debris, a second bang.
			"These negotiations were supposed to be way less-"\\
			Dwayne's joke is interrupted by his screaming, as a bullet pierces his armor and embeds itself in his abdomen.\\
			"11mm hollow point, haven't felt that in a while."\\
			Blood drips to the floor, no bandages in sight - and no bullets left. Another shot barely misses his head, leaving a massive hole as it breaks through a weaker part of the table he is hiding behind. He lobs his gun into the middle of the room as his last way out of this mess. Larry approaches.\\
			"Look at you, like a beaten puppy, lying on the floor in a pool of your own blood and tears. You should have just given me the case."\\
			Dwayne stares him down, pulls the pin on a grenade, grasping it with the last ounce of his strength.
			"Piss off now or neither of us will make it."}
	\end{exampleblock}
	\ul{Mechanical Solution:}
	\begin{exampleblock}
		Quite obviously this situation calls for an Intimidation (courage) roll which Dwayne is \emph{experienced} in, netting him a +20 bonus. Due to his severe injuries, he will assist himself with his \emph{trained} Restraint (Constitution) - representing him powering through his pain. Under the effects of strong painkiller - granting him +10 - he scores one degree of success, netting him another +15 to the test. His courage of 33 and modifiers add up to a target value of 78.\\
		He is opposed by Larry's Restraint (courage). He is not in a particularly good spot - not under any enhancing influence and not very steeled to the horrors of war. Therefore his \emph{known} Restraint and his courage of 38 is all he can fall back on, making this opposition a very one-sided affair.
	\end{exampleblock}
	\columnbreak
	\ul{Situation:}
	\begin{exampleblock}
		\textit{"Get your sorry asses up!" Ness' voice was booming, grabbing attention of the rundown mercenaries he once called comrades.\\
			"You look miserable. Oh, how the mighty have fallen. If we can do it, why can't you?! Do you want to live forever!"}
	\end{exampleblock}
	\columnbreak
	\ul{Mechanical Solution:}
	\begin{exampleblock}
		Trying to motivate the mercs to pick up arms again and fight for their cause should be easier than introducing them to something new; so let's assume a +20 modifier. Additionally the mercs already know him; while not quite being friends, a friendly relationship grants another +20 for a total +40. The amount of people following after his call would then depend on the DoS - possibly around 4 and 2 more per DoS, in case numbers are necessary.
	\end{exampleblock}
\end{multicols}

	\chapter{Movement}
	\section{Walking, running, sprinting}
	Characters move at different speeds, starting with a casual stroll and ending in a full-power short distance sprint.\\
	\textbf{Walking speed} is about 3-4km/h, 1m/s or 3m per turn for a normal human being. Being larger or smaller might affect this but to a generally negligible degree. It is not exhausting and most things are not impaired at this speed.\\
	\textbf{Running speed} is half a character’s movement speed in meters per second. At this speed most things and most notably combat actions take a -20 penalty. It is also exhausting after a while, so every ~5 minutes a jogging character takes a level of exhaustion.\\
	A normal character’s \textbf{sprint speed} in meters per second is given by their derived value movement speed. At this speed most actions and most notably all combat actions take a -40 penalty or might, as per GM discretion, be completely impossible - firing machine guns while sprinting might be difficult due to bulk for example. A running character takes a level of exhaustion every other round.
	\section{Marching}
	For the most part getting access to vehicular transport is easy enough. In some instances though it might be necessary to take longer marches on foot. Normal groups travel at roughly 30 kilometers a day or 40 if they push themselves. Marching is exhausting and every traveler will gain D5 levels of exhaustion (or D5+2 if they're pushing), adjusted depending on their journey.
	\section{Climbing, jumping, leaping}
	Sometimes getting from point A to point B becomes difficult when they are on different elevations. It might also be necessary to cross over obstacles in one’s way and do so fast.\\
	\textbf{Climbing} works just as walking or running would, only that speed is halved. Since usually the climbers hands are occupied, what actions can be taken is up to GM discretion.\\
	Actions that need footwork or are taken twisted behind one’s back suffer at least a -20 penalty - potentially in addition to -20 for actions during movement - if they are possible at all. When in doubt, ask for a Climbing check.\\
	\textbf{Leaping} is another important form of horizontal coverage. Leap length is technically a characteristic-derived value, though it decreases with exhaustion: (AB + SB - Exhaustion)/2. Leap length can be doubled with proper run-up, a minimum of about 4 meters.\\
	Finally \textbf{jumping} is probably the most basic form of vertical coverage. Jump height is a quarter of leaping distance, follows the same rules for run-up and - if the character has something to grip and at least one free hand - may end in a climb.
	\section{Swimming}
	Swimming is the last form of human movement and quite possibly the slowest. For most people swim speed is roughly a tenth of a character’s movement speed on land. Depending on equipment load, armor and special augmentation it may be even lower or completely impossible, per GM discretion when in question.
	\section{Movement and size}
	Due to longer extremities larger characters are generally faster, as per the size table (see p. \pageref{sizestable}).\\
	Additionally characters that are at least 2 categories apart can temporarily occupy the same space, as the small character runs through the larger character’s legs or climbs on their back. Use common sense to determine where and when that is and is not possible. In any case it should slow down both characters.
	\section{Encumbrance: bulk and weight}
	When a character is carrying too much, he will eventually be \emph{over-encumbered}. Any object that is carried at one's person but outside of a proper pouch will cause one point of encumbrance from \emph{bulk}.\\
	Any excess \emph{weight} will also cause encumbrance. One point is gained for surpassing half the character’s strength in kilograms and for a quarter of the character’s strength each after that point. Worn armor is ignored, so long as it fits.\\
	While over-encumbered, he takes a -10 on all physical actions - most notably combat - for every point of encumbrance he has. Speed is reduced by 1 for every two points of encumbrance.

	\chapter{Combat}
	\section{Initiative and combat speed}
Before combat can start, every character needs to have a total initiative value.
Every creature takes its \emph{base initiative} value
	and adds 1D5,
	as well as potentially other circumstantial modifiers to it.\\
Every combat round is about 3 seconds long and consists of a \textit{declaration phase} and an \textit{action phase}. Having higher initiative means declaring later, when that character knows what slower characters will do this round, and resolving their actions before those of slower characters.
\subsection{Declaration Phase}
In \emph{ascending} order all characters exclaim their plans for the combat round. This includes movement, converting actions and reactions, attacks and their targets, drawing and readying equipment - pretty much everything besides reactions and exact maneuvers.
\begin{exampleblock}
	\itshape
	Three contenders are involved in a fight.\\
	Jane has 8 initiative after hitting her head multiple times, making her the slowest, therefore first to declare. She intends to pick up a weapon from the floor and whack Paul over the head.\\
	Anne, with an initiative of 12 is next in line. Since she fears being attacked and severely injured before she can attack, she will block twice. As to not cause attacks of opportunity, she does not move.\\
	Paul is the fastest with an initiative of 16 and therefore declares last. Knowing he's at a disadvantage, he disengages and moves away, keeping one reaction for the attack he knows is coming.
\end{exampleblock}
\subsection{Action Phase}
In \emph{descending} order all characters resolve their actions. Depending on the actions that resolve earlier, some declared actions may become invalid due to ranges and line of sight - this is the main advantage of being faster in combat.
\begin{exampleblock}
	\itshape
	First Paul, the fastest combatant, resolves his actions. He takes a disengaging stance as to not suffer attacks of opportunity and runs.\\
	Anne, having prepared to block or evade, essentially skips her turn.\\
	Lastly, Jane moves after Paul and attacks. She's attacking in movement unexpectedly and therefore has to deal with a -20 in addition to the modifiers she already has.
\end{exampleblock}
\subsection{Variable Initiative}
\label{subsec:variableini}
Initiative is variable. When a character is disorientated by being blinded, stunned or falling to the ground, he loses initiative. How much is up to the GM but should be between a D5 for taking a wounding hit to the head and a D10 for dropping to the floor.
\\%
Unlike a reduced Base Initiative, this may be regained by taking the \emph{Gather Senses} action.
\section{Control area}
Every combatant has a \emph{control area}, which is 120° in front of the character and within melee range. \\
When fighting multiple enemies in melee
	that are not all within the character's control area,
	he takes -20 to all tests in that combat encounter.
This is in addition to being outnumbered.
\section{Actions}
Every character has 2 actions per combat round, \emph{in addition} to being allowed to move. Remember that all actions are happening simultaneously and some actions might be more difficult to do while moving.\\
There are two types of combat actions: active actions and reactions.
By default one of the character’s actions is active, the other is reactive.
Reactions are parries and dodges - everything else are generally active actions.
Actions are interchangeable:
	when turning an action into a reaction or vice versa,
	any test in the turned action will be at a -20 penalty.
Turning actions has to be \emph{declared}.
\section{Free actions}
Free actions can be used at any time and don't use up an action. Free actions cannot be used to interrupt another character that has higher initiative. Any character is generally limited to one of any free action, e.g. a character cannot drop to the ground, stand up using an action and drop back down again. Remember that a round of combat is about 3 seconds long - when in doubt, apply common sense.
\section{Maneuver basics}
A maneuver describes a combat action a character can take.
These actions might look differently or be executed in a different way,
	yet mechanically act the exact same way:
	a skillful attack to the most vulnerable parts of the human body
	and a particularly powerful blow
	will both increase the impact’s damage
	at a cost to accuracy
	and therefore are both mechanically \textit{power strikes}.\\
Unless specified otherwise any character can use any maneuver even without being properly trained (i.e. knowing the ability) but all penalties are twice as high to achieve the same results.
\section{Voluntary penalties}
Most maneuvers take great risk and the bigger the risk, the higher the reward. Voluntary penalties are taken by the acting character and have some sort of effect on maneuvers. The maneuvers that profit from voluntary penalties are designated with a \emph{-X} modifier.\\
Voluntary penalties can’t be taken to reduce the target value below 20. If multiple maneuvers profiting from voluntary penalties are combined - e.g. a \emph{power strike} and a \emph{faint} - the benefits are split freely among them.
\begin{exampleblock}
	\itshape
	Nick attacks with a spear (value of 60). He has both the ability Power Strike and Faint.\\
	He takes a voluntary penalty of 30 - imposing a -20 penalty to his opponent's parry and increasing his damage by 2.\\
	His penalty would be limited to 40.
\end{exampleblock}

\section{Shooting size \& multiple targets}
When attacking a group of targets or one large target in \emph{ranged} combat,
	hitting becomes easier.
Depending on the overall target size a bonus to hit given on the Size-table (p. \pageref{sizestable}) is inferred.
If the target is a group, the individual target is determined randomly if necessary.\\
Attacking a single target in a large, moving group is impossible and the target will have to either be lured out or hit by chance or explosives.
\section{Dual-wielding}
When wielding two weapons,
	the character may attack or defend using both weapons pending just one action.
Using one-handed weapons infers a -30,
	using two-handed weapons infers -60.
Additionally, attacking different targets suffers a -10 penalty.\\
Attacking or parrying twice is not possible in turned actions,
	i.e. dual-wielders cannot attack or parry 4 times per turn.
\section{Combined actions}
\label{combinedaction}
To take combined actions, the involved members need some way to communicate - preferable privately - and all take their go on the lowest involved initiative. They don't have to test against the same characteristic or skill, but they should in some way support each other.\\
The team may then distribute all Degrees of Success among each other freely, allowing them to cancel out Degrees of Failure and making failed tests succeed.
\begin{exampleblock}
	\itshape
	Three people participate in a combined action. Two of them succeed, with 3 and 4 DoS respectively. The last one fails with 3 DoF.\\
	In total the group scores (7-3=) 4 DoS,
		which they can distribute among themselves freely.
	Any test that scores 0 DoS after distribution indicates an irrelevant failure.
	\\%
	If the total DoS score less than 0,
		the action results in some undesirable outcome.
	Details should depend on how bad the result was
		and are up to the GM.
\end{exampleblock}
\section{Damage and armor}
When a character takes a hit that is not defended against, first worn armor and cover are reduced by the attack's armor penetration, down to a minimum of 0. \\
The leftover armor and the target’s injury threshold then reduce the attack’s damage, again down to a minimum of 0. \\
Any leftover damage is then applied to the hit location
	and the target now takes penalties in accordance.
\begin{exampleblock}
	\itshape
	An attack deals 2D10+5 damage (rolling 15 in total) and has 6 AP.\\
	The defender wears "armor" with an armor value of 4 and sits behind a thin wall with armor value of 8, for a total of 12. His Injury Threshold is 3.\\
	The effective armor is reduced to 6. Now both the armor and the Injury Threshold is deducted from the damage, so the hit deals (15-6-3=) 6 damage in total.
	\par
	\vspace{2mm}
	\hrule
	\vspace{2mm}
	Another hit from the same weapon against the target without cover:
	The attack deals 15 damage and has 6 AP.
	The defender still wears 4 points of armor and that is his total this time.
	\\%
	His armor is reduced by 6, down to (a minimum of) 0.
	Now only his Injury Threshold is deducted from the damage,
		meaning the hit deals 12 damage in total.
\end{exampleblock}
\section{Cover and concealment}
Any object blocking line of sight is considered \emph{Concealment}. Attacking a target behind such concealment comes with a -30 penalty as if blindfiring.
\emph{Cover} on the other hand provides additional armor points depending on material and thickness.
\\%
Cover degrades. For every 5 points of damage a section of cover blocks, the armor it provides is reduced by 1. If a section of cover functions as both cover and concealment, the concealing effect stops when roughly 25\% of armor remains. When in doubt, apply common sense and let the GM make an executive decision.
\par
Examples for cover by material:
\\
\begin{tabular}{ll}
	Material & Armor per cm \\ \hline
	Light: Wood, Plastic & 2 \\
	Medium: Glass, Soil & 5 \\
	Heavy: Concrete, Steel & 8
\end{tabular}

	\newpage
\section{General combat actions}
\subsection*{Stand up / Hop up / Drop}
Unless restrained a prone character can spend one action to start kneeling, a kneeling character can spend an action to stand up. A prone character may choose to make an agility check instead to stand up fully in one action. On a success he stands up, on a fail he stays prone.\\
Dropping lower is a free action, no matter if to the knees or prone.
\subsection*{Jump to Cover}
When announcing actions for the combat round a character may choose to jump into cover. At that moment he dashes a distance up to twice his movement characteristic and counts as being in cover effective immediately. This counts as moving and cannot be combined with other movement.
\subsection*{Gather senses}
The character gets a grip of the battlefield again after losing her overview. She re-rolls initiative.
\subsection*{Draw / Ready}
Draw a weapon or ready an equipment piece.
\subsection*{Defensive stance}
A combat stance to increase one’s defensive capabilities. Grants a -30 to offensive actions but +10 to defensive actions. Unlike other defensive actions this must be declared during the declaration phase but it may be taken without an ability.
\subsection*{Assistance / Cooperation}
In combat assistance should be handled as a combined action (see \ref{combinedaction}, page \pageref{combinedaction}).
\subsection*{Spotting}
Spotting is a special type of assistance. The shooter and the spotter take a combined action with the spotter using Perception. 10 times his DoS will be added to the shooter’s attack roll instead of 5 times.
\subsection*{Suppress pain}
Penalties from being injured are in large parts due to pain. Trained individuals may attempt to suppress pain and carry on for a while.\\
As an action the character may make a Restraint test. This test is at a penalty equal to the penalty the character wants to suppress. If the test succeeds, the character may ignore said penalties for the combat or scene or until he takes further damage.
\begin{exampleblock}
	In a firefight Nathan was injured in his right arm and left leg, having taken 7 points of damage each. To get the hell outta dodge he tries to push through the pain: He makes a \emph{Restraint} test at a penalty of 20 for his leg and 20 for his arm for 40 in total.
\end{exampleblock}
	\section{Melee actions}
\subsection*{Standard attack}
A single armed or unarmed melee attack. Attacking a specific location is modified by its size. This is compatible with any maneuver and does not require an ability. 
\subsection*{Sure strike (+20, also uses up reaction)}
A single melee attack with higher chances to hit. This maneuver \emph{cannot} be combined with any other and does not require an ability.
\subsection*{Attack of opportunity (-10, free action)}
When an enemy passes through or leaves the character’s control area, the character may make a single attack as a free action with a -10 modifier in addition to any other maneuver. This may not be any maneuver requiring more than one action.\\
One attack of opportunity may be made against any enemy once per combat round.
\subsection*{Disengage}
\label{action:disengage}
When disengaging, no attacks are allowed to be made that round but no attacks of opportunity are invoked when passing through other characters' area of control.
\subsection*{Feint (-X)}
A skillful attack that is harder to defend against. Defending against this maneuver is at a penalty equal to the voluntary penalty.
\subsection*{Called jab (-20 -opponent’s armor)}
A strike against an armor’s weak points. The attacker picks the location without additional penalty. The attack ignores half the target’s armor.
\subsection*{Disarm (-20 -X)}
An attack against the opponent’s grip to get control of his weapon. The defender makes a dexterity check at a penalty equal to the voluntary penalty of the maneuver. On a fail the defender loses her weapon, dropping on the ground. Failing by at least 3 degrees gives the attacker control over the weapon.\\
This maneuver does not deal damage.
\subsection*{Flurry (-10)}
Two attacks in quick succession using both hands. Defending against this attack requires two reactions. On a hit the attacker deals damage once with her primary weapon and once with the weapon in her offhand. If she is not carrying a weapon in her offhand, she deals unarmed damage instead. 
\subsection*{Piercing thrust (-40 -half of opponent’s armor, also uses up reaction)}
An all-out stab. The attacker picks the location without additional penalty. The attack ignores both half the target's armor and its full injury threshold.\\
Cannot be performed without the corresponding ability.
\subsection*{Take-down (-10 -X)}
A sweeping and pushing attack to take the opponent off his legs. If the attack hits, it deals no damage. Instead the defender makes an agility check at a penalty equal to the voluntary penalty on the maneuver. If that fails, he falls on the ground. If it fails by 3 or more degrees, he also drops his weapon.
\subsection*{Shift (-10 -X)}
Superior footwork or raw force moves an opponent. If the attack hits, the opponent makes a strength check at a penalty equal to the voluntary penalty on the maneuver or be moved by up to a meter per degree of failure in a direction chosen by the attacker.
\subsection*{Powerstrike (-X)}
A strong blow that increases the hit’s damage. For every 5 points of voluntary penalty the damage is increased by one.
\subsection*{Cleave (-15 per target, also uses up the reaction)}
A half circle of steel pushing multiple enemies back. This maneuver can target up to three enemies. A success indicates hits on all targets. A target hit will - in addition to normal effects - be pushed back by the attacker’s StrB in meters -1 per size category the target is bigger. This maneuver is always also a Knockdown.
\subsection*{Stunning blow (-20 -X)}
A blunt attack to stun a target. If the hit is successful and the attack would deal damage, then deal half of the effective damage and the target makes a Constitution check at a penalty equal to the voluntary penalty on the maneuver. By GM discretion the target may also fall prone.
\subsection*{Knockdown (-20 -X)}
A forceful attack to knock the target off their feet. If the attack hits, the target makes a Strength check at a penalty equal to the voluntary penalty on the maneuver. On a fail the target falls prone.
\subsection*{Charge (-20, also uses up reaction)}
An attack utilizing the momentum of movement. The attack gains a bonus to damage equal to the attacker’s current speed, usually his sprint speed. He needs a minimum of 4 meters of runup.
\subsection*{Crushing blow (-40 -X, also uses up the reaction)}
A powerful, all-out attack that deals greatly increased damage. If the attack is a success, the damage is first increased by one per full 5 points of voluntary penalty and then damage and AP are doubled. Cannot be performed without the corresponding ability.

\section{Melee reactions}
\subsection*{Dodge (-10; -0 against unarmed)}
Evasion without weapons touching. Evasion has to be declared before the attack is rolled. On a success the attack is evaded. This does not require an ability.
\subsection*{Parry}
A basic defense against a melee attack. Parrying has to be declared before the attack is rolled. On a success the attack is evaded. This does not require an ability.
\subsection*{Bind (-X)}
Enter a bind to lead one’s opponent’s weapon and gain an advantage. The next action the character takes has a bonus equal to the voluntary penalty.
\subsection*{Disarm from reaction (-30 -X)}
A very risky defense to rid the opponent of their weapon during her attack. If the defense is successful, the attacker makes a dexterity check at a penalty equal to the voluntary penalty of the maneuver. On a fail the attacker loses her weapon, dropping on the ground. Failing by at least 3 degrees gives the defender control over the weapon.
\subsection*{Intercept (-20, uses attack value)}
A reckless attack into the opponent's attack. If only one attack is successful, that attack deals damage as normal. If both succeed, the one that succeeded better (see Dice checks and stats-Opposed Checks) deals full damage, while the other one only deals half damage.\\
Intercepting a Charge is at an additional penalty equal to half the character’s courage below 80 but deals additional damage equal to the attacker’s movement speed.
\subsection*{Reversal (min -30)}
A skillful counterattack that exploits an overswing. This maneuver can only be used against any maneuver based on \emph{Powerstrike} and grants a penalty equal to the attacking powerstrike penalty, at least -30. If the defense is successful, the maneuver also counts as an immediate powerstrike against the original attacker with a voluntary penalty equal to this maneuver’s penalty. The attack may be parried as normal and may also be subject to Reversal. Cannot be performed without the corresponding ability.

\section{Grappling}
Grappling may be somewhat more complicated than striking due to the amount of options one has and how different these options are to striking: control over someone else and dislocation of opponents or their limbs.
\subsection*{Hold}
A hold is the basis of all grappling. By itself it only stops the opponent from leaving but it is necessary for most other maneuvers to initiate.\\
To initiate a hold, the character makes a grappling check which may be opposed by the target as a reaction. If the character wins, a hold is engaged.\\
To break free the target makes a grappling check which may be opposed by the character as a reaction. If the target wins, it breaks free.
\subsection*{Throw}
The character attempts to lift the target off the ground and forcefully put it back.\\
The character makes a grappling check which may be opposed by the target as a reaction. If the character wins, the target takes D10+StrB damage with the trauma quality and falls prone (therefore also losing D5 initiative).
\subsection*{Lock}
The character attempts to prevent his opponent from moving freely. Arm locks, leg locks and head locks are common but bear hugs also fall into this category.\\
To initiate a lock, the character makes a grappling attack roll which may be opposed by the target as a reaction.\\
While the lock is active, the character may use an action to interfere with any action taken by the target: the character takes a grappling check as an action and the target’s action is at a -10 penalty per degree of success.\\
To break free the target may take a -10 grappling check, which may be opposed by the character as a reaction. If the target wins, it breaks free, back into the initial hold.
\subsection*{Lever hold}
While holding an opponent in a lock of some kind, the character may choose to initiate a lever hold as an action. This forfeits the option to control any action in favor of dealing damage to a location every turn.\\
To initiate a lever hold, the character has to have his target in a lock already and takes a grappling check which may be opposed by the target as a reaction. If the character is successful, a location is determined and a lever hold is initiated.\\
While the lever hold is active, the character deals 1D5-2+StrB damage to the location, ignoring armor.\\
To break free the target may make a -20 grappling check, which may be opposed by the character as a reaction. If the target wins, it breaks free.
\subsection*{Choke hold}
Another common grappling target besides joints are the airways.\\
To initiate a chokehold the character makes a grappling check which may be opposed by the target as a reaction. If the character wins, the chokehold is initiated.\\
The chokehold works like a lever hold but instead of dealing damage, it causes suffocation.\\
To break free the target may make a -20 grappling check, which may be opposed by the character as a reaction. If the target wins, it breaks free.
\subsection*{Move}
Moving while grappling is hard, even more so if only one party intends to move.\\
The character makes a grappling check which may be opposed by the target as a reaction. If the character wins, the target is moved up to the difference of DoS in meters or half the amount if on the ground.

\section{Melee dancing}
When attacking or defending in melee combat, the combatants are rarely stationary. After an attack that does not include intentional movement the character and his opponent move a meter into a random direction within the same control areas. Roll a D10:
\par
\begin{tabular}{ll}
	1-2: & towards the opponent\\
	3:   & forward right\\
	4:   & right\\
	5:   & backwards right\\
	6-7: & away from the opponent\\
	8:   & backwards left\\
	9:   & left\\
	10:  & forward left
\end{tabular}

	\section{Ranged}
\subsection{Weapons}
\vspace{4mm}
\begin{multicols}{2}
\subsubsection{Pistol}
\vspace{2mm}
\rangedweapon{Obsidian Bull}{A hand cannon fed from a cylinder.}{Shots: 1; Damage: 2D10+8; AP: 1}{25m}{6 (11.8x45mm)}{6}{Reliable, Single Loader}{cr 50}{1.7 kg}
\rangedweapon{Wasteland Condor}{A powerful handgun with an extremely heavy cartridge.}{Shots: 1; Damage: 2D10+8; AP: 1}{25m}{7 (11.8x45mm)}{2}{}{cr 50}{1.9 kg}
\rangedweapon{Amber Bull}{A revolver made to fire low caliber rounds.}{Shots: 1; Damage: 2D10+3; AP: 0}{20m}{8 (7.2mm)}{8}{Reliable, Single Loader}{cr 30}{0.5 kg}
\rangedweapon{M10}{A small, low caliber, semiautomatic pistol. Very light and decently easy to conceal.}{Shots: 1; Damage: 2D10+3; AP: 0}{20m}{9 (7.2mm)}{1}{}{cr 30}{0.6 kg}
\rangedweapon{mOP-3}{A hand-sized 20mm grenade hind loader}{Shots: 1; Damage: ; AP: depending on ammo}{20m}{1 (20mm grenade)}{4}{minimum arming distance of 10m}{cr 40}{1.3 kg}
\rangedweapon{"Little Lucifer" Hand Flamer}{A one-handed flamethrower. While it holds "flame" in its name, it can be used for all fueled spray weapons.}{Shots: 5; Damage: 1D10+2; AP: 0}{12m}{5 (small fuel tank)}{8}{Flame, Scatter, Spray}{cr 80}{1.6 kg}
\rangedweapon{OH-VLS}{A small, adjustable laser pistol}{Shots: 2; Damage: 1D10+10; AP: 1}{35m}{16 (small fuel cell)}{4}{Adjustable}{cr 70}{0.6 kg}
\rangedweapon{Newton Pistol Mk5}{A forearm-sized magnetic accelerator shooting metal fragments}{Shots: 1; Damage: 1D10+2; AP: 12}{35m}{30 (small mass driver shard block)}{8}{}{cr 110}{2.6 kg}
\rangedweapon{Newton Pistol Mk2 }{The smallest mass drivers ever built. Shoots very small metal fragments}{Shots: 2; Damage: 1D10+2; AP: 8}{30m}{30 (small mass driver shard block)}{8}{}{cr 95}{2.1 kg}
\rangedweapon{Mk1 Shock pistol}{A modified taser. Very much lethal.}{Shots: 1; Damage: 2D5; AP: 0}{20m}{8 (small fuel cell)}{5}{EMP, Stun (4)}{cr 50}{0.4 kg}
\rangedweapon{Scattershock}{A one-handed Tesla gun}{Shots: 2; Damage: 3D5; AP: 0}{15m}{6 (small fuel cell)}{5}{Scatter, EMP, Stun (3)}{cr 80}{0.6 kg}
\end{multicols}

\subsubsection{Short Rifle}
\vspace{8mm}
\begin{multicols}{2}
\rangedweapon{MS shotgun}{A double-barrel shotgun with shortened barrels to facilitate pellet spread}{Shots: 1; Damage: 2D10+3; AP: 1}{40m}{2 (12gauge)}{4}{Reliable, Scatter, Single loader}{cr 25}{3.0 kg}
\rangedweapon{short-barrel AR-22}{A rifle caliber fired from a short system. Warning: only buy this if you don't own a dog!}{Shots: 3; Damage: 2D10+2; AP: 6}{50m}{30 (6.3mm)}{4}{}{cr 110}{3.5 kg}
\rangedweapon{MP15}{A pistol caliber fired from a small rifle frame.}{Shots: 4; Damage: 2D10+3; AP: 1}{30m}{25 (7.2mm)}{3}{}{cr 65}{2.8 kg}
\rangedweapon{"Thumper"}{A small CQC grenade launcher meant for utility grenades. Exists as an under barrel version for rifles.}{Shots: 1; Damage: ; AP: depending on ammo}{20m}{1 (20mm grenade)}{4}{minimum arming distance of 10m}{cr 70}{4.8 kg}
\rangedweapon{OH-VLS conversion kit}{A conversion kit to turn an OH-VLS into a small laser carbine.}{Shots: 3; Damage: 1D10+12; AP: 1}{70m}{16 (medium fuel cell)}{5}{Adjustable, Flame, Upgrade (OH-VLS)}{cr 140}{2.8 kg}
\rangedweapon{Newton Pistol Repeater kit}{A conversion kit for low powered Newton Pistols to increase fire rate.}{Shots: 4; Damage: 1D10+3; AP: 12}{45m}{30 (small mass driver shard block)}{8}{Upgrade (Mk2 Newton Pistol)}{cr 140}{3.6 kg}
\rangedweapon{"Junk Jet" Mass Driver Shotgun}{A mass driver with increased bore to achieve shotgun-like ballistics}{Shots: 1; Damage: 2D10+2; AP: 8}{55m}{6 (small mass driver shard block)}{3}{Scatter}{cr 170}{4.2 kg}
\rangedweapon{Crossbow / Harpoon}{A very short bow with a trigger }{Shots: 1; Damage: 1D10+4; AP: 0}{50m}{1 (crossbow bolt)}{6}{Silenced, ignores shields}{cr 25}{3.0 kg}
\rangedweapon{Scattershock rifle conversion kit}{A conversion kit for the Scattershock to feed from larger fuel cells. Increases size, mag size and range.}{Shots: 2; Damage: 3D5; AP: 0}{45m}{30 (medium fuel cell)}{5}{Scatter, EMP, Stun (3), Upgrade (Scattershock)}{cr 125}{2.1 kg}
\end{multicols}

\subsubsection{Long Rifle}
\vspace{8mm}
\begin{multicols}{2}
\vspace{2mm}
\rangedweapon{AR-22}{A full sized rifle}{Shots: 3; Damage: 2D10+3; AP: 11}{120m}{30 (6.3mm)}{6}{}{cr 150}{3.8 kg}
\rangedweapon{M282 SSW}{A bulky, belt-fed fire support weapon with high ammunition capacity}{Shots: 6; Damage: 2D10+3; AP: 11}{120m}{120 (6.3mm)}{14}{Bulky}{cr 220}{7.0 kg}
\rangedweapon{Mk3 Tankgewehr}{An insanely high-caliber, long barrel rifle made to fight heavy armor. Carries a lot of energy of comparatively small distances.}{Shots: 1; Damage: 3D10+3; AP: 30}{250m}{1 (15x132mm)}{10}{}{cr 210}{16.6 kg}
\rangedweapon{A3R}{The "Makar Advanced Anti-Armor Rifle": A long range rifle made to take out high value targets or light armor. Not an overly clever name but don't tell the snipers.}{Shots: 1; Damage: 2D10+7; AP: 18}{800m}{5 (11.8x112mm)}{6}{}{cr 210}{12.0 kg}
\rangedweapon{M5 Shotgun}{A scatter gun firing multiple pellets}{Shots: 2; Damage: 4D5+6; AP: 4}{80m}{8 (12gauge)}{16}{Single Loader, Scatter}{cr 130}{3.8 kg}
\rangedweapon{OP-2}{A 20mm single shot grenade launcher with increased barrel length}{Shots: 1; Damage: -; AP: depending on ammo}{50m}{1 (20mm grenade)}{4}{minimum arming distance of 10m}{cr 60}{1.9 kg}
\rangedweapon{"Rathalos" Flamethrower}{A flame thrower with an integrated tank to profit from familiar, proven rifle ergonomics.}{Shots: 5; Damage: 2D10+4; AP: 0}{35m}{15 (medium fuel tank)}{8}{Flame, Scatter, Spray}{cr 120}{4.0 kg}
\rangedweapon{BR-56}{A semiautomatic laser rifle}{Shots: 2; Damage: 1D10+18; AP: 3}{500m}{10 (medium fuel cell)}{12}{Adjustable}{cr 230}{4.3 kg}
\rangedweapon{Incinerator Rifle}{A long range laser rifle. Instead of piercing armor it burns it away.}{Shots: 1; Damage: 2D5+30; AP: 4}{700m}{1 (large fuel cell)}{16}{Adjustable, Flame}{cr 220}{9.7 kg}
\rangedweapon{M4 "Silver Crow"}{A handheld, semi automatic mass driver}{Shots: 2; Damage: 2D10+4; AP: 22}{550m}{16 (large mass driver shard block)}{10}{}{cr 180}{11.2 kg}
\rangedweapon{R235 Squad Support Weapon}{A fully automatic mass driver}{Shots: 5; Damage: 2D10+4; AP: 16}{180m}{180 (large mass driver shard block)}{18}{}{cr 250}{15.6 kg}
\rangedweapon{R13 "Armor Cracker"}{A handheld, long range mass driver}{Shots: 1; Damage: 1D5+18; AP: 35}{1400m}{7 (large mass driver shard block)}{16}{}{cr 260}{23.9 kg}
\rangedweapon{FG-9}{A rifle firing high energetic, self-stabilized plasma that explodes on impact. The energy release is highly unpredictable}{Shots: 1; Damage: 4D10+6; AP: 1}{40m}{10 (large fuel cell)}{16}{Blast (3), EMP, Stun (2)}{cr 310}{6.3 kg}
\rangedweapon{Choke M3 Arc Rifle}{An arc rifle with little spread to increase range}{Shots: 2; Damage: 3D10; AP: 0}{60m}{10 (large fuel cell)}{12}{EMP, Stun (3), Scatter}{cr 180}{4.1 kg}
\rangedweapon{Choke F3 Arc Thrower}{An arc rifle with a large spread made for crowd suppression}{Shots: 2; Damage: 5D5; AP: 0}{30m}{10 (large fuel cell)}{10}{EMP, Stun (2), Spray}{cr 150}{3.9 kg}
\end{multicols}

\subsubsection{Heavy}
\vspace{8mm}
\begin{multicols}{2}
\rangedweapon{M2 Brass Storm}{A high-caliber machine gun with high fire-rate.}{Shots: 8; Damage: 2D10+4; AP: 10}{220m}{120 (11.8x112mm)}{18}{}{cr 280}{42.7 kg}
\rangedweapon{M36 Chain Cannon}{A heavy machine gun with an external cycler. Reduced fire rate when compared to the M2 but renowned for unrivaled reliability. It is extremely heavy and cannot reasonably be called portable.}{Shots: 5; Damage: 2D10+4; AP: 10}{240m}{120 (11.8x112mm)}{18}{Reliable}{cr 295}{69.2 kg}
\rangedweapon{RM4 Chain Shotgun}{An automatic, belt fed shotgun known for its high maintenance need.}{Shots: 4; Damage: 4D5+6; AP: 4}{80m}{80 (12 gauge)}{18}{Scatter}{cr 280}{28.4 kg}
\rangedweapon{"Forge Fire" Heavy Flamer}{A large flamethrower made for vehicles and emplacements}{Shots: 5; Damage: 3D10+3; AP: 0}{50m}{30 (large fuel tank)}{40}{Flame, Scatter, Spray}{cr 220}{35.4 kg}
\rangedweapon{"Slicer" Laser Cannon}{A vehicle laser cannon, powerful enough to break bunker doors. Requires a massive power source to equalize its needs and is overly affected by refraction, limiting its range.}{Shots: 1; Damage: 5D10+20; AP: 8}{40m}{5 (vehicle power cell)}{40}{Powered}{cr 450}{231.0 kg}
\rangedweapon{Model 6 "Gladiator"}{A heavy mass driver sporting 5 rotating accelerators to prevent overheating and increase fire rate}{Shots: 12; Damage: 3D10+8; AP: 12}{500m}{500 (vehicle mass driver shard block)}{40}{Powered}{cr 420}{169.0 kg}
\rangedweapon{Mag Rail Cannon}{A massive mass driver built for anti armor LAVs in the days of war}{Shots: 1; Damage: 3D10+18; AP: 40}{6000m}{20 (vehicle mass driver shard block)}{40}{Powered}{cr 450}{143.0 kg}
\rangedweapon{"Zeushammer"}{Also referred to as the "Ball Lightning Cannon" this weapon is a long range Tesla cannon firing arcing ball lightnings.}{Shots: 1; Damage: 9D10-4; AP: }{150m}{5 (vehicle power cell)}{40}{Blast (6), EMP, Stun (5), Spray, Powered}{cr 410}{197.0 kg}
\end{multicols}

\subsubsection{Launcher}
\vspace{8mm}
\begin{multicols}{2}
\rangedweapon{MML-4}{A multi rocket launcher built for vehicles}{Shots: 2; Damage: -; AP: depending on ammo}{350m}{6 (rocket propelled grenade)}{40}{Powered, Backblast (4)}{cr 220}{36.9 kg}
\rangedweapon{WTFBRR}{A massive recoilless rifle made for giant loads to be fired from vehicles}{Shots: 1; Damage: -; AP: depending on ammo}{700m}{1 (large rocket propelled grenade)}{20}{Backblast (8)}{cr 360}{45.7 kg}
\rangedweapon{Hail Fire}{A micro missile launcher firing multiple rockets on multiple targets simultaneously}{Shots: 4; Damage: 3D10+2; AP: 4}{250m}{4 (micro missile packs)}{40}{Powered, Multitargeting, Backblast (2)}{cr 200}{26.7 kg}
\rangedweapon{M64}{A 40mm rotary grenade launcher}{Shots: 1; Damage: -; AP: depending on ammo}{100m}{6 (40mm grenade)}{36}{Single Loader, minimum arming distance of 15m}{cr 180}{5.4 kg}
\rangedweapon{ERPG}{A rocket propelled grenade launcher}{Shots: 1; Damage: -; AP: depending on ammo}{400m}{1 (rocket propelled grenade)}{6}{Backblast (4)}{cr 210}{6.2 kg}
\rangedweapon{OP-4}{A 40mm single shot grenade launcher}{Shots: 1; Damage: -; AP: depending on ammo}{70m}{1 (40mm grenade)}{6}{minimum arming distance of 15m}{cr 90}{2.2 kg}
\rangedweapon{"Thunderstrike" Grenade Launcher}{A grenade launcher fed from a belt. Very useful for crowd control but complete overkill in most situations. It is also extremely cumbersome due to the massive belt or belt box, much too large to be used on foot.}{Shots: 4; Damage: -; AP: -}{100m}{30 (40mm grenade)}{9}{minimum arming distance of 15m}{cr 136}{14.2 kg}
\end{multicols}

\pagebreak
\subsubsection{Bow}
\vspace{8mm}
\begin{multicols}{2}
\rangedweapon{collapsible bow}{A bow that can be collapsed to fit small cases for transportation and concealment}{Shots: 1; Damage: 1D10+3; AP: 1}{50m}{1 (arrow)}{1}{Silenced}{cr 25}{1.5 kg}
\rangedweapon{recurve bow}{A full-sized bow}{Shots: 1; Damage: 1D10+4; AP: 2}{75m}{1 (arrow)}{1}{Silenced, -10 to hit per StrB below 4}{cr 20}{3.2 kg}
\end{multicols}

\subsubsection{Throwing}
\vspace{8mm}
\begin{multicols}{2}
\rangedweapon{Asphyxiation Grenade}{A chemical weapon that burns away oxygen in a living room sized area. This reaction is not strongly exothermic, making it hard to detect before it's too late.}{Does not deal damage but causes suffocation.}{15m}{1 (Grenade)}{1}{Blast (5)}{cr 31}{0.3 kg}
\rangedweapon{EMP grenade}{Very expensive and very rare, this new type of explosive stuns or even damages unshielded electronics. It also causes piezoelectric exoskeletons to spasm and hurt the wearer.}{Shots: 1; Damage: 0; AP: 0}{15m}{1 (Grenade)}{1}{Blast (6), EMP}{cr 28}{0.1 kg}
\rangedweapon{Flashbang}{Popular among law enforcement personnel this non-lethal grenade blinds and deafens anyone not wearing appropriate protective gear with a loud and bright flash.}{Shots: 1; Damage: 0; AP: 0}{15m}{1 (Grenade)}{1}{Blast (12), Flash}{cr 14}{0.1 kg}
\rangedweapon{Frag grenade}{An explosive that spreads a large amount of shrapnel in a decently large area. Used against groups of hostiles and to clear rooms since the dawn of gunpowder.}{Shots: 1; Damage: 2D10+2; AP: 2}{15m}{1 (Grenade)}{1}{Blast (6)}{cr 22}{0.1 kg}
\rangedweapon{Incendiary grenade}{A fire-based explosive known to be easy and cheap to improvise and to cause a lot of collateral damage. Less useful against energy shielding and metal constructions.}{Shots: 1; Damage: 1D10+2; AP: 0}{15m}{1 (Grenade)}{1}{Blast (8), ignites flammable surfaces in the area}{cr 22}{0.1 kg}
\rangedweapon{Riot Foam}{A canister of sticky foam that heavily impairs movement of everyone caught up in it. Can safely but slowly be burnt away at low temperatures.}{Shots: 1; Damage: 0; AP: 0}{15m}{1 (Grenade)}{1}{Blast (12), Sticky}{cr 12}{0.1 kg}
\rangedweapon{Smoke grenade}{Favored in open places this non-lethal grenade grants a large amount of artificial concealment to otherwise clear lines of fire.}{Shots: 1; Damage: 0; AP: 0}{15m}{1 (Grenade)}{1}{Blast (8), Smoke}{cr 14}{0.1 kg}
\rangedweapon{Tear gas grenade}{Non-lethal irritant grenades usually employed to stop riots}{Shots: 1; Damage: 0; AP: 0}{15m}{1 (Grenade)}{1}{Blast (8), Smoke, Stun (0), Stun is applied every round in the Smoke}{cr 16}{0.1 kg}
\rangedweapon{Blunt throwing weapon}{Blunt instruments like rocks and bottles being thrown. Very common when nothing else is at hand.}{Shots: 1; Damage: 1D5+7; AP: 6}{15m}{1 (Blunt instruments)}{1}{Trauma, Piercing, Improvised}{Rocks are free}{1.0 kg}
\rangedweapon{Throwing axe}{Sharp, heavy, bladed weapon weighted for throwing}{Shots: 1; Damage: 1D10+3; AP: 0}{15m}{1 (Axes)}{1}{}{cr 12}{1.1 kg}
\rangedweapon{Throwing knives}{Sharp, bladed weapons weighted for throwing}{Shots: 1; Damage: 1D5+3; AP: 3}{15m}{1 (Knives)}{1}{}{cr 10}{0.4 kg}
\end{multicols}

\subsubsection{Traps}
\vspace{8mm}
\begin{multicols}{2}
\rangedweapon{Bear trap}{A strong, spring loaded clamp made to injure legs and stop creatures from fleeing or approaching.}{Shots: 1; Damage: 2D10+12}{0.3m}{-}{10}{}{cr 8}{1.1 kg}
\rangedweapon{C-7}{A block of plastic explosives. Usually made to be set up but can the thrown in a pinch}{Shots: 1; Damage: 2D10+18; AP: 16}{5m}{1 (C-7)}{2}{Blast (6)}{cr 22}{0.2 kg}
\rangedweapon{Cloak Mine}{A landmine with an integrated cloaking field. Makes an obvious noise when stepped upon and takes a decent amount of time to arm for safety reasons.}{Shots: 1; Damage: 2D10+5; AP: 4}{-}{1 (Mine)}{10}{Blast (3); detecting the noise when stepped upon takes a +0 Perception test}{cr 33}{1.2 kg}
\rangedweapon{Exploding Fuel Cell, Small}{A small fuel cell, usually built very sturdily, may be rigged as an improvised explosive device. Unstable cells however may explode when handled too roughly in high temperatures.}{Shots: 1; Damage: 2D10+4; AP: 6}{15m}{1 (small fuel cell)}{1}{Blast (1)}{cr 9}{0.3 kg}
\rangedweapon{Exploding Fuel Cell, Medium}{A fuel cell, usually built very sturdily, may be rigged as an improvised explosive device. Unstable cells however may explode when handled too roughly in high temperatures.}{Shots: 1; Damage: 3D10+6; AP: 9}{15m}{1 (medium fuel cell)}{1}{Blast (1)}{cr 15}{0.8 kg}
\rangedweapon{Exploding Fuel Cell, Large}{A large fuel cell, usually built very sturdily, may be rigged as an improvised explosive device. Unstable cells however may explode when handled too roughly in high temperatures.}{Shots: 1; Damage: 4D10+8; AP: 12}{15m}{1 (large fuel cell)}{1}{Blast (2)}{cr 21}{1.5 kg}
\end{multicols}

\subsection{Mods}
\subsubsection{Constraints}
Ranged weapons have a limited number of mod slots which can each only be used once - at some point the weapon becomes unwieldy or the space is simply used up. When in question, apply common sense.\par\vspace{10mm}

\begin{multicols}{2}
Pistols:
\vspace{-8mm}
\begin{enumerate}
	\setlength\itemsep{-8mm}
	\item Upper Rail
	\item Lower Rail
	\item Barrel
	\item Muzzle
	\item Core
\end{enumerate}

Rifle:
\vspace{-8mm}
\begin{enumerate}
	\setlength\itemsep{-8mm}
	\item Upper Rail
	\item Lower Rail
	\item Two Side Rails
	\item Barrel
	\item Muzzle
	\item Core
\end{enumerate}

Heavy Weapon / Launcher:
\vspace{-8mm}
\begin{enumerate}
	\setlength\itemsep{-8mm}
	\item Upper Rail
	\item Lower Rail
	\item Lower Rail (restricted to bipod / tripod)
	\item Two Side Rails
	\item Barrel
	\item Muzzle
	\item Core
\end{enumerate}

Bow:
\vspace{-8mm}
\begin{enumerate}
	\setlength\itemsep{-8mm}
	\item Upper Rail
	\item Core
\end{enumerate}
\end{multicols}

\subsubsection{Optics}
\weaponmod{Red Dot Sight}{A simple 1x scope with good eye relief, barely obstructing peripheral vision.}{Upper Rail}{cr 5}{0.15 kg}
\weaponmod{Holographic Sight}{A non-magnifying 1x sight that is only projected when the character makes an Aim action. Only takes up minimal space on the rail it is mounted, allowing for another item to be mounted without taking this sight off.}{Upper Rail}{cr 7}{0.2 kg}
\weaponmod{Flip Scope}{A 3x magnifier to be used in conjunction with a non-magnifying sight.}{Upper Rail}{cr 7}{0.32 kg}
\weaponmod{AOG (Advanced Optical Gun sight)}{A common, analog 4x combat scope.}{Upper Rail}{cr 12}{0.35 kg}
\weaponmod{TEOG (Tactical Electronic Optical Gun sight)}{A common variable 4-6x combat scope. Due to using electronic magnification it is vulnerable to EMP.}{Upper Rail}{cr 32}{0.38 kg}
\weaponmod{LoRa Sight}{An 8x scope - the cheapest long range sight but usually absolutely sufficient.}{Upper Rail}{cr 16}{0.4 kg}
\weaponmod{LS-Marksman Scope}{A massive 14x scope used for the longest ranges.}{Upper Rail}{cr 28}{0.42 kg}
\weaponmod{OMS (Oculus Marksman System)}{A variable 8-14x electronic scope. Like other electronic magnifiers it is vulnerable to EMP.}{Upper Rail}{cr 51}{0.43 kg}
\weaponmod{TS-CQC Gun sight}{A big non-magnifying thermal imaging sight with a sleek profile created for CQC.}{Upper Rail}{cr 42}{0.52 kg}
\weaponmod{TS-BR Gun sight}{A 5x battle rifle scope with thermal imaging capabilities.}{Upper Rail}{cr 68}{0.67 kg}
\weaponmod{TS-M Gun sight}{A 10x marksman scope with thermal imaging capabilities.}{Upper Rail}{cr 78}{0.67 kg}
\weaponmod{Canted Irons}{A 45-degree offset rail mount is added to the top of the weapon for the use of a secondary Iron sight/Optic/Light.}{-}{cr 8}{0.1 kg}

\subsubsection{Lights}
\weaponmod{Laser Module}{Gives a +5 bonus to hit when not using aim actions. A target spotting the dot with a -10 Perception test cannot be surprised.}{Any rail}{cr 14}{0.1 kg}
\weaponmod{Infrared Laser Module}{Works like a laser module but it can only be seen with thermal imaging.}{Any rail}{cr 28}{0.2 kg}
\weaponmod{Flashlight}{A flashlight is attached to the weapon, able to light wherever the wielder is aiming.}{Any rail}{cr 6}{0.26 kg}

\subsubsection{Muzzles}
\weaponmod{Muzzle Brake}{Reduces recoil, thereby halving penalties from Burst actions.}{Muzzle}{cr 20}{0.2 kg}
\weaponmod{Flash Suppressor}{Reduces muzzle flash, inferring a -30 penalty to visual perception checks to spot the weapon firing.}{Muzzle}{cr 22}{0.2 kg}
\weaponmod{Sound Suppressor}{Reduces both noise and muzzle flash, granting a -30 penalty to all visual perception checks to spot the weapon firing and -10 to hearing based tests; can be combined with cold load ammunition}{Muzzle}{cr 28}{0.2 kg}
\weaponmod{Shotgun Choke, wide}{Reduces pellet spread on shotguns somewhat. Scatter rule only gives a hit bonus from normal range onward. Range is increased by 50\%.}{Muzzle}{cr 12}{0.2 kg}
\weaponmod{Shotgun Choke, narrow}{Reduces pellet spread on shotguns a lot. Scatter rule only gives a hit bonus from long range onward. Range is increased by 100\%.}{Muzzle}{cr 12}{0.2 kg}

\subsubsection{Underslung Weapons}
\weaponmod{Skeleton Key}{Adds an underslung MS shotgun with half range. Due to increased weight at the front this infers a -5 to hit when using aim actions without a rest.}{Lower Rail}{cr 28}{2.8 kg}
\weaponmod{Underslung CQC Grenade Launcher "Thumper"}{Adds an underslung "Thumper" CQC Grenade Launcher with half range. Due to increased weight at the front this infers a -5 to hit when using aim actions without a rest.}{Lower Rail}{cr 70}{3.2 kg}

\subsubsection{Braces / Grips}
\weaponmod{Angled Grip}{A fast-pull grip that grants +2 initiative when using both hands and no rest.}{Lower Rail}{cr 8}{0.1 kg}
\weaponmod{Pistol Grip}{A stable grip that increases aim bonus by 5 when using both hands and no rest.}{Lower Rail}{cr 8}{0.1 kg}
\weaponmod{Bipod/Tripod}{A stable rest that increases aim bonus by 10 when deployed. A bipod is short and to be used over cover or while prone. Tripods are large and to be used while standing.}{Lower Rail}{cr 10}{0.15 kg}

\subsubsection{Barrel}
\weaponmod{Extended Barrel}{A longer barrel that increases optimum range and armor penetration by 25\%. Such a long weapon may be difficult to use in confined spaces.}{Barrel}{cr 23}{0.2 kg}
\weaponmod{Heavy Barrel}{A heavy, more stable barrel. It halves penalties from shooting while on the move but decreases aim bonus by 10 when not using a rest.}{Barrel}{cr 11}{0.2 kg}

\subsubsection{Ammunition}
\weaponmod{Magazine}{A simple magazine for a ballistic weapon. It holds ammunition in the gun.}{Ballistic Magazine}{cr 12}{-}
\weaponmod{Extended-Mag/Belt}{A larger magazine that doubles mag size but is more prone to error, causing jams on 96 and above.}{Ballistic Magazine}{cr 26}{-}
\weaponmod{Drum-Mag}{A massive spiral magazine that triples mag size but jams on 96+ and also doesn't fit in normal mag pouches, therefore need to be carried in backpacks.}{Ballistic Magazine}{cr 34}{-}
\weaponmod{Dual-Mag}{Taped magazines make swapping from one to the other take half as long, however it doesn't fit in mag pouches anymore. Can easily be improvised with 2 magazines and some duct tape.}{Ballistic Magazine}{-}{-}
\weaponmod{Shell Holder}{A clip attached to a rail holding up to 8 shotgun shells. Very useful for special ammunition.}{Any Rail}{cr 2}{-}
\weaponmod{Bolt Holder}{A container holding up to 8 bolts attached below a crossbow.}{Core}{cr 3}{-}

\subsubsection{Misc}
\weaponmod{Camouflage Paint}{If the pattern matches the environment, grants a -20 penalty to visually perceive the weapon. However if the pattern doesn't match, it grants a +5 to visually perceive the weapon instead.}{Paint}{cr 3}{0 kg}
\weaponmod{Ghillie Cover}{If the pattern matches the environment, grants a -30 penalty to visually perceive the weapon. However if the pattern doesn't match, it grants a +10 to visually perceive the weapon instead.\\
	Unlike with camouflage paint patterns only exist for desert and forest.}{Paint}{cr 4}{0.15 kg}
	
\subsubsection{Core}
\weaponmod{Offload Cycler}{A technological marvel increasing a weapon's fire rate by 50\%, rounded down, if it already has a fire rate of 2 or more.}{Core}{80\%}{0.9 kg}
\weaponmod{Multi Barrel}{Adds an additional barrel feeding from the same magazine to double the amount of hits and ammo spent at an additional -10 penalty to hit. Only works if the weapon has a capacity of at least 2.}{Core}{60\%}{0.55 kg}
\weaponmod{Heatsink}{A heat sink removing the drawbacks from the Adjustable rule. It increases the weapon's weight by 20\%.}{Core}{40\%}{20\%}
\weaponmod{Backblast Refunneling}{It halves the Backblast range to increase the weapon's range by 50\%. The additional rumble gives a penalty equal to half of the user's strength difference to 70.}{Core}{60\%}{0.4 kg}
\weaponmod{Gim-barrel}{A self-stabilizing barrel halves penalties from shooting while moving as well as Burst actions. It doubles the weapon's weight and decreases aim bonus by 5 when not using a rest.}{Core, Barrel}{80\%}{100\%}
\weaponmod{Biocoding}{A trigger lock reading biometric data of the user. The weapon can only be used when wielded by a user who was registered during building of the weapon. It cannot be removed or reprogrammed without destroying the weapon.}{none}{cr 20}{0.24 kg}
\weaponmod{Vehicle Lock-on}{Adds vehicle lock-on to a Launcher or Heavy weapon; grants +30 to hitting vehicles and powered armor when taking at least one round to aim. Flares don't disrupt targeting but static smoke will.}{Core}{80\%}{3.4 kg}
\weaponmod{Heat Seeker}{Adds heat-based targeting assistance systems, thereby granting +15 to shoot targets warmer than their surroundings but -15 to shoot targets colder than their surroundings.}{Core}{cr 280}{2.3 kg}
\weaponmod{Energy Shield}{Adds a directional PES that covers the user's head and torso. Covers the whole user instead if used with a tripod.\\
	Armor: 15 / Threshold: 4}{Core}{cr 180}{3.8 kg}
\weaponmod{Smart Gun}{Sophisticated aiming assistance takes care of all the hard work. The gun always uses a target value of 65 when taking shots, regardless of modifiers or the shooter. It still requires to be pointed in roughly the right direction though and will completely fail under EMP.}{Core}{cr 190}{6.9kg}

\subsection{Ammunition}
In the following section ammo types are given in this format:\\
First, in bold, there is the type designation. This is little more than an identifier.\\
Secondly after the colon you find the effect. These are all changes this type of ammunition has as opposed to the basic weapon profile.\\
The price is the price per box off ammunition; every box contains an amount of ammunition equal to the value given at "Unit of sale". For ballistic ammo a single unit is one bullet, for others it is one full load of the weapon.\\
Weight and bulk are given in the weight that one "standard magazine" - as used in storage capacities - weighs.
%	Pistols
\subsubsection{7.2mm}
\ammo{FMJ}{-}{cr 2}{30}{4g}{40}
\ammo{AP}{-1 dmg, +4 AP}{cr 3}{30}{4g}{40}
\ammo{Hot Load}{+1 dmg, +4 AP, +15 on hearing checks to identify the shooter's position}{cr 4}{30}{4g}{40}
\ammo{Hot AP}{+10 AP, +15 on hearing checks to identify the shooter's position}{cr 6}{30}{4g}{40}
\ammo{JSP}{+2 dmg, -3 AP}{cr 3}{30}{4g}{40}
\ammo{JHP}{+4 dmg, -6 AP}{cr 4}{30}{4g}{40}
\ammo{Shock}{-4 dmg, -5 AP, Stun (4), EMP}{cr 6}{30}{4g}{40}
\ammo{Cold Load}{-3 dmg, -3 AP, -15 on hearing checks to identify the shooter's position}{cr 3}{30}{4g}{40}
\ammo{Cold AP}{-4 dmg, -15 on hearing checks to identify the shooter's position}{cr 4}{30}{4g}{40}
\ammo{Cold JSP}{-6 AP, -15 on hearing checks to identify the shooter's position}{cr 4}{30}{4g}{40}
\ammo{Cold JHP}{+2 dmg, -8 AP, -15 on hearing checks to identify the shooter's position}{cr 6}{30}{4g}{40}
\ammo{Incendiary}{-2 AP, will ignite flammable materials hit}{cr 3}{30}{4g}{40}
\ammo{Tracer}{grants a +10 bonus to hit for others aiming at the same target during this turn}{cr 4}{30}{4g}{40}
\ammo{Rubber Rounds}{-10 dmg, 0 AP, Stun (2), every hit against the target's head reduces it's initiative by D5-1}{cr 2}{30}{4g}{40}
\subsubsection{11.8x45mm}
\ammo{FMJ}{-}{cr 3}{20}{22g}{40}
\ammo{AP}{-1 dmg, +4 AP}{cr 5}{20}{22g}{40}
\ammo{Hot Load}{+1 dmg, +4 AP, +15 on hearing checks to identify the shooter's position}{cr 6}{20}{22g}{40}
\ammo{Hot AP}{+10 AP, +15 on hearing checks to identify the shooter's position}{cr 9}{20}{22g}{40}
\ammo{JSP}{+2 dmg, -3 AP}{cr 5}{20}{22g}{40}
\ammo{JHP}{+4 dmg, -6 AP}{cr 6}{20}{22g}{40}
\ammo{Shock}{-4 dmg, -5 AP, Stun (4), EMP}{cr 9}{20}{22g}{40}
\ammo{Cold Load}{-3 dmg, -3 AP, -15 on hearing checks to identify the shooter's position}{cr 5}{20}{22g}{40}
\ammo{Cold AP}{-4 dmg, -15 on hearing checks to identify the shooter's position}{cr 6}{20}{22g}{40}
\ammo{Cold JSP}{-6 AP, -15 on hearing checks to identify the shooter's position}{cr 6}{20}{22g}{40}
\ammo{Cold JHP}{+2 dmg, -8 AP, -15 on hearing checks to identify the shooter's position}{cr 9}{20}{22g}{40}
\ammo{Incendiary}{-2 AP, will ignite flammable materials hit}{cr 5}{20}{22g}{40}
\ammo{Tracer}{grants a +10 bonus to hit for others aiming at the same target during this turn}{cr 6}{20}{22g}{40}
\ammo{Rubber Rounds}{-10 dmg, 0 AP, Stun (2), every hit against the target's head reduces it's initiative by D5-1}{cr 3}{20}{22g}{40}

%	Rifles
\subsubsection{6.3mm}
\ammo{FMJ}{-}{cr 2}{30}{8.6g}{30}
\ammo{AP}{-1 dmg, +4 AP}{cr 3}{30}{8.6g}{30}
\ammo{Hot Load}{+1 dmg, +4 AP, +15 on hearing checks to identify the shooter's position}{cr 4}{30}{8.6g}{30}
\ammo{Hot AP}{+10 AP, +15 on hearing checks to identify the shooter's position}{cr 6}{30}{8.6g}{30}
\ammo{JSP}{+2 dmg, -3 AP}{cr 3}{30}{8.6g}{30}
\ammo{JHP}{+4 dmg, -6 AP}{cr 4}{30}{8.6g}{30}
\ammo{Shield Breaker}{-3 dmg, Disrupt}{cr 5}{30}{8.6g}{30}
\ammo{Shock}{-4 dmg, -5 AP, Stun (4), EMP}{cr 6}{30}{8.6g}{30}
\ammo{Cold Load}{-3 dmg, -3 AP, -15 on hearing checks to identify the shooter's position}{cr 3}{30}{8.6g}{30}
\ammo{Cold AP}{-4 dmg, -15 on hearing checks to identify the shooter's position}{cr 4}{30}{8.6g}{30}
\ammo{Cold JSP}{-6 AP, -15 on hearing checks to identify the shooter's position}{cr 4}{30}{8.6g}{30}
\ammo{Cold JHP}{+2 dmg, -8 AP, -15 on hearing checks to identify the shooter's position}{cr 6}{30}{8.6g}{30}
\ammo{Incendiary}{-2 AP, will ignite flammable materials hit}{cr 3}{30}{8.6g}{30}
\ammo{Tracer}{grants a +10 bonus to hit for others aiming at the same target during this turn}{cr 4}{30}{8.6g}{30}
\ammo{Rubber Rounds}{-10 dmg, 0 AP, Stun (2), every hit against the target's head reduces it's initiative by D5-1}{cr 2}{30}{8.6g}{30}
\subsubsection{11.8x112mm}
\ammo{FMJ}{-}{cr 4}{20}{50g}{5}
\ammo{AP}{-1 dmg, +4 AP}{cr 6}{20}{50g}{5}
\ammo{Hot Load}{+1 dmg, +4 AP, +15 on hearing checks to identify the shooter's position}{cr 8}{20}{50g}{5}
\ammo{Hot AP}{+10 AP, +15 on hearing checks to identify the shooter's position}{cr 12}{20}{50g}{5}
\ammo{JSP}{+2 dmg, -3 AP}{cr 6}{20}{50g}{5}
\ammo{JHP}{+4 dmg, -6 AP}{cr 8}{20}{50g}{5}
\ammo{Shield Breaker}{-3 dmg, Disrupt}{cr 8}{20}{50g}{5}
\ammo{Shock}{-4 dmg, -5 AP, Stun (4), EMP}{cr 12}{20}{50g}{5}
\ammo{Cold Load}{-3 dmg, -3 AP, -15 on hearing checks to identify the shooter's position}{cr 6}{20}{50g}{5}
\ammo{Cold AP}{-4 dmg, -15 on hearing checks to identify the shooter's position}{cr 8}{20}{50g}{5}
\ammo{Cold JSP}{-6 AP, -15 on hearing checks to identify the shooter's position}{cr 8}{20}{50g}{5}
\ammo{Cold JHP}{+2 dmg, -8 AP, -15 on hearing checks to identify the shooter's position}{cr 12}{20}{50g}{5}
\ammo{Incendiary}{-2 AP, will ignite flammable materials hit}{cr 6}{20}{50g}{5}
\ammo{Tracer}{grants a +10 bonus to hit for others aiming at the same target during this turn}{cr 8}{20}{50g}{5}
\ammo{Rubber Rounds}{-10 dmg, 0 AP, Stun (2), every hit against the target's head reduces it's initiative by D5-1}{cr 4}{20}{50g}{5}
\subsubsection{15x132mm}
\ammo{FMJ}{-}{cr 5}{5}{85g}{4}
\ammo{AP}{-1 dmg, +4 AP}{cr 8}{5}{85g}{4}
\ammo{Hot Load}{+1 dmg, +4 AP, +15 on hearing checks to identify the shooter's position}{cr 10}{5}{85g}{4}
\ammo{Hot AP}{+10 AP, +15 on hearing checks to identify the shooter's position}{cr 15}{5}{85g}{4}
\ammo{JSP}{+2 dmg, -3 AP}{cr 8}{5}{85g}{4}
\ammo{JHP}{+4 dmg, -6 AP}{cr 10}{5}{85g}{4}
\ammo{Shock}{-4 dmg, -5 AP, Stun (4), EMP}{cr 15}{5}{85g}{4}
\ammo{Cold Load}{-3 dmg, -3 AP, -15 on hearing checks to identify the shooter's position}{cr 8}{5}{85g}{4}
\ammo{Cold AP}{-4 dmg, -15 on hearing checks to identify the shooter's position}{cr 10}{5}{85g}{4}
\ammo{Cold JSP}{-6 AP, -15 on hearing checks to identify the shooter's position}{cr 10}{5}{85g}{4}
\ammo{Cold JHP}{+2 dmg, -8 AP, -15 on hearing checks to identify the shooter's position}{cr 15}{5}{85g}{4}
\ammo{Incendiary}{-2 AP, will ignite flammable materials hit}{cr 8}{5}{85g}{4}
\ammo{Tracer}{grants a +10 bonus to hit for others aiming at the same target during this turn}{cr 10}{5}{85g}{4}
\ammo{Rubber Rounds}{-10 dmg, 0 AP, Stun (2), every hit against the target's head reduces it's initiative by D5-1}{cr 5}{5}{85g}{4}

%	Shotgun
\subsubsection{12gauge}
\ammo{Buckshot}{-}{cr 3}{24}{42g}{8}
\ammo{Birdshot}{halves damage, ignores all negative size modifiers}{cr 2}{24}{42g}{8}
\ammo{Incendiary}{-2 dmg, -4 AP, ignites flammable materials hit}{cr 3}{24}{42g}{8}
\ammo{Breaching Round}{+3 dmg, +4 AP, halved range}{cr 5}{24}{42g}{8}
\ammo{Slug}{+2 AP, removes Scatter, doubles range}{cr 3}{24}{42g}{8}
\ammo{Fragmentation Slug}{removes Scatter but adds Blast (2) rule, doubles range}{cr 6}{24}{42g}{8}
\ammo{Incendiary Slug}{+1 AP, removes Scatter, doubles range, ignites flammable objects hit}{cr 4}{24}{42g}{8}
\ammo{EMP Slug}{-2 dmg, EMP, removes Scatter, doubles range}{cr 9}{24}{42g}{8}
\ammo{Rubber Slug}{-2 dmg, -6 AP, removes Scatter, doubles range, Stun (2)}{cr 3}{24}{42g}{8}

%	Fuel Tank
\subsubsection{Small Fuel Tanks}
\ammo{Napalm}{ignites flammable surfaces and sticks even to non-flammable ones}{cr 32}{1}{10kg}{}
\ammo{White Phosphorus}{halves range, burns for a D5 minutes, ignites flammable surfaces and sticks even to non-flammable ones, cannot be extinguished by water}{cr 64}{1}{20kg}{}
\ammo{Water}{1 dmg, 0 AP, Trauma, Stun (*the original amount of damage dice)}{cr 8}{1}{13kg}{}
\ammo{Liquid Helium}{halves range, causes D5 exhaustion per hit}{cr 64}{1}{13kg}{}

\subsubsection{Medium Fuel Tanks}
\ammo{Napalm}{ignites flammable surfaces}{cr 61}{1}{20kg}{}
\ammo{White Phosphorus}{halves range, burns for a D5 minutes, ignites flammable surfaces and sticks even to non-flammable ones}{cr 122}{1}{40kg}{}
\ammo{Water}{1 dmg, 0 AP, Trauma, Stun (*the original amount of damage dice)}{cr 17}{1}{25kg}{}
\ammo{Liquid Helium}{halves range, causes D5 exhaustion per hit}{cr 122}{1}{25kg}{}

\subsubsection{Large Fuel Tanks}
\ammo{Napalm}{ignites flammable surfaces}{cr 150}{1}{50kg}{}
\ammo{White Phosphorus}{halves range, burns for a D5 minutes, ignites flammable surfaces and sticks even to non-flammable ones}{cr 300}{1}{100kg}{}
\ammo{Water}{1 dmg, 0 AP, Trauma, Stun (*the original amount of damage dice)}{cr 32}{1}{62.5kg}{}
\ammo{Liquid Helium}{halves range, causes D10 exhaustion per hit}{cr 300}{1}{62.5kg}{}

%	Fuel Cells
\subsubsection{Small Fuel Cells}
\ammo{Standard Issue}{-}{cr 9}{1}{250g}{2}
\ammo{Unstable}{On a hit roll above 95 causes the weapon's damage with Blast (1) around the shooter.}{cr 5}{1}{250g}{2}
\subsubsection{Medium Fuel Cells}
\ammo{Standard Issue}{-}{cr 15}{1}{750g}{}
\ammo{Unstable}{On a hit roll above 95 causes the weapon's damage with Blast (1) around the shooter.}{cr 8}{1}{750g}{}
\subsubsection{Large Fuel Cells}
\ammo{Standard Issue}{-}{cr 21}{1}{1.5kg}{0.5}
\ammo{Unstable}{On a hit roll above 95 causes the weapon's damage with Blast (1) around the shooter.}{cr 11}{1}{1.5kg}{0.5}

%	Mass Driver Blocks
\subsubsection{Small Shard Blocks}
\ammo{Standard Issue}{-}{cr 2}{1}{300g}{1}
\ammo{Crude}{-50\% mag size, jams on 91 and above}{cr 1}{2}{300g}{1}
\subsubsection{Large Shard Blocks}
\ammo{Standard Issue}{-}{cr 5}{1}{1.8kg}{0.5}
\ammo{Crude}{-50\% mag size, jams on 91 and above}{cr 2}{2}{1.8kg}{0.5}
\ammo{Osmium Alloy}{+4 damage, -2 AP}{cr 10}{1}{3.6kg}{0.5}
\subsubsection{Vehicle Shard Blocks}
\ammo{Standard Issue}{-}{cr 11}{1}{3.2kg}{0.2}
\ammo{Crude}{-50\% mag size, jams on 91 and above}{cr 5}{2}{3.2kg}{0.2}
\ammo{Osmium Alloy}{+4 damage, -2 AP}{cr 21}{2}{6.2kg}{0.5}


%	Grenades
\subsubsection{20mm grenades}
\ammo{Fragmentation}{2D10+2 | 2 AP | Blast (6)}{cr 60}{1}{120g}{0.5}
\ammo{High Explosive}{2D10+6 | 5 AP | Blast (2)}{cr 60}{1}{120g}{0.5}
\ammo{Thermobaric}{3D10 | 2 AP | Blast (10) | heat-based, ignores all cover and non-vacuum-sealed or natural armor}{cr 96}{1}{120g}{0.5}
\ammo{Incendiary}{D10+2 | 0 AP | Blast (8) | ignites all flammable objects in the radius}{cr 60}{1}{120g}{0.5}
\ammo{Buckshot Canister}{2D10+3 | 5 AP | Scatter}{cr 75}{1}{120g}{0.5}
\ammo{Slug}{2D5+8 | 8 AP}{cr 75}{1}{120g}{0.5}
\ammo{Smoke}{0 | 0 AP | Blast (8), Smoke}{cr 45}{1}{120g}{0.5}
\ammo{Flashbang}{0 | 0 AP | Blast (12), Flash}{cr 45}{1}{120g}{0.5}
\ammo{EMP}{0 | 0 AP | Blast (6), EMP}{cr 80}{1}{120g}{0.5}
\ammo{Tear Gas}{0 | 0 AP | Blast (6), Smoke, Stun (0), ignores cover and non-vacuum-sealed armor, Stun is applied every round in the Smoke}{cr 45}{1}{120g}{0.5}
\ammo{Rubber Slug}{1D10+3 | 2 AP | Stun (5), hits to the head cause 2D5 initiative loss}{cr 50}{1}{120g}{0.5}
\ammo{Riot Foam}{0 | 0 AP | Blast (12), Sticky}{cr 20}{1}{120g}{0.5}
\ammo{Signal Smoke}{0 | 0 AP | Blast (4) | a smoke spewer to signal allies; comes in various colors}{cr 38}{1}{120g}{0.5}
\ammo{Parachute Flare}{D5 | 0 AP | a signal flare that burns for 3 minutes and falls at 1 meter per 3 seconds (or 1 round)}{cr 40}{1}{120g}{0.5}
\ammo{Infrared Illuminator}{1 | 0 AP | a large chemlight that emits infrared light in a very large area}{cr 42}{1}{120g}{0.5}
\subsubsection{40mm grenades}
\ammo{Fragmentation}{2D10+7 | 2 AP | Blast (9)}{cr 78}{1}{230g}{1}
\ammo{High Explosive}{3D10+6 | 5 AP | Blast (3)}{cr 78}{1}{230g}{1}
\ammo{Thermobaric}{4D10 | 2 AP | Blast (15) | heat-based, ignores all cover and non-vacuum-sealed armor}{cr 125}{1}{230g}{1}
\ammo{Incendiary}{D10+3 | 0 AP | Blast (12) | ignites all flammable objects in the radius}{cr 78}{1}{230g}{1}
\ammo{Buckshot Canister}{3D10+3 | 5 AP | Scatter}{cr 98}{1}{230g}{1}
\ammo{Slug}{2D10+9 | 9 AP}{cr 98}{1}{230g}{1}
\ammo{Smoke}{0 | 0 AP | Blast (12), Smoke}{cr 59}{1}{230g}{1}
\ammo{Flashbang}{0 | 0 AP | Blast (18), Flash}{cr 59}{1}{230g}{1}
\ammo{EMP}{0 | 0 AP | Blast (9), EMP}{cr 104}{1}{230g}{1}
\ammo{Tear Gas}{0 | 0 AP | Blast (8), Smoke, Stun (0), ignores cover and non-vacuum-sealed armor, Stun is applied every round in the Smoke}{cr 59}{1}{230g}{1}
\ammo{Rubber Slug}{2D10+3 | 2 AP | Stun (7), hits to the head cause 2D10 initiative loss}{cr 65}{1}{230g}{1}
\ammo{Signal Smoke}{0 | 0 AP | Blast (4) | a smoke spewer to signal allies that lasts for a few minutes; comes in various colors}{cr 49}{1}{230g}{1}
\ammo{Parachute Flare}{D5+1 | 0 AP | a signal flare that burns for 5 minutes and falls at 1 meter per 3 seconds (or 1 round)}{cr 52}{1}{230g}{1}
\ammo{Thermite}{6D10+10 | 25 AP | Blast (3) | takes two rounds to get going, then burns for one minute dealing its damage every round; sticks to rough surfaces}{cr 156}{1}{230g}{1}

%	Rockets
\subsubsection{Micro Missiles}
\ammo{Fragmentation}{-}{cr 84}{4}{1kg}{0.5}
\ammo{High Explosive}{-1D10 dmg | +8 AP}{cr 88}{4}{1kg}{0.5}
\ammo{Incendiary}{-3 dmg | -3 AP | ignites flammable objects}{cr 76}{4}{1kg}{0.5}
\subsubsection{Small Rockets}
\ammo{HEAT}{4D10+5 dmg | 6 AP | Blast (5)}{cr 124}{1}{2kg}{0.5}
\ammo{High Explosive}{4D10+2 dmg | 3 AP | Blast (6)}{cr 99}{1}{2kg}{0.5}
\ammo{Fragmentation}{3D10+8 dmg | 2 AP | Blast (8)}{cr 99}{1}{2kg}{0.5}
\ammo{Incendiary}{2D10+5 dmg | 2 AP | Blast (6) | ignites flammable objects}{cr 112}{1}{2kg}{0.5}
\ammo{Incendiary Fragmentation}{3D10+5 dmg | 2 AP | Blast (8) | ignites flammable objects}{cr 134}{1}{2kg}{0.5}
\ammo{Thermobaric}{4D10+8 dmg | 3 AP | Blast (12) | heat-based, ignores all cover and non-vacuum-sealed armor}{cr 201}{1}{2kg}{0.5}
\ammo{Thermite}{6D10+10 | 25 AP | Blast (4) | takes two rounds to get going, then burns for one minute dealing its damage every round; sticks to rough surfaces}{cr 179}{1}{2kg}{0.5}
\ammo{Kinetic Penetrator}{4D10+5 dmg | 14 AP}{cr 201}{1}{2kg}{0.5}
\subsubsection{Large Rockets}
\ammo{HEAT}{4D10+6 dmg | 8 AP | Blast (6)}{cr 168}{1}{3.5kg}{0.25}
\ammo{High Explosive}{4D10+4 dmg | 5 AP | Blast (8)}{cr 134}{1}{3.5kg}{0.25}
\ammo{Fragmentation}{3D10+11 dmg | 3 AP | Blast (10)}{cr 134}{1}{3.5kg}{0.25}
\ammo{Incendiary}{2D10+8 dmg | 2 AP | Blast (8) | ignites flammable objects}{cr 151}{1}{3.5kg}{0.25}
\ammo{Incendiary Fragmentation}{3D10+8 dmg | 3 AP | Blast (10) | ignites flammable objects}{cr 181}{1}{3.5kg}{0.25}
\ammo{Thermobaric}{5D10+10 dmg | 5 AP | Blast (18) | heat-based, ignores all cover and non-vacuum-sealed armor}{cr 272}{1}{3.5kg}{0.25}
\ammo{Thermite}{6D10+10 | 25 AP | Blast (6) | takes two rounds to get going, then burns for one minute dealing its damage every round; sticks to rough surfaces}{cr 242}{1}{3.5kg}{0.25}
\ammo{Kinetic Penetrator}{4D10+8 dmg | 18 AP}{cr 272}{1}{3.5kg}{0.25}

%	Arrows / Bolts
\subsubsection{Arrows}
\ammo{Broadhead}{-}{cr 1}{25}{30g}{10}
\ammo{AP}{-1 dmg, +3 AP}{cr 1}{25}{30g}{10}
\ammo{Shield Breaker}{-1 dmg, Disrupt}{cr 2}{25}{30g}{10}
\ammo{Syringer}{-3 dmg, +1 AP, contains an empty vial to be filled with poisons or other injections that are released into a biological target when hit}{cr 3}{25}{30g}{10}
\ammo{Grapple}{-4 dmg, 0 AP, half range; the mechanized grapple attaches to a rope and holds on most surfaces but e.g. very sturdy metals or fragile materials like glass}{cr 6}{1}{30g}{10}
\subsubsection{Bolts}
\ammo{Broadhead}{-}{cr 1}{25}{30g}{10}
\ammo{AP}{-1 dmg, +3 AP}{cr 1}{25}{30g}{10}
\ammo{Shield Breaker}{-1 dmg, Disrupt}{cr 2}{25}{30g}{10}
\ammo{Syringer}{-3 dmg, +1 AP, contains an empty vial to be filled with poisons or other injections that are released into a biological target when hit}{cr 3}{25}{30g}{10}
\ammo{Grapple}{-4 dmg, 0 AP, half range; the mechanized grapple attaches to a rope and holds on most surfaces but e.g. very sturdy metals or fragile materials like glass}{cr 6}{1}{30g}{10}


	\chapter{Healing}
	\vspace{-10mm} %fixes pagebreak
	\section{Anatomy}
	The human body is no monolith; it will be regarded in 6 parts: two legs, two arms, the torso and the head. To determine a random location, consult the chart below. When a regular hit is scored, the two dice are swapped to determine the location.\\
	Example: 
	\begin{exampleblock}
		\itshape
		A simple ranged attack. The hit roll is a 37, so the location is a 73, therefore hitting the right leg.\\
		Attacks scoring multiple hits may use the same location for convenience or roll different locations for all additional hits. You should probably decide which way you take \emph{before} it comes up at the table.
	\end{exampleblock}
	\par
	\begin{minipage}{\columnwidth}
		\begin{tabularx}{\columnwidth}{Xcr}
%			\hline
			Locations & HP & Random \\ \hline
			Head      & 15 &   1-10 \\ %\hline
			Right Arm & 15 &  11-25 \\ %\hline
			Left Arm  & 15 &  26-40 \\ %\hline
			Body      & 25 &  41-70 \\ %\hline
			Right Leg & 15 &  71-85 \\ %\hline
			Left Leg  & 15 & 86-100 \\ %\hline
		\end{tabularx}
	\end{minipage}
	\section{Degradation}
	For every full 20\% of location HP lost, the character takes a -10 penalty to all tests when using this body part. For the head and body locations that is every action.
	When the head is struck, base initiative is reduced by 1 and D5 temporary initiative is lost, which may be regained through the “Gather Senses” action.
	\par
	When 60\% is lost, light bleeding begins and at 80\% severe bleeding sets in. At more than 100\% the body part is lost or crippled, more in need of repair rather than healing. Details are up to GM discretion.
	\begin{exampleblock}
		\itshape
		Natalie has taken heavy damage - 4 to the head, 7 to the body, 7 to her right arm.
		Any attack action using her pistol suffers a -40 (with her right arm) or a -50 (with her off hand).
		Any calculation or charming attempts suffer -20 from head wounds and general pain and fatigue (the body wounds).
	\end{exampleblock}
	
	\section{Lethality}
	\label{sec:lethality}
	There are multiple ways for any living character to die.
	\paragraph{Critical organ failure}is first. This is achieved by destruction or removal of essential organs. When the torso or head body location is destroyed in a spectacular fashion, the GM may rule that this character has died on the spot.
	\paragraph{Blood loss}is another way a character may die. When characters start \emph{bleeding lightly} as determined by degradation or GM discretion, they have ConB minutes (or Con*2 rounds) before unconsciousness sets in.
	They then have another ConB*9 seconds (or ConB*3 rounds) before they irrevocably bleed out. When \emph{bleeding severely} the character only has ConB*3 seconds (or ConB rounds) before unconsciousness sets in.
	After that he can be saved for only ConB+D5 rounds or bleed to death. Note that “light” and “severe” bleeding are distinctions within game mechanics; both are actually quite severe, a matter of life and death.\\
	With basic medical knowledge (a medication check +40, if in doubt) a character can stop another from bleeding out. That character needs to be right next to the patient and cannot perform any other actions other than holding the wound shut.
	\paragraph{Overexertion} is often overlooked but can very well be lethal. Generally if a character’s exhaustion surpasses their ConB, they collapse, recovering from one level per 2 hours of rest. If the character manages to surpass 3 times her ConB, the character collapses but does not recover naturally anymore. Whether the character can be saved at that point, is up to the GM.\\
	Note that overexertion is generally a very slow process and - unless he is forced to continue - anyone can slow down before reaching dangerous levels of exhaustion.
	\section{Natural healing}
	Characters recover from wounds naturally. The wounded character takes a constitution test at a modifier according to the table below. He heals one wound per Degree of Success, but reduced by the total average armor worn during sleep including PES, to a minimum of 0 - staying combat-ready all night is not the epitome of relaxation.\\
	Recovery is split evenly among damaged zones, prioritizing more severe injuries. This recovery only happens if the wounds can at all heal and are not infected.
	\section{Medical attention}
	If a character gets medical attention, he recovers in addition to his natural healing. The medical check is at a -5 penalty for every location that is being treated beyond the first, so up to a -25 if the whole body is being treated. If the performing doctor is treating himself, the test takes an additional \mbox{-15}. %prevent line break
	Every location treated this way will recover from one wound and the tending doctor can split their DoS between all treated locations to recover from even more wounds. Any character can benefit from this once per day.
	\section{Recovery time}
	Certain conditions will require recovery time.\\
	After \emph{implant surgery} a character will have to recover for two weeks. If he is to act during this time he suffers penalties at the GM’s discretion, at least a -20 to all mental characteristics and a -2 to initiative.\\
	The second, more common recovery will be from \emph{exhaustion}. A character recovers roughly 2 levels of exhaustion per 3 hours of proper rest.

	\chapter{Hazards}
	\section{Common hazards}
	\paragraph{Deafening} may be painful but is not lethal. It is very common when using firearms without ear protection or a silencer, especially in enclosed spaces.\\
	As a rule of thumb a character is deafened for one combat round for every shot fired within 2 meters or 4 meters indoors. Explosives cause deafening for a number of rounds equal to twice (or four times when indoors) the Blast radius, in an area equal to roughly twice the Blast.\\
	When the number of remaining rounds of deafening exceeds 25, the character is likely to permanently lose his hearing.
	\paragraph{Exhaustion} is easily the most common threat to any character after physical damage. Exhaustion is gathered through many ways. Every level of exhaustion gives the character a -10 penalty to any check. If exhaustion reaches too high, the character is at risk of overexertion (see p. \pageref{sec:lethality}).
	\paragraph{Uncontrolled falling} is the most common way the environment will try to kill a character. By default the character will take damage that ignores armor but not injury threshold. The damage is an amount of D10 equal to a third of the distance fallen in meters and hits an additional zone per full 5 meters, not hitting any zones twice.\\
	For example hitting the ground after falling 12 meters would deal 4D10 damage to 2 locations. If the fall was e.g. 65 meters instead, it would deal 22D10 damage to all locations.\\
	This is always the fall onto solid ground. Falling onto softer grounds, like snow, will reduce the damage by up to half.
	\paragraph{Objects falling} onto a character deal half as much damage for every kilogram they weigh. For example a flower pot weighing three kilos falling from 12 meters would deal 6D10 damage.\\
	Unless the object is particularly large, it only hits one location. The aforementioned flower pot will only hit one location, a fridge might hit 3 or more.\\
	\emph{Note: These calculations can be unnecessarily complicated. If you don't absolutely need precise values, estimate the results.}
	\label{hazards:suffocation}
	\paragraph{Suffocation}- commonly through drowning or smoke inhalation - is an often underestimated threat to most characters. If a character consciously holds her breath, she can hold it for three times her Constitution in seconds (or her Constitution in combat rounds).
	While holding her breath, she can’t fight or do other very strenuous tasks. If she cannot, or can no longer, hold her breath, the effects of suffocation set in the next round.\\
	The afflicted character will gain exhaustion every round starting with 1 on the first round, 2 on the second round and so on.
	\paragraph{Fire} and the burns associated with it are a minor threat to characters. Fire does minor damage, generally up to D5 per round, but it deals its damage to every location ignoring three quarters of any non-sealed armor. Fire generally comes with smoke and eats away at the oxygen in the air - both of which can be very lethal in enclosed spaces, see \emph{Suffocation}.
	\paragraph{Freezing} is relatively rare. Should a character be submitted to cold temperatures without proper equipment however, she will suffer a level of exhaustion every hour. If temperatures are freezing cold, she will suffer 2 levels per hour.
%	\paragraph{Starvation} is even rarer and purely optional, food is widely available. Should the party be confronted with the wasteland however, starvation suddenly enters the ring. Starvation progresses in 5 steps, advancing one per day of no or inadequate food. Eating around half as much as required advances the effects by one step only every two days.\\
%	After the first step the character will be distracted by hunger, increasing all disabling characteristics by 15. After two steps natural healing is stopped and recovery times are twice as long. The character also shows the first visual signs of malnourishment. After step three she takes -15 to every test and exhaustion does not regenerate naturally anymore, while after the fourth one level of exhaustion is also gained every day. Five days in the character will start hallucinating.\\
%	Starvation starts recovering when the character has eaten enough again by one step every rest phase.
	
	\section{CBRN}
	\label{hazards:cbrn}
	While not quite an everyday occurrence, CBRN hazards are also quite common - chemical terrorism or dirty bombs, ruins of ancient broken power plants or engineered animals; our protagonists are at a constant risk to face such hazards.
	\paragraph{Biological and chemical} hazards are likely the most common special hazards. \emph{Biological} includes disease, as well as animal and plant toxins. \emph{Chemical} encompasses synthetic poison, chemical fires and fumes as well as drug intake. In both cases the inflicted victim makes a constitution check at a specified penalty.\\
	If it passes, it suffers no ill effects. If it fails, it is afflicted with the hazard’s full effect.
	\paragraph{Radiological} hazards, encompassing everything between radiation sickness and cancer, is a constant threat in fallout zones.\\
	After about an hour or two of exposure the afflicted makes a constitution check. Failing advances the ill effects by the Degrees of Failure.
	Effects are cumulative.
	\begin{enumerate}
		\setlength\itemsep{-10mm}
		\item \textbf{Nausea, Fatigue} - The character gains a level of fatigue that will not recover.
		\item \textbf{Headache} - He takes a -10 to all physical activities.
		\item \textbf{Vomiting} - The character takes a Constitution-based Restraint test every round in combat. Failing means losing his actions.
		\item \textbf{Dehydration} - The character requires three times as much water every day.
		\item \textbf{Fever} - The character develops a fever during the next rest, preventing regeneration and recovery.
		\item \textbf{Weakening} - His physical characteristics are reduced by 10.
		\item \textbf{Anemia} - The character will bleed out twice as fast.
		\item \textbf{Confusion} - Mental characteristics are reduced by 10.
		\item \textbf{Ataxia} - All characteristics are reduced by a further 10.
		\item \textbf{Internal damages} - The character permanently loses 1 RI or MT. If implants require more than remains available, the character both experiences severe graft rejection and immediately advances radiation sickness by one.
		\item \textbf{Recovery or fatality} - The character collapses. If he is rescued from the premises before the next test would be made, he may recover as usually, though failing to reduce the severity means death.
	\end{enumerate}
	\par
	When outside the radiation the character may take a Constitution test every day. On a success the radiation sickness reduces by one.\\
	Afterwards roll D10: if the result is lower than the character's level of radiation sickness, the character develops cancer in a random location.
	Of course this may happen repeatedly.
	\paragraph{Nuclear} hazards are essentially a radiological hazard following an explosive and should therefore be handled as an \emph{explosive}, an \emph{EMP} and a \emph{radiological} hazard.
	\paragraph{EMPs} are capable of disabling improperly shielded technology and can either be used voluntarily by educated personnel, or be created as a side product from orbital mining operations, terraforming or high-powered plasma lances.\\
	For every meter the machine is inside the EMP’s radius, it is disabled for one round or 3 seconds narrative time. Once the time reaches 30 seconds, all stored data is wiped.\\
	Inner cybernetics affected by EMP cause one level of exhaustion to the wearer every round. Every cybernetic that was "wiped" causes an immediate D5.\\
	Proper shielding reduces the time of the effect and certain data drives can be immune to wipes by means of EMP.
%	
	\section{Shock, fear and trauma}
	When a character is exposed to stressful situations, he will suffer from shock. The character makes a courage check with a modifier according to the source and severity of shock.\\
	This ranges from +0 for large amounts of blood or an unexpected dead body, over -20 for watching someone die in front of oneself, to -60 in truly desperate situations or near-death experiences.
	Generally the better the reputation with the injured person and the less predictable the situation, the higher the penalty. It also takes a penalty equal to the sum of all “Fear” or “Phobia” disabling characteristics that fit the situation.
	\par \vspace{-5mm}
	In case the character meets multiple panic situations in a day, add a -10 penalty for each after the first.
	\par \vspace{-5mm}
	If the check is failed, roll a D10, add the Degrees of Failure and apply the according result from the table on page \pageref{feartable}.
	
	\chapter{Transhumanism}
	\section{Philosophies}
\paragraph{Transhumanist}
beliefs are deeply intertwined with the fusion of humanity with machines.
They advocate for extensive body modifications and augmentations.\\
Most are looking for a viable, invasive MCI - Mind-Computer Interface -
	or even to digitalize themselves,
	looking to escape reality.
\paragraph{Body purists} reject or resist the widespread adoption of advanced body modifications and enhancements.\\
The beliefs of body purists often stem from a variety of philosophical, cultural, or ethical perspectives, like preservation of humanity or fear of loss of individuality.

\section{Cybernetics}
Cybernetic augmentation is widespread.
Missing or deformed limbs are replaced,
	or existing body parts are enhanced to help with work or social life.
They don’t naturally regenerate anymore,
	they are often expensive
	and create a huge load on the patient’s nervous system,
	as indicated by Rayleigh index values.
However they are generally more resilient than most human limbs
	and most of them offer additional special features.
Obviously it takes very invasive surgery to install.
\section{Biotech}
Biotechnological engineering has come a long way and engineered body parts are rather common. Vat-grown clones of original human body parts or extra dexterous arms; slim, fine legs or extremely dense back muscle implants; whatever it may be, it falls into this category.\\
While assuring compatibility with the host's nervous system is easier, fitting it to the rest of a human host body is even harder to assure using this younger technology than it is with cybernetic implants, so biotech implants put a great amount of stress on the rest of the body, as indicated by medical toughness.
%TODO: add after chem rework
%then probably rephrase this whole thing
%\section{Chemicals}
%Chemical augmentation is a powerful way to enhance one’s abilities for a short period of time.
%It is common among thugs that cannot afford proper bionics but also among military
%	and mercenary personnel that need to surpass human peak performance.
%Chemical augmentation comes with a temporary benefit
%	and some sort of drawback when it runs out.
%Overuse and overdose can be harmless or extremely dangerous
%	depending heavily on the substance at hand.


	\chapter{Hacking}
	\section{Basics}
	Any piece of cyberware that in some way has a connection to another device may attempt a hacking attack. If someone is monitoring the target device, that person may defend or counterattack.\\
	The attacker’s goal is to reduce the target devices’ \emph{integrity} to 0, working much like damage in combat. If the integrity reaches 0, the attacker gains access to read, write and execute. Attacking gains a modifier called the attacking device’s \emph{processing power}.\\
	The device protects itself by means summarized as “\emph{firewall}”, working much like a character’s damage threshold.
	\section{Actions}
	Any action when hacking is an extended action, requiring a full combat round or more. They can also often times not be interrupted.
	\subsection*{Attack}
	When attacking a device, the character may once per turn spend an action to attack. He makes a Computer Operation check, dealing 1D5+DoS to the target’s integrity, reduced by the target’s firewall. Failing by 10-firewall degrees of failure (or critically) locks the attacking device out of the target device indefinitely.
	\subsection*{Restore}
	As a defender - instead of counterattacking - may restore integrity to the target of the attack. As an action once per turn the defender may make a Computer Operation test to restore 2+DoS integrity.
	\subsection*{Pull the plug}
	Obviously any contender can disconnect the device from the network if it is physically accessible. This immediately stops the hacking attempt and may have - potentially devastating - side effects, which are up to the GM.

	\chapter{Character Creation}
	\section{Prefabs}
	If you want to skip character creation and dive right into play, choose one of the prefab characters below.
	They use the same rules any other character is created with and they have no leftover XP.
	\\
	You can find them at page \pageref{sec:pfchars}.
	
	\section{Resources}
	\label{sec:ccResources}
	Character creation uses two different point pools: Generation Points, or \emph{GP} for short, and Experience Points, or \emph{XP} for short. 
	GP describe the most basic attributes that make up your character - mostly the traits he was born with - while XP depict everything the character has learned.\\
	Every character begins with \emph{100 GP} and \emph{2500 XP}. While you can earn and use XP later on in the campaign, make sure you have exactly no GP left when you finish character creation as any leftover is wasted.
	
	\section{Basic Questions}
	Before anyone determines \emph{what} the character is - i.e. create a stat sheet - first it needs to be decided \emph{who} said character is. The following list of 20 questions covers all questions a player or a GM should have concerning a character and will form the basis for character creation from this point forward. Keep them close at every step, as they define who your character is at its core.

	\pagebreak

	\begin{enumerate}
		\setlength\itemsep{-6mm}
		\item What does the character look like?\\
			\textit{Are you large or small? Are you male or female? Can people even tell? What color are your hair and his eyes? How are you built?}
		\item What first impression does the character leave on strangers?\\
			\textit{Do you look trustworthy or shady? Are others easily intimidated by you? Do you look deceptively smart or stupid?}
		\item How did the character grow up?\\
			\textit{What did you do in your past? What memories will you never forget?}
		\item Does the character still have relations to people from his past?\\
			\textit{Do you have parents or siblings and are they alive? What about other relatives, friends or comrades in arms? A rival perhaps? Or what about a lover?}
		\item Why did the character join the party?\\
			\textit{What event caused you to give up your previous life? What are you after now; what drives you?}
		\item Where has the character been already?\\
			\textit{I know that people rarely come around these days. But maybe you have. And if you haven't: what parts of town have you seen?}
		\item How does the character regard transhumanism?\\
			\textit{Are you chromed up? Would you like to? Do you prefer tech or biotech?}
		\item What’s the character’s opinion on chemical enhancements and recreation?\\
			\textit{Do you like chems? Do you take some? Or do you think they're dangerous, too many side effects?}
		\item For who or what would the character risk his life?\\
			\textit{What is so dear to you that you would risk the only life you have? A person maybe? Or an idea, a dream?}
		\item What’s the character’s biggest wish?\\
			\textit{Do you have a dream, something you want to do above all else? Tell me about what you are living for.}
		\item What’s the character’s biggest fear?\\
			\textit{What do you fear more than anything? Existential dread? Dying without having made a difference? Losing a loved one? You gotta have something, I know it.}
		\item What does the character’s morality look like and how law-abiding is he?\\
			\textit{What do you say, are laws made to be followed or to be broken? Do you rip off others to gain an advantage or do you live for the greater good?}
		\item Is the character open towards strangers?\\
			\textit{Do you trust others? Or are they out to get you? What do they have to offer you?}
		\item How important is life to him?\\
			\textit{Do you value life? Others or just your own? Would you walk over dead bodies to achieve your goals?}
		\item What does the character think of animals?\\
			\textit{Do you like animals? Do you like one in particular? And what about them? Do animals like you?}
		\item What does the character regard as beautiful?\\
			\textit{Everyone has something they really like to surround themselves with. What is it for you - paintings, sculptures, music... certain people? What do you like about it?}
		\item What does the character like to eat and drink? What would he like to try?\\
			\textit{Have you ever tried anything besides nutrient paste? Do you eat cooked food, perhaps even on a regular basis? And is there something that you'd definitely want to try?}
		\item What does love look like to the character?\\
			\textit{What do you feel when you love? Do you even know? Is it something mundane and commonplace or extremely special and precious?}
		\item Does the character have a dark secret?\\
			\textit{What is the worst thing you have done or seen? Why did you do it? Why does it haunt you?}
		\item What character traits define the character?\\
			\textit{If you had to describe yourself is just a handful of words, which would you choose?}
	\end{enumerate}
	
	\section{Race \& Background}
	Player characters are not the average Joe; they are heroes, villains, the protagonists of their story. However before they became these prodigies, they had to have done something, been someone. This is represented by race and background, the first step of character creation, which every character needs to have.\par
	Races and backgrounds have an associated GP cost and grant various bonuses and potentially some penalties. Comprehensive lists can be found in sections \ref{sec:racelist} and \ref{sec:backgroundlist}.
	\section{Characteristics}
	GP can be spent to increase characteristics. Every characteristic can receive up to 25 points and up to 100 can be spent on characteristics overall.\\
	They obviously don’t have to be exhausted.
	\section{Boons \& Banes}
	The leftover GP can be spent on Boons and missing GP can be compensated with Banes.\\
	Boons describe an inherent advantage this character has over others. This is nothing that can be acquired after character creation but it might be at least partially substituted with augments.\\
	Banes are disadvantages that are either innate or stem from the character’s background and are extremely hard to get rid of. They don’t cost GP but grant them instead. Up to 40 GP can be generated from banes.
	\par
	You should assure that all boons and banes don’t exist purely for mechanical reasons but that they fit the character.\\
	If you are looking for additional mechanics, try reflavoring existing boons and banes instead of inventing new ones. There is usually something to fit most visions.\\
	You can find the lists in sections \ref{sec:boonlist} and \ref{sec:banelist}.
	
	\subsection{Disabling Characteristics}
	A special type of Banes are \emph{Disabling Characteristics}. They behave much like characteristics in that a test may be made against them to determine whether they come up. If the test is successful, i.e. the dice show less or equal to the value of the disabling characteristic, the character suffers from whatever it describes, e.g. fear, arrogance, greed or megalomania.\\
	Alternatively whenever a character is making a test they would be affected by a disabling characteristic, e.g. climbing with a fear of heights, that test might be at a penalty equal to the disabling characteristic. They range from 20 to 70 only grant additional GP per full 5 or full 10, as specified.
	
	\section{Skills}
	The first thing to spend experience on are skills. Some skills might already have advancements from race or background. If those have prerequisites that the character does not fulfill, they are dormant until the character fulfills the prerequisites. See a full list of skills in section \ref{skilllist}.
	\section{Educations}
	Next every character has some sort of knowledge, represented by educations. Some educations might be given by background. If those have prerequisites that the character does not fulfill, they are dormant until the character fulfills the prerequisites.\\
	See an example list of Educations in section \ref{eds-explanation} or make some up.
	\section{Abilities}
	Some abilities might be interesting for the character as well. Some abilities might already be known from race or background. If those have prerequisites that the character does not fulfill, they are dormant until the character fulfills the prerequisites. A full list of abilities can be found in section \ref{abilitylist}.
	\section{Shopping trip}
	Lastly, a character needs some equipment. The amount of money a character starts with is equal to 100 plus their charisma. \\
	By default every character will start with clothing fitting of their social standing, former profession and to a degree place of origin. Visit the equipment section below to buy anything beyond that.

	\chapter{Equipment}
	\section{Special Rules}
\vspace{10mm}
\begin{multicols}{2}
%\specialrule{Adjustable}{Allows spending double ammunition for double range. Shooting two full rounds like this deals the weapon's damage to the weapon and the user.}
\specialrule{Backblast (X)}{Everyone behind the shooter in a cone - X meters in length and X/2 meters in base diameter - takes half the weapon damage as if hit by a weapon with the Blast rule. If a wall blocks the cone, the area becomes Blast(X) centered on the user instead.}
\specialrule{Blast}{Every character inside the area takes a hit. For every 2 meters inside the Blast radius the target is, one additional location is hit, up to every location once.}
\specialrule{Bulky}{Cannot be carried in holsters, slings or on mag holsters.}
\specialrule{Cerberus Infection}{When touching an open wound, such as when dealing damage with an attack, the target makes a Constitution test at a -20 penalty. On a failure the target is infected by the Cerberus fungus. Small creatures like mice will simply die.}
\specialrule{Cumbersome}{Halves speed of the wearer/user and grants -10 to all physical actions.}
\specialrule{Disrupt}{Turns off PES for D5 rounds after a hit.}
\specialrule{EMP}{Causes 2D10 rounds of EMP (p. \pageref{hazards:cbrn}).}
\specialrule{Flame}{Sets ablaze any flammable surfaces and objects it hits.}
\specialrule{Flash}{Any affected character looking at the source makes an Ins test at -10 or becomes Blinded for DoF+2 rounds. In addition any affected character within half range makes a Con test at -10 or becomes deafened for DoF+2 rounds. Any affected character with a third becomes Stunned instead.}
\specialrule{Flexible}{The weapon cannot be parried. It only gains half of the normal melee damage bonus.}
\specialrule{Improvised}{Does not require familiarity but instead always invokes -10 penalty and fumbles on 98+, usually breaking apart.}
\specialrule{Multitargeting}{Only one attack is rolled and can be split freely among many targets.}
\specialrule{Oversized}{Built for battle mechs, cannot be used by normal people.}
\specialrule{Piercing}{The user's StrB is added to the weapon's AP.}
\specialrule{Powered}{Requires an external power source to function.}
\specialrule{Protective}{Made for defense: -10 to hit, -2 damage, but +20 to parry.}
\specialrule{Reliable}{Can't jam and is more resilient to damage.}
\specialrule{Rending}{When the attack is blocked by armor (after AP),
	it deals no damage to the target.
	Instead the armor is burnt away and reduced by the damage that would have been dealt until it is repaired.}
\specialrule{Scatter}{+10 to hit at short to extreme range, no AP at long and extreme range, +1d10 dmg when shooting in point blank range}
\specialrule{Single Loader}{Reloading may be interrupted after each bullet. The given reload time is a single round of ammunition. \\
	A Reload test - instead of reducing the reload time - increases the amount of shells reloaded during the loading time, but is at a penalty equal to 5 times the reload time.}
\specialrule{Smoke}{The area inside and behind the smoke is fully concealed. Smoke usually remains dense for a minute and lasts for another as slightly obscuring; this will be reduced in wind or active ventilation, or increased in a fully sealed room.}
\specialrule{Spray}{Ignores half of non-sealed armor and all but full cover.}
\specialrule{Stun (X)}{If the target fails a constitution check at a penalty equal to 10 times X, he becomes stunned for DoF rounds.}
\specialrule{Tearing}{Roll an additional damage dice and drop the lowest. Armor (after AP) is twice as effective.}
\specialrule{Trauma}{When attacking the head location, halve the target's armor. Also halves Stunning Strike penalties.}
\specialrule{Unwieldy}{The weapon disallows transforming reactions to actions, or vice versa.}
\specialrule{Upgrade Kit (X)}{The weapon is an upgrade kit for another weapon X. Price and weight include X.}
\end{multicols}

		\section{Supply}
\subsection{Communication}
\vspace{10mm}
\begin{multicols}{2}
\supply{EC-Comm}{Two-way emergency communicator that fits in the ear. There are variants for all types of trans-species; vat-grown are assumed to have normal human proportions.}{cr 18}{0,01 kg}
\supply{Emergency Location Beacon}{An emergency beacon used for creating distress signals. Able to send an emergency signal on multiple open channels up to 15 kilometers.}{cr 5}{0,03 kg}
\supply{Marker/Transmitter}{Small Tracking device and location transmitter. Can be tethered to almost any smart device.}{cr 22}{0,005 kg}
\supply{Radio Backpack}{An old backpack-carried radio that can transmit any small amount of data like speech up to 5 KM in clear conditions.}{cr 28}{26 kg}
\end{multicols}

\subsection{Cyberware}
\vspace{10mm}
\begin{multicols}{2}
    \supply{AI Docking Station}{By itself this device does nothing but it fulfills the requirements to hold an AI.\\
	Integrity: 15\\
	Firewall: 6\\
	Processing Power: +25}{cr 512}{12 kg}
\supply{Business/Lab Server}{A large device with a lot of processing power and storage but without a direct user interface; either needs a direct connection or access via a network.\\
	Integrity: 40\\
	Firewall: 6\\
	Processing Power: +30}{cr 420}{26 kg}
\supply{Cell phone}{A communication device small enough to fit into one's pocket.\\
	Integrity: 5\\
	Firewall: 1\\
	Processing power: 0}{cr 18}{0,6 kg}
\supply{Cheap Router/Switch}{A connection device to create or scale networks.\\
	Integrity: 11\\
	Firewall: 2 (may work for connected devices, if the attack passes through)\\
	Processing Power: +5}{cr 12}{0,7 kg}
\supply{Cracker}{An expensive machine with incredible processing power.\\
	Integrity: 40\\
	Firewall: 8\\
	Processing Power: +30}{cr 108}{9 kg}
\supply{Portable AI Docking Station}{By itself this device does nothing but it has the requirements to hold an AI. It's small enough to be carried around.\\
	Integrity: 15\\
	Firewall: 5\\
	Processing Power: +20}{cr 896}{5 kg}
\supply{Secure Router/Switch}{A connection device to create or scale networks. Comes with a few security measures and proper configuration.\\
	Integrity: 25\\
	Firewall: 8 (may work for connected devices)\\
	Processing Power: +10}{cr 50}{0,8 kg}
\supply{Smart Fridge}{Barely anyone needs a fridge anymore. But those who do tend to pay extra for completely unnecessary features.\\
	Integrity: 5\\
	Firewall: 0\\
	Processing Power: -5}{cr 333}{170 kg}
\supply{Tablet}{A portable data processing device with a touch screen.\\
	Integrity: 7\\
	Firewall: 2\\
	Processing Power: +5}{cr 23}{1,2 kg}
\supply{Tactical Data Pad}{A tablet that's has 6 armor and shields against 6 rounds of EMP.\\
	Integrity: 8\\
	Firewall: 4\\
	Processing Power: +5}{cr 35}{2,2 kg}
\supply{Workstation}{A personal computer made to work found in many white collar businesses.\\
	Integrity: 20\\
	Firewall: 4\\
	Processing Power: +10}{cr 43}{8 kg}
\end{multicols}

\subsection{Eyewear}
\vspace{10mm}
\begin{multicols}{2}
\supply{Flash Glass Goggles}{Made from special glass, these goggles normalize incoming light. This removes all bright-light based penalties and makes the wearer immune to the Flash rule but the user counts as colorblind while wearing it.}{cr 30}{0,45 kg}
\supply{Holographic Eyepiece}{Allows a HUD to be displayed over one eye.}{cr 28}{0,3 kg}
\supply{Infrared Goggles}{Goggles that give the ability to see in Infrared, allowing to see heat signatures and infrared devices such as lasers.\\
	Infrared Goggles give Character a +15 Bonus to opposing Camouflage Tests. Infrared Goggles halve all Penalties from Darkness.}{cr 32}{0,45 kg}
\supply{Night Vision Device}{Grants Boon: Night Vision but doubles all penalties to resist being blinded by bright light.}{cr 18}{0,6 kg}
\supply{Sun Glasses}{Pair of goggles that halve any Penalties from bright lights but increases low-light penalties by 10 and darkness penalties by 20.}{cr 2}{0,15 kg}
\supply{Aerial Descent Unit}{Standard issue military parachute. Maneuvering uses Pilot: Air but anyone can decent moderately safely using this.}{cr 10}{9,5 kg}
\end{multicols}

\subsection{Hazards}
\vspace{10mm}
\begin{multicols}{2}
\supply{CBRN Unit}{This unit detects and gives information on chemical and radioactive materials, or any other airborne or grounded contaminant. Will get overworked after 60 minutes in strong radiation, thick gases or similar effects.}{cr 12}{0,9 kg}
\supply{Climbing Kit}{A climbing harness kit that offers the Character a +40 Bonus to climbing tests. Contains a Harness, rappelling kit, and other various climbing tools.}{cr 7}{2,2 kg}
\supply{Earplugs}{A set of Military-Grade earplugs that allow Characters to ignore the hearing-based Penalties given by Flashbangs or other loud noises. \\
	Gives a -25 Penalty to Hearing-based Investigation and Perception Tests.}{cr 2}{0,004 kg}
\supply{Explosives Detection Assistant}{A pioneer support device assisting in detecting known explosive compounds. Grants a +30 bonus to detecting and identifying explosives.}{cr 24}{3,4 kg}
\supply{Fire Extinguisher}{Puts out any fire using foamed dry chemicals. Has 10 uses.}{cr 3}{15 kg}
\supply{Metal Detector}{A metal detection unit that is a small metallic plate on an adjustable pole. The Metal Detector\\
	has a screen at the top of the pole that displays the types of metal found underneath, within\\
	3 Meters underground, and how far underground the object is.}{cr 9}{8 kg}
\supply{Proximity detonator}{An expensive proximity trigger mechanism. A specialist can rig most explosives with this trigger, causing it to go off when a human or vehicles passes 1 to 3 meters depending on configuration.}{cr 12}{0,25 kg}
\supply{Remote Detonator}{A small hand-held device that can be synced up to Demolition and Satchel Charge explosives. Has a range of 400 Meters. At the press of a button or an input of an up to 16 digit pass code, the chosen synced explosives will detonate. Can also be set to timed detonation.}{cr 5}{0,45 kg}
\supply{Sound Filter Plugs}{Small speakers, a microphone array and an equalizer, in a small enough package to fit a person's ear. Ignore all hearing based-penalties.\\
	When broken down, infers a -30 to hearing based tests but still protects from loud sounds.}{cr 8}{0,01 kg}
\supply{Tripwire}{A kit containing all necessary components to rig a grenade into an IED triggered by a tripwire.}{cr 4}{0,7 kg}
\end{multicols}

\subsection{Medical}
\vspace{10mm}
\begin{multicols}{2}
\supply{Field surgery kit}{A kit containing scalpels, syringes and other equipment to do basic field surgery like removing bullets or even remove augments from corpses provided the necessary expertise.\\
	Grants +10 to medical tests.}{cr 34}{5,6 kg}
\supply{Field surgery tent}{A very simple, packaged tent that can be used as an almost sterile field surgery room. Does not protect from the elements much.\\
	Grants +10 to medical tests.}{cr 35}{1,5 kg}
\supply{Splint}{When applied this reduces penalties from cracked and broken bones to a quarter.}{cr 3}{0,01 kg}
\supply{Stitch kit}{A more elaborate needle and thread that will close wounds for good. Takes a minute (20 combat rounds) and a Medication check +20 for light bleeding or two minutes and an unmodified Medication check for heavy bleeding. Cannot be applied by the bleeding character on his own.}{cr 22}{0,6 kg}
\supply{Thermal blanket}{Prevents hypothermia or overheating.}{cr 2}{0,9 kg}
\supply{Wound dressing}{Will stop bleeding when applied. Takes 6 seconds (2 combat rounds) but already stops the target from bleeding out when started. Cannot be applied by the bleeding character on his own. Comes in a pack of 20.}{cr 5}{0,2 kg}
\supply{Wound sealant foam}{Will stop bleeding. Takes about 12 seconds (4 combat rounds) to use, 2 of which are the application, the other 2 are hardening. Natural healing of the given zone is stopped until the foam is removed by a physician. Comes in packs of 5.}{cr 4}{0,8 kg}
\end{multicols}

\subsection{Navigation}
\vspace{10mm}
\begin{multicols}{2}
\supply{Compass}{A standard magnetic compass that is not effected by EMP or any other sort of digital or electrical attack.}{cr 1}{0,2 kg}
\supply{Road ware}{Weak AI based GPS-reading software that assists drivers at high speeds and off-roads. The unit gives a +20 Bonus to high-speed driving Tests and Stunting Tests.}{cr 14}{4,8 kg}
\end{multicols}

\subsection{Protection}
\vspace{10mm}
\begin{multicols}{2}
\supply{Balaclava}{Face mask for face protection from harsh winds or being identified.}{cr 1}{0,3 kg}
\supply{Cover Foam Canister}{A cylindrical 2 liter canister holding pressurized, quick hardening foam. When primed and thrown the foam will escape and expand to roughly one cubic meter. Within a minute it will harden to something akin to shale rock and while it hardens it may be molded. It does not gain any mass, of course, only volume.}{cr 28}{4,2kg}
\supply{Clothing, Simple}{Simple clothing to blend into ghettos.}{cr 6}{1,1 kg}
\supply{Clothing, Expensive}{Expensive clothing to blend into high society.}{cr 140}{1,1 kg}
\supply{Flotation Vest}{An adjustable flotation vest that allows up to 21 kg float in water.}{cr 2}{1 kg}
\supply{Gas Mask}{Allows to breathe in toxic environments. Comes with 5 disposable filters, each able to last 24 hours.}{cr 18}{1,1 kg}
\supply{Ghillie Cloak}{The Ghillie Cloak offers a +20 Bonus to Camouflage when used in matching surroundings.}{cr 6}{2,4 kg}
\supply{Handcuffs}{Used to bind an individual’s hands or ankles. To escape, a character must roll a -40 security test or a -60 strength test. The handcuffs have an armor rating of 10 and 50 break points.}{cr 12}{0,5 kg}
\supply{Rain Poncho}{Protects the user from rain.}{cr 2}{0,8 kg}
\supply{Rape Whistle}{A small, loud whistle. Reportedly used by some "special" forces alongside foam swords.}{cr 1}{0,05 kg}
\supply{Riot Shield}{The Anti-Ballistics Riot Shield acts as mobile cover for a Character. The Riot Shield is made of heavy mineral carbon with a hardened plastic window to see through. \\
	The Shield covers normal-sized or smaller peoples' full bodies, while larger people must crouch to cover their whole body. \\
	The Riot Shield functions like any normal cover worth 22 armor points.}{cr 28}{5,9 kg}
\end{multicols}

\subsection{Software}
\vspace{10mm}
\begin{multicols}{2}
\supply{Security Measures}{Decreases a device's processing power by 10 but increases its firewall by 3. Can be deactivated if required for some reason.}{cr 75}{0 kg}
\end{multicols}

\subsection{Storage}
\vspace{10mm}
\begin{multicols}{2}
\supply{Ammunition Bandolier}{A bandolier that is worn over armor/clothing that can hold up to 100 shotgun shells, 150 rifle and pistol Rounds, \\
	50 Sniper Rounds, and 10 Grenades, mass driver blocks, filled magazines, and 40mm Grenades.}{cr 7}{0,3 kg}
\supply{Ammunition Pouch}{Ammunition pouch. Carries 8 magazines, grenades or anything of equivalent size.}{cr 5}{0,35 kg}
\supply{Blast-proof Clear Casing}{Blast-proof casing. Can hold up to 5 magazines, grenades or size equivalents but is generally used for portable electronic devices. Offers 18 points of cover that also blocks damage from the EMP special rule to devices on the inside. Damage from EMP cannot damage the case.}{cr 21}{2,8 kg}
\supply{Duffle Bag}{Soft bagged carrying device like a sports- or backpack. Carries 25 magazines, grenades or size equivalent.}{cr 9}{2,9 kg}
\supply{Ergonomic Weapon Holster}{Holds weapons in completely over-engineered ways. Time to draw is reduced by one action to a free action minimum.}{cr 16}{0,35 kg}
\supply{Hardcase/Tactical Hardcase}{An armored carrying device the size of a small backpack / lunchbox that can be placed virtually anywhere on the body. Capable of carrying up to 14/5 Magazines, Grenades, and Clips, or anything of equivalent size. \\
	The Hardcase has 10 Armor Rating protecting both itself and its content but can only take 30 damage before it's so broken down that it won't hold anything anymore. }{cr 30/18}{1,2 kg / 0,7 kg}
\supply{Softcase/Tactical Softcase}{Soft bagged carrying device the size of a small backpack / lunchbox that can be placed on virtually anywhere on the body. Carries 20/7 magazines or grenades, or anything of equivalent size.\\
	The softcase has no armor protecting itself or its content and 25 Break Points before it's so broken down that it won't hold anything anymore. }{cr 25/13}{0,9 kg / 0,5 kg}
\supply{Tactical Thigh Rig}{Small, thigh-rigged casing. Carries 3 magazines or grenades, or anything of equivalent size.}{cr 9}{0,2 kg}
\supply{Weapon Holster}{Holds a small weapon. Legally bought or issued weapons are generally assumed to come with a proper holster.}{cr 5}{0,3 kg}
\supply{Weapon Sling}{Holds a larger weapon to keep the hands free.}{cr 5}{0,4 kg}
\end{multicols}

\subsection{Survival}
\vspace{10mm}
\begin{multicols}{2}
\supply{Area Heater}{A small heater that can heat an area large enough to fill a dome tent.}{cr 24}{2,3 kg}
\supply{Camping Stool}{A stool for sitting that has a setting that allows the stool to fit small to large person.}{cr 6}{1,5 kg}
\supply{Entrenching Tool}{A compact package functioning as a shovel, an axe or a crowbar. It is much more convenient than carrying three equipment pieces but does none quite as well as the the real thing:\\
	The crowbar end only grants 20 bonus.\\
	The axe head deals 4 damage less to structures.\\
	The small shovel takes 10\% longer.}{cr 25}{2,5 kg}
\supply{Fishing Kit}{A fishing kit that contains a folding fishing rod, ten hooks and bobbers, a spool of 1550 meters of fishing line that can withstand up to 6 kg of tension, 50 pieces of bait, a fishing net that can hold up to 6 kg of weight and a fitting hat.}{cr 7}{2,8 kg}
\supply{Military Shovel}{A standard issue sharpened shovel for entrenchment and other uses. Folds down to the size of the shovel’s head. Rust-proof, sturdy and reliable.}{cr 7}{1,4 kg}
\supply{MRE / Nutrient solution}{Disgusting nutrient solution that never goes bad - or worse for that matter... \\
	One tube is enough to sustain normal people for a day and one tube is government issued to all registered civilians each day.}{cr 1}{0,3 kg}
\supply{Multi-Tool Kit}{A briefcase sized kit of tools for repair and building.}{cr 19}{8 kg}
\columnbreak %remove if it becomes unnecessary
\supply{Portable Electricity Generator}{A portable generator that is capable of powering equipment and structures who either don't have their own or whose generators have broken down.}{cr 42}{26 kg}
\supply{Portable Food Stove}{A small stove that can prepare food. Since the advent of large scale nutrient solution supply this piece of equipment has become very rare.}{cr 64}{3,5 kg}
\supply{Prepared Meal}{A luxury only very few people choose to afford. Generally a waste of biomass and more of a treat than survival equipment.}{cr 12}{0,5 kg}
\supply{Proper Meal Ingredients}{Expensive ingredients that have short shelf life, need the time, skill and installations to be cooked and don't always offer all nutrients a person needs but if prepared correctly can be a rare, tasty treat.}{cr 4}{0,6 kg}
\supply{Singles/Large/Dome Tent}{A weather-resistant tent for one/five/ten people to fit in.}{cr 8/22/40}{1,2 / 6 / 12 kg}
\supply{Sleep Cot}{A small cot bed for a single person.}{cr 12}{4,5 kg}
\supply{Sleeping Bag}{A wind-proof sleeping bag.}{cr 10}{1,2 kg}
\supply{Survival Blanket}{A wind and water-proof blanket that uses a thermal lining to keep the user in a more comfortable temperature, closer to their body-heat.}{cr 10}{1,1 kg}
\supply{Water Canteen}{A leak-proof survival canteen with a lock top and 2-liter capacity.}{cr 4}{0,6 kg}
\supply{Water Purification System}{A portable water purification system, favored by wastelanders. It can process 1 liter of water every two hours. Must be cleaned after each use.}{cr 32}{24 kg}
\end{multicols}

\subsection{Tools}
\vspace{10mm}
\begin{multicols}{2}
\supply{Blowtorch}{A blowtorch that can be used in any atmosphere and underwater. This blowtorch is plasma based and will sear through any metal, given enough time.\\
	For every 2 points of Armor or Cover Points something gives, it takes 1 Half Action to cut through a small portion of it}{cr 18}{1,8 kg}
\supply{Bolt Cutters}{Powerful bolt cutters. Strong enough to cut through even metal sheets and locks. Pincers may need to be replaced after heavy duty as determined by the GM.}{cr 12}{1,1 kg}
\supply{Crowbar}{A proven tool from ages past. Grants a +30 bonus to strength based tests to pry open containers.}{cr 8}{2,6 kg}
\supply{Duct Tape}{A simple, strong water-proof tape. A thousand years don't change everything.}{cr 1}{0,3 kg}
\supply{Fire Starter}{Small dark-green cubes that when cracked open, begin to spark and get hot. After 8 Seconds (3 Rounds) it begins to burn. Works in wind and in rain. Comes with 8 cubes.}{cr 2}{0,1 kg}
\supply{Grappling Hook}{A metal rod with hooks on them that allow for easier climbing when attached to a rope. +10 to Climbing Tests.}{cr 4}{1 kg}
\supply{Gun Swivel}{A swivel arm that holds the gun. It reduces recoil but is rather cumbersome; it halves top movement speed and the Burst penalty. It is not an exoskeleton and does not allow the use of much heavier guns.}{cr 42}{1,4kg}
\supply{Lighter}{A lighter with a flip-off top that is able to produce flame. Has enough fuel for 300 uses}{cr 1}{0,1 kg}
\supply{Omnidirectional Wave Emitter}{A fancy music box that generates synthwave music on the spot.}{cr 12}{6kg}
\supply{Rope}{A carbon fiber rope that can hold up to 5 kg. Comes in a bundle of 35 meters.}{cr 6}{4,2 kg}
\supply{SOS Knife Kit}{A special multi-tool knife that contains the following tools: Screwdriver, corkscrew, small blade, sharpening stone, toothpick, tweezers, wood saw, large blade, metal saw, scissor.}{cr 6}{0,2 kg}
\supply{Weapon Cleaning Kit}{A kit that can be used to clean weapons and perform maintenance to remove any adverse Penalties gained from the weapon being dirty.}{cr 2}{0,7 kg}
\end{multicols}

\subsection{Vision \& Detection}
\vspace{10mm}
\begin{multicols}{2}
\supply{Binoculars}{An electronic Binocular. Has 6 zoom functions, 2x/4x/6x/10x/20x.}{cr 4}{0,8 kg}
\supply{Chemical Light Sticks}{A Chemical Light Stick is a bendable tube, that when bent, begins to glow. Gives a +10 in any Dark or Pitch-Black scenarios. Lasts for 8 hours. Comes in packs of 10.}{cr 2}{0,2 kg}
\supply{Fiber Optic Probe}{A small camera on the end of a 4-meter wire.}{cr 42}{0,6 kg}
\supply{Flare Gun}{A small firearm that fires a 40mm Parachute Flare up to 150 meters}{cr 30}{0,9 kg}
\supply{Flashlight}{Helps to see in the dark. Lowers any Low-light or Darkness penalties by up to 30 but makes it impossible to hide.}{cr 5}{0,6 kg}
\supply{Helmet Recorder}{Records every instance of what happens. Able to store 80 hours of higher quality footage.}{cr 16}{0,4 kg}
\supply{Signal Flare}{A bright flare that is activated by snapping the top off. Can be seen from 2 KM away and lasts up to 1 hour.}{cr 3}{0,35 kg}
\supply{Spotter Target System}{A scoped spotting system that has multiple zoom variants. X10, x15, x20, x25, x30, and x40 scopes that allow a Spotter to assist a sniper in extreme-ranged combat.}{cr 25}{1,1 kg}
\supply{Stationary Motion Tracker}{Scans movement of the environment up to 120 meters and sends it to tethered display devices. \\
	Must be stationary for 1 Minute (20 Rounds) to calibrate and work. If moved, the Stationary Motion Tracker must be re-calibrated again.}{cr 88}{29 kg}
\end{multicols}
	\section{Services}
\vspace{10mm}
\begin{multicols}{2}
%Manpower
\service{Goon}{Hire a simple goon.}{cr 18}{day}
\service{Bodyguard}{Hire a trained bodyguard.}{cr 32}{day}
\service{Fire Team}{Employ a fire team of mercenaries.}{cr 100}{soldier}
\service{Strike Team}{Employ a strike team.}{cr 700}{soldier}
\service{Spec Ops}{Call in a special operations team.}{cr 320}{soldier}

%Fire support
\service{CAS}{Call in close air-support on a marked location.}{cr 200}{run}
%todo Bombing run
%todo MAC strike

%Assassination
\service{Hitman (Simple target)}{Set a hitman on a rather simple target like a former soldier.}{cr 250}{target}
\service{Hitman (Advanced Target)}{Set a hitman on a difficult target like an active general.}{cr 700}{target}

%Medical
\service{Back Alley Doc}{The character is healed D5+1 wounds from medical attention and no word is lost about where those injuries came from.}{cr 25}{session}
\service{Back Alley Surgery}{One of:\\
	a) Install an augment, no questions asked. Recovery time is doubled.\\
	b) The character undergoes facial reconstruction, becoming unrecognizable from sight alone.}{cr 45}{session}
\service{Hospital}{One of:\\
	a) The character is healed 2D5+2 wounds from medical attention.\\
	b) Install an augment.}{cr 55}{session}
\service{Medical Tank}{The character is put into a medical tank. He doesn't bleed, poisons stop and he is granted one natural regeneration phase per hour. Since the setup and post-regeneration cleanup are the difficult and time consuming things, no time frame under 24 hours is usually at offer.}{cr 180}{24 hours}

%Transportation
\service{Taxi (Safe Area)}{The character is taken to a desired location. The ride goes through only safe neighborhoods.}{cr 7}{10 km}
\service{Taxi (Dangerous Area)}{The character is taken to a desired location. The ride goes through a dangerous neighborhood.}{cr 4}{km}
\service{VTOL Ride}{The character is taken to a desired location by a VTOL aircraft, like a helicopter.}{cr 40}{10 km}
\service{Orbital Drop}{Drop from low orbit onto a desired location. This can obviously not be taken for inside underground complexes but may dig through a bunker's ceiling and trash buildings.}{cr 375}{jump}
\end{multicols}

	\section{Ranged}
\subsection{Weapons}
\vspace{4mm}
\begin{multicols}{2}
\subsubsection{Pistol}
\vspace{2mm}
\rangedweapon{Obsidian Bull}{A hand cannon fed from a cylinder.}{Shots: 1; Damage: 2D10+8; AP: 1}{25m}{6 (11.8x45mm)}{6}{Reliable, Single Loader}{cr 50}{1.7 kg}
\rangedweapon{Wasteland Condor}{A powerful handgun with an extremely heavy cartridge.}{Shots: 1; Damage: 2D10+8; AP: 1}{25m}{7 (11.8x45mm)}{2}{}{cr 50}{1.9 kg}
\rangedweapon{Amber Bull}{A revolver made to fire low caliber rounds.}{Shots: 1; Damage: 2D10+3; AP: 0}{20m}{8 (7.2mm)}{8}{Reliable, Single Loader}{cr 30}{0.5 kg}
\rangedweapon{M10}{A small, low caliber, semiautomatic pistol. Very light and decently easy to conceal.}{Shots: 1; Damage: 2D10+3; AP: 0}{20m}{9 (7.2mm)}{1}{}{cr 30}{0.6 kg}
\rangedweapon{mOP-3}{A hand-sized 20mm grenade hind loader}{Shots: 1; Damage: ; AP: depending on ammo}{20m}{1 (20mm grenade)}{4}{minimum arming distance of 10m}{cr 40}{1.3 kg}
\rangedweapon{"Little Lucifer" Hand Flamer}{A one-handed flamethrower. While it holds "flame" in its name, it can be used for all fueled spray weapons.}{Shots: 5; Damage: 1D10+2; AP: 0}{12m}{5 (small fuel tank)}{8}{Flame, Scatter, Spray}{cr 80}{1.6 kg}
\rangedweapon{OH-VLS}{A small, adjustable laser pistol}{Shots: 2; Damage: 1D10+10; AP: 1}{35m}{16 (small fuel cell)}{4}{Adjustable}{cr 70}{0.6 kg}
\rangedweapon{Newton Pistol Mk5}{A forearm-sized magnetic accelerator shooting metal fragments}{Shots: 1; Damage: 1D10+2; AP: 12}{35m}{30 (small mass driver shard block)}{8}{}{cr 110}{2.6 kg}
\rangedweapon{Newton Pistol Mk2 }{The smallest mass drivers ever built. Shoots very small metal fragments}{Shots: 2; Damage: 1D10+2; AP: 8}{30m}{30 (small mass driver shard block)}{8}{}{cr 95}{2.1 kg}
\rangedweapon{Mk1 Shock pistol}{A modified taser. Very much lethal.}{Shots: 1; Damage: 2D5; AP: 0}{20m}{8 (small fuel cell)}{5}{EMP, Stun (4)}{cr 50}{0.4 kg}
\rangedweapon{Scattershock}{A one-handed Tesla gun}{Shots: 2; Damage: 3D5; AP: 0}{15m}{6 (small fuel cell)}{5}{Scatter, EMP, Stun (3)}{cr 80}{0.6 kg}
\end{multicols}

\subsubsection{Short Rifle}
\vspace{8mm}
\begin{multicols}{2}
\rangedweapon{MS shotgun}{A double-barrel shotgun with shortened barrels to facilitate pellet spread}{Shots: 1; Damage: 2D10+3; AP: 1}{40m}{2 (12gauge)}{4}{Reliable, Scatter, Single loader}{cr 25}{3.0 kg}
\rangedweapon{short-barrel AR-22}{A rifle caliber fired from a short system. Warning: only buy this if you don't own a dog!}{Shots: 3; Damage: 2D10+2; AP: 6}{50m}{30 (6.3mm)}{4}{}{cr 110}{3.5 kg}
\rangedweapon{MP15}{A pistol caliber fired from a small rifle frame.}{Shots: 4; Damage: 2D10+3; AP: 1}{30m}{25 (7.2mm)}{3}{}{cr 65}{2.8 kg}
\rangedweapon{"Thumper"}{A small CQC grenade launcher meant for utility grenades. Exists as an under barrel version for rifles.}{Shots: 1; Damage: ; AP: depending on ammo}{20m}{1 (20mm grenade)}{4}{minimum arming distance of 10m}{cr 70}{4.8 kg}
\rangedweapon{OH-VLS conversion kit}{A conversion kit to turn an OH-VLS into a small laser carbine.}{Shots: 3; Damage: 1D10+12; AP: 1}{70m}{16 (medium fuel cell)}{5}{Adjustable, Flame, Upgrade (OH-VLS)}{cr 140}{2.8 kg}
\rangedweapon{Newton Pistol Repeater kit}{A conversion kit for low powered Newton Pistols to increase fire rate.}{Shots: 4; Damage: 1D10+3; AP: 12}{45m}{30 (small mass driver shard block)}{8}{Upgrade (Mk2 Newton Pistol)}{cr 140}{3.6 kg}
\rangedweapon{"Junk Jet" Mass Driver Shotgun}{A mass driver with increased bore to achieve shotgun-like ballistics}{Shots: 1; Damage: 2D10+2; AP: 8}{55m}{6 (small mass driver shard block)}{3}{Scatter}{cr 170}{4.2 kg}
\rangedweapon{Crossbow / Harpoon}{A very short bow with a trigger }{Shots: 1; Damage: 1D10+4; AP: 0}{50m}{1 (crossbow bolt)}{6}{Silenced, ignores shields}{cr 25}{3.0 kg}
\rangedweapon{Scattershock rifle conversion kit}{A conversion kit for the Scattershock to feed from larger fuel cells. Increases size, mag size and range.}{Shots: 2; Damage: 3D5; AP: 0}{45m}{30 (medium fuel cell)}{5}{Scatter, EMP, Stun (3), Upgrade (Scattershock)}{cr 125}{2.1 kg}
\end{multicols}

\subsubsection{Long Rifle}
\vspace{8mm}
\begin{multicols}{2}
\vspace{2mm}
\rangedweapon{AR-22}{A full sized rifle}{Shots: 3; Damage: 2D10+3; AP: 11}{120m}{30 (6.3mm)}{6}{}{cr 150}{3.8 kg}
\rangedweapon{M282 SSW}{A bulky, belt-fed fire support weapon with high ammunition capacity}{Shots: 6; Damage: 2D10+3; AP: 11}{120m}{120 (6.3mm)}{14}{Bulky}{cr 220}{7.0 kg}
\rangedweapon{Mk3 Tankgewehr}{An insanely high-caliber, long barrel rifle made to fight heavy armor. Carries a lot of energy of comparatively small distances.}{Shots: 1; Damage: 3D10+3; AP: 30}{250m}{1 (15x132mm)}{10}{}{cr 210}{16.6 kg}
\rangedweapon{A3R}{The "Makar Advanced Anti-Armor Rifle": A long range rifle made to take out high value targets or light armor. Not an overly clever name but don't tell the snipers.}{Shots: 1; Damage: 2D10+7; AP: 18}{800m}{5 (11.8x112mm)}{6}{}{cr 210}{12.0 kg}
\rangedweapon{M5 Shotgun}{A scatter gun firing multiple pellets}{Shots: 2; Damage: 4D5+6; AP: 4}{80m}{8 (12gauge)}{16}{Single Loader, Scatter}{cr 130}{3.8 kg}
\rangedweapon{OP-2}{A 20mm single shot grenade launcher with increased barrel length}{Shots: 1; Damage: -; AP: depending on ammo}{50m}{1 (20mm grenade)}{4}{minimum arming distance of 10m}{cr 60}{1.9 kg}
\rangedweapon{"Rathalos" Flamethrower}{A flame thrower with an integrated tank to profit from familiar, proven rifle ergonomics.}{Shots: 5; Damage: 2D10+4; AP: 0}{35m}{15 (medium fuel tank)}{8}{Flame, Scatter, Spray}{cr 120}{4.0 kg}
\rangedweapon{BR-56}{A semiautomatic laser rifle}{Shots: 2; Damage: 1D10+18; AP: 3}{500m}{10 (medium fuel cell)}{12}{Adjustable}{cr 230}{4.3 kg}
\rangedweapon{Incinerator Rifle}{A long range laser rifle. Instead of piercing armor it burns it away.}{Shots: 1; Damage: 2D5+30; AP: 4}{700m}{1 (large fuel cell)}{16}{Adjustable, Flame}{cr 220}{9.7 kg}
\rangedweapon{M4 "Silver Crow"}{A handheld, semi automatic mass driver}{Shots: 2; Damage: 2D10+4; AP: 22}{550m}{16 (large mass driver shard block)}{10}{}{cr 180}{11.2 kg}
\rangedweapon{R235 Squad Support Weapon}{A fully automatic mass driver}{Shots: 5; Damage: 2D10+4; AP: 16}{180m}{180 (large mass driver shard block)}{18}{}{cr 250}{15.6 kg}
\rangedweapon{R13 "Armor Cracker"}{A handheld, long range mass driver}{Shots: 1; Damage: 1D5+18; AP: 35}{1400m}{7 (large mass driver shard block)}{16}{}{cr 260}{23.9 kg}
\rangedweapon{FG-9}{A rifle firing high energetic, self-stabilized plasma that explodes on impact. The energy release is highly unpredictable}{Shots: 1; Damage: 4D10+6; AP: 1}{40m}{10 (large fuel cell)}{16}{Blast (3), EMP, Stun (2)}{cr 310}{6.3 kg}
\rangedweapon{Choke M3 Arc Rifle}{An arc rifle with little spread to increase range}{Shots: 2; Damage: 3D10; AP: 0}{60m}{10 (large fuel cell)}{12}{EMP, Stun (3), Scatter}{cr 180}{4.1 kg}
\rangedweapon{Choke F3 Arc Thrower}{An arc rifle with a large spread made for crowd suppression}{Shots: 2; Damage: 5D5; AP: 0}{30m}{10 (large fuel cell)}{10}{EMP, Stun (2), Spray}{cr 150}{3.9 kg}
\end{multicols}

\subsubsection{Heavy}
\vspace{8mm}
\begin{multicols}{2}
\rangedweapon{M2 Brass Storm}{A high-caliber machine gun with high fire-rate.}{Shots: 8; Damage: 2D10+4; AP: 10}{220m}{120 (11.8x112mm)}{18}{}{cr 280}{42.7 kg}
\rangedweapon{M36 Chain Cannon}{A heavy machine gun with an external cycler. Reduced fire rate when compared to the M2 but renowned for unrivaled reliability. It is extremely heavy and cannot reasonably be called portable.}{Shots: 5; Damage: 2D10+4; AP: 10}{240m}{120 (11.8x112mm)}{18}{Reliable}{cr 295}{69.2 kg}
\rangedweapon{RM4 Chain Shotgun}{An automatic, belt fed shotgun known for its high maintenance need.}{Shots: 4; Damage: 4D5+6; AP: 4}{80m}{80 (12 gauge)}{18}{Scatter}{cr 280}{28.4 kg}
\rangedweapon{"Forge Fire" Heavy Flamer}{A large flamethrower made for vehicles and emplacements}{Shots: 5; Damage: 3D10+3; AP: 0}{50m}{30 (large fuel tank)}{40}{Flame, Scatter, Spray}{cr 220}{35.4 kg}
\rangedweapon{"Slicer" Laser Cannon}{A vehicle laser cannon, powerful enough to break bunker doors. Requires a massive power source to equalize its needs and is overly affected by refraction, limiting its range.}{Shots: 1; Damage: 5D10+20; AP: 8}{40m}{5 (vehicle power cell)}{40}{Powered}{cr 450}{231.0 kg}
\rangedweapon{Model 6 "Gladiator"}{A heavy mass driver sporting 5 rotating accelerators to prevent overheating and increase fire rate}{Shots: 12; Damage: 3D10+8; AP: 12}{500m}{500 (vehicle mass driver shard block)}{40}{Powered}{cr 420}{169.0 kg}
\rangedweapon{Mag Rail Cannon}{A massive mass driver built for anti armor LAVs in the days of war}{Shots: 1; Damage: 3D10+18; AP: 40}{6000m}{20 (vehicle mass driver shard block)}{40}{Powered}{cr 450}{143.0 kg}
\rangedweapon{"Zeushammer"}{Also referred to as the "Ball Lightning Cannon" this weapon is a long range Tesla cannon firing arcing ball lightnings.}{Shots: 1; Damage: 9D10-4; AP: }{150m}{5 (vehicle power cell)}{40}{Blast (6), EMP, Stun (5), Spray, Powered}{cr 410}{197.0 kg}
\end{multicols}

\subsubsection{Launcher}
\vspace{8mm}
\begin{multicols}{2}
\rangedweapon{MML-4}{A multi rocket launcher built for vehicles}{Shots: 2; Damage: -; AP: depending on ammo}{350m}{6 (rocket propelled grenade)}{40}{Powered, Backblast (4)}{cr 220}{36.9 kg}
\rangedweapon{WTFBRR}{A massive recoilless rifle made for giant loads to be fired from vehicles}{Shots: 1; Damage: -; AP: depending on ammo}{700m}{1 (large rocket propelled grenade)}{20}{Backblast (8)}{cr 360}{45.7 kg}
\rangedweapon{Hail Fire}{A micro missile launcher firing multiple rockets on multiple targets simultaneously}{Shots: 4; Damage: 3D10+2; AP: 4}{250m}{4 (micro missile packs)}{40}{Powered, Multitargeting, Backblast (2)}{cr 200}{26.7 kg}
\rangedweapon{M64}{A 40mm rotary grenade launcher}{Shots: 1; Damage: -; AP: depending on ammo}{100m}{6 (40mm grenade)}{36}{Single Loader, minimum arming distance of 15m}{cr 180}{5.4 kg}
\rangedweapon{ERPG}{A rocket propelled grenade launcher}{Shots: 1; Damage: -; AP: depending on ammo}{400m}{1 (rocket propelled grenade)}{6}{Backblast (4)}{cr 210}{6.2 kg}
\rangedweapon{OP-4}{A 40mm single shot grenade launcher}{Shots: 1; Damage: -; AP: depending on ammo}{70m}{1 (40mm grenade)}{6}{minimum arming distance of 15m}{cr 90}{2.2 kg}
\rangedweapon{"Thunderstrike" Grenade Launcher}{A grenade launcher fed from a belt. Very useful for crowd control but complete overkill in most situations. It is also extremely cumbersome due to the massive belt or belt box, much too large to be used on foot.}{Shots: 4; Damage: -; AP: -}{100m}{30 (40mm grenade)}{9}{minimum arming distance of 15m}{cr 136}{14.2 kg}
\end{multicols}

\pagebreak
\subsubsection{Bow}
\vspace{8mm}
\begin{multicols}{2}
\rangedweapon{collapsible bow}{A bow that can be collapsed to fit small cases for transportation and concealment}{Shots: 1; Damage: 1D10+3; AP: 1}{50m}{1 (arrow)}{1}{Silenced}{cr 25}{1.5 kg}
\rangedweapon{recurve bow}{A full-sized bow}{Shots: 1; Damage: 1D10+4; AP: 2}{75m}{1 (arrow)}{1}{Silenced, -10 to hit per StrB below 4}{cr 20}{3.2 kg}
\end{multicols}

\subsubsection{Throwing}
\vspace{8mm}
\begin{multicols}{2}
\rangedweapon{Asphyxiation Grenade}{A chemical weapon that burns away oxygen in a living room sized area. This reaction is not strongly exothermic, making it hard to detect before it's too late.}{Does not deal damage but causes suffocation.}{15m}{1 (Grenade)}{1}{Blast (5)}{cr 31}{0.3 kg}
\rangedweapon{EMP grenade}{Very expensive and very rare, this new type of explosive stuns or even damages unshielded electronics. It also causes piezoelectric exoskeletons to spasm and hurt the wearer.}{Shots: 1; Damage: 0; AP: 0}{15m}{1 (Grenade)}{1}{Blast (6), EMP}{cr 28}{0.1 kg}
\rangedweapon{Flashbang}{Popular among law enforcement personnel this non-lethal grenade blinds and deafens anyone not wearing appropriate protective gear with a loud and bright flash.}{Shots: 1; Damage: 0; AP: 0}{15m}{1 (Grenade)}{1}{Blast (12), Flash}{cr 14}{0.1 kg}
\rangedweapon{Frag grenade}{An explosive that spreads a large amount of shrapnel in a decently large area. Used against groups of hostiles and to clear rooms since the dawn of gunpowder.}{Shots: 1; Damage: 2D10+2; AP: 2}{15m}{1 (Grenade)}{1}{Blast (6)}{cr 22}{0.1 kg}
\rangedweapon{Incendiary grenade}{A fire-based explosive known to be easy and cheap to improvise and to cause a lot of collateral damage. Less useful against energy shielding and metal constructions.}{Shots: 1; Damage: 1D10+2; AP: 0}{15m}{1 (Grenade)}{1}{Blast (8), ignites flammable surfaces in the area}{cr 22}{0.1 kg}
\rangedweapon{Riot Foam}{A canister of sticky foam that heavily impairs movement of everyone caught up in it. Can safely but slowly be burnt away at low temperatures.}{Shots: 1; Damage: 0; AP: 0}{15m}{1 (Grenade)}{1}{Blast (12), Sticky}{cr 12}{0.1 kg}
\rangedweapon{Smoke grenade}{Favored in open places this non-lethal grenade grants a large amount of artificial concealment to otherwise clear lines of fire.}{Shots: 1; Damage: 0; AP: 0}{15m}{1 (Grenade)}{1}{Blast (8), Smoke}{cr 14}{0.1 kg}
\rangedweapon{Tear gas grenade}{Non-lethal irritant grenades usually employed to stop riots}{Shots: 1; Damage: 0; AP: 0}{15m}{1 (Grenade)}{1}{Blast (8), Smoke, Stun (0), Stun is applied every round in the Smoke}{cr 16}{0.1 kg}
\rangedweapon{Blunt throwing weapon}{Blunt instruments like rocks and bottles being thrown. Very common when nothing else is at hand.}{Shots: 1; Damage: 1D5+7; AP: 6}{15m}{1 (Blunt instruments)}{1}{Trauma, Piercing, Improvised}{Rocks are free}{1.0 kg}
\rangedweapon{Throwing axe}{Sharp, heavy, bladed weapon weighted for throwing}{Shots: 1; Damage: 1D10+3; AP: 0}{15m}{1 (Axes)}{1}{}{cr 12}{1.1 kg}
\rangedweapon{Throwing knives}{Sharp, bladed weapons weighted for throwing}{Shots: 1; Damage: 1D5+3; AP: 3}{15m}{1 (Knives)}{1}{}{cr 10}{0.4 kg}
\end{multicols}

\subsubsection{Traps}
\vspace{8mm}
\begin{multicols}{2}
\rangedweapon{Bear trap}{A strong, spring loaded clamp made to injure legs and stop creatures from fleeing or approaching.}{Shots: 1; Damage: 2D10+12}{0.3m}{-}{10}{}{cr 8}{1.1 kg}
\rangedweapon{C-7}{A block of plastic explosives. Usually made to be set up but can the thrown in a pinch}{Shots: 1; Damage: 2D10+18; AP: 16}{5m}{1 (C-7)}{2}{Blast (6)}{cr 22}{0.2 kg}
\rangedweapon{Cloak Mine}{A landmine with an integrated cloaking field. Makes an obvious noise when stepped upon and takes a decent amount of time to arm for safety reasons.}{Shots: 1; Damage: 2D10+5; AP: 4}{-}{1 (Mine)}{10}{Blast (3); detecting the noise when stepped upon takes a +0 Perception test}{cr 33}{1.2 kg}
\rangedweapon{Exploding Fuel Cell, Small}{A small fuel cell, usually built very sturdily, may be rigged as an improvised explosive device. Unstable cells however may explode when handled too roughly in high temperatures.}{Shots: 1; Damage: 2D10+4; AP: 6}{15m}{1 (small fuel cell)}{1}{Blast (1)}{cr 9}{0.3 kg}
\rangedweapon{Exploding Fuel Cell, Medium}{A fuel cell, usually built very sturdily, may be rigged as an improvised explosive device. Unstable cells however may explode when handled too roughly in high temperatures.}{Shots: 1; Damage: 3D10+6; AP: 9}{15m}{1 (medium fuel cell)}{1}{Blast (1)}{cr 15}{0.8 kg}
\rangedweapon{Exploding Fuel Cell, Large}{A large fuel cell, usually built very sturdily, may be rigged as an improvised explosive device. Unstable cells however may explode when handled too roughly in high temperatures.}{Shots: 1; Damage: 4D10+8; AP: 12}{15m}{1 (large fuel cell)}{1}{Blast (2)}{cr 21}{1.5 kg}
\end{multicols}

\subsection{Mods}
\subsubsection{Constraints}
Ranged weapons have a limited number of mod slots which can each only be used once - at some point the weapon becomes unwieldy or the space is simply used up. When in question, apply common sense.\par\vspace{10mm}

\begin{multicols}{2}
Pistols:
\vspace{-8mm}
\begin{enumerate}
	\setlength\itemsep{-8mm}
	\item Upper Rail
	\item Lower Rail
	\item Barrel
	\item Muzzle
	\item Core
\end{enumerate}

Rifle:
\vspace{-8mm}
\begin{enumerate}
	\setlength\itemsep{-8mm}
	\item Upper Rail
	\item Lower Rail
	\item Two Side Rails
	\item Barrel
	\item Muzzle
	\item Core
\end{enumerate}

Heavy Weapon / Launcher:
\vspace{-8mm}
\begin{enumerate}
	\setlength\itemsep{-8mm}
	\item Upper Rail
	\item Lower Rail
	\item Lower Rail (restricted to bipod / tripod)
	\item Two Side Rails
	\item Barrel
	\item Muzzle
	\item Core
\end{enumerate}

Bow:
\vspace{-8mm}
\begin{enumerate}
	\setlength\itemsep{-8mm}
	\item Upper Rail
	\item Core
\end{enumerate}
\end{multicols}

\subsubsection{Optics}
\weaponmod{Red Dot Sight}{A simple 1x scope with good eye relief, barely obstructing peripheral vision.}{Upper Rail}{cr 5}{0.15 kg}
\weaponmod{Holographic Sight}{A non-magnifying 1x sight that is only projected when the character makes an Aim action. Only takes up minimal space on the rail it is mounted, allowing for another item to be mounted without taking this sight off.}{Upper Rail}{cr 7}{0.2 kg}
\weaponmod{Flip Scope}{A 3x magnifier to be used in conjunction with a non-magnifying sight.}{Upper Rail}{cr 7}{0.32 kg}
\weaponmod{AOG (Advanced Optical Gun sight)}{A common, analog 4x combat scope.}{Upper Rail}{cr 12}{0.35 kg}
\weaponmod{TEOG (Tactical Electronic Optical Gun sight)}{A common variable 4-6x combat scope. Due to using electronic magnification it is vulnerable to EMP.}{Upper Rail}{cr 32}{0.38 kg}
\weaponmod{LoRa Sight}{An 8x scope - the cheapest long range sight but usually absolutely sufficient.}{Upper Rail}{cr 16}{0.4 kg}
\weaponmod{LS-Marksman Scope}{A massive 14x scope used for the longest ranges.}{Upper Rail}{cr 28}{0.42 kg}
\weaponmod{OMS (Oculus Marksman System)}{A variable 8-14x electronic scope. Like other electronic magnifiers it is vulnerable to EMP.}{Upper Rail}{cr 51}{0.43 kg}
\weaponmod{TS-CQC Gun sight}{A big non-magnifying thermal imaging sight with a sleek profile created for CQC.}{Upper Rail}{cr 42}{0.52 kg}
\weaponmod{TS-BR Gun sight}{A 5x battle rifle scope with thermal imaging capabilities.}{Upper Rail}{cr 68}{0.67 kg}
\weaponmod{TS-M Gun sight}{A 10x marksman scope with thermal imaging capabilities.}{Upper Rail}{cr 78}{0.67 kg}
\weaponmod{Canted Irons}{A 45-degree offset rail mount is added to the top of the weapon for the use of a secondary Iron sight/Optic/Light.}{-}{cr 8}{0.1 kg}

\subsubsection{Lights}
\weaponmod{Laser Module}{Gives a +5 bonus to hit when not using aim actions. A target spotting the dot with a -10 Perception test cannot be surprised.}{Any rail}{cr 14}{0.1 kg}
\weaponmod{Infrared Laser Module}{Works like a laser module but it can only be seen with thermal imaging.}{Any rail}{cr 28}{0.2 kg}
\weaponmod{Flashlight}{A flashlight is attached to the weapon, able to light wherever the wielder is aiming.}{Any rail}{cr 6}{0.26 kg}

\subsubsection{Muzzles}
\weaponmod{Muzzle Brake}{Reduces recoil, thereby halving penalties from Burst actions.}{Muzzle}{cr 20}{0.2 kg}
\weaponmod{Flash Suppressor}{Reduces muzzle flash, inferring a -30 penalty to visual perception checks to spot the weapon firing.}{Muzzle}{cr 22}{0.2 kg}
\weaponmod{Sound Suppressor}{Reduces both noise and muzzle flash, granting a -30 penalty to all visual perception checks to spot the weapon firing and -10 to hearing based tests; can be combined with cold load ammunition}{Muzzle}{cr 28}{0.2 kg}
\weaponmod{Shotgun Choke, wide}{Reduces pellet spread on shotguns somewhat. Scatter rule only gives a hit bonus from normal range onward. Range is increased by 50\%.}{Muzzle}{cr 12}{0.2 kg}
\weaponmod{Shotgun Choke, narrow}{Reduces pellet spread on shotguns a lot. Scatter rule only gives a hit bonus from long range onward. Range is increased by 100\%.}{Muzzle}{cr 12}{0.2 kg}

\subsubsection{Underslung Weapons}
\weaponmod{Skeleton Key}{Adds an underslung MS shotgun with half range. Due to increased weight at the front this infers a -5 to hit when using aim actions without a rest.}{Lower Rail}{cr 28}{2.8 kg}
\weaponmod{Underslung CQC Grenade Launcher "Thumper"}{Adds an underslung "Thumper" CQC Grenade Launcher with half range. Due to increased weight at the front this infers a -5 to hit when using aim actions without a rest.}{Lower Rail}{cr 70}{3.2 kg}

\subsubsection{Braces / Grips}
\weaponmod{Angled Grip}{A fast-pull grip that grants +2 initiative when using both hands and no rest.}{Lower Rail}{cr 8}{0.1 kg}
\weaponmod{Pistol Grip}{A stable grip that increases aim bonus by 5 when using both hands and no rest.}{Lower Rail}{cr 8}{0.1 kg}
\weaponmod{Bipod/Tripod}{A stable rest that increases aim bonus by 10 when deployed. A bipod is short and to be used over cover or while prone. Tripods are large and to be used while standing.}{Lower Rail}{cr 10}{0.15 kg}

\subsubsection{Barrel}
\weaponmod{Extended Barrel}{A longer barrel that increases optimum range and armor penetration by 25\%. Such a long weapon may be difficult to use in confined spaces.}{Barrel}{cr 23}{0.2 kg}
\weaponmod{Heavy Barrel}{A heavy, more stable barrel. It halves penalties from shooting while on the move but decreases aim bonus by 10 when not using a rest.}{Barrel}{cr 11}{0.2 kg}

\subsubsection{Ammunition}
\weaponmod{Magazine}{A simple magazine for a ballistic weapon. It holds ammunition in the gun.}{Ballistic Magazine}{cr 12}{-}
\weaponmod{Extended-Mag/Belt}{A larger magazine that doubles mag size but is more prone to error, causing jams on 96 and above.}{Ballistic Magazine}{cr 26}{-}
\weaponmod{Drum-Mag}{A massive spiral magazine that triples mag size but jams on 96+ and also doesn't fit in normal mag pouches, therefore need to be carried in backpacks.}{Ballistic Magazine}{cr 34}{-}
\weaponmod{Dual-Mag}{Taped magazines make swapping from one to the other take half as long, however it doesn't fit in mag pouches anymore. Can easily be improvised with 2 magazines and some duct tape.}{Ballistic Magazine}{-}{-}
\weaponmod{Shell Holder}{A clip attached to a rail holding up to 8 shotgun shells. Very useful for special ammunition.}{Any Rail}{cr 2}{-}
\weaponmod{Bolt Holder}{A container holding up to 8 bolts attached below a crossbow.}{Core}{cr 3}{-}

\subsubsection{Misc}
\weaponmod{Camouflage Paint}{If the pattern matches the environment, grants a -20 penalty to visually perceive the weapon. However if the pattern doesn't match, it grants a +5 to visually perceive the weapon instead.}{Paint}{cr 3}{0 kg}
\weaponmod{Ghillie Cover}{If the pattern matches the environment, grants a -30 penalty to visually perceive the weapon. However if the pattern doesn't match, it grants a +10 to visually perceive the weapon instead.\\
	Unlike with camouflage paint patterns only exist for desert and forest.}{Paint}{cr 4}{0.15 kg}
	
\subsubsection{Core}
\weaponmod{Offload Cycler}{A technological marvel increasing a weapon's fire rate by 50\%, rounded down, if it already has a fire rate of 2 or more.}{Core}{80\%}{0.9 kg}
\weaponmod{Multi Barrel}{Adds an additional barrel feeding from the same magazine to double the amount of hits and ammo spent at an additional -10 penalty to hit. Only works if the weapon has a capacity of at least 2.}{Core}{60\%}{0.55 kg}
\weaponmod{Heatsink}{A heat sink removing the drawbacks from the Adjustable rule. It increases the weapon's weight by 20\%.}{Core}{40\%}{20\%}
\weaponmod{Backblast Refunneling}{It halves the Backblast range to increase the weapon's range by 50\%. The additional rumble gives a penalty equal to half of the user's strength difference to 70.}{Core}{60\%}{0.4 kg}
\weaponmod{Gim-barrel}{A self-stabilizing barrel halves penalties from shooting while moving as well as Burst actions. It doubles the weapon's weight and decreases aim bonus by 5 when not using a rest.}{Core, Barrel}{80\%}{100\%}
\weaponmod{Biocoding}{A trigger lock reading biometric data of the user. The weapon can only be used when wielded by a user who was registered during building of the weapon. It cannot be removed or reprogrammed without destroying the weapon.}{none}{cr 20}{0.24 kg}
\weaponmod{Vehicle Lock-on}{Adds vehicle lock-on to a Launcher or Heavy weapon; grants +30 to hitting vehicles and powered armor when taking at least one round to aim. Flares don't disrupt targeting but static smoke will.}{Core}{80\%}{3.4 kg}
\weaponmod{Heat Seeker}{Adds heat-based targeting assistance systems, thereby granting +15 to shoot targets warmer than their surroundings but -15 to shoot targets colder than their surroundings.}{Core}{cr 280}{2.3 kg}
\weaponmod{Energy Shield}{Adds a directional PES that covers the user's head and torso. Covers the whole user instead if used with a tripod.\\
	Armor: 15 / Threshold: 4}{Core}{cr 180}{3.8 kg}
\weaponmod{Smart Gun}{Sophisticated aiming assistance takes care of all the hard work. The gun always uses a target value of 65 when taking shots, regardless of modifiers or the shooter. It still requires to be pointed in roughly the right direction though and will completely fail under EMP.}{Core}{cr 190}{6.9kg}

\subsection{Ammunition}
In the following section ammo types are given in this format:\\
First, in bold, there is the type designation. This is little more than an identifier.\\
Secondly after the colon you find the effect. These are all changes this type of ammunition has as opposed to the basic weapon profile.\\
The price is the price per box off ammunition; every box contains an amount of ammunition equal to the value given at "Unit of sale". For ballistic ammo a single unit is one bullet, for others it is one full load of the weapon.\\
Weight and bulk are given in the weight that one "standard magazine" - as used in storage capacities - weighs.
%	Pistols
\subsubsection{7.2mm}
\ammo{FMJ}{-}{cr 2}{30}{4g}{40}
\ammo{AP}{-1 dmg, +4 AP}{cr 3}{30}{4g}{40}
\ammo{Hot Load}{+1 dmg, +4 AP, +15 on hearing checks to identify the shooter's position}{cr 4}{30}{4g}{40}
\ammo{Hot AP}{+10 AP, +15 on hearing checks to identify the shooter's position}{cr 6}{30}{4g}{40}
\ammo{JSP}{+2 dmg, -3 AP}{cr 3}{30}{4g}{40}
\ammo{JHP}{+4 dmg, -6 AP}{cr 4}{30}{4g}{40}
\ammo{Shock}{-4 dmg, -5 AP, Stun (4), EMP}{cr 6}{30}{4g}{40}
\ammo{Cold Load}{-3 dmg, -3 AP, -15 on hearing checks to identify the shooter's position}{cr 3}{30}{4g}{40}
\ammo{Cold AP}{-4 dmg, -15 on hearing checks to identify the shooter's position}{cr 4}{30}{4g}{40}
\ammo{Cold JSP}{-6 AP, -15 on hearing checks to identify the shooter's position}{cr 4}{30}{4g}{40}
\ammo{Cold JHP}{+2 dmg, -8 AP, -15 on hearing checks to identify the shooter's position}{cr 6}{30}{4g}{40}
\ammo{Incendiary}{-2 AP, will ignite flammable materials hit}{cr 3}{30}{4g}{40}
\ammo{Tracer}{grants a +10 bonus to hit for others aiming at the same target during this turn}{cr 4}{30}{4g}{40}
\ammo{Rubber Rounds}{-10 dmg, 0 AP, Stun (2), every hit against the target's head reduces it's initiative by D5-1}{cr 2}{30}{4g}{40}
\subsubsection{11.8x45mm}
\ammo{FMJ}{-}{cr 3}{20}{22g}{40}
\ammo{AP}{-1 dmg, +4 AP}{cr 5}{20}{22g}{40}
\ammo{Hot Load}{+1 dmg, +4 AP, +15 on hearing checks to identify the shooter's position}{cr 6}{20}{22g}{40}
\ammo{Hot AP}{+10 AP, +15 on hearing checks to identify the shooter's position}{cr 9}{20}{22g}{40}
\ammo{JSP}{+2 dmg, -3 AP}{cr 5}{20}{22g}{40}
\ammo{JHP}{+4 dmg, -6 AP}{cr 6}{20}{22g}{40}
\ammo{Shock}{-4 dmg, -5 AP, Stun (4), EMP}{cr 9}{20}{22g}{40}
\ammo{Cold Load}{-3 dmg, -3 AP, -15 on hearing checks to identify the shooter's position}{cr 5}{20}{22g}{40}
\ammo{Cold AP}{-4 dmg, -15 on hearing checks to identify the shooter's position}{cr 6}{20}{22g}{40}
\ammo{Cold JSP}{-6 AP, -15 on hearing checks to identify the shooter's position}{cr 6}{20}{22g}{40}
\ammo{Cold JHP}{+2 dmg, -8 AP, -15 on hearing checks to identify the shooter's position}{cr 9}{20}{22g}{40}
\ammo{Incendiary}{-2 AP, will ignite flammable materials hit}{cr 5}{20}{22g}{40}
\ammo{Tracer}{grants a +10 bonus to hit for others aiming at the same target during this turn}{cr 6}{20}{22g}{40}
\ammo{Rubber Rounds}{-10 dmg, 0 AP, Stun (2), every hit against the target's head reduces it's initiative by D5-1}{cr 3}{20}{22g}{40}

%	Rifles
\subsubsection{6.3mm}
\ammo{FMJ}{-}{cr 2}{30}{8.6g}{30}
\ammo{AP}{-1 dmg, +4 AP}{cr 3}{30}{8.6g}{30}
\ammo{Hot Load}{+1 dmg, +4 AP, +15 on hearing checks to identify the shooter's position}{cr 4}{30}{8.6g}{30}
\ammo{Hot AP}{+10 AP, +15 on hearing checks to identify the shooter's position}{cr 6}{30}{8.6g}{30}
\ammo{JSP}{+2 dmg, -3 AP}{cr 3}{30}{8.6g}{30}
\ammo{JHP}{+4 dmg, -6 AP}{cr 4}{30}{8.6g}{30}
\ammo{Shield Breaker}{-3 dmg, Disrupt}{cr 5}{30}{8.6g}{30}
\ammo{Shock}{-4 dmg, -5 AP, Stun (4), EMP}{cr 6}{30}{8.6g}{30}
\ammo{Cold Load}{-3 dmg, -3 AP, -15 on hearing checks to identify the shooter's position}{cr 3}{30}{8.6g}{30}
\ammo{Cold AP}{-4 dmg, -15 on hearing checks to identify the shooter's position}{cr 4}{30}{8.6g}{30}
\ammo{Cold JSP}{-6 AP, -15 on hearing checks to identify the shooter's position}{cr 4}{30}{8.6g}{30}
\ammo{Cold JHP}{+2 dmg, -8 AP, -15 on hearing checks to identify the shooter's position}{cr 6}{30}{8.6g}{30}
\ammo{Incendiary}{-2 AP, will ignite flammable materials hit}{cr 3}{30}{8.6g}{30}
\ammo{Tracer}{grants a +10 bonus to hit for others aiming at the same target during this turn}{cr 4}{30}{8.6g}{30}
\ammo{Rubber Rounds}{-10 dmg, 0 AP, Stun (2), every hit against the target's head reduces it's initiative by D5-1}{cr 2}{30}{8.6g}{30}
\subsubsection{11.8x112mm}
\ammo{FMJ}{-}{cr 4}{20}{50g}{5}
\ammo{AP}{-1 dmg, +4 AP}{cr 6}{20}{50g}{5}
\ammo{Hot Load}{+1 dmg, +4 AP, +15 on hearing checks to identify the shooter's position}{cr 8}{20}{50g}{5}
\ammo{Hot AP}{+10 AP, +15 on hearing checks to identify the shooter's position}{cr 12}{20}{50g}{5}
\ammo{JSP}{+2 dmg, -3 AP}{cr 6}{20}{50g}{5}
\ammo{JHP}{+4 dmg, -6 AP}{cr 8}{20}{50g}{5}
\ammo{Shield Breaker}{-3 dmg, Disrupt}{cr 8}{20}{50g}{5}
\ammo{Shock}{-4 dmg, -5 AP, Stun (4), EMP}{cr 12}{20}{50g}{5}
\ammo{Cold Load}{-3 dmg, -3 AP, -15 on hearing checks to identify the shooter's position}{cr 6}{20}{50g}{5}
\ammo{Cold AP}{-4 dmg, -15 on hearing checks to identify the shooter's position}{cr 8}{20}{50g}{5}
\ammo{Cold JSP}{-6 AP, -15 on hearing checks to identify the shooter's position}{cr 8}{20}{50g}{5}
\ammo{Cold JHP}{+2 dmg, -8 AP, -15 on hearing checks to identify the shooter's position}{cr 12}{20}{50g}{5}
\ammo{Incendiary}{-2 AP, will ignite flammable materials hit}{cr 6}{20}{50g}{5}
\ammo{Tracer}{grants a +10 bonus to hit for others aiming at the same target during this turn}{cr 8}{20}{50g}{5}
\ammo{Rubber Rounds}{-10 dmg, 0 AP, Stun (2), every hit against the target's head reduces it's initiative by D5-1}{cr 4}{20}{50g}{5}
\subsubsection{15x132mm}
\ammo{FMJ}{-}{cr 5}{5}{85g}{4}
\ammo{AP}{-1 dmg, +4 AP}{cr 8}{5}{85g}{4}
\ammo{Hot Load}{+1 dmg, +4 AP, +15 on hearing checks to identify the shooter's position}{cr 10}{5}{85g}{4}
\ammo{Hot AP}{+10 AP, +15 on hearing checks to identify the shooter's position}{cr 15}{5}{85g}{4}
\ammo{JSP}{+2 dmg, -3 AP}{cr 8}{5}{85g}{4}
\ammo{JHP}{+4 dmg, -6 AP}{cr 10}{5}{85g}{4}
\ammo{Shock}{-4 dmg, -5 AP, Stun (4), EMP}{cr 15}{5}{85g}{4}
\ammo{Cold Load}{-3 dmg, -3 AP, -15 on hearing checks to identify the shooter's position}{cr 8}{5}{85g}{4}
\ammo{Cold AP}{-4 dmg, -15 on hearing checks to identify the shooter's position}{cr 10}{5}{85g}{4}
\ammo{Cold JSP}{-6 AP, -15 on hearing checks to identify the shooter's position}{cr 10}{5}{85g}{4}
\ammo{Cold JHP}{+2 dmg, -8 AP, -15 on hearing checks to identify the shooter's position}{cr 15}{5}{85g}{4}
\ammo{Incendiary}{-2 AP, will ignite flammable materials hit}{cr 8}{5}{85g}{4}
\ammo{Tracer}{grants a +10 bonus to hit for others aiming at the same target during this turn}{cr 10}{5}{85g}{4}
\ammo{Rubber Rounds}{-10 dmg, 0 AP, Stun (2), every hit against the target's head reduces it's initiative by D5-1}{cr 5}{5}{85g}{4}

%	Shotgun
\subsubsection{12gauge}
\ammo{Buckshot}{-}{cr 3}{24}{42g}{8}
\ammo{Birdshot}{halves damage, ignores all negative size modifiers}{cr 2}{24}{42g}{8}
\ammo{Incendiary}{-2 dmg, -4 AP, ignites flammable materials hit}{cr 3}{24}{42g}{8}
\ammo{Breaching Round}{+3 dmg, +4 AP, halved range}{cr 5}{24}{42g}{8}
\ammo{Slug}{+2 AP, removes Scatter, doubles range}{cr 3}{24}{42g}{8}
\ammo{Fragmentation Slug}{removes Scatter but adds Blast (2) rule, doubles range}{cr 6}{24}{42g}{8}
\ammo{Incendiary Slug}{+1 AP, removes Scatter, doubles range, ignites flammable objects hit}{cr 4}{24}{42g}{8}
\ammo{EMP Slug}{-2 dmg, EMP, removes Scatter, doubles range}{cr 9}{24}{42g}{8}
\ammo{Rubber Slug}{-2 dmg, -6 AP, removes Scatter, doubles range, Stun (2)}{cr 3}{24}{42g}{8}

%	Fuel Tank
\subsubsection{Small Fuel Tanks}
\ammo{Napalm}{ignites flammable surfaces and sticks even to non-flammable ones}{cr 32}{1}{10kg}{}
\ammo{White Phosphorus}{halves range, burns for a D5 minutes, ignites flammable surfaces and sticks even to non-flammable ones, cannot be extinguished by water}{cr 64}{1}{20kg}{}
\ammo{Water}{1 dmg, 0 AP, Trauma, Stun (*the original amount of damage dice)}{cr 8}{1}{13kg}{}
\ammo{Liquid Helium}{halves range, causes D5 exhaustion per hit}{cr 64}{1}{13kg}{}

\subsubsection{Medium Fuel Tanks}
\ammo{Napalm}{ignites flammable surfaces}{cr 61}{1}{20kg}{}
\ammo{White Phosphorus}{halves range, burns for a D5 minutes, ignites flammable surfaces and sticks even to non-flammable ones}{cr 122}{1}{40kg}{}
\ammo{Water}{1 dmg, 0 AP, Trauma, Stun (*the original amount of damage dice)}{cr 17}{1}{25kg}{}
\ammo{Liquid Helium}{halves range, causes D5 exhaustion per hit}{cr 122}{1}{25kg}{}

\subsubsection{Large Fuel Tanks}
\ammo{Napalm}{ignites flammable surfaces}{cr 150}{1}{50kg}{}
\ammo{White Phosphorus}{halves range, burns for a D5 minutes, ignites flammable surfaces and sticks even to non-flammable ones}{cr 300}{1}{100kg}{}
\ammo{Water}{1 dmg, 0 AP, Trauma, Stun (*the original amount of damage dice)}{cr 32}{1}{62.5kg}{}
\ammo{Liquid Helium}{halves range, causes D10 exhaustion per hit}{cr 300}{1}{62.5kg}{}

%	Fuel Cells
\subsubsection{Small Fuel Cells}
\ammo{Standard Issue}{-}{cr 9}{1}{250g}{2}
\ammo{Unstable}{On a hit roll above 95 causes the weapon's damage with Blast (1) around the shooter.}{cr 5}{1}{250g}{2}
\subsubsection{Medium Fuel Cells}
\ammo{Standard Issue}{-}{cr 15}{1}{750g}{}
\ammo{Unstable}{On a hit roll above 95 causes the weapon's damage with Blast (1) around the shooter.}{cr 8}{1}{750g}{}
\subsubsection{Large Fuel Cells}
\ammo{Standard Issue}{-}{cr 21}{1}{1.5kg}{0.5}
\ammo{Unstable}{On a hit roll above 95 causes the weapon's damage with Blast (1) around the shooter.}{cr 11}{1}{1.5kg}{0.5}

%	Mass Driver Blocks
\subsubsection{Small Shard Blocks}
\ammo{Standard Issue}{-}{cr 2}{1}{300g}{1}
\ammo{Crude}{-50\% mag size, jams on 91 and above}{cr 1}{2}{300g}{1}
\subsubsection{Large Shard Blocks}
\ammo{Standard Issue}{-}{cr 5}{1}{1.8kg}{0.5}
\ammo{Crude}{-50\% mag size, jams on 91 and above}{cr 2}{2}{1.8kg}{0.5}
\ammo{Osmium Alloy}{+4 damage, -2 AP}{cr 10}{1}{3.6kg}{0.5}
\subsubsection{Vehicle Shard Blocks}
\ammo{Standard Issue}{-}{cr 11}{1}{3.2kg}{0.2}
\ammo{Crude}{-50\% mag size, jams on 91 and above}{cr 5}{2}{3.2kg}{0.2}
\ammo{Osmium Alloy}{+4 damage, -2 AP}{cr 21}{2}{6.2kg}{0.5}


%	Grenades
\subsubsection{20mm grenades}
\ammo{Fragmentation}{2D10+2 | 2 AP | Blast (6)}{cr 60}{1}{120g}{0.5}
\ammo{High Explosive}{2D10+6 | 5 AP | Blast (2)}{cr 60}{1}{120g}{0.5}
\ammo{Thermobaric}{3D10 | 2 AP | Blast (10) | heat-based, ignores all cover and non-vacuum-sealed or natural armor}{cr 96}{1}{120g}{0.5}
\ammo{Incendiary}{D10+2 | 0 AP | Blast (8) | ignites all flammable objects in the radius}{cr 60}{1}{120g}{0.5}
\ammo{Buckshot Canister}{2D10+3 | 5 AP | Scatter}{cr 75}{1}{120g}{0.5}
\ammo{Slug}{2D5+8 | 8 AP}{cr 75}{1}{120g}{0.5}
\ammo{Smoke}{0 | 0 AP | Blast (8), Smoke}{cr 45}{1}{120g}{0.5}
\ammo{Flashbang}{0 | 0 AP | Blast (12), Flash}{cr 45}{1}{120g}{0.5}
\ammo{EMP}{0 | 0 AP | Blast (6), EMP}{cr 80}{1}{120g}{0.5}
\ammo{Tear Gas}{0 | 0 AP | Blast (6), Smoke, Stun (0), ignores cover and non-vacuum-sealed armor, Stun is applied every round in the Smoke}{cr 45}{1}{120g}{0.5}
\ammo{Rubber Slug}{1D10+3 | 2 AP | Stun (5), hits to the head cause 2D5 initiative loss}{cr 50}{1}{120g}{0.5}
\ammo{Riot Foam}{0 | 0 AP | Blast (12), Sticky}{cr 20}{1}{120g}{0.5}
\ammo{Signal Smoke}{0 | 0 AP | Blast (4) | a smoke spewer to signal allies; comes in various colors}{cr 38}{1}{120g}{0.5}
\ammo{Parachute Flare}{D5 | 0 AP | a signal flare that burns for 3 minutes and falls at 1 meter per 3 seconds (or 1 round)}{cr 40}{1}{120g}{0.5}
\ammo{Infrared Illuminator}{1 | 0 AP | a large chemlight that emits infrared light in a very large area}{cr 42}{1}{120g}{0.5}
\subsubsection{40mm grenades}
\ammo{Fragmentation}{2D10+7 | 2 AP | Blast (9)}{cr 78}{1}{230g}{1}
\ammo{High Explosive}{3D10+6 | 5 AP | Blast (3)}{cr 78}{1}{230g}{1}
\ammo{Thermobaric}{4D10 | 2 AP | Blast (15) | heat-based, ignores all cover and non-vacuum-sealed armor}{cr 125}{1}{230g}{1}
\ammo{Incendiary}{D10+3 | 0 AP | Blast (12) | ignites all flammable objects in the radius}{cr 78}{1}{230g}{1}
\ammo{Buckshot Canister}{3D10+3 | 5 AP | Scatter}{cr 98}{1}{230g}{1}
\ammo{Slug}{2D10+9 | 9 AP}{cr 98}{1}{230g}{1}
\ammo{Smoke}{0 | 0 AP | Blast (12), Smoke}{cr 59}{1}{230g}{1}
\ammo{Flashbang}{0 | 0 AP | Blast (18), Flash}{cr 59}{1}{230g}{1}
\ammo{EMP}{0 | 0 AP | Blast (9), EMP}{cr 104}{1}{230g}{1}
\ammo{Tear Gas}{0 | 0 AP | Blast (8), Smoke, Stun (0), ignores cover and non-vacuum-sealed armor, Stun is applied every round in the Smoke}{cr 59}{1}{230g}{1}
\ammo{Rubber Slug}{2D10+3 | 2 AP | Stun (7), hits to the head cause 2D10 initiative loss}{cr 65}{1}{230g}{1}
\ammo{Signal Smoke}{0 | 0 AP | Blast (4) | a smoke spewer to signal allies that lasts for a few minutes; comes in various colors}{cr 49}{1}{230g}{1}
\ammo{Parachute Flare}{D5+1 | 0 AP | a signal flare that burns for 5 minutes and falls at 1 meter per 3 seconds (or 1 round)}{cr 52}{1}{230g}{1}
\ammo{Thermite}{6D10+10 | 25 AP | Blast (3) | takes two rounds to get going, then burns for one minute dealing its damage every round; sticks to rough surfaces}{cr 156}{1}{230g}{1}

%	Rockets
\subsubsection{Micro Missiles}
\ammo{Fragmentation}{-}{cr 84}{4}{1kg}{0.5}
\ammo{High Explosive}{-1D10 dmg | +8 AP}{cr 88}{4}{1kg}{0.5}
\ammo{Incendiary}{-3 dmg | -3 AP | ignites flammable objects}{cr 76}{4}{1kg}{0.5}
\subsubsection{Small Rockets}
\ammo{HEAT}{4D10+5 dmg | 6 AP | Blast (5)}{cr 124}{1}{2kg}{0.5}
\ammo{High Explosive}{4D10+2 dmg | 3 AP | Blast (6)}{cr 99}{1}{2kg}{0.5}
\ammo{Fragmentation}{3D10+8 dmg | 2 AP | Blast (8)}{cr 99}{1}{2kg}{0.5}
\ammo{Incendiary}{2D10+5 dmg | 2 AP | Blast (6) | ignites flammable objects}{cr 112}{1}{2kg}{0.5}
\ammo{Incendiary Fragmentation}{3D10+5 dmg | 2 AP | Blast (8) | ignites flammable objects}{cr 134}{1}{2kg}{0.5}
\ammo{Thermobaric}{4D10+8 dmg | 3 AP | Blast (12) | heat-based, ignores all cover and non-vacuum-sealed armor}{cr 201}{1}{2kg}{0.5}
\ammo{Thermite}{6D10+10 | 25 AP | Blast (4) | takes two rounds to get going, then burns for one minute dealing its damage every round; sticks to rough surfaces}{cr 179}{1}{2kg}{0.5}
\ammo{Kinetic Penetrator}{4D10+5 dmg | 14 AP}{cr 201}{1}{2kg}{0.5}
\subsubsection{Large Rockets}
\ammo{HEAT}{4D10+6 dmg | 8 AP | Blast (6)}{cr 168}{1}{3.5kg}{0.25}
\ammo{High Explosive}{4D10+4 dmg | 5 AP | Blast (8)}{cr 134}{1}{3.5kg}{0.25}
\ammo{Fragmentation}{3D10+11 dmg | 3 AP | Blast (10)}{cr 134}{1}{3.5kg}{0.25}
\ammo{Incendiary}{2D10+8 dmg | 2 AP | Blast (8) | ignites flammable objects}{cr 151}{1}{3.5kg}{0.25}
\ammo{Incendiary Fragmentation}{3D10+8 dmg | 3 AP | Blast (10) | ignites flammable objects}{cr 181}{1}{3.5kg}{0.25}
\ammo{Thermobaric}{5D10+10 dmg | 5 AP | Blast (18) | heat-based, ignores all cover and non-vacuum-sealed armor}{cr 272}{1}{3.5kg}{0.25}
\ammo{Thermite}{6D10+10 | 25 AP | Blast (6) | takes two rounds to get going, then burns for one minute dealing its damage every round; sticks to rough surfaces}{cr 242}{1}{3.5kg}{0.25}
\ammo{Kinetic Penetrator}{4D10+8 dmg | 18 AP}{cr 272}{1}{3.5kg}{0.25}

%	Arrows / Bolts
\subsubsection{Arrows}
\ammo{Broadhead}{-}{cr 1}{25}{30g}{10}
\ammo{AP}{-1 dmg, +3 AP}{cr 1}{25}{30g}{10}
\ammo{Shield Breaker}{-1 dmg, Disrupt}{cr 2}{25}{30g}{10}
\ammo{Syringer}{-3 dmg, +1 AP, contains an empty vial to be filled with poisons or other injections that are released into a biological target when hit}{cr 3}{25}{30g}{10}
\ammo{Grapple}{-4 dmg, 0 AP, half range; the mechanized grapple attaches to a rope and holds on most surfaces but e.g. very sturdy metals or fragile materials like glass}{cr 6}{1}{30g}{10}
\subsubsection{Bolts}
\ammo{Broadhead}{-}{cr 1}{25}{30g}{10}
\ammo{AP}{-1 dmg, +3 AP}{cr 1}{25}{30g}{10}
\ammo{Shield Breaker}{-1 dmg, Disrupt}{cr 2}{25}{30g}{10}
\ammo{Syringer}{-3 dmg, +1 AP, contains an empty vial to be filled with poisons or other injections that are released into a biological target when hit}{cr 3}{25}{30g}{10}
\ammo{Grapple}{-4 dmg, 0 AP, half range; the mechanized grapple attaches to a rope and holds on most surfaces but e.g. very sturdy metals or fragile materials like glass}{cr 6}{1}{30g}{10}

	\section{Melee actions}
\subsection*{Standard attack}
A single armed or unarmed melee attack. Attacking a specific location is modified by its size. This is compatible with any maneuver and does not require an ability. 
\subsection*{Sure strike (+20, also uses up reaction)}
A single melee attack with higher chances to hit. This maneuver \emph{cannot} be combined with any other and does not require an ability.
\subsection*{Attack of opportunity (-10, free action)}
When an enemy passes through or leaves the character’s control area, the character may make a single attack as a free action with a -10 modifier in addition to any other maneuver. This may not be any maneuver requiring more than one action.\\
One attack of opportunity may be made against any enemy once per combat round.
\subsection*{Disengage}
\label{action:disengage}
When disengaging, no attacks are allowed to be made that round but no attacks of opportunity are invoked when passing through other characters' area of control.
\subsection*{Feint (-X)}
A skillful attack that is harder to defend against. Defending against this maneuver is at a penalty equal to the voluntary penalty.
\subsection*{Called jab (-20 -opponent’s armor)}
A strike against an armor’s weak points. The attacker picks the location without additional penalty. The attack ignores half the target’s armor.
\subsection*{Disarm (-20 -X)}
An attack against the opponent’s grip to get control of his weapon. The defender makes a dexterity check at a penalty equal to the voluntary penalty of the maneuver. On a fail the defender loses her weapon, dropping on the ground. Failing by at least 3 degrees gives the attacker control over the weapon.\\
This maneuver does not deal damage.
\subsection*{Flurry (-10)}
Two attacks in quick succession using both hands. Defending against this attack requires two reactions. On a hit the attacker deals damage once with her primary weapon and once with the weapon in her offhand. If she is not carrying a weapon in her offhand, she deals unarmed damage instead. 
\subsection*{Piercing thrust (-40 -half of opponent’s armor, also uses up reaction)}
An all-out stab. The attacker picks the location without additional penalty. The attack ignores both half the target's armor and its full injury threshold.\\
Cannot be performed without the corresponding ability.
\subsection*{Take-down (-10 -X)}
A sweeping and pushing attack to take the opponent off his legs. If the attack hits, it deals no damage. Instead the defender makes an agility check at a penalty equal to the voluntary penalty on the maneuver. If that fails, he falls on the ground. If it fails by 3 or more degrees, he also drops his weapon.
\subsection*{Shift (-10 -X)}
Superior footwork or raw force moves an opponent. If the attack hits, the opponent makes a strength check at a penalty equal to the voluntary penalty on the maneuver or be moved by up to a meter per degree of failure in a direction chosen by the attacker.
\subsection*{Powerstrike (-X)}
A strong blow that increases the hit’s damage. For every 5 points of voluntary penalty the damage is increased by one.
\subsection*{Cleave (-15 per target, also uses up the reaction)}
A half circle of steel pushing multiple enemies back. This maneuver can target up to three enemies. A success indicates hits on all targets. A target hit will - in addition to normal effects - be pushed back by the attacker’s StrB in meters -1 per size category the target is bigger. This maneuver is always also a Knockdown.
\subsection*{Stunning blow (-20 -X)}
A blunt attack to stun a target. If the hit is successful and the attack would deal damage, then deal half of the effective damage and the target makes a Constitution check at a penalty equal to the voluntary penalty on the maneuver. By GM discretion the target may also fall prone.
\subsection*{Knockdown (-20 -X)}
A forceful attack to knock the target off their feet. If the attack hits, the target makes a Strength check at a penalty equal to the voluntary penalty on the maneuver. On a fail the target falls prone.
\subsection*{Charge (-20, also uses up reaction)}
An attack utilizing the momentum of movement. The attack gains a bonus to damage equal to the attacker’s current speed, usually his sprint speed. He needs a minimum of 4 meters of runup.
\subsection*{Crushing blow (-40 -X, also uses up the reaction)}
A powerful, all-out attack that deals greatly increased damage. If the attack is a success, the damage is first increased by one per full 5 points of voluntary penalty and then damage and AP are doubled. Cannot be performed without the corresponding ability.

\section{Melee reactions}
\subsection*{Dodge (-10; -0 against unarmed)}
Evasion without weapons touching. Evasion has to be declared before the attack is rolled. On a success the attack is evaded. This does not require an ability.
\subsection*{Parry}
A basic defense against a melee attack. Parrying has to be declared before the attack is rolled. On a success the attack is evaded. This does not require an ability.
\subsection*{Bind (-X)}
Enter a bind to lead one’s opponent’s weapon and gain an advantage. The next action the character takes has a bonus equal to the voluntary penalty.
\subsection*{Disarm from reaction (-30 -X)}
A very risky defense to rid the opponent of their weapon during her attack. If the defense is successful, the attacker makes a dexterity check at a penalty equal to the voluntary penalty of the maneuver. On a fail the attacker loses her weapon, dropping on the ground. Failing by at least 3 degrees gives the defender control over the weapon.
\subsection*{Intercept (-20, uses attack value)}
A reckless attack into the opponent's attack. If only one attack is successful, that attack deals damage as normal. If both succeed, the one that succeeded better (see Dice checks and stats-Opposed Checks) deals full damage, while the other one only deals half damage.\\
Intercepting a Charge is at an additional penalty equal to half the character’s courage below 80 but deals additional damage equal to the attacker’s movement speed.
\subsection*{Reversal (min -30)}
A skillful counterattack that exploits an overswing. This maneuver can only be used against any maneuver based on \emph{Powerstrike} and grants a penalty equal to the attacking powerstrike penalty, at least -30. If the defense is successful, the maneuver also counts as an immediate powerstrike against the original attacker with a voluntary penalty equal to this maneuver’s penalty. The attack may be parried as normal and may also be subject to Reversal. Cannot be performed without the corresponding ability.

\section{Grappling}
Grappling may be somewhat more complicated than striking due to the amount of options one has and how different these options are to striking: control over someone else and dislocation of opponents or their limbs.
\subsection*{Hold}
A hold is the basis of all grappling. By itself it only stops the opponent from leaving but it is necessary for most other maneuvers to initiate.\\
To initiate a hold, the character makes a grappling check which may be opposed by the target as a reaction. If the character wins, a hold is engaged.\\
To break free the target makes a grappling check which may be opposed by the character as a reaction. If the target wins, it breaks free.
\subsection*{Throw}
The character attempts to lift the target off the ground and forcefully put it back.\\
The character makes a grappling check which may be opposed by the target as a reaction. If the character wins, the target takes D10+StrB damage with the trauma quality and falls prone (therefore also losing D5 initiative).
\subsection*{Lock}
The character attempts to prevent his opponent from moving freely. Arm locks, leg locks and head locks are common but bear hugs also fall into this category.\\
To initiate a lock, the character makes a grappling attack roll which may be opposed by the target as a reaction.\\
While the lock is active, the character may use an action to interfere with any action taken by the target: the character takes a grappling check as an action and the target’s action is at a -10 penalty per degree of success.\\
To break free the target may take a -10 grappling check, which may be opposed by the character as a reaction. If the target wins, it breaks free, back into the initial hold.
\subsection*{Lever hold}
While holding an opponent in a lock of some kind, the character may choose to initiate a lever hold as an action. This forfeits the option to control any action in favor of dealing damage to a location every turn.\\
To initiate a lever hold, the character has to have his target in a lock already and takes a grappling check which may be opposed by the target as a reaction. If the character is successful, a location is determined and a lever hold is initiated.\\
While the lever hold is active, the character deals 1D5-2+StrB damage to the location, ignoring armor.\\
To break free the target may make a -20 grappling check, which may be opposed by the character as a reaction. If the target wins, it breaks free.
\subsection*{Choke hold}
Another common grappling target besides joints are the airways.\\
To initiate a chokehold the character makes a grappling check which may be opposed by the target as a reaction. If the character wins, the chokehold is initiated.\\
The chokehold works like a lever hold but instead of dealing damage, it causes suffocation.\\
To break free the target may make a -20 grappling check, which may be opposed by the character as a reaction. If the target wins, it breaks free.
\subsection*{Move}
Moving while grappling is hard, even more so if only one party intends to move.\\
The character makes a grappling check which may be opposed by the target as a reaction. If the character wins, the target is moved up to the difference of DoS in meters or half the amount if on the ground.

\section{Melee dancing}
When attacking or defending in melee combat, the combatants are rarely stationary. After an attack that does not include intentional movement the character and his opponent move a meter into a random direction within the same control areas. Roll a D10:
\par
\begin{tabular}{ll}
	1-2: & towards the opponent\\
	3:   & forward right\\
	4:   & right\\
	5:   & backwards right\\
	6-7: & away from the opponent\\
	8:   & backwards left\\
	9:   & left\\
	10:  & forward left
\end{tabular}

	\section{Armor}
\subsection{Introduction}
Armor is wearable cover - very convenient for anyone not looking to die.\\
Every part of the body can be covered once, meaning that armor pieces which overlap or even cover the same locations altogether cannot be combined.

\subsection{Casual wear}
This section contains clothing that is slightly armored but looks inconspicuous.\par
\begin{multicols}{2}
\armor{Biker helmet}{3}{0}{0}{0}{head}{2 mods; -5 to sight-based perception, +5 to Drive(Bike) checks}{cr 26}{1.5 kg}
\armor{Armored suit}{0}{4}{2}{3}{torso, arms, legs}{1 mod; inconspicuous}{cr 79}{2 kg}
\armor{Tough coat}{0}{2}{1}{1}{torso, arms}{0 mods}{cr 35}{2 kg}
\armor{Flak jacket}{0}{5}{5}{1}{torso, arms}{1 mod; inconspicuous}{cr 96}{2 kg}
\armor{Tough pants}{0}{0}{0}{2}{legs}{0 mods; inconspicuous}{cr 18}{1.5 kg}
\end{multicols}

\subsection{Helmets}
A section on protective head wear, armoring the arguably most vulnerable body location.\par
\begin{multicols}{2}
\armor{Blast helmet}{14}{0}{0}{0}{head}{1 mod; -10 to sight-based perception}{cr 123}{8 kg}
\armor{Improvised helmet}{5}{0}{0}{0}{head}{1 mod; -5 Ag, Improvised}{cr 44}{3 kg}
\armor{Banshee helmet}{6}{1}{0}{0}{head}{2 mods; inbuilt spotter target assist system}{cr 61}{2 kg}
\armor{Flak helmet}{8}{0}{0}{0}{head}{1 mod; -5 to sight-based perception}{cr 70}{1.5 kg}
\armor{Ghost helmet}{6}{0}{0}{0}{head}{2 mods; projects a holographic image onto the helmet, granting +10 to intimidation when active and visible to the opponent}{cr 53}{5 kg}
\armor{EVA helmet}{5}{0}{0}{0}{head}{3 mods; vac-sealed}{cr 44}{2 kg}
\armor{Argus helmet}{10}{0}{0}{0}{head}{2 mods; -10 to sight-based perception, inbuilt explosives detection assistant}{cr 88}{9 kg}
\end{multicols}

\subsection{Body Armor}
Armor pieces below are about protecting the largest location of the human body.\par
\begin{multicols}{2}
\armor{EOD armor}{3}{20}{18}{2}{torso, arms}{2 mods; -2 movement speed, -10 Ag, -10 to Dex and shooting tests}{cr 376}{36 kg}
\armor{Banshee suit}{0}{10}{7}{0}{torso, arms}{2 mods; inbuilt climbing gear}{cr 149}{4 kg}
\armor{Improvised vest}{0}{9}{0}{0}{torso}{1 mod; -5 Ag, Improvised}{cr 79}{3.5 kg}
\armor{Stab vest}{0}{9}{0}{0}{torso}{1 mod; -5 Ag, immune to piercing rule}{cr 79}{3.5 kg}
\armor{Flak vest}{0}{12}{0}{0}{torso}{1 mod; -5 Ag}{cr 105}{4 kg}
\armor{UA pauldrons}{0}{0}{10}{0}{arms}{2 mods; -5 Ag}{cr 88}{3.5 kg}
\armor{Ghost suit}{0}{10}{7}{0}{torso, arms}{3 mods; does not stop a skin-close holoprojector, includes piezoelectric exo-frame}{cr 149}{12 kg}
\armor{Locust armor}{0}{8}{0}{2}{torso}{2 mods; includes jump pack}{cr 172}{38 kg}
\armor{Node armor}{0}{14}{4}{0}{torso, arms}{3 mods; can carry AI\\
	includes piezoelectric exo-frame}{cr 158}{14 kg}
%\armor{Strike suit}{16}{24}{20}{20}{all}{? mods; has own chapter}{cr 1.200}{300 kg}
\armor{EVA suit}{0}{3}{3}{3}{torso, arms, legs}{4 mods; vac-sealed}{cr 79}{2.5 kg}
\armor{Shadow suit}{0}{3}{3}{3}{torso, arms, legs}{0 mods; the wearer gains +20 to hide and is at a -10 to hit in low-light and darkness}{cr 132}{2.8 kg}
\end{multicols}

\subsection{Leg Armor}
This section is the result of a fine balance act between all-important movement and protection.\par
\begin{multicols}{2}
\armor{Blast greaves}{0}{1}{0}{18}{legs}{1 mod; -1 movement speed, -10 Ag}{cr 166}{10 kg}
\armor{Banshee greaves}{0}{0}{0}{6}{legs}{2 mods}{cr 53}{2 kg}
\armor{Flak greaves}{0}{0}{0}{10}{legs}{2 mods; -5 Ag}{cr 88}{2.5 kg}
\armor{Ghost greaves}{0}{0}{0}{6}{legs}{1 mod; halves penalties from noise on sneaking}{cr 103}{6 kg}
\armor{Locust greaves}{0}{0}{0}{6}{legs}{2 mods; reduces fall damage by 1/4}{cr 128}{5.5 kg}
\end{multicols}

\subsection{Mods}
In the following section there are lists for all the modifications that can be added to armor pieces. The amount of mods that can be added to any one armor piece or combination is given at the piece of armor itself.
\par
\vspace{2mm}
\begin{multicols}{2}
\armormod{Ablative coating}{A heat dispersing coat, layered upon the armor, that increases armor against laser and other heat-based weaponry by 20\%.}{cr 40}{1}{N}{0.6 kg}
\armormod{Auto-injector}{On the press of a button, detection of a certain type of stress or a similar trigger, this mod introduces a prepared dosage of an injection-type chemical into the wearer's bloodstream. Can hold up to 3 cartridges.}{cr 24}{1}{N}{0.8 kg}
\armormod{Auto-loader}{A reload assistance system added to the outer thighs which halves reload times for pistols. When present on both legs, this will allow reloading both guns while dual wielding.}{cr 14}{1}{N}{1.2 kg}
\armormod{Booster Sheath}{A melee weapon can be draw from this sheath in a free action. When combined with Quick Draw a weapon draw from this sheath can be combined with one non-flurry attack as a free action. When the armor mod is also powered, this attack gains 4 damage.}{cr 24}{1}{N}{1.4 kg}
\armormod{Brace Assist}{The character may brace as a free action. Moving still stops bracing.}{cr 34}{2}{N}{2.6 kg}
\armormod{CBRN (Helmet)}{A hazard prevention system containing a CBRN unit, rebreather and redundancies for each subsystem.}{cr 112}{2}{N}{1.4 kg}
\armormod{Cigarette Lighter Adapter}{Adds an outlet to light cigarettes on.}{cr 3}{1}{Y}{0 kg}
\armormod{Deafen Suppressor}{Grants +20 to tests to resist deafening and disorientation. Grants -5 to auditory perception.}{cr 18}{1}{N}{0.2 kg}
\armormod{Discardable Plates}{Adds an outer layer that can be discarded easily. Useful when on fire or drenched in acid.}{cr 12}{1}{N}{1.4 kg}
\armormod{Energy Rotella}{Projects a small shield over the user's arm that can be used to cover one location in addition to the arm it's attached to. Has 12 armor and a threshold of 7.}{cr 50}{1}{Y}{1.1 kg}
\armormod{Faraday cage}{Reduces EMP duration by 8 rounds and grants +30 to resist electricity-related Stun rules.}{cr 62}{1}{N}{18.8 kg}
\armormod{Final hour chip}{Once the wearer's vital signs cease, the armor self-destructs violently, like a 40mm thermite grenade with a 6m Blast radius instead. Requires a vital monitor.}{cr 170}{0}{Y}{0.1 kg}
\armormod{Flare Launcher}{Turns Vehicle Lock-on bonuses into penalties. Has 3 flares and takes a minute to refill. }{cr 48}{3}{Y}{2.1 kg}
\armormod{Flash Visor}{Adds flash glass goggles to the helmet.}{cr 31}{1}{N}{0,6 kg}
\armormod{Flashlight}{Adds a flashlight and its related bonuses. Hitting the location the active flashlight is attached to is at no penalty for low light or location. }{cr 65}{1}{N}{1.1 kg}
\armormod{Ghillie}{Adds a Ghillie Cloak to the armor, offering a +20 Bonus to Camouflage when used in matching surroundings.}{cr 6}{0}{N}{1.5 kg}
\armormod{Grappling hook}{Adds a reusable grappling hook to the armor. Works like an integrated crossbow using grappling hook ammunition.\\
	This is a very niche upgrade, as the grappling hook requires power but can only hold up to 250 kg of weight.}{cr 55}{1}{Y}{5.9 kg}
\armormod{HUD}{Adds a heads up display to the helmet visor. Can be tied to display any device but has no input device included.}{cr 4}{1}{N}{0.8 kg}
\armormod{Inertia Dampeners}{A marvel of technology allowing rapid changes in velocity. Fall damage is reduced and the Trauma rule is ignored. Depending on the situation this may have other bonuses, e.g. in fast vehicles.}{cr 189}{3}{Y}{26 kg}
\armormod{Tacpad}{Adds a data pad the the armor's forearm.}{cr 22}{1}{N}{1.1 kg}
\armormod{Integrated Weapon}{Any one-handed melee or pistol type weapon can be affixed to an arm. Pistols now use Ranged Combat Training (Integrated Weaponry). One-handed melee weapons may use Melee Combat Training (Striking) or their normal skill depending on how they are installed.\\
	Removes the Inconspicuous rule.}{weapon + 10\%}{1}{N}{weapon +10\%}
\armormod{Intercom}{An inbuilt communication device using very strong encryption. Different intercoms need to be grouped at a computing device equipped for the task before being able to communicate among each other safely.}{cr 28}{1}{N}{0.2 kg}
\armormod{Jump pack}{A backpack like armor attachment containing boosters to increase jump height, leap length, allow some amount of control in the air or stop hitting the floor at a violently rapid pace. Requires Pilot (Jetpack) to steer.\\
	Height: increases the jump height by a factor of 3, reduced by 0.5 for every full 100 kg the wearer weighs, becoming useless at 400kg\\
	Control: can decrease fall damage by 4 dice, reduced by 1 for every full 100 kg the wearer weighs, becoming useless at 400kg\\
	Ground Slam: doubles fall damage but enables a Charge with a speed of 10 times meters fallen and Blast (1)}{cr 82}{2}{N}{32 kg}
\armormod{Medic Container}{Fits one pack of wound sealant foam or wound dressing, a field surgery or stitch kit, 5 doses of any medicine and a thermal blanket. Any equipment stored here can be drawn as a free action.}{cr 21}{1}{N}{1.2 kg}
\armormod{Medical HUD}{When a patient has an accessible vital monitor, this device gives much more detailed information, thereby improving medical tests by +20.}{cr 36}{2}{N}{0.8 kg}
\armormod{Nanomesh Undersuit}{The minimum armor the wearer can be reduced to by e.g. abilities is 8.}{cr 43}{2}{N}{2.3 kg}
\armormod{Night vision}{A night vision device included in the helmet.}{cr 14}{1}{N}{0.7 kg}
\armormod{Odysseus system}{A display that includes many forms of navigation systems, granting a +10 to driving and +20 to navigation tests.}{cr 28}{1}{N}{1.3 kg}
\armormod{Oversized Mag Holster}{Hooks to hold a taped or drum mag}{cr 16}{1}{N}{0.2 kg}
\armormod{Piezoelectric Exo-Frame}{carries the armor's weight\\
	When struck by an EMP, it grants a -20 penalty to all physical actions taken and deals D10+6 damage ignoring armor to all the wearer's locations every round.}{cr 190}{2}{Y}{18 kg}
\armormod{Power Pack}{Adds a generator to non-powered armor to accommodate armor mods requiring power.}{cr 45}{2}{N}{14 kg}
\armormod{Radar}{A small radar device detecting moving objects in the vicinity. Detecting someone sneaking around is at a +40 bonus. Has a range of 40m.}{cr 78}{2}{N}{11 kg}
\armormod{Reactive Shields}{Shield arms controlled by a small AI. Grants an additional block reaction at a target value of 45. If an attack is blocked with this shield, it triggers an explosion equal to a 20mm HE grenade directed outwards. The mod contains 6 shields, each can trigger once before having to be replaced for cr 15.}{cr 180}{3}{Y}{24 kg}
\armormod{Servo Exo-Frame}{A replacement for the more prevalent piezoelectric exo-frame. It is less potent, carrying only 3/4 of the armor's weight but won't harm the wearer in case of an EMP and instead just shut off.}{cr 8}{0}{Y}{18 kg}
\armormod{Spring Charge (arm)}{Increases Strength for quick tasks like punching or throwing by 20.}{cr 54}{1}{Y}{1.2 kg}
\armormod{Spring Charge (leg)}{Increases Strength for quick tasks like jumping or kicking by 20.}{cr 58}{1}{Y}{1.5 kg}
\armormod{Temperature regulators}{Tries to keep the armor's internals at around room temperature, preventing heat stroke and hypothermia in all but the harsh wastelands. }{cr 24}{1}{N}{0.8 kg}
\armormod{TROPHY Active Defense System}{Shoots projectiles out of the air that travel at least 5m/s, are between 2cm and 10cm across and within 10m of the wearer. Explosives are usually just destroyed while incendiaries still spill. Holds 6 cartridges of 11.8x45mm ammo and takes a minute to refill.}{cr 54}{4}{N}{2.1 kg}
\armormod{Up Armor}{Additional armor plates that increase the location's armor by 1.}{cr 22}{1}{N}{2 kg}
\armormod{Vacuum seal}{Seals the armor to prevent any gas leaking, like air flowing outside or hydrogen cyanide inside.}{cr 32}{0}{N}{1.1 kg}
\armormod{Vital monitor}{Tracks the wearer's vital data. A abbreviation can be made accessible over network.}{cr 8}{0}{N}{1.6 kg}
\end{multicols}

\section{PES}
Personal energy shields are widely available and protect the body without being cumbersome, large or particularly heavy. It behaves like a non-Newtonian fluid, so slow, low-energy attacks are not blocked.\par
\begin{multicols}{2}
\pes{Mk IIa PES Prototype}{8}{3}{cr 96}{1.9 kg}{}
\pes{Mk IIIa Retail-Ready PES}{15}{4}{cr 126}{1.2 kg}{}
\pes{Mk III Projected Armor}{24}{5}{cr 144}{2.6 kg}{}
\pes{Mk IV Heavy PES}{32}{5}{cr 156}{5.8 kg}{Cumbersome}
\pes{Projected Bunker Shielding}{64}{7}{cr 186}{350 kg}{incredibly large, 8m radius}
\pes{Rumble Energy Shielding}{12}{4}{cr 114}{1.4 kg}{triggers on melee attacks as well}
\pes{MagMA Heat Shield}{16}{4}{cr 132}{1.4 kg}{halves fire damage}
\pes{EOD Energy Shielding}{19}{4}{cr 138}{1.2 kg}{armor doubled against dispersed damage (all attacks with Scatter or Blast rules)}
\pes{Unstable PES}{24}{5}{cr 144}{0.7 kg}{shocks surroundings (excluding the wearer) when triggered: Blast (1), EMP, Stun (2)}
\pes{Energy Rotella}{13}{4}{cr 120}{1.1 kg}{}
\pes{Mk IVa Heavy Prototype}{30}{5}{cr 156}{1.2 kg}{blocks ranged attacks going out}
\end{multicols}


		\section{Implants}
\subsection{Limits}
Implantation of augments into the human body is an immense opportunity - to fix, to improve, to optimize. To the displeasure of employers everywhere however humans reach their limits much faster than implant technology does. These limits are measurable:
	\subsubsection{Money}
	Chrome generally doesn't come cheap. Someone looking to get augmented needs to have funds from successful business, life-long savings - or crime. Because they are so expensive, many people have started stealing implants from corpses, for profits or for themselves.
	\subsubsection{Space}
	A human body only has so many parts one can replace before starting to replace other implants. These parts will be generalized as follows.
	\begin{multicols}{2}
	\begin{itemize}
		\setlength\itemsep{-10mm}
		\item Eyes \textsubscript{\itshape Senses}
		\item Brain
		\item Nervous system
		\item Skin \textsubscript{\itshape Senses}
		\item Two arms \textsubscript{\itshape Bones \& Extremities}
		\item Chest \textsubscript{\itshape Bones}
		\item Lungs %(inner organ)
		\item Two legs \textsubscript{\itshape Bones \& Extremities}
	\end{itemize}
	\end{multicols}
	\subsubsection{Load}
	The human nervous and cardiovascular system can only take so much chrome before starting to break down.
	Put into scale by Dr Edmunth Rayleigh, the limit on one's nerves carries his name, while the second - due to its broader nature - is simply named after its purpose.
	Cybernetics are generally harder on the user's nervous system, while biotechnological implants tend to be rejected by the natural flesh instead.\\
	Overdoing it on augments has immediate and usually very severe consequences - among them are total implant rejection, heavy brain damage, permanent paralysis, sudden cardiac arrest.
\subsection{Cybernetics}
\vspace{10mm}
\begin{multicols}{2}
\implant{Ablative Skin}{A mesh, added on top of the skin, that helps resisting against heat and fire based injury.}{Halves heat and fire damage (except lasers). Cannot catch on fire.}{skin}{1 RI}{cr 180}{Wrongly fitted, Perfect fit, Inconspicuous, Aesthetic}
%TODO: anchor legs - brace anywhere, move while bracing
\implant{Bionic Cardiovascular Enhancements}{A cybernetic cardiovascular replacement. Can withstand more strain than the body's natural systems.}{Increases MT by 1. May be implanted up to 3 times.}{-}{1 RI}{cr 40}{Wrongly fitted, Inconspicuous, Perfect fit}
\implant{Bionic Extremity}{A cybernetic body part to replace a lost one}{The body part won't bleed and can't receive biotech implants or benefit from them.}{one extremity}{1 RI}{cr 50}{Cheap, Compartment, Wrongly fitted, Powerful, Tough, Inconspicuous, Perfect fit, Integrated Weaponry}
\implant{Bone Casing}{Half cybernetic, half bioengineered hardened casing for the human skeleton}{Grants 6 armor to the location. Swimming tests suffer a -15 penalty and weight is increased by 10\%.}{one tagged bone}{1 RI, 1 MT}{cr 120}{Wrongly fitted; Perfect fit}
\implant{Climbing Arms}{Cybernetic arms with a strong grip that can lock their fingers in position to prevent all exhaustion from holding on to ledges.}{Grants a +15 bonus to climbing tests and tests made to resist being disarmed.}{both arms}{2 RI}{cr 145}{Cheap, Compartment, Wrongly fitted, Powerful, Tough, Inconspicuous, Perfect fit, Integrated Weaponry}
\implant{Deadeye}{A holographic eye implant that grants many features like magnification and a range finder to assist in marksmanship}{Aim actions grant an additional +10 to the following shot.}{eyes}{4 RI}{cr 140}{Cheap, Wrongly fitted, Perfect fit}
\implant{Decentralized Muscle Processors}{Muscle memory was taken literally with this enhancement. Reactions are now partially calculated near the acting muscle fibers.}{Grants +15 to all reaction tests. Enables the user to parry bullets.
	Feinting against the user is twice as effective. Noticing the implant requires a successful Perception check at -20.}{nervous system}{5 RI}{cr 210}{Cheap, Wrongly fitted, Inconspicuous, Perfect fit}
\implant{Jump Joints}{Cybernetic loaded joints and leg bones that greatly increase jump height. No world records have been made using a set of these however - not for a lack of effectiveness, they were banned in sports before they were ever used.}{Jump and leap distances are doubled. Halves distance for calculation of fall damage when landing on both feet.}{both legs}{2 RI}{cr 190}{Aesthetic, Cheap, Inconspicuous, Perfect fit, Powerful, Tough, Wrongly fitted, Compartment}
%TODO: standard equipment!!
%TODO: Level One Neuro Calculator (L1NC)
\implant{Lifter Arms}{Arm and back implants that improve slow, sustained strength feats like lifting objects}{StrB and DoS on Str tests for determining carry weight and slow strength feats are increased by 4.}{arms, chest}{3 RI}{cr 150}{Cheap, Compartment, Wrongly fitted, Powerful, Tough, Perfect fit}
\implant{Neural Vault}{A brain implant that represses all but the most necessary emotions}{The character gains the benefits of \emph{Broken}, stacking with the ability. He may never benefit from any positive disposition.}{brain}{4 RI}{cr 220}{Wrongly fitted, Inconspicuous, Perfect fit}
\implant{Pain Inhibitors}{An implant that suppresses all pain reception, enabling fighting way beyond mortal limits.}{The user suffers no pain or related effects.
	Due to removal of a natural inhibitor, the character suffers D5-2 dmg to a random location and D5 exhaustion at the end of every day he is active, ignoring injury threshold.}{brain}{4 RI}{cr 140}{Wrongly fitted, Inconspicuous, Perfect fit}
\implant{Protected senses}{Sensory overload can be a detriment to many professionals, from soldiers to thieves. This augment protects from such occurrences at some loss to the sense's accuracy}{Grants +20 to resist overstimulation like blinding, deafening or pain. Using the associated sense suffers a -15.}{one tagged senses}{1 RI}{cr 80}{Cheap, Wrongly fitted, Perfect fit}
\implant{Quick Replacement Extremity}{A cheap cybernetic arm that can quickly be replaced with a new, undamaged one without taking much time getting used to}{The body part won't bleed and can't receive biotech implants or benefit from them.
	Zone HP is reduced to 20\%. As long as the interface is not destroyed, an extremity can be replaced in 15 seconds (5 rounds), taking another minute getting used to (-10 to all action taken with this limb during that time).
	Note: most mods are applied to the mount point but some may be applied to the limb instead as per GM discretion}{one extremity}{1 RI}{cr 50 to install the interface; cr 20 per limb}{Cheap, Compartment, Wrongly fitted, Powerful, Tough, Inconspicuous, Perfect fit}
\implant{Rayleigh Fibers}{An enhancement over the natural nervous system to massively increase reaction speed}{Grants +10 to all reaction tests. The character can dodge or parry non-primitive ranged weapons.}{nervous system}{5 RI}{cr 230}{Cheap, Wrongly fitted, Perfect fit}
\implant{Reprocessor Stomach}{A recycler squeezing the last nutrients from food waste. One of the only cybernetics straining medical toughness.}{The speed of starvation is reduced to a quarter.}{-}{2 MT}{cr 88}{Cheap, Wrongly fitted, Perfect fit}
\implant{Skin-close Holoprojector}{Technically not one but a set of holographic projectors that obscure the user's movements and make it hard to track exactly what someone is doing.}{Grants -20 to attacks and parries against the user.	The user suffers -20 to any tests made to hide. \\
	It has no effect when wearing armor or thick clothing.}{skin}{4 RI}{cr 180}{Wrongly fitted, Perfect fit, Aesthetic}
\implant{Subdermal Data Storage}{A large data storage implanted within the body. Nearly impossible to detect, great for smuggling data.}{The user always has data storage at hand. The connector can be found with a -30 Investigation test - if it is technically possible.}{-}{1 RI}{cr 50}{Cheap, Inconspicuous, Perfect fit, Wrongly fitted}
\implant{Thermal Vision}{An implant that allows interpretation of infrared light as visible light}{Allows the user to see IR light like heat sources and IR lasers. This may grant +20 to finding hidden people.}{eyes}{3 RI}{cr 160}{Cheap, Inconspicuous, Perfect fit, Wrongly fitted}
\implant{Voice Modulator}{Adjustments to the voice box allow a user to reform the vocal cords, allowing for potentially flawless imitation - at least in terms of sound.}{The user may make a +20 Voice Athlete (Instinct) test to gain a bonus of 10 + 5 per DoS to following Deceive or Intimidate tests. This does not count as an assist.}{brain}{1 RI}{cr 190}{Cheap, Inconspicuous, Perfect fit, Wrongly fitted}
\end{multicols}
\subsection{Biotech}
\vspace{10mm}
\begin{multicols}{2}
\implant{Compound Eyes}{Fly-like eyes that, while reducing some detail, give the user a much larger field of view}{Grants the user +20 to tests against being ambushed or surprised. The user's control area increases to 270°. Visual perception when looking for details suffers a -10.}{eyes}{2 MT}{cr 100}{Cheap, Wrongly fitted, Perfect fit}
\implant{Extradextrous Arm}{An arm featuring increased flexibility and improved tactile sense but somewhat lacking in raw strength}{The user gains +10 Dex, -10 Str and +20 to tactile perception using this arm.}{one arm}{1 MT}{cr 150}{Cheap, Wrongly fitted, Perfect fit, Powerful, Tough, Inconspicuous, Integrated Weaponry}
\implant{Flexible Bone Mesh}{An injection that adds some flex to the user's bones, granting better movement but an overall weaker frame}{Grants +25 to most acrobatics checks and all tests to squeeze through small spaces or out of grapples.
	The user suffers -15 to resist getting knocked down. The zone's HP are halved.}{one tagged bone}{1 MT}{cr 120}{Wrongly fitted, Perfect fit, Tough}
\implant{Gills}{Implanted additional gills to provide another source of oxygen}{Allows the character to breathe under water.}{lungs}{2 MT}{cr 130}{Wrongly fitted, Perfect fit, Inconspicuous}
\implant{Gripping Legs}{Flexible legs and dexterous feet with opposable thumbs are installed into the user's lower body.}{The character can now stand up in half the time (an action turns into a free action), won't fall over by himself and gains a +20 bonus to resist being knocked over. However he can no longer wear normal shoes.}{both legs}{3 MT}{cr 220}{Cheap, Wrongly fitted, Perfect fit, Powerful, Tough}
\implant{High-performance senses}{In the animal kingdom heightened senses are frequently the difference between life and death. Generally human brains cannot handle the sudden increase in stimulus that well.}{Grants +20 to perception tests using one particular sense.
	The user suffers -20 to tests resisting overstimulation (like blinding, deafening or pain).}{one tagged senses}{1 MT}{cr 100}{Cheap, Wrongly fitted, Perfect fit, Inconspicuous}
\implant{Metabolism Booster}{Efficiently closing wounds always was import to humans. This rare implant can enhance natural healing but is quite inefficient and requires a lot of additional chemical energy.}{Once per day, if the character is not already malnourished or starving, the character may take 15 minutes of rest to gain natural healing and restore 2D5 exhaustion.
	The user will become extremely hungry afterwards and take a -20 penalty to most actions until he gets a meal.}{all inner organs}{3 MT}{cr 182}{Wrongly fitted, Perfect fit}
\implant{Mucus Gland}{An engineered gland producing thick mucus. While it stinks, it is very resistant.}{The character gains 3 temporary HP on every zone, which don't contribute to degradation. When they are lost, it takes 15 minutes  for them to regenerate. The user suffers Bane: Terrible Odor}{skin}{3 MT}{cr 195}{Wrongly fitted, Perfect fit}
\implant{Oxygen Tanks}{Additional organs that save oxygen and reduce the feeling of exhaustion}{The user's exhaustion threshold is doubled.}{all inner organs}{1 MT}{cr 75}{Wrongly fitted, Perfect fit}
\implant{Pheromone Glands}{Airborne pheromones as a means of communication are a staple in the animal kingdom. While humans can't quite use it to communicate, this augment allows subtle manipulation during swaying attempts.}{Grants +20 to all friendly social tests - the bonus is +40 instead in enclosed rooms without airflow.
	If either the target or the user wears vac-sealed armor, the augment has no effect.}{skin}{3 MT}{cr 155}{Wrongly fitted, Perfect fit, Inconspicuous, Aesthetic}
\implant{Radiation proofing}{An under-skin mesh that blocks many forms of harmful radiation}{Grants +40 to tests to resist radiation poisoning and +30 to tests against getting shocked. The user's augments are immune to EMP.}{skin}{2 MT}{cr 135}{Wrongly fitted, Perfect fit, Inconspicuous, Aesthetic}
\implant{Shadow Runners}{Very springy leg replacements, making it much easier to walk silently.}{Sound-related penalties on sneaking tests are halved.}{both legs}{3 MT}{cr 144}{Wrongly fitted, Perfect fit, Tough, Powerful, Inconspicuous}
\implant{Shielded Nerves}{An enhancement to the user's nervous system designed to bear larger loads. While it replaces a large part of the nervous system, it is compatible with any other augment.}{Increases RI by 1. May be implanted up to 3 times.}{-}{1 MT}{cr 40}{Wrongly fitted, Perfect fit}
\implant{Tenadermis}{An engineered gland that creates a thick, tough layer just below the skin}{Grants +8 natural armor.
	It removes all sense of touch, effectively granting \emph{Bane: No tactile sense}.}{skin}{3 MT}{cr 205}{Wrongly fitted, Perfect fit, Inconspicuous, Aesthetic}
\implant{Toxin Filters}{A very large group of various replacements that remove toxic substances from all parts of the body}{The user becomes immune to almost all toxins.}{all inner organs}{5 MT}{cr 110}{Wrongly fitted, Perfect fit, Tough, Inconspicuous}
\implant{Vat-grown Body part}{A biological body part grown in a vat, usually grown from the patient's own DNA. Just as good as the original.}{A lost or broken body part is restored back to normal. Growing an exact clone takes very long (4 months) but does not use any MT.}{one tagged extremity}{1 MT}{cr 50}{Cheap, Wrongly fitted, Perfect fit, Powerful, Tough, Integrated Weaponry}
\implant{Wound Sealant Foam Gland}{An engineered gland that releases blood clotting foam when injured to prevent large amounts of blood loss}{All bleeding is stopped the moment it begins. Venoms last twice as long. These effects are inhibited by most common anesthetics.}{skin}{4 MT}{cr 220}{Wrongly fitted, Perfect fit}
\end{multicols}

\subsection{Mods}
\vspace{10mm}
\begin{multicols}{2}
\augmod{Aesthetic}{The replacement is very visually pleasing. This increases Ch by 5 (essentially increasing looks by one step).}{+35\%}
\augmod{Cheap}{Bonuses are halved. If no bonuses are granted or the bonus is below 10, this also adds a -5 penalty to all actions utilizing this body part.}{-25\%}
\augmod{Inconspicuous}{Identifying the augment as such requires a -35 Perception check. If the augment is already not obvious, the penalties are added.}{+20\%}
\augmod{Integrated Weaponry}{A small weapon is hidden in the augment. Biotech may only include organic versions of melee weapons while Cybernetics may also include pistols, though they then use Ranged Combat Training (Integrated Weaponry). If the character is giant or bigger, they may include small rifles as well. If the weapon is either ranged or a retractable melee weapon, it takes +1 RI (for cybernetics) or MT (for biotech). These weapons cannot be disarmed or removed without invasive surgery.}{+25\% + the weapon's cost}
\augmod{Perfect fit}{RI and MT costs are halved.}{+40\%}
\augmod{Powerful}{Offers more physical strength, increasing Str by 5.}{+60\%}
\augmod{Tough}{Denser, harder, tougher. Grants 2 natural armor and +33\% zone HP.}{+40\%}
\augmod{Wrongly fitted}{RI and MT costs are increased by 25\%, at least by 1. Generally speaking this is applied to all scavenged augments - if this modification is not available for some augment, those scavenged implants are unusable.}{-25\%}
\augmod{Compartment}{A small compartment is added to the augment. It is very limited in space but nearly impossible to detect unless someone has deep knowledge of cybernetics and likely augmented eyes themselves.}{+25\%}
\end{multicols}

%	TODO: add after chemical rework
%	\section{Chemical}
\label{sec:chems}

\subsection{Effects}
Every chem provides one or more benefits and detriments.
They last for a given time before the effects run out.
\\%
Kicking back multiple doses increases the effects.
After the given duration, only one dose runs out.
\par%
\begin{exampleblock}
	Mei hits 3 doses of liquid armor.
	For 3 hours she has 3D10 natural armor,
	her Ranged Base, Dex and Ag are reduced by 3D10,
	and she suffers an additional load of 6 MT.
	\\%
	After 3 hours the bonuses and penalties are reduced to 2D10 and the load to 4 MT, and so forth.
\end{exampleblock}
Like more permanent augmentation chems also put a load on the body, using the same metrics (RI and MT).
This load normalizes quickly once the effects subside.

\subsection{Risks}
\subsubsection*{Overdose}
Every chem produces load - just like implants
	- for the time of its effect.
A user who surpasses his RI or MT using chems,
	will risk long term side effects (see below).
\\%
A user surpassing his RI or MT by more than 2 suffers overdose effects
	of all chems currently active in his system.
Some chems may not have any - these are safe.
\subsubsection*{Long Term Side Effects}
When a user risks long term side effects,
	he rolls a D10.
If the result is
	less than or equal to
	his current RI or MT load
		(whichever one is higher and above his limit),
	that value will permanently be reduced by \emph{0.1}.
\\%
This effect can occur once for every dose of any chem he takes.

\subsection{Substances}
\luaimport{equipment/chems.csv}{chem.tpl}{augs/chem}


	\chapter{Lists}
	\vspace{-10mm}
\section{Status effects}
\vspace{8mm}
\begin{multicols}{2}
\statuseffect{Blinded}{The character cannot see and suffers according penalties, most notably to Perception and combat. He also cannot Gather Senses if the status effect is only temporary or brand new.}
\statuseffect{Deafened}{The character cannot hear. She suffers according penalties and can't Gather Senses if the condition is only temporary or brand new.}
\statuseffect{Frenzied}{The character has entered a vicious rage.
	She no longer takes penalties from pain and damage aside from complete destruction of body parts.
	Melee damage bonus is doubled.
	She will throw herself into combat and needs to make Int checks to distinguish friend from foe.
	These checks are at a +30 bonus for very close friends and a -30 penalty for completely unknown people.}
\statuseffect{Nausea}{The character gains exhaustion for every minute of strenuous activity. In addition, on a failed Con check for any strenuous action, he starts vomiting, getting Stunned for D5 rounds.}
\statuseffect{Prone}{The character is granted bonuses and suffers penalties as described in Situational Modifiers. If falling prone was involuntary, she also loses D10 initiative.}
\statuseffect{Stunned}{The character cannot act except for moving with a speed of ~1m/s and loses 1 initiative for every combat round he is stunned. Tests to prevent falling over automatically fail.}
\statuseffect{Unconscious}{The character cannot act at all.}
\end{multicols}

	\section{Skills}
\vspace{10mm}
\label{skilllist}
\begin{multicols}{2}
\skill{Acrobatics}{Agility}{Acrobatics describes the ability to control one's body. It encompasses moving elegantly, gymnastics, breaking falls and dancing.}
\skill{Animal Handling}{Charisma}{Animal Handling is any interaction with animals; calming them down, taming and training them.}
\skill{Appeal}{Charisma}{Appeal encompasses any attempt to charm or convince someone that at least outwardly happens in good blood.}
\skill{Athletics}{Strength}{Athletics is used for any task built on directing one's strength, be it lifting, pushing, running or jumping.}
\skill[Advanced]{Brewing}{Intelligence}{Making any sort of alcoholic beverage is considered brewing. All differences between creating beer, wine and whatever else is neglected.}
\skill[Advanced]{Chemistry}{Intelligence}{Chemistry is necessary for identifying and creating chemicals. It also includes knowledge of recipes and estimation of effects and side effects for unknown recipes.}
\skill{Climbing}{Strength}{Climbing generally means scaling heights. Holding oneself at the ceiling might be considered climbing as well.}
\skill{Command}{Charisma}{Command describes military social interactions that feature a clear hierarchy. It also represents knowledge about military leadership and command structure.}
\skill[Advanced]{Commerce}{Intelligence}{Anything concerned with the exchange of money falls under the Commerce skill. This obviously includes trading but also bribery and extortion.}
\skill[Advanced]{Computer Operation}{Intelligence}{Computer Operation means using a computer. That technically includes even the simplest tasks but only advanced operations will require a test, such as getting access to files that one should not have access to.}
\skill[Advanced]{Construction}{Intelligence}{Construction is concerned with creating buildings and understanding buildings; that means probable layouts or weaknesses.}
\skill[Advanced]{Cryptography}{Intelligence}{Cryptography is the knowledge about en- and deciphering data of any type. If the data was human-readable, a cunning character might identify simple probable cipher algorithms.}
\skill{Deceive}{Charisma}{Deceiving means telling lies. Making half-truths believable generally falls under this skill as well.}
\skill{Dodge}{Agility}{Dodging is physically evading harm. This can be an attack or something more natural like falling rocks or sudden vehicles.}
\skill[Advanced]{Drive: Bike}{Agility}{Drive (Bike) encompasses the knowledge of how motorbikes and similar vehicles work and how to get them to move in a desired direction.}
\skill[Advanced]{Drive: Boat}{Agility}{Drive (Boat) encompasses the knowledge of how motorboats and similar vehicles work and how to get them to move in a desired direction. Rowboats, canoes and so on should be covered with athletics instead.}
\skill[Advanced]{Drive: Car}{Agility}{Drive (Car) encompasses the knowledge of how cars and similar vehicles work and how to get them to move in a desired direction.}
\skill[Advanced]{Drive: Heavy}{Agility}{Drive (Heavy) encompasses the knowledge of how trucks and similar vehicles work and how to get them to move in a desired direction. It also includes knowledge about APCs, IFVs or tanks and other heavy armored vehicles.}
\skill{Estimate}{Instinct}{Estimation is the ability to guess amounts, weight, size, distance and whatever else might come up; prices are not covered however, and instead fall under the crafting skill or \emph{Commerce}.}
\skill{Gambling}{Instinct}{Gambling is betting, card and board games: knowing how to play, knowing how to win, by all means if necessary.}
\skill[Advanced]{Geography}{Intelligence}{Geography describes the knowledge of the world and its places.}
\skill[Advanced]{Herbalism}{Intelligence}{Herbalism is knowledge of plants and plant-based products, formal or by experience.}
\skill{History}{Intelligence}{History is the knowledge of past events. Low values don't necessarily mean no knowledge but might include false information as well.}
\skill{Interrogation}{Instinct}{Interrogation encompasses asking the right questions as well as reading facial expressions and voice tone to determine truth or lie; also includes understanding the opposition's true meaning.}
\skill{Intimidation}{Courage}{Intimidation means skillfully threatening others to avoid actually having to go through with the threat.}
\skill[Advanced]{Law}{Charisma}{Law means the knowledge about rights and also the ability to accuse and defend properly.}
\skill[Advanced]{Logic}{Intelligence}{Logic is the ability to dissect and combine information rationally and quickly or detect patterns where others do not.}
\skill[Advanced]{Mechanics}{Dexterity}{Mechanics encompasses the knowledge of how technology works and the skill to fix or build machinery.}
\skill[Advanced]{Medication}{Dexterity}{Medication is the necessary knowledge of anatomy and medical chemicals and the skill of how to apply them, give first aid and do surgery.}
\skill{Navigation: Air}{Instinct}{Navigation in the air is difficult and relies on good sense of both distance and direction. It may be easier when sun, moon, stars or large landmarks on the ground are visible.}
\skill{Navigation: Ground}{Instinct}{Navigation on the ground is the most common type of navigation. It helps finding one's way on foot or in earthbound vehicles.}
\skill{Navigation: Sea}{Instinct}{Navigation on the sea relies on reading maps and a compass. It is possible to make good educated guesses by using celestial bodies as well.}
\skill[Advanced]{Navigation Space}{Intelligence}{Navigation through the void of space without equipment is almost impossible due to how vast space is. This skill represents finding the shortest and safest path given such instruments.}
\skill{Navigation: Underground}{Instinct}{Navigating through underground complexes is quite difficult, as there are very few landmarks. It requires good memory or a map and good sense of distance.}
\skill{Perception}{Instinct}{Perception is essential for noticing small details. Someone properly educated with trained sense of hearing for example might make out the type of gun or ammunition used in a shot from the sound alone.}
\skill[Advanced]{Philosophy}{Intelligence}{Philosophy is what drives many people in the shadows of neon lights. Existential questions rarely find clear answers but one might come close.}
\skill[Advanced]{Pilot: Air}{Agility}{Piloting aircraft means understanding basics of aerodynamics and controls of different aircraft, usually but not necessarily it also encompasses basic understanding of flight control procedures.}
\skill[Advanced]{Pilot: Jet pack}{Agility}{While not piloting in the classical sense, using jet packs effectively is an extremely difficult discipline of aeronautics.}
\skill[Advanced]{Pilot: Space}{Intelligence}{Piloting spacecraft does not happen alone but this skill is representative of any position spacecraft may have.}
\skill[Advanced]{Quaffing}{Constitution}{Quaffing, getting wrecked, hammered, tanked, any of the above. This is a character's ability to hold his liquor.}
\skill{Restraint}{Courage}{Restraint is the ability to push through and ignore both pain and strong emotions. Trying to suppress disabling characteristics is at a penalty equal to the disabling characteristic.}
\skill[Advanced]{Security}{Intelligence}{Security encompasses anything that is involved in defense: Personally setting and defusing traps, choosing defensive positions, how to employ choke points and the exploitation of any mistakes the opposition might have made.}
\skill{Sleight of hand}{Dexterity}{Sleight of hand covers everything that requires delicate fingers and a good distraction. This may be pickpocketing or card tricks or whatever someone might come up with. Picking locks is part of this skill as well.}
\skill[Advanced]{Smithing}{Dexterity}{Smithing is the creation of new objects from metal. Casting also falls under this skill, even though it is not technically smithing.}
\skill{Stealth}{Agility}{Stealth describes the ability to hide, be it oneself or objects, as well as moving quietly. It is usually heavily modified and often opposed with Perception.}
\skill{Stories}{Charisma}{Stories means the knowledge of stories, the ability to make them up on the spot and the capability to tell them interestingly.}
\skill{Survival}{Instinct}{Survival is a catchall skill encompassing anything required for wilderness or wasteland survival: finding something edible and clean water, building improvised shelters and the most basic equipment, as well as cooking.}
\skill{Swimming}{Constitution}{Swimming is the capacity to cross through waters without drowning. Due to how rare clean open bodies of water are, this skill is surprisingly rare.}
\skill{Tailoring}{Dexterity}{Tailoring is the knowledge of how to make objects from cloth or leathers. Very few people still do this themselves and their works are revered and sold for high prices.}
\skill{Tracking}{Instinct}{Tracking means finding and following a trail. This might be footprints of some wild animal or disturbances left by fleeing suspects.}
\skill[Advanced]{Voice Athlete}{Charisma}{Voice athletes are capable of changing their voices to sound a certain way. This can be used for ventriloquism, impersonation or something as simple as singing.}
\skill[Advanced]{Zoology}{Intelligence}{Zoology is knowledge about animals. This may be from formal education or experience and can sometimes be restricted when dealing with rare or exotic animals.}
\end{multicols}

\subsection{Melee Combat Training}
\vspace{10mm}
\begin{multicols}{2}
\melee{Striking}{Punching, kicking, headbutting: Short contact unarmed combat.}
\melee{Grappling}{Holds, locks, chokes: All long contact combat. This includes unarmed and armed techniques that don't cause damage from direct impact.}
\melee{One-handed Blade}{Swords starting with very large knives and ending at bastard swords.}
\melee{One-handed Club}{Short blunt weapons with a weighted tip, as well as axes.}
\melee{One-handed Piercing}{Knives. Really these are mostly knives (and some improvised weaponry).}
\melee{Whip}{Flexible weaponry that cannot easily be parried.}
\melee{Two-handed Blade}{Long swords that require two hands, up to half a meter longer than body height and 8 kilograms at most.}
\melee{Two-handed Club}{Long blunt weapons with a weighted tip, as well as axes.}
\melee{Two-handed Thrusting}{Spears and staves, weapons with a reach advantage.}
\end{multicols}

\subsection{Ranged Combat Training}
\vspace{10mm}
\begin{multicols}{2}
\ranged{Bow}{Flexing weapons shooting projectiles from a string.}
\ranged{Heavy Weapon}{Heavy weaponry usually fired from vehicles or emplacements.}
\ranged{Launcher}{Launcher weapons are heavy weapons that shoot large projectiles and are especially affected by gravity.}
\ranged{Long Rifle}{Any full-length assault rifle or larger, up to anti materiel rifles.}
\ranged{Short Rifle}{Any ranged weapon requiring two hands but still in a CQC configuration. Also includes pistol caliber rifles and submachineguns.}
\ranged{Pistol}{Pistols generally use small calibers, are concealable and can be fired using one hand decently easily.}
\ranged{Throwing}{Any thrown weapon. This may be grenades, knives, rocks or the kitchen sink.}
\skill{Reload}{Dexterity}{Quickly reloading a weapon. Pretty self-explanatory.}
\end{multicols}
		\section{Abilities}
\label{abilitylist}
\vspace{3mm}
\begin{multicols}{2}
\ability{Advanced Parry}{400}{Overly Cautious}{When parrying multiple hits, distribute DoS among any amount of attackers. Still requires more DoS than the opponent for single hits. Can also parry Flexible weapons.}
\ability{Alpha Strike}{500}{From the shadows, no Beta Strike}{Successfully attacking an unaware target automatically hits critically.}
\ability{Assassin Strike}{350}{Acrobatics Trained}{Scoring a critical hit or deathblow in melee allows the character to move up to his speed in meters right away without provoking opportunity attacks.}
\ability{Bar Fighter}{250}{Quaffing Trained}{Ignore \emph{Improvised} penalties.}
\ability{Battle Mind}{400}{Intelligence 40}{The character may use intelligence instead of instinct for initiative calculation.}
\ability{Battle Sense}{500}{Perception Trained}{Adds a fifth of Perception Bonus (2, 4, or 6) to initiative.
	Doubles penalties to attacks of opportunity against the character.}
%TODO: rework!
\ability{Beta Strike}{500}{Constitution 40, no Alpha Strike}{Cause a level of exhaustion to an enemy for every 3 attacks targeted at him.}
\ability{Blind Fighting}{300}{Trained Sense: Hearing}{Limits the penalties for being blinded in melee combat to -10.}
\ability{Blurry}{250}{Acrobatics Trained, Athletics Trained}{While sprinting the character imposes a -20 to everyone trying to hit him.}
\ability{Broken}{450}{Courage 40}{Grants +30 to fear and pinning tests but doubles penalties from bad reputation or disposition.}
\ability{Butcher}{400}{no Earth Shaker}{Damage dealt with blades is increased by 1 for one-handed weapons and 2 for two-handed weapons.}
\ability{Calculated Pilot}{300}{Instinct 30, Dexterity 30}{Halves penalties for piloting tests in demanding, stressful situations.}
\ability{Calming Focus}{350}{Courage 40}{Re-roll one failed fear or pinning test per turn.}
\ability{Calming Presence}{200}{Command Trained, Calming Focus}{The character may take an action to make a command test at a +20 bonus. For every degree of success all allies he addresses gain a +10 to tests to resist fear and pinning.}
\ability{Clear Target}{300}{Ranged Base 30}{Halves penalties for calling hit locations on ranged attacks.}
\ability{Clotter}{300}{Constitution 55}{When starting to bleed, make a toughness check at a -10 penalty. On a success the character doesn't bleed.}
\ability{Companion}{200}{per GM discretion}{The character may adopt a small companion. The GM will explain the details.}
\ability{Crack Shot}{500}{Perception Trained, Clear Target}{Before taking a shot, make a perception check at a penalty equal to twice the target location armor. On a success halve the armor for this attack. Has no effect on shields.}
\ability{Dominion}{350}{Supreme Awareness}{When taking a Defensive Stance and not moving, the control area is increased to 360°. The character may also gain attacks of opportunity against disengaging enemies.}
\ability{Drunken Master}{300}{Bar Fighter}{Limits the penalty to melee combat resulting from intoxication to -10.
	While intoxicated he himself is at a -20 to hit
	and can use maneuvers as normal.\\
	Costs half as much for \textit{balanced} characters.}
\ability{Dual Wielding}{250}{Instinct 30, Dexterity 40}{Halves penalties for dual wielding. Has to be taken once for melee and once for ranged separately.}
\ability{Earth Shaker}{400}{no Butcher}{Damage dealt with blunt weapons is increased by 1 for one-handed weapons or 2 for two-handed weapons.}
\ability{Executioner}{350}{Trained Sense: Smell}{Grants +10 to hit bleeding targets.}
\ability{Expert Wrestler}{350}{Melee Combat Training (Grappling) Experienced}{The character is incredibly well-trained in grappling that he can initiate a Hold as a free action.}
\ability{Face Breaker}{300}{Dexterity 30, Agility 30, Stunning Blow}{Grants unarmed strikes the Trauma special rule.}
\ability{Fighting Retreat}{400}{Battle Sense}{Allows a single attack to be made when disengaging. This attack counts as being performed in movement.}
\ability{Flip Up}{300}{Acrobatics Trained}{When standing up, reduce the time it takes by an action: prone to standing now takes one action or a free action and a successful agility test.}
\ability{Focused Frenzy}{350}{Instinct 30, Intelligence 40, no Unbridled Rage}{When the character is made to go berserk by some means, they can still identify friend and foe as well as take defensive actions, albeit at a -15 penalty.}
\ability{Formation}{300}{Intelligence 35}{When taking Combined Actions, everyone involved is granted a +10,
	up to +30 when three or more characters have Formation.
	When \emph{Unbreakable Bond} partners also have Formation,
	those two gain an additional +10.}
\ability{From The Shadows}{250}{Overly Cautious}{During the first round of combat the character has 6 additional initiative, unless the character was ambushed or similarly caught off guard.}
\ability{Gruesome}{450}{Intimidate Experienced}{When destroying body locations or landing a killing blow, the character makes an Intimidate test with a bonus equal to twice the last attack's damage. All fear tests suffer an additional penalty equal to 10 + 5 per DoS.}
\ability{Hand-To-Hand Basics}{250}{Combat Training (Striking or Grappling) Trained}{When fighting armed enemies, the character counts as armed as well.}
\ability{Hax0r e1337}{400}{Computer Operation Experienced}{Add two degrees of success to any Computer Operation tests to crack other devices or protect your own.}
\ability{Heroic}{400}{Courage 40}{Grants +15 to fear and pinning tests.}
\ability{High Noon}{500}{Combat Training (Pistol) Experienced, Quick Draw}{When drawing a pistol, the character may take a single shot as a free action.}
\ability{Iron Fist}{450}{Hand-to-hand Basics, Maneuver: Stunning Blow}{Every unarmed strike may now count as a having Stun(0) without inferring a penalty.}
\ability{Lead Storm}{350}{Strength 45}{Weapons with a fire rate of more than 1 which are not launchers always count as braced.}
\ability{Light Sleeper}{250}{Overly Cautious}{Overly Cautious now also works when sleeping or intoxicated.}
\ability{Marksman}{300}{Perception Experienced}{Halves penalties for long ranges.}
\ability{Medical Insight}{400}{Medication Experienced}{Doubles the degrees of success on Medication tests to help others recover.}
\ability{Mighty Companion}{500}{per GM discretion}{The character may adopt a powerful companion. The GM will explain the details.}
\ability{Mobile Combat}{250}{Dexterity 30, Agility 40}{Halves penalties for fighting while moving. This includes a Charge maneuver's flat penalty if the character also has Maneuver: Charge.}
\ability{Natural Geek}{350}{Intelligence 45}{Halves penalties to mental tasks resulting from not knowing what you're doing.
	This may include Weapon Training and Skills
	but does not include tasks in stressful situations (like open combat)
	or purely physical actions
	- no matter how apt,
	thinking won't make him lift heavier weights.}
\ability{On Your Feet!}{100}{Command Trained, Heroic}{The character may take an action to make a command check. On a success an ally gains a +5 per degree of success to any action following the character's orders.}
\ability{One-Man Artillery}{350}{Strength 45}{Launchers always count as being braced.}
\ability{Overly Cautious}{300}{Instinct 30}{The character becomes jittery or nervous,
	but he can ignore Unaware conditions by passing a simple Perception check,
	so long as he is not asleep, intoxicated or otherwise impaired.}
\ability{Peer}{100}{Charisma 30}{Gain +15 to interaction tests with a specified group of people.
	May be chosen once per group of people.}
\ability{Perfectionist Craftsman}{400}{Experienced in a crafting skill}{Doubles the degrees of success on crafting tests when the specific skill is used.\\
	May (theoretically) be taken once per skill.}
\ability{Personified Inspiration}{500}{Charisma 40, Heroic}{When the character does a daring or heroic deed, all allies present are immune to fear and pinning for the rest of the encounter or scene.}
\ability{Precision Strike}{300}{Melee Base 30}{Halves penalties for attacking small targets with melee attacks.}
\ability{Protective Stance}{250}{Instinct 40, Constitution 40}{When taking a Defensive Stance, the character may turn active actions into reactions without penalties and he may parry for allies within his melee range.}
\ability{Provoke}{350}{Courage 45}{Spend an action to taunt an enemy and make a skill test opposed by the target's restraint. If the character wins, the target must attack the character at the next opportunity.}
\ability{Quick Draw}{250}{Dexterity 30, Agility 30}{Draw smaller equipment pieces, pistols and one-handed melee weapons as a free action.}
\ability{Rapid Reload}{300}{Dexterity 40, Reload Experienced}{When reducing reload time to below 1 action, reload as a free action.}
\ability{Reliable Reputation}{200}{Charisma 40, no Broken}{Halves penalties from bad disposition.}
\ability{Render}{400}{Strength 55}{When dealing damage with the Tearing rule, roll an additional dice. Still only keep one.}
\ability{Revenge}{400}{Broken}{When taking damage from a direct attack, you may forego one of your actions later in the combat round to immediately attack the offending enemy.}
\ability{Ripper}{350}{Render}{When dealing damage with the Tearing rule and rolling doubles,
	you may choose those two dice
	(instead of keeping just one).}
\ability{Sidearm}{250}{Dual Wielding Melee \& Ranged}{When attacking in melee combat with a melee weapon in one hand and a pistol in the other, the character takes no penalties from shooting in melee combat.}
%I have determined that there are enough ways to bypass armor using maneuvers
%\ability{Skinner}{500}{Perception Trained}{Before taking a melee attack, make a perception check at a penalty equal to twice the target location armor. On a success halve the armor for this attack.}
\ability{Skull Taker}{500}{Clear Target or Precision Strike}{Removes penalties from calling hit locations for ranged attacks (when combined with Clear Target) or melee attacks (when combined with Precision Strike).}
\ability{Snapshot}{350}{Steady Aim}{Aiming takes half as long.}
\ability{Stabilization}{300}{Medication Trained}{The character is able to staunch light bleeding on others in one round without equipment.}
\ability{Steadfast}{200}{Agility 30, Constitution 30}{Tests to prevent falling over are at a +10.}
\ability{Steady Aim}{250}{Intelligence 40, Strength 30}{When aiming, line of sight may be broken for a round without aiming being interrupted. By GM discretion this may count even longer if the target is moving particularly predictably.}
\ability{Strong Back}{350}{Strength 55}{The character's carry weight is increased by half.}
\ability{Supreme Awareness}{350}{Instinct 55 or Light Sleeper}{Roll twice when determining initiative or gathering senses, picking the higher result.}
\ability{Tactical Advance}{350}{Command Known}{While advancing carefully from one cover to the next over a distance of MS meters in one turn, the character counts as in cover for the whole way.}
\ability{The Best Defense}{300}{Courage 40}{Reduces offensive penalties of Defensive Stance by 20.}
\ability{Thick Skull}{400}{Constitution 50}{Grants immunity to the Trauma special rule.}
\ability{Threatening Presence}{450}{Intimidation +10, no Calming Presence}{Impose a -15 penalty to enemies' tests against fear and intimidation.}
\ability{Too Angry To Die}{200}{Courage 40, Constitution 40}{While frenzied,
	the character has to take at least 8 damage at once or lose his head in order to die.
	Bleeding out only starts when the character stops frenzying,
	but he will in this case be unconscious when frenzy ends.}
\ability{Trained Sense}{250}{Perception Trained}{Grants a +10 bonus to sensory checks using that sense. Can be taken once for every sense.}
\ability{Unbreakable Bond}{250}{being good friends; if one is Clumsy and the other is Cursed by Fortuna, the cost is only 100 XP}{When around one particular close friend, that friend may make a +20 charisma check as a free action and every degree of success gives the character +10 to resist fear and pinning,
	as well as potentially the effects of disabling characteristics.
	If both of them have Unbreakable Bond,
	the bonus to the charisma check is +40 instead.}
\ability{Unquestioned}{500}{Reliable Reputation}{Penalties to Command tests resulting from disbelief in authority (e.g. represented by Rebel or some Bound by Principle) do not apply.}
\ability{Unrestrained Reflexes}{300}{Agility 35, Dodge Trained}{The character may make an additional dodge action using the same reaction. This may happen once per turn.}
\ability{Vault}{300}{Agility 30, Strength 30}{The character can use suitable objects in the environment to double jump height and leap length during movement without slowing down.}
\ability{Volume of Fire}{350}{Ranged Combat Training (Heavy) Known}{When using a weapon with a rate of fire above 1 to take a standard shot, the character may spend more ammunition to gain a +5 to hit per shot fired after the first, up to the weapon's rate of fire.}
\ability{Walk It Off}{350}{Constitution 40}{The character can continue acting while bleeding until completely bled out. If combined with Tough as nails, the character may enter a frenzy when bleeding as well.}
\ability{Where It Hurts}{400}{Hand-to-hand Basics, Strength 40}{The character's unarmed attacks may gain the Tearing special rule.}
\end{multicols}

\vspace{-5mm}
\subsection{Maneuvers}
\vspace{8mm}
\begin{multicols}{2}
\maneuver{Attack of Opportunity}{150}{Instinct 30, Agility 30}
\maneuver{Bind}{200}{Instinct 35, Dexterity 35}
\maneuver{Burst}{200}{Dexterity 30, Strength 35}
\maneuver{Called Jab}{100}{Maneuver: Feint, Perception Known}
\maneuver{Charge}{100}{Maneuver: Knockdown, Agility 40}
\maneuver{Cleave}{100}{Maneuver: Knockdown, Courage 40, Instinct 40}
\maneuver{Crushing Blow}{300}{Maneuver: Powerstrike, Courage 50, Strength 60}
\maneuver{Disarm}{100}{Maneuver: Feint, Maneuver: Bind}
\maneuver{Feint}{200}{Dexterity 35, Agility 35}
\maneuver{Flurry}{100}{Maneuver: Feint, Dual Wield (Melee)}
\maneuver{Intercept}{100}{Maneuver: Bind, Courage 50}
\maneuver{Knockdown}{100}{Maneuver: Powerstrike, Strength 45}
\maneuver{Piercing Thrust}{300}{Maneuver: Called Jab, Courage 40, Dexterity 40}
\maneuver{Powerstrike}{200}{Strength 35}
\maneuver{Reversal}{300}{Maneuver: Intercept, Courage 60, Instinct 50}
\maneuver{Shift}{100}{Maneuver: Feint, Instinct 40, Dexterity 35, Agility 35}
\maneuver{Stunning Blow}{100}{Maneuver: Powerstrike, Intelligence 35}
\maneuver{Takedown}{100}{Maneuver: Feint, Agility or Strength 40}
\end{multicols}


	\section{Races}
\label{sec:racelist}

\subsection*{At a glance}
\providecommand{\tmplastgroup}{}
\begin{center}
\begin{tabularx}{0.7\textwidth}{X|rr}
	Race & Cost & Page \\
	\hline%
	\luaimport{lists/races.csv}{race-sum.tpl}{races-summary}
\end{tabularx}
\end{center}
\subsection*{In detail}
\luaimport{lists/races.csv}{race.tpl}{races}

		\section{Backgrounds}
\label{sec:backgroundlist}
\paragraph*{Academic (9 GP)}
\textit{Formerly educated in a single or many different things this character mainly knows a lot of things but might lack in the necessary skills to apply them properly.}\par
\begin{tabular}{|l|l|l|l|l|l|l|l|}
	\hline
	Cr & Int & Ins & Ch & Dex & Ag & Con & Str \\ \hline
	& 5 &  &  &  &  &  &  \\ \hline
\end{tabular}\par
\noindent\textbf{Other modifiers:} \\
\textbf{Skills:} +6, maximum of +2 across: Chemistry, Computer Operation, Cryptography, Herbalism, History, Law, Mechanics, Medication, Philosophy, Smithing, Stories, Zoology;
one Education +2,
one Education +1\\
\textbf{Abilities:} Natural Geek\\
\textbf{Boons:} \\
\textbf{Banes:} \\

\hrulefill
\paragraph*{Bodyguard (7 GP)}
\textit{Sworn to protect this character is versed in how to identify and eliminate threats to another person.}\par
\begin{tabular}{|l|l|l|l|l|l|l|l|}
	\hline
	Cr & Int & Ins & Ch & Dex & Ag & Con & Str \\ \hline
	5 &  &  &  &  &  &  &  \\ \hline
\end{tabular}\par
\noindent\textbf{Other modifiers:} \\
\textbf{Skills:} Ranged Combat Training (Pistol or Short rifle) +2,
One armed Melee Combat Training +2,
Combat Training (Striking or Grappling) +1,
Perception +1,
One Drive +2,
a fitting Navigation +2\\
\textbf{Abilities:} \\
\textbf{Boons:} \\
\textbf{Banes:} \\

\hrulefill
\paragraph*{Bouncer (1 GP)}
\textit{This character was employed to stop intrusions. Usually this happens rather peacefully but if necessary this might include force as well.}\par
\begin{tabular}{|l|l|l|l|l|l|l|l|}
	\hline
	Cr & Int & Ins & Ch & Dex & Ag & Con & Str \\ \hline
	&  &  &  &  &  &  &  \\ \hline
\end{tabular}\par
\noindent\textbf{Other modifiers:} \\
\textbf{Skills:} Combat Training (Striking AND Grappling) +2,
Intimidate +2,
Interrogation +1,
Restraint +1\\
\textbf{Abilities:} \\
\textbf{Boons:} \\
\textbf{Banes:} \\

\hrulefill
\paragraph*{Child Soldier (15 GP)}
\textit{Naturally born children are far cheaper than vat-grown soldiers but, if trained correctly, can be just as effective. These children are raised mentally scarred but to be very efficient and disciplined killing machines.}\par
\begin{tabular}{|l|l|l|l|l|l|l|l|}
	\hline
	Cr & Int & Ins & Ch & Dex & Ag & Con & Str \\ \hline
	2 &  & 2 & -5 & 2 & 2 & 2 & 2 \\ \hline
\end{tabular}\par
\noindent\textbf{Other modifiers:} \\
\textbf{Skills:} Command +1,
Ranged Combat Training (Pistol and two others) +2,
Melee Combat Training (Any one) +2,
Melee Combat Training (Another unarmed) +1,
Security +2\\
\textbf{Abilities:} Broken\\
\textbf{Boons:} Tough as nails\\
\textbf{Banes:} Ignorant (Horrors of war) 40\\

\hrulefill
\paragraph*{Courier (8 GP)}
\textit{Delivering packages in the darker parts can be easier said than done. It takes precision, cunning and obviously a fit pair of legs. Such a character knows the darkest secrets of his city and those secrets are safe with him.}\par
\begin{tabular}{|l|l|l|l|l|l|l|l|}
	\hline
	Cr & Int & Ins & Ch & Dex & Ag & Con & Str \\ \hline
	&  &  &  &  & 5 &  &  \\ \hline
\end{tabular}\par
\noindent\textbf{Other modifiers:} \\
\textbf{Skills:} Athletics +2,
Climbing +2,
Deceive +1,
One Drive or Pilot +2,
Sleight of hand +1,
Stealth +1\\
\textbf{Abilities:} Flip Up\\
\textbf{Boons:} \\
\textbf{Banes:} \\

\hrulefill
\paragraph*{Doctor (7 GP)}
\textit{A medical professional is always welcome. He has learned not just first aid but may also be capable of performing surgery.}\par
\begin{tabular}{|l|l|l|l|l|l|l|l|}
	\hline
	Cr & Int & Ins & Ch & Dex & Ag & Con & Str \\ \hline
	& 5 &  &  &  &  &  &  \\ \hline
\end{tabular}\par
\noindent\textbf{Other modifiers:} \\
\textbf{Skills:} Appeal +1,
Medication +3,
Perception +1,
Restraint +2\\
\textbf{Abilities:} Stabilization\\
\textbf{Boons:} \\
\textbf{Banes:} \\

\hrulefill
\paragraph*{Driver (8 GP)}
\textit{Professionally trained in the art of moving a vehicle. She keeps her calm in stressful situations and gets her contractor from point A to B safely.}\par
\begin{tabular}{|l|l|l|l|l|l|l|l|}
	\hline
	Cr & Int & Ins & Ch & Dex & Ag & Con & Str \\ \hline
	&  & 5 &  &  &  &  &  \\ \hline
\end{tabular}\par
\noindent\textbf{Other modifiers:} \\
\textbf{Skills:} One Drive +2,
One Drive or Pilot +3,
One fitting Navigation +2,
Restraint +2\\
\textbf{Abilities:} \\
\textbf{Boons:} \\
\textbf{Banes:} \\

\hrulefill
\paragraph*{Engineer (0 GP)}
\textit{Constructing buildings is incredibly important, people who understand how to are vital to society.}\par
\begin{tabular}{|l|l|l|l|l|l|l|l|}
	\hline
	Cr & Int & Ins & Ch & Dex & Ag & Con & Str \\ \hline
	&  &  &  &  &  &  &  \\ \hline
\end{tabular}\par
\noindent\textbf{Other modifiers:} \\
\textbf{Skills:} Computer Operation +1,
Construction +3,
Mechanics +2\\
\textbf{Abilities:} \\
\textbf{Boons:} \\
\textbf{Banes:} \\

\hrulefill
\paragraph*{Escort (4 GP)}
\textit{Capable of talking through almost every situation and usually chosen for their looks, such people managed to turn their bodies and tongues into money.}\par
\begin{tabular}{|l|l|l|l|l|l|l|l|}
	\hline
	Cr & Int & Ins & Ch & Dex & Ag & Con & Str \\ \hline
	&  &  & 5 &  &  &  &  \\ \hline
\end{tabular}\par
\noindent\textbf{Other modifiers:} \\
\textbf{Skills:} Appeal +2,
Deceive +1,
Interrogation +1,
Quaffing +1,
Restraint +2,
Sleight of hand +1\\
\textbf{Abilities:} \\
\textbf{Boons:} \\
\textbf{Banes:} \\

\hrulefill
\paragraph*{Fence (2 GP)}
\textit{Stolen goods are not terribly uncommon, one just needs to know where to sell them to. Some of those fences are even capable of stealing their own goods.}\par
\begin{tabular}{|l|l|l|l|l|l|l|l|}
	\hline
	Cr & Int & Ins & Ch & Dex & Ag & Con & Str \\ \hline
	&  &  &  &  &  &  &  \\ \hline
\end{tabular}\par
\noindent\textbf{Other modifiers:} \\
\textbf{Skills:} Commerce +3,
Deceive +2,
Sleight of hand +1,
Stealth +1\\
\textbf{Abilities:} \\
\textbf{Boons:} Good Reputation 2\\
\textbf{Banes:} \\

\hrulefill
\paragraph*{Fixer (4 GP)}
\textit{There are a lot of people dependent on drugs, be it Flow addicted members of illegal fight clubs, corporate hackers working under Speed influence or young couples experimenting with Raze, someone has to make their substances. These people know how to make it and know what it does to one's body, in the short as well as long term. Whether they care is an individual question.}\par
\begin{tabular}{|l|l|l|l|l|l|l|l|}
	\hline
	Cr & Int & Ins & Ch & Dex & Ag & Con & Str \\ \hline
	&  &  &  &  &  &  &  \\ \hline
\end{tabular}\par
\noindent\textbf{Other modifiers:} \\
\textbf{Skills:} Appeal +1,
Brewing +1,
Chemistry +3,
Commerce +2,
Stealth +1,
Medication +2\\
\textbf{Abilities:} \\
\textbf{Boons:} \\
\textbf{Banes:} \\

\hrulefill
\paragraph*{Gambler (2 GP)}
\textit{Somehow Lady Fortuna is with this one and he made his money by gambling. In many cases such players are accused of cheating, sometimes banned from their dens and other times wanted by crime families.}\par
\begin{tabular}{|l|l|l|l|l|l|l|l|}
	\hline
	Cr & Int & Ins & Ch & Dex & Ag & Con & Str \\ \hline
	&  &  &  &  &  &  &  \\ \hline
\end{tabular}\par
\noindent\textbf{Other modifiers:} \\
\textbf{Skills:} Gambling +3,
Interrogation +2\\
\textbf{Abilities:} \\
\textbf{Boons:} Fortune's Child\\
\textbf{Banes:} Debt (2000cr)\\

\hrulefill
\paragraph*{Hacker - Black Hat (1 GP)}
\textit{Corporate hacking, blackmailing or juvenile curiosity and too much contact with cyberspace can lead someone to become a black hat. Their acts are malicious and their allegiance lies with money. More often than not they are in the process of fleeing from authorities or corporations.}\par
\begin{tabular}{|l|l|l|l|l|l|l|l|}
	\hline
	Cr & Int & Ins & Ch & Dex & Ag & Con & Str \\ \hline
	&  &  &  &  &  &  &  \\ \hline
\end{tabular}\par
\noindent\textbf{Other modifiers:} \\
\textbf{Skills:} Commerce +1,
Computer Operation +3,
Security +1,
Stealth +2\\
\textbf{Abilities:} \\
\textbf{Boons:} \\
\textbf{Banes:} \\

\hrulefill
\paragraph*{Hacker - White Hat (0 GP)}
\textit{As opposed to black hats, white hats are trying to find weaknesses in systems so they can fix them. Their actions are generally benevolent in nature but they also rarely work unpaid.}\par
\begin{tabular}{|l|l|l|l|l|l|l|l|}
	\hline
	Cr & Int & Ins & Ch & Dex & Ag & Con & Str \\ \hline
	&  &  &  &  &  &  &  \\ \hline
\end{tabular}\par
\noindent\textbf{Other modifiers:} \\
\textbf{Skills:} Computer Operation +3,
Law or Security +2,
the other +1\\
\textbf{Abilities:} \\
\textbf{Boons:} \\
\textbf{Banes:} \\

\hrulefill
\paragraph*{Investigator (4 GP)}
\textit{Be it private investigators or police officials, many unsolved cases drift through any city's archive. Some investigators are interested in these old, unsolved cases, others are committed to reducing the amount of new ones.}\par
\begin{tabular}{|l|l|l|l|l|l|l|l|}
	\hline
	Cr & Int & Ins & Ch & Dex & Ag & Con & Str \\ \hline
	&  &  &  &  &  &  &  \\ \hline
\end{tabular}\par
\noindent\textbf{Other modifiers:} \\
\textbf{Skills:} Interrogation +2,
Perception +2,
Law +2,
Security +2\\
\textbf{Abilities:} Overly Cautious\\
\textbf{Boons:} \\
\textbf{Banes:} \\

\hrulefill
\paragraph*{Laborer (1 GP)}
\textit{A lot of work can be taken over by machines. Yet sometimes a machine can't be used or is simply more expensive than a low wage laborer or a slave. This usage of human resources is not officially endorsed and even technically illegal, yet when a contract is formed (or forged...), the endeavor might be considered legitimate.}\par
\begin{tabular}{|l|l|l|l|l|l|l|l|}
	\hline
	Cr & Int & Ins & Ch & Dex & Ag & Con & Str \\ \hline
	&  &  &  &  &  &  &  \\ \hline
\end{tabular}\par
\noindent\textbf{Other modifiers:} \\
\textbf{Skills:} Athletics +2,
Medication +1,
Navigation (Ground) +1,
Restraint +2,
Survival +2\\
\textbf{Abilities:} \\
\textbf{Boons:} \\
\textbf{Banes:} \\

\hrulefill
\paragraph*{Mercenary (2 GP)}
\textit{Born for fighting and fighting for pay; the core principles of mercenaries haven't changed. Loyalty belongs to the first or highest bidder and as long as the price fits, the dirty work will be taken care of.}\par
\begin{tabular}{|l|l|l|l|l|l|l|l|}
	\hline
	Cr & Int & Ins & Ch & Dex & Ag & Con & Str \\ \hline
	&  &  &  &  &  &  &  \\ \hline
\end{tabular}\par
\noindent\textbf{Other modifiers:} \\
\textbf{Skills:} Intimidation +1,
Ranged Combat Training (Pistol and one other) +2,
Melee Combat Training (Any one) +1,
Security +2\\
\textbf{Abilities:} \\
\textbf{Boons:} Stoic or Tough as nails\\
\textbf{Banes:} Bound by duty or Wanted II\\

\hrulefill
\paragraph*{Performer (5 GP)}
\textit{Performers are paid to sing, dance, balance and whatever might be entertaining to watch or listen to. Payment is usually slim or depending on the circumstances as simple as continuing to live.}\par
\begin{tabular}{|l|l|l|l|l|l|l|l|}
	\hline
	Cr & Int & Ins & Ch & Dex & Ag & Con & Str \\ \hline
	&  &  & 3 &  & 3 &  &  \\ \hline
\end{tabular}\par
\noindent\textbf{Other modifiers:} \\
\textbf{Skills:} Acrobatics +2,
Appeal +1,
Athletics +1,
Education: Etiquette +2,
Voice Athlete +2\\
\textbf{Abilities:} \\
\textbf{Boons:} \\
\textbf{Banes:} \\

\hrulefill
\paragraph*{Shark (0 GP)}
\textit{Some people have enough money to make more money by loaning money and having it paid back at ludicrous interest rates. These people are usually versed in all forms of trade and bending the law.}\par
\begin{tabular}{|l|l|l|l|l|l|l|l|}
	\hline
	Cr & Int & Ins & Ch & Dex & Ag & Con & Str \\ \hline
	&  &  &  &  &  &  &  \\ \hline
\end{tabular}\par
\noindent\textbf{Other modifiers:} \\
\textbf{Skills:} Commerce +3,
Intimidation +1,
Law +2\\
\textbf{Abilities:} \\
\textbf{Boons:} \\
\textbf{Banes:} \\

\hrulefill
\paragraph*{Soldier (3 GP)}
\textit{Harsh training and incredible equipment make them a force to be reckoned with. Most are only employed to defend borders but the best of them are chosen to sabotage terrorists, either by infiltration or via an orbital drop in a strike suit.}\par
\begin{tabular}{|l|l|l|l|l|l|l|l|}
	\hline
	Cr & Int & Ins & Ch & Dex & Ag & Con & Str \\ \hline
	&  &  &  &  &  &  &  \\ \hline
\end{tabular}\par
\noindent\textbf{Other modifiers:} \\
\textbf{Skills:} Command +2,
Ranged Combat Training (Pistol and one other) +2,
Melee Combat Training (Any one and another unarmed) +1,
Security +2\\
\textbf{Abilities:} \\
\textbf{Boons:} Stoic or Tough as nails\\
\textbf{Banes:} Belief in authority 30
Bound by duty\\

\hrulefill
\paragraph*{Terrorist (3 GP)}
\textit{There are many reasons to be unhappy with the country's situation but only few take such radical measures. These people have reached the point of either complete devotion to a cause and an according enemy or of having absolutely nothing to lose anymore.}\par
\begin{tabular}{|l|l|l|l|l|l|l|l|}
	\hline
	Cr & Int & Ins & Ch & Dex & Ag & Con & Str \\ \hline
	&  &  &  &  &  &  &  \\ \hline
\end{tabular}\par
\noindent\textbf{Other modifiers:} \\
\textbf{Skills:} Chemistry or Construction +2,
Deceive +1,
History or Philosophy +2,
Melee Combat Training (Any one) +1,
Ranged Combat Training (Any one) +1,
Security +1\\
\textbf{Abilities:} \\
\textbf{Boons:} Social Chameleon\\
\textbf{Banes:} Rebel 30\\

\hrulefill
\paragraph*{Trafficker (3 GP)}
\textit{Anything can be traded, so long as you aren't caught. A particularly hot good is giving the term "human resources" a new meaning. People like this are cunning, cold and the toughest tradesmen.}\par
\begin{tabular}{|l|l|l|l|l|l|l|l|}
	\hline
	Cr & Int & Ins & Ch & Dex & Ag & Con & Str \\ \hline
	&  &  &  &  &  &  &  \\ \hline
\end{tabular}\par
\noindent\textbf{Other modifiers:} \\
\textbf{Skills:} Appeal +1,
Commerce +2,
Deceive +2,
Intimidation +1,
Security +1\\
\textbf{Abilities:} Broken\\
\textbf{Boons:} \\
\textbf{Banes:}

\hrulefill
\paragraph*{Urchin (0 GP)}
\textit{A child that lost its guardians and maybe its living place very early in life needs to learn a lot of things in a short amount of time in order to survive. There is not much time for youth nor slacking off and it will understand that acting slowly is very dangerous.}\par
\begin{tabular}{|l|l|l|l|l|l|l|l|}
	\hline
	Cr & Int & Ins & Ch & Dex & Ag & Con & Str \\ \hline
	&  &  &  &  &  &  &  \\ \hline
\end{tabular}\par
\noindent\textbf{Other modifiers:} \\
\textbf{Skills:} Appeal +1,
Athletics +1,
Climbing +1,
Deceive +1,
Dodge +1,
Perception +1,
Restraint +1,
Sleight of hand +1,
Stealth +1,
Survival +1\\
\textbf{Abilities:} \\
\textbf{Boons:} \\
\textbf{Banes:} \\

\hrulefill
\paragraph*{Wastelander (1 GP)}
\textit{Not everyone grows up in flashing cities. Many people grow up in the far outskirts, places where barely anything grows and that are by their very existence hostile towards life, yet somehow small communities arise in them again and again. 
	Growing up in such places takes a toll on someone and it requires quickly learning a very particular set of skills.}\par
\begin{tabular}{|l|l|l|l|l|l|l|l|}
	\hline
	Cr & Int & Ins & Ch & Dex & Ag & Con & Str \\ \hline
	&  &  &  &  &  &  &  \\ \hline
\end{tabular}\par
\noindent\textbf{Other modifiers:} \\
\textbf{Skills:} Cooking +2,
Perception +2,
Survival +2\\
\textbf{Abilities:} Trained sense (any one)\\
\textbf{Boons:} \\
\textbf{Banes:} \\

\hrulefill
\paragraph*{Technician (8 GP)}
\textit{}\par
\begin{tabular}{|l|l|l|l|l|l|l|l|}
	\hline
	Cr & Int & Ins & Ch & Dex & Ag & Con & Str \\ \hline
	& 3 &  &  & 3 &  &  &  \\ \hline
\end{tabular}\par
\noindent\textbf{Other modifiers:} \\
\textbf{Skills:} Computer Operation +2,
Construction +1,
Mechanics +3,
Perception +1,
Security +1,
two Educations +1\\
\textbf{Abilities:} \\
\textbf{Boons:} \\
\textbf{Banes:} \\

\hrulefill
\paragraph*{Street Fighter (8 GP)}
\textit{}\par
\begin{tabular}{|l|l|l|l|l|l|l|l|}
	\hline
	Cr & Int & Ins & Ch & Dex & Ag & Con & Str \\ \hline
	&  &  &  &  &  & 3 & 3 \\ \hline
\end{tabular}\par
\noindent\textbf{Other modifiers:} \\
\textbf{Skills:} One armed Melee Combat Training +2,
Melee Combat Training (Striking or Grappling) +2,
Melee Combat Training (the other) +1,
Restraint +2\\
\textbf{Abilities:} Melee Maneuvers worth 400 XP\\
\textbf{Boons:} \\
\textbf{Banes:} \\

	\section{Boons}
\begin{multicols}{2}
\boon[each]{Adaptable Circulation}{Medical Toughness is increased by 1. Can be taken multiple times.}{2}
\boon{Ambidextrous}{The character can use either hand equally good. Dual Wielding and Maneuver: Flurry only cost half. When combined with Dual Wielding, removes penalties for dual wielding melee weapons and pistols completely.}{12}
\boon{Aptitude}{Adds +5 to a characteristic which doesn't count against investment limits. Can be taken multiple times but only once per characteristic.}{10}
\boon{Balance}{Tests to prevent falling over are at a +20. Improves on Steadfast and doesn't stack.}{10}
\boon{Battle Trance}{The character can enter a battle trance. This takes two actions and causes one level of exhaustion. The character becomes frenzied but can identify friend and foe and return to normal by spending two actions.}{12}
\boon[each]{Big Spender}{The character gains additional equipment worth up to \(5 * Ch * level\) of this boon. The money can only be spent at character creation and unused credits are wasted. Equipment may as always be restricted by the GM.}{3}
\boon{Born Hero}{The character counts as having all prerequisites for the Abilities Heroic, On your feet! and Personified Inspiration. These abilities only cost half as much.}{6}
\boon[+ new background]{Broad Education}{The character may choose a second background and gain its skill bonuses as well.}{17}
\boon{Combat Academic}{All maneuver abilities only cost 75\% of their original cost. Penalties from missing combat training are halved.}{12}
\boon{Dark Vision}{The character only takes half penalties in low light but normal penalties in complete darkness.}{10}
\boon{Diver's Lung}{The character may hold her breath twice as long.}{4}
\boon{Euphonia}{The character has a beautiful voice. He gains a +10 to social interaction tests when the other party can hear him. This also makes Voice Athlete a basic skill.}{10}
\boon[each]{Formal Education}{Grants 50 XP to be spent only on educations. Can be taken multiple times.}{1}
\boon{Fortune's Child}{Every session the character may re-roll one of her dice rolls or one that she is directly affected by.}{10}
\boon{Giant}{The character becomes two categories larger. He is easier to hit, has more melee range, is faster and more imposing.}{10}
\boon[each]{Good Reputation}{The character has some sort of good reputation, either from well known deeds or from known genetic markers. If he is talking to someone who knows of this, he gains a bonus equal to 5 times the boon's level. Can be taken up to level 10.}{2}
\boon{Iron Resilience}{Degradation from wounds is reduced by one step.}{10}
\boon{Large}{The character becomes a category larger. He is easier to hit, has more melee range, is faster and more imposing.}{5}
\boon{Looker}{The character is beautiful. She gains +5 to social interaction tests. Not compatible with Smoking, Ugly and Physically repulsive.}{5}
\boon{Low Libido}{The character has very little interest in getting frisky. This makes it near impossible for him to be involved in seduction attempts on either side, adding a -25 penalty to seduction tests targeting him or being performed by him.}{5}
\boon[each]{Natural Armor}{The character has natural armor covering her skin. Adds +1 armor to every location and can be taken multiple times.}{6}
\boon[each]{Nerves Of Steel}{Rayleigh Index is increased by 1. Can be taken multiple times.}{2}
\boon{Night Vision}{The character only takes half penalties in complete darkness and no penalties in low light.}{18}
\boon{Nimble}{Speed is increased by 1 and dodging penalties are reduced by 10.}{10}
\boon{Outstanding Sense}{Grants +20 to tests based on the chosen sense. Counts as a Trained Sense for the purposes of prerequisites. Not compatible with Impaired Sense for the same sense.}{7}
\boon{Personal Possession}{The character owns a very special, valuable (up to 800cr) item with not only large material but also high sentimental value. This may include sets of items such as two bionic arms but are up to GM discretion even more than usual.}{8}
\boon{Pleasant Smell}{If the target of a character's charming social test can smell the character, the character gains +10. Not compatible with Terrible odor.}{7}
\boon{Poison Resistance}{Grants +20 to resist the effects of poison or venom.}{4}
\boon{Privileged Access}{The character gains a 25\% discount for license-free equipment made by a certain company. During character creation this bonus is 50\% instead to account for sales in the past. Requires Good Reputation with that company / salesman.}{7}
\boon{Quick Regeneration}{Grants +30 to natural regeneration tests.}{10}
\boon{Smoking}{The character is extremely physically attractive or otherwise imposing. He gains +15 to social interaction tests but -20 to hide because his appearance is extremely memorable and outstanding. Not compatible with Looker, Ugly and Physically repulsive. If the character already is a Looker, this can replace it for 7 GP.}{12}
\boon{Social Chameleon}{Halves all penalties from unknown cultures and improves NPCs' default disposition to the character.}{12}
\boon{Stoic}{Halves all penalties and double all bonuses on fear tests.}{18}
\boon{Superior Balance}{Tests to prevent falling over are at a +30. Improves on Balance and doesn't stack.}{16}
\boon{Thick Skull}{Tests against being stunned gain a +10 and all penalties are halved.}{6}
\boon{Tough As Nails}{The character can continue acting while bleeding until completely bled out.}{12}
\boon{Unbridled Rage}{Any check to enter a berserk or frenzy state is at a +20 and any check to resist or return from such a state is at a -20.}{7}
\boon{Unrivaled Flexibility}{Agile without peer, the character gains a +30 when attempting to release himself from holds or to escape from being tied up.}{15}
\boon[+ background]{Veteran}{The character is older and more experienced. He gains his background's skill bonuses again, up to the skill's maximum.}{17}
\end{multicols}
	\pagebreak
\section{Banes}
\label{sec:banelist}
\vspace{4mm}
\begin{multicols}{2}
\bane{Addiction}{The character is addicted to a substance. A week of withdrawal will cause a level of exhaustion every day which are not naturally regenerated anymore. Other exhaustion still regenerates naturally until two weeks of withdrawal. This only applies to uncontrolled withdrawal, not to therapy.}{10}{}
\bane{Agoraphobia (DC)}{The character is afraid of wide, open places.}{1}{per 5}
\bane{Annoying Voice}{The character has an annoying voice. This grants -10 to social tests that involve talking.}{5}{}
\bane{Arrogance (DC)}{The character believes to be superior to others.}{1}{per 5}
\bane{Backwoods Customs}{Even though most cultures have melted, the character has grown up outside the core of society. He has internalized customs that some would call barbaric.}{5}{}
\bane{Bad Reputation}{The character has some sort of bad reputation, either from well known deeds or from known genetic markers. If he is talking to someone who knows of this, he gains a penalty equal to 5 times the boon's level. Can be taken up to level 10.}{2}{per level}
\bane{Bathophobia}{The character is afraid of depths. This may trigger at the bottom of the ocean just as much as it could in a massive canyon.}{1}{per 10}
\bane{Belief in Authority (DC)}{The character believes in authority and those who represent it.
	Not compatible with Rebel.}{1}{per 10}
\bane{Berserk Rush}{When the character is thoroughly enraged
	or he is influenced by substances that boost courage,
	he falls into a frenzy.
	Restraint tests to resist frenzy suffer a -20.}{15}{}
\bane{Blind}{The character can't see. This is due to nervous or brain damage and can't be fixed via augments.}{40}{}
\bane{Blood Lust (DC)}{Lust for bloody combat drives the character. She likes taking the time to make her victims suffer and might end up causing trouble for herself and her group due to this.}{1}{per 10}
\bane{Bound by Duty}{The character is obliged to follow a certain person's or group's commands.
	If she disobeys, there might be severe consequences.}{12}{}
\bane{Bound by Principle}{The character is bound by her own principles. They are central moral values that may rarely be bent and almost never broken. If they are, doubts will follow the character for at least a week, inferring a -40 penalty to anything requiring focus. By GM discretion these penalties might function slightly differently depending on the situation.}{12}{}
\bane{Caring (DC)}{The character is overly caring for his peers. He will take great risks to help others.}{1}{per 5}
\bane{Claustrophobia}{The character is afraid of small, narrow places.}{3}{per 10}
\bane{Clumsy}{Critical failures occur on a roll of 96 and above.}{20}{}
\bane{Colorblind}{The character is incapable of seeing colors. When she wants to identify something that is only discernible by color, she takes a -20 penalty. This also counts for ranged attacks against large or smaller targets.}{5}{}
\bane{Curiosity (DC)}{When there is something to know, the character will want to know, whether it is of concern or not.}{1}{per 5}
\bane{Cursed by Fortuna}{When a bad event hits the group, this character is much more likely to be hit in particular. Also once per session the GM may declare a success to be a failure instead.}{15}{}
\bane{Debt}{The character is indebted to someone. The loaner will require regular payments or the character will suffer consequences depending on their relationship and the type of people they are.}{1}{per 500 cr}
\bane{Delusions of Grandeur (DC)}{The character tends to overstate and overestimate her own ability and take on monumental tasks with delusional confidence.}{1}{per 5}
\bane{Destitute}{The character has made some bad choices regarding their money. While not knee-deep in debt, the character does not have fund and currently lives from paycheck to paycheck. The character starts with only half the cash. Cannot be combined with Big Spender.}{5}{}
\bane{Dietary Rules}{The character will only eat certain things due to personal reasons. Whenever the character is forced to eat something that does not align with his values, e.g. as to not starve to death, this infers penalties equal to the bane's value.}{5}{}
\bane{Directionally Challenged}{The character has extreme difficulty distinguishing left from right. This makes navigation quite difficult: navigation and navigation-related tests - like tests made in a chase - are at a -10.}{5}{}
\bane{Dwarfish}{The character is tiny. He is harder to hit, has less melee range and is slower. He is also rarely taken seriously.}{8}{}
\bane{Envy (DC)}{When another has something that the character does not, she will overstep boundaries to get some of her own.}{1}{per 10}
\bane{Evil Monologue}{The character tends to talk to herself. This may lead to embarrassing social encounters or even to spilling information about secret plans.}{5}{}
\bane{Fear of Heights (DC)}{The character is followed by an everlasting fear of falling.}{3}{per 10}
\bane{Flimsy}{Medical Toughness reduced by 1. Can be taken multiple times.}{2}{each}
\bane{Food Restrictions}{Due to mutation or specific engineering, the character can only consume certain types of food. Anything else leaves the body undigested; often times violently.}{5}{}
\bane{Forgetful}{The character tends to forget even the most important details.
	Any checks made to remember specific details suffer a -20.}{5}{}
\bane{Gambler (DC)}{The character will not pass up on a good bet. When he can play a more or less fair game for something, he has a hard time passing up the chance.}{1}{per 5}
\bane{Greed (DC)}{Money makes the world go round. Opportunities to make your world go just that bit rounder are very hard to pass by.}{1}{per 5}
\bane{Hallucinations}{Hallucinating characters live in a notably vastly different reality, e.g. thinking that milk is a deadly poison, that he is haunted by demons only vulnerable to supersonic cheese and so on. Every person will notice and this will infer penalties as per GM discretion.}{10}{}
\bane{Hesitant}{The character suffers a -3 to initiative.}{5}{}
\bane{Hungry I/II/III}{The character needs 1.5/2/3 times as much food.}{5}{per level}
\bane{Impaired Sense}{One of the character's senses functions improperly, incurring -20 on checks based on that sense. This may be offset with augments but never restored to normal levels. Not compatible with Outstanding Sense for the same sense.}{5}{}
\bane{Impatient (DC)}{The character does not like waiting. Tasks that take a long time of not doing much will be severely harder.}{1}{per 5}
\bane{Impulsive}{The character tends to act on stupid decisions in the heat of the moment. When such an idea comes up, it takes a -20 intelligence check to resist acting on it.}{5}{}
\bane{Insomnia}{Getting a good night's rest is hard for the character. In roughly a quarter of all nights (whether randomly or following a pattern) the character takes a -30 penalty to regenerating naturally.}{7}{}
\bane{Laziness (DC)}{The character is a slob and doesn't put more effort into things than absolutely necessary.}{1}{per 5}
\bane{Light Sensitivity}{The character hates the light.
	Every action taken in light is at a -30 penalty.
	Sufficient protection from the light is generally enough work and requires enough equipment that the sheer bulk grants similar penalties.}{10}{}
\bane{Ignorant (DC)}{The character has a hard time understanding a certain topic.
	This topic should be specified when the Bane is chosen.
	All tests checking knowledge of that topic suffer a penalty equal to this disabling characteristic.}{1}{per 10}
\bane{Naive (DC)}{The character easily trusts strangers. Resisting deception and persuasion is at a penalty equal to the Bane.}{1}{per 5}
\bane{Necrophobia (DC)}{The character has fear of death and dead creatures and will be unable to focus properly when on battlefields, murder or slaughter scenes or graveyards.}{1}{per 5}
\bane{Needy (DC)}{The character's strong physical desires become distracting to everyday life.}{1}{per 10}
\bane{Nervous}{Rayleigh Index reduced by 1. Can be taken multiple times.}{2}{each}
\bane{Neurotically Hygienic (DC)}{The character has a need to keep clean.}{1}{per 5}
\bane{Night Blind}{Doubles all penalties for low light.}{5}{}
\bane{No Tactile Sense}{Anything requiring sense of touch takes a -40 penalty, this includes the usage of any weapon with a trigger or a blade.}{15}{}
\bane{Nyctophobia (DC)}{The character has a strong fear of the dark.}{2}{per 5}
\bane{Obese}{The character is heavily overweight. This comes with multiple drawbacks like a -20 penalty to strength based checks that have to do with body weight (e.g. climbing), -3 to medical toughness, as well as stigmatization and other drawbacks by GM discretion.}{15}{}
\bane{One-Armed}{The character has lost an arm and due to nerve damage can't accept replacements.}{15}{}
\bane{One-Eyed}{The character has lost an eye and due to nerve damage can't accept replacements.}{5}{}
\bane{One-Handed}{The character has lost a hand and due to nerve damage can't accept replacements.}{10}{}
\bane{One-Legged}{The character has lost a leg and due to nerve damage can't accept replacements.}{25}{}
\bane{Paralyzed Leg}{Due to irreversible nervous damage the character can barely move one leg. This reduces speed by 2 and grants penalties around -15 to tests as per GM discretion.}{15}{}
\bane{Personal Enemy}{An important person or group has set its eyes on the character and will take measures to foil his plans.}{7}{}
\bane{Physically Repulsive}{The character is staggeringly ugly. This infers a -20 penalty to social tests when the character can be seen. Not compatible with Smoking, Looker and Ugly.}{12}{}
\bane{Playful (DC)}{The character loves playing. That can include simple things like chess or video games, but also "games" of predator and prey.}{1}{per 5}
\bane{Portly}{The character is stocky or otherwise slower than her peers. Her speed is reduced by 1.}{5}{}
\bane{Prejudice (DC)}{The character is biased against a certain group of people and has a hard time interacting with them.}{1}{per 5}
\bane{Rebel (DC)}{The character has a hard time trusting authority and tends to revolt. Not compatible with Belief in Authority.}{1}{per 10}
\bane{Righteous (DC)}{Some laws are absolute and they must be followed. The character will work against any injustice to the order she believes in.}{1}{per 5}
\bane{Scrooge (DC)}{Spending money is painful.}{1}{per 10}
\bane{Second-class Citizen}{The character is oppressed for one reason or another. This may e.g. be due to stigmatization of a part of his lifestyle or being born with extremely few resources. In any case this will have consequences in social situations and encounters with law enforcement.
	Cannot be taken by Vat-grown.}{5}{}
\bane{Sensitive Smell (DC)}{When around something that smells particularly nasty, he takes penalties to everything that requires concentration or a healthy stomach.}{3}{per 10}
\bane{Short}{The character is small. He is harder to hit, has less melee range and is slower. He is also rarely taken seriously.}{5}{}
\bane{Shortcoming}{A characteristic, that hasn't been invested in, is reduced by 5.}{7}{}
\bane{Sickly}{Grants -15 penalty to resist disease.}{2}{}
\begin{exampleblock}
	Banes like \emph{Sickly} may have no chance to come up in game.
	In this case such Banes should be disallowed by the GM.
\end{exampleblock}
\bane{Slave Mentality (DC)}{The character does not like making his own decisions. Such situations put a lot of stress on the character.}{1}{per 5}
\bane{Slow Healer}{The character only heals naturally at a -20 penalty and tests taken to heal this character are at a -10.}{10}{}
\bane{Speech Disorder}{The character has some sort of speaking disorder. What it may be, the character gains a -10 to tests requiring to speak and takes longer to finish sentences which might cause problems in stressful situations like combat or a heated chase.}{7}{}
\bane{Split Consciousness}{Multiple consciousnesses fight over control of a single body. The character is represented by multiple character sheets, sharing physical characteristics, boons and banes, health conditions and their Athletics level. Mental characteristics boons and banes, conditions and all other skills are individual to every consciousness.}{10}{per additional consciousness}
\bane{Squeamish (DC)}{The character takes a penalty to tests to resist fear and panic.}{2}{per 5}
\bane{Technophobia (DC)}{The character is suspicious of technological advancements, especially solutions to things that could easily be done by humans.}{3}{per 10}
\bane{Temperamental (DC)}{Small things may make the character unreasonably angry.}{3}{per 10}
\bane{Terrible Odor}{The character has bad smell. This grants -10 to social tests when the character is in smelling distance and hiding from watchdogs is essentially impossible. Not compatible with Pleasant smell.}{5}{}
\bane{Trigger Phrase}{Sleeper agents were rumored to have trigger phrases carved into their mind. For the longest time that was purely a myth but now many vat-grown are programmed with such a phrase to force certain behavior. When the phrase is heard by the character, all tests to resist persuasion and commands suffer at least a -40 penalty.}{7}{}
\bane{Trouble Magnet (DC)}{Some people actively look for trouble. This character has the tendency, willingly or not, to provoke and attract the wrath of others.}{1}{per 10}
\bane{Ugly}{The character is unshapely. This grants -10 to social tests while the character is being seen. Not compatible with Smoking, Looker and Physically repulsive.}{5}{}
\bane{Uneducated}{The character has not learned much and loses 50 XP. Can be taken multiple times but XP may never be negative.}{1}{per level}
\bane{Vanity (DC)}{The character is posh to an extreme. He might refuse tube food, simple and cheap clothing or getting dirty.}{1}{per 5}
\bane{Vengeful (DC)}{The character does not forget. When someone wronged him, he will make sure to pay them back tenfold.}{1}{per 5}
\bane{Wanted I-V}{The character is being looked for by an organization. This can be due to criminal record, corporate espionage or involvement in gang wars. The first step roughly describes being the target of a small mercenary or criminal group, a level 3 target would be a low-priority military target and a level 5 target is on LEGION's radar.}{5}{per level}
\end{multicols}


	\section{Prefab Characters}
\label{sec:pfchars}

\def\pfcname{The Pack Man}
\subsection{Low, "\pfcname"}
\vspace{10mm} %fixes a label jumping between two coordinates at the bottom of the enumeration
\subsubsection{Basic Questions}
\begin{enumerate}
	\setlength\itemsep{-8mm}
	\item I'm a short guy. Chinese descent. Short dark hair. Really hate fluttering coats, prefer something tighter.
	\item Usually I like not to leave an impression. Most people would probably call me shady.
	\item Grew up in the city sprawl. Liked climbing, mainly into abandoned buildings. Never got caught.
	\item Had to break most ties. Would like to see some of my old gang again. Maybe some day. The one tie I would like to break but can't is to "Rust" Golotov. Owe him a whole bunch of dough.
	\item Stellar people. They make up for the little things I'm missing. Ideal support.
	\item Didn't come around much. Saw my fair share of the city though.
	\item Love myself some chrome, would love some more.
		Too expensive though and cheap goods don't scratch that itch.
		Also paints a massive target on your ass,
			running around with fancy gear like that.
	\item Took a bunch when I was younger. Trying to stay clean now, shit breaks you.
	\item This is where I'd say I love my life. But time again I almost kick it just for the rush.
	\item Would love to pay off my debt one day, leave my past behind.
	\item Golotov will poison me one day, I just know it. Or murk me in my sleep. Can't run, can't hide. Really got me by the balls.
	\item I'm flexible. Gotta do what it takes to do the right thing.
	\item Can't be too careful or you get bitten. Takes a while to earn my trust.
	\item You take it as easy as you give it. What's the big deal?
	\item Annoying things. They see you, they smell you - they make my job that much harder.
	\item I mean... girls? Tech'd up. Proud.... kinda like Wire.
	\item Anything works, really. But I still like my booze.
	\item I just know it ends in the morning.
	\item People like me don't live through dark secrets.
	\item A bit secretive perhaps. Careful with people, but not myself cuz I don't have to be.
\end{enumerate}

\luaimport{prefabs/prefab-stats-\pfcname.csvin}{prefabchar.tpl}{\pfcname}

\pagebreak
\def\pfcname{Wire}
\subsection{\pfcname}
\subsubsection{Basic Questions}
\begin{enumerate}
	\setlength\itemsep{-8mm}
	\item Well... guess you'd call me a cat girl. The small ears, soft and light hair, the tail. I have very light skin, pronouncing the permanent rings under my eyes.
	\item People will probably think me a common girl who can't really think ahead. You know... the cat parts and all.
	\item I grew up in the middle of the city. Went to a tech collage, got myself some good side gigs but... on one job I got blown up. Still don't know where the blast came from. I just know that I needed most of my cash to pay for fix ups.
	\item Well, yeah. 2 of them in fact - Doc and CJ. I've worked with them before and I will again for as long as I can.
	\item Doc Jones patched be back together when... you know... anyway, CJ has my back now and I finally feel safe. I can't really read that courier guy but he seems decently chill when it's just us.
	\item I've seen... server rooms? Very little to actually talk about. I saw as many places as security measures.
	\item Gotta love my new parts. I don't like chrome for the augs' sake. But the things I got are just amazing.
	\item They have great effect, could kick it all day... and sometimes I do... I just work better that way, okay?
	\item I wouldn't risk my life willingly, however... sometimes I don't think to the very end, when my curiosity gets the better of me.
	\item I wanna break into a mega-corp for no reason other than to see what's going on in there. Just to prove I can.
	\item The unknown depths below. I hate deep, dark holes.
	\item I don't want to hurt anyone! But, you know, sometimes people are bad and leave you with no choice. Better be safe than sorry.
	\item I like meeting new people, seeing new ways. Each brings some different tragedy.
	\item Again, I don't want to hurt anyone. I prefer not getting into conflict in the first place. If I have to, I want to cause as little damage as necessary.
	\item Pets are cute but anything else is either disgusting or dangerous.
	\item I like machines. I like things with purpose. Also, brains in a guy are neat.
	\item Energy drinks and crystal for breakfast.
	\item It's like trying to decrypt complex code and all you got is a rain-soaked circuit board. But it's the most pleasant glitch I know.
	\item I don't work right without my juice, you know? I don't like it but I sure as hell won't change it.
	\item I'm a smart but physically weak body modder. Knowledge is power and I want it all.
\end{enumerate}

\luaimport{prefabs/prefab-stats-\pfcname.csvin}{prefabchar.tpl}{\pfcname}

\pagebreak
\def\pfcname{Doc Jones}
\subsection{\pfcname}
\subsubsection{Basic Questions}
\begin{enumerate}
	\setlength\itemsep{-8mm}
	\item I'm a scientist and I wear it on my sleeve. Because I'm only biologically 22, I had to dye my hair gray and my beard white, and I comb my hair up so onlookers would immediately understand my brilliance.
	\item I've been called mad scientist, which of course is preposterous. Curious and engaged perhaps, but I digress.
	\item I was grown and educated in a vat. They say I didn't come out right but I'll show 'em... I'll show 'em all!
	\item I... never really saw people outside the corp. And most patients didn't make it.
	\item This is the best way to make a name for myself. Also, they need my help.
	\item My lab and the ER, really nothing past it. I DID however visit a lecture hall once!
	\item Very useful, very useful indeed. Considering I was purpose-made in the first place, why would I not aug every part I can?
	\item Makes you work longer and more efficient. But my opinion as a medical professional should be clear, right?
	\item No, no, no. The medic dies last, that's just common sense.
	\item I'd love to have my own lab. Like, my \textit{own} own.
	\item I have a bad feeling about Golotov. He wouldn't just let me go, that's not the type of man he is. He probably has some control mechanism left in place and the feeling won't let me go.
	\item Yes, yes, it's always about morals. But what are morals really if not your upbringing imprinting on your mind? So I'm not evil - I'm just different and your argument is invalid.
	\item Strangers usually want something from me. At best I will be professional, at worst dismissive. Depends on how much cash they bring.
	\item I swore an oath to preserve... \textit{however} there is a lot to be learned people suffering and dying and who would let that go to waste?
	\item I love animals. They are premium test subjects.
	\item Beauty is such a strange concept, so superficial. So let's say health to get it over with.
	\item Only the most elegant. Like... hot dogs and cheap wine. Or tap water.
	\item I- I'm not good with these flowery mumbo jumbo. It's just a nice feeling.
	\item I've grown up in a vat. What do you think? I didn't have the time for a dark secret.
	\item I'm a proud medical scientist, a bit too invested into surgery. I'm also born to be compliant and don't like spending money.
\end{enumerate}

\luaimport{prefabs/prefab-stats-\pfcname.csv}{prefabchar.tpl}

\pagebreak
\def\pfcname{CJ}
\subsection{\pfcname}
\subsubsection{Basic Questions}
\vspace{4mm}
\begin{multicols}{2}
\begin{enumerate}
	\setlength\itemsep{-9mm}
	\item Black man. Short, dark hair. Muscular. Tall.
	\item Look like a bouncer at best, killer at worst. People around me look important.
	\item Street fighter down town. Body guard for a while. Special training in stunt driving.
	\item Bunch of fighters and cops. A cute nurse.
	\item Reliable. Skilled. Willing.
	\item Few samey cities. Many corpo meetings.
	\item Like my own parts. Know what I'm at.
	\item Unpleasant side effects. Can't argue with the results though.
	\item My job. Loyalty.
	\item My own tank.
	\item Getting attached.
	\item Hard job. No time for morals.
	\item No.
	\item I'm getting paid to protect very important people. Life is exactly what I preserve.
	\item Like dogs. They're loyal.
	\item Armor. Or armored vehicles.
	\item Anything quick.
	\item Just something that gets in my way. Gotta block off.
	\item Got my bad rep when I beat one of my bosses. Was selling off kids out of town. Still don't know if I did the right thing betraying him.
	\item Direct and stern.
\end{enumerate}
\end{multicols}
\vspace{-18mm}

\luaimport{prefabs/prefab-stats-\pfcname.csvin}{prefabchar.tpl}{\pfcname}

\pagebreak

		\section{Fear}
\label{feartable}
\begin{tabularx}{\textwidth}{|l|X|}
	\hline
	2-3    & Shivering, the character takes a -10 penalty on his tests until he passes a courage check. He can spend one action to attempt this check.                                                                                                                                                                                                                                            \\ \hline
	4-5   & Severely shaken, the character takes a -15 penalty on all tests for the rest of the encounter or scene and can't use maneuvers or transform actions.                                                                                                                                                                                                                                 \\ \hline
	6-7   & Dumbstruck by fear, the character freezes in place and can’t act until he passes a courage check at the end of one of his turns. This check begins at the same penalty as the initial one and gets a cumulative +10 for every previous attempt. He may take reactions to defend himself or run away (at an increased risk of stumbling as quickly avoiding obstacles is “acting”...) \\ \hline
	8-9   & Frozen by terror, the character is completely unable to move until he passes a courage check at the end of one of his turns. This check begins at the same penalty as the initial one and gets a cumulative +5 for every previous attempt. Courage checks are allowed 1D5-1 rounds after fear strikes. He may take reactions to defend himself.                                      \\ \hline
	10-11  & Overcome by horror, the character loses consciousness for 1D10-CrB minutes, minimum of 1.                                                                                                                                                                                                                                                                                            \\ \hline
	12-13  & Pure horror pierces the character’s heart. His courage is reduced by 2D10+5 for a week, recovering by 1 each day afterwards. He takes a second shock effect at a value of D100/2.                                                                                                                                                                                                    \\ \hline
	14-15  & Stuck in their worst nightmare, the character falls comatose for a day.                                                                                                                                                                                                                                                                                                           \\ \hline
	16-19  & Fear breaks the character’s mind. He falls into a coma for half a week and his memory of the last 2D10 days is erased. After such a traumatic experience, the character likely develops a mental disorder (a Bane worth 2-3 GP). Player and GM should discuss what this might be.                                                                                                    \\ \hline
	20+    & The character dies from sudden cardiac arrest.                                                                                                                                                                                                                                                                                                                                       \\ \hline
\end{tabularx}

	\section{Sizes}
\label{sizestable}
\begin{tabularx}{\columnwidth}{|l|r|r|X|}
	\hline
	\ul{Name} & \ul{Hit} & \ul{MS} & \ul{Example} \\ \hline
	Miniscule & -40 & -4 & a knife, pistol or mobile phone \\ \hline
	Puny & -20 & -2 & a limb, rifle or sword \\ \hline
	Tiny & -10 & -1 & a child \\ \hline
	Small & -5 & -0.5 & small people, tall children \\ \hline
	Average & +/-0 & +/-0 & most humans \\ \hline
	Tall & +5 & +0.5 & imposing vat-grown soldiers and bodyguards \\ \hline
	Huge & +10 & +1 & horses, bikes \\ \hline
	Hulking & +20 & +2 & cars \\ \hline
	Enormous & +40 & +4 & buses, trucks \\ \hline
\end{tabularx}

	\section{Modifiers}
\label{situationalmodifiers}
\subsection{General}
\begin{center}
	\begin{tabular}{|l|l|l|}
		\hline
		\ul{Situation}         & \ul{Modifier} \\ \hline
		Difficulty: Difficult   & -10            \\ \hline
		Difficulty: Hard        & -20            \\ \hline
		Difficulty: Very Hard   & -30            \\ \hline
		Difficulty: Grueling    & -40            \\ \hline
		Difficulty: Punishing   & -50            \\ \hline
		Difficulty: Hellish     & -60            \\ \hline
		Difficulty: Challenging & +0             \\ \hline
		Difficulty: Ordinary    & +10            \\ \hline
		Difficulty: Routine     & +20            \\ \hline
		Difficulty: Easy        & +30            \\ \hline
		Difficulty: Simple      & +40            \\ \hline
		Difficulty: Elementary  & +50            \\ \hline
		Difficulty: Trivial     & +60            \\ \hline
	\end{tabular}
\end{center}
\subsection{Melee Combat}
\begin{tabularx}{\textwidth}{|X|l|l|}
	\hline
	\ul{Situation} & \ul{Modifier} & \ul{Keyword} \\ \hline
	The character attacks below his optimum & -20 & Distance \\ \hline
	The character is attacking slightly outside of reach. & -20 & Distance \\ \hline
	The character is fighting in strong winds, heavy rain or on slippery ground. & -10 & Environment \\ \hline
	The character tries to attack or parry while prone. & -10 & Environment \\ \hline
	The character is fighting a character in mist, fog or smoke. & -20 & Environment \\ \hline
	The character tries to dodge while prone. & -20 & Environment \\ \hline
	The character is fighting in knee deep snow or water. & -30 & Environment \\ \hline
	Fighting an armed opponent while unarmed. & -10 & Equipment \\ \hline
	The character is attempting to use two knives or a sword and a knife at once. & -10 & Equipment \\ \hline
	The character attempts to use a spear one-handed. & -20 & Equipment \\ \hline
	The character is attempting to use two spears at once. & -40 & Equipment \\ \hline
	The character is outnumbered 2 to 1. & -10 & Numbers \\ \hline
	The character and his allies outnumber the enemy 2 to 1. & +10 & Numbers \\ \hline
	The character is attacking a kneeling target. & +10 & Target \\ \hline
	The character attacks a stunned target & +20 & Target \\ \hline
	The character is attacking a prone target. & +20 & Target \\ \hline
	The character attacks an unaware target. & +30 & Target \\ \hline
\end{tabularx}
\subsection{Ranged Combat}
\begin{tabularx}{\textwidth}{|X|l|l|}
	\hline
	\ul{Situation} & \ul{Modifier} & \ul{Keyword} \\ \hline
	The character is shooting over extreme range. & -50 & Distance \\ \hline
	The character is shooting over long range. & -20 & Distance \\ \hline
	The character shoots over short range. & +10 & Distance \\ \hline
	The character shoots at point blank range. & +30 & Distance \\ \hline
	The character attempts to use a pistol one-handed. & -10 & Dual-wielding \\ \hline
	The character is attempting to use two pistols at once. & -20 & Dual-wielding \\ \hline
	The character attempts to use a small rifle one-handed. & -30 & Dual-wielding \\ \hline
	The character attempts to use another weapon one-handed. & -60 & Dual-wielding \\ \hline
	The character is attempting to use two small rifles at once. & -40 & Dual-wielding \\ \hline
	The character is fighting in strong winds, heavy rain or on slippery ground. Multiple may stack. & -10 & Environment \\ \hline
	The character tries to dodge while prone. & -20 & Environment \\ \hline
	The character is attacking a specific enemy caught up in melee combat or he himself is caught in melee combat. & -30 & Environment \\ \hline
	The character is attempting to fire blindly at an enemy he knows the rough location of. & -30 & Environment \\ \hline
	The character is fighting in knee deep snow or water. & -30 & Environment \\ \hline
	The character is shooting at a character in mist, fog or smoke. & -30 & Environment \\ \hline
	The character is using a long weapon in somewhat confined spaces like a house. & -20 & Environment \\ \hline
	The character is using smgs in very tight corridors or tunnels. & -20 & Environment \\ \hline
	The character shoots from the higher ground. & +10 & Environment \\ \hline
	The character is shooting a crouching target. & -10 & Target \\ \hline
	The character is shooting a prone target. & -20 & Target \\ \hline
	The character attacks a stunned target. & +20 & Target \\ \hline
	The character attacks an unaware target. & +30 & Target \\ \hline
\end{tabularx}
\subsection{Social}
\begin{tabularx}{\textwidth}{|X|X|l|}
	\hline
	\ul{Situation} & \ul{Modifier} & \ul{Focused skill} \\ \hline
	Performing a simple dance. & -10 / +0 if the character is trained in both etiquette as well as acrobatics & Acrobatics / Etiquette \\ \hline
	Performing a difficult, intricate dance. & -40 / -30 if the character is trained in both etiquette as well as acrobatics & Acrobatics / Etiquette \\ \hline
	The target has something else to do or does not want to be bothered. & -10 & Appeal \\ \hline
	The target is suspicious and/or has had bad experiences with the character. & -20 & Appeal \\ \hline
	The target is hostile but not aggressive yet. & -30 & Appeal \\ \hline
	The target is a devoted enemy and currently in mortal combat with the character. & -60 & Appeal \\ \hline
	The target and the character are acquainted. & +10 & Appeal \\ \hline
	The target is friendly. & +20 & Appeal \\ \hline
	The target is devoted to or close friends with the character. & +30 & Appeal \\ \hline
	Pickpocketing something obvious like an ID card or a phone sticking out of someone's pocket & -10 & Sleight of hand \\ \hline
	Performing a simple magic trick to draw one person's attention. & +10 & Sleight of hand \\ \hline
\end{tabularx}
\subsection{Exploration}
\begin{tabularx}{\textwidth}{|X|l|l|}
	\hline
	\ul{Situation} & \ul{Modifier} & \ul{Focused skill} \\ \hline
	Driving under fire or on difficult terrain. & -20 & Drive / pilot \\ \hline
	Driving a damaged vehicle, still under heavy fire. & -40 & Drive / pilot \\ \hline
	Driving a vehicle for the first time. & +0 & Drive / pilot \\ \hline
	Driving in familiar terrain and regions. & +10 & Drive / pilot \\ \hline
	Controlling a civilian vehicle at moderate speeds or at low speeds in fairly close quarters. & +20 & Drive / pilot \\ \hline
	Controlling a civilian vehicle at low speeds. & +40 & Drive / pilot \\ \hline
	Making out sounds with an engine humming on an idle car right next to the character or light animal chatter in the background. & -10 & Perception \\ \hline
	Trying to make out sounds in a firefight or riot. & -20 & Perception \\ \hline
	Trying to see something or fight in low light or against light. & -20 & Perception \\ \hline
	Trying to make out sounds in a street war. & -40 & Perception \\ \hline
	Trying to fight in total darkness or while blinded. & -60 & Perception \\ \hline
	Trying to make out sounds with large explosions going off. & -60 & Perception \\ \hline
	Listening for footsteps on sandy ground. & +0 & Perception \\ \hline
	Listening for footsteps on wet ground. & +15 & Perception \\ \hline
	Listening for footsteps in tall grass, on gravel or snow. & +20 & Perception \\ \hline
	Listening for footsteps on metal flooring. & +30 & Perception \\ \hline
	Listening for footsteps on grassy ground. & +5 & Perception \\ \hline
	Listening for sounds in echoing environments. & +100\% & Perception \\ \hline
	Hide or draw a small object. & -10 & Sleight of hand \\ \hline
	Hide or draw a fairly large object, like a submachinegun. & -30 & Sleight of hand \\ \hline
	Trying to hide while moving slowly. & -10 & Stealth \\ \hline
	Trying to sneak in armor or heavy footwear. & -10 & Stealth \\ \hline
	Trying to sneak in a strike suit. & -15 & Stealth \\ \hline
	Trying to hide while running. & -40 & Stealth \\ \hline
	Trying to sneak while barefoot or wearing very light footwear. & +0 & Stealth \\ \hline
	Trying to hide while crouching in place. & +10 & Stealth \\ \hline
	Trying to hide while laying prone in place. & +30 & Stealth \\ \hline
\end{tabularx}
\subsection{Craftsmanship}
\begin{tabularx}{\textwidth}{|X|l|l|}
	\hline
	\ul{Situation} & \ul{Modifier} & \ul{Focused skill} \\ \hline
	Creating or repairing equipment with only insufficient or improvised tools. & -20 & Mechanics \\ \hline
	Trying to fix unknown technology. & -40 & Mechanics \\ \hline
	Attempting to repair and use damaged weapon attachments. & +10 & Mechanics \\ \hline
	Repairing a damaged vehicle with the necessary tools. & +20 & Mechanics \\ \hline
	Repairing simple circuitry or devices like antennas. & +30 & Mechanics \\ \hline
	Picking more secure locks like found on motorized vehicles. & -10 & Security \\ \hline
	Defusing an old landmine. & -30 & Security \\ \hline
	Setting up a tripwire or trapping a door. & +10 & Security \\ \hline
	Picking a simple lock like found on a bicycle. & +20 & Security \\ \hline
	Strategically placing a simple lock or chain. & +60 & Security \\ \hline
\end{tabularx}
	
	\chapter{Credits}
	\section{Images}
While most images were, again, generated using Stable Diffusion,
	the images of the Tarot cards used were taken from \url{https://pixabay.com/}, created by \emph{giftedMG}.

\end{document}