\section{Possessed Weapons}
\label{sec:possessedweapon}
\subsection{Binding}
Unlike living creatures an inanimate object always has to have a demon bound to it ritually
- as Demons are not interested in objects and objects cannot invite Demons.
\par
First a demon needs to running free, e.g. from Summoning.
The free demon is then distracted or incapacitated for 5 minutes while the object is prepared to house the spirit.
\par
Once the object is ready, the ritual leader must win a contest of wills
	(an opposed Courage-based Philosophy test)
	against the demon, forcing it into the object and sealing it away.
\\%
The ritual leader may repeat this test as often as he likes and collect the difference in DoS.
Once he is satisfied, the total is noted down as the \emph{Binding Strength}.
If the total ever falls below 0, however,
	the leader is stunned for D5 rounds
	and the demon is \emph{frenzied}.

\subsection{Distortions}
When a demon possesses a weapon, it gains unfathomable new abilities and its outward appearance is drastically warped.
It can no longer be disassembled, becomes very resistant or outright immune to most damage, regenerates like a living creature and requires no maintenance.
\par
In addition draw a number of tarot cards equal to
	the sum of the demon's combat and social rating.
Then remove an amount of cards equal to the Binding Strength.
\begin{exampleblock}
	\itshape
	At this point the ritual leader (picking the cards) is supposed to know the cards' meanings already.
	More Binding Strength is supposed to give an advantage.
\end{exampleblock}
Go through all leftover cards and apply the results in \ref{sec:traits}.
\\%
Apply all of the remaining effects.
\begin{exampleblock}
	\itshape
	Ideally the players know only the meaning of each tarot card
	(e.g. \emph{Youth} for a straight Fool),
	not the the associated weapon trait.
\end{exampleblock}

\subsection{Traits}
\label{sec:traits}
\luaimport{components/weapontraits.csv}{arcanum.tpl}{possessedweapontraits}
%TODO: page filler - "demonic possessed sword"

\subsection{Erosion}
The demon can gradually break the Binding
	or the user can loosen it voluntarily,
	making the weapon more powerful by setting free more of the demon's might.
\\%
\begin{sitemize}
	\item Whenever the wielder suffers \emph{Corruption},
		he needs to win an opposed Courage test against the demon.
		He gains +5 for every Binding Strength currently on the weapon.
		\\%
		On a failure the Binding Strength is reduced by 1.
	\item The same applies when the wielder suffers from \emph{Shock}
		or other, equally strong emotions.
	\item The wielder may focus to release more power as an action.
		\\%
		He needs to win an opposed Courage test against the demon,
		gaining the same bonus.
		If he wins, the Binding Strength is reduced by 1.
		If he fails, the demon uses the opportunity
		and the Binding Strength is reduced by 2 instead.
\end{sitemize}
\par%
\needspace{20mm}
When the Binding Strength erodes,
	the wielder immediately draws a Tarot card
	and applies the result.
\\%
Any rules that allow re-rolls of dice
	allow discarding and re-drawing this card.
Just like re-rolling dice, the card may not be re-drawn more than once.
And just like when creating the weapon in the first place,
	the wielder does not learn the effect before it is applied.
\par%
When the Binding Strength reaches 0, the demon is set free.
When the Binding is broken voluntarily in a prepared ritual area,
	the demon may immediately be bound again to the same object.
When the Binding is broken involuntarily though,
	the weapon breaks apart.
