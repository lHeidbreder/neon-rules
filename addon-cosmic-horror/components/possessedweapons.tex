\section{Possessed Weapons}
\label{sec:possessedweapon}
\subsection*{Binding}
Unlike living creatures an inanimate object always has to have a demon bound to it ritually
- as Demons are not interested in objects and objects cannot invite Demons.
\par
First a demon needs to running free, e.g. from Summoning.
The free demon is then distracted or incapacitated for 5 minutes while the object is prepared to house the spirit.
\par
Once the object is ready, the ritual leader must win a contest of wills
	(an opposed Courage-based philosophy test)
	against the demon, forcing it into the object and sealing it away.
\\%
The ritual leader may repeat this test as often as he likes and collect the difference in DoS.
Once he is satisfied, the total is noted down as the \emph{Binding Strength}.
If the total ever falls below 0, however, the leader is stunned for D5 rounds and the demon is frenzied.

\subsection*{Distortions}
When a demon possesses a weapon, it gains unfathomable new abilities and its outward appearance is drastically warped.
It can no longer be disassembled, becomes very resistant or outright immune to most damage, regenerates like a living creature and requires no maintenance.
\par
In addition draw a number of tarot cards equal to twice the summoning's DoS.
Then remove an amount of cards equal to the Binding Strength.\\
Go through all leftover cards and apply the results in \ref{sec:traits}.
\\%
Apply all of the remaining effects.
\begin{exampleblock}
	\itshape
	Ideally the players know only the meaning of each tarot card
	(e.g. \emph{Youth} for a straight Fool),
	not the the associated weapon trait.
\end{exampleblock}

\subsection{Traits}
\label{sec:traits}
\luaimport{components/weapontraits.csv}{arcanum.tpl}{possessedweapontraits}
