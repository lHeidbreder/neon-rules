\section{Force}
\label{possession:force}
A demon goes to forcefully possess a target against his will.
This is a grueling endeavor to behold,
and a scarringly painful experience for the host.
\paragraph{Advantages:}
None.
It's a painful, grueling experience and leaves most scarred for life, if they even regain themselves at all.
\paragraph{Disadvantage:}
The target suffers corruption from the exposure to the demon.
He also has very little control, once the demon takes over.
\paragraph{The Path:}
When the target is within arm's reach, and either unaware, grappled by the demon or otherwise incapacitated, the demon may initiate a Possession attempt.
It immediately makes a Courage test and collects the DoS.
It may spend an action each turn to repeat this test and collect the DoS.
\par \vspace{-5mm}
During this time the target can use an action once each turn to resist the attacking demon
	(one of them if multiple are attacking).
He makes a Courage test at a bonus equal to his Insight (as opposed to at a penalty).
Every DoS reduces the demon's DoS, one by one.
Failing the test means the target suffers 1 Corruption and the collected DoS remain untouched.
\par \vspace{-5mm}
If the demon's DoS fall below 0 (or the hold is broken), the demon is repelled and cannot attempt another possession until the next sundown (and may completely be banished).
Notably whenever the vessel-to-be itself intends to attack the demon, he immediately suffers from \emph{Shock}.
\par \vspace{-5mm}
If, however, the DoS reach more than twice the target's Toughness bonus, the target becomes possessed.
It also suffers 1 Corruption for every round the attack took.
\par \vspace{-5mm}
While possessed the character can still use an action to reduce the Demon's DoS and the Demon can still use its to increase the connection.
If the target reaches his limit of either Corruption or Insight, his soul is completely lost and his body turns into a mutant.
\paragraph{Effects:}
The target gains the following alterations:
\begin{itemize}
	\vspace{-10mm}
	\setlength\itemsep{-10mm}
	\item the DEMONIC, DEADLY MELEE and VOLATILE trait
	\item +6 Natural Armor
	\item To act at all, the character must take an Opposed Courage test against the Demon during Declaration.
	If he wins or there's a stalemate, he gets to act as normal.
	If he fails, the demon acts instead.
\end{itemize}

\section{Invitation}
A human invites a demon voluntarily and traps it in his mind.
A risky endeavor to be sure,
but craving power often leads to poor judgment.
\paragraph{Advantages:}
Having a very loose connection with the demon allows to use its powers easily.
As long as the demon's and the vessel's intentions align,
the vessel is free to act.
\paragraph{Disadvantages:}
Demons are hard to understand.
Their desires are alien and unpredictable and ever-changing.
This makes it very hard to act in alignment with them.
\paragraph{The Path \& Effects:}
Invitation works mostly like forced possession.
The core differences are:
\begin{itemize}
	\setlength\itemsep{-8mm} \vspace{-8mm}
	\item The vessel cannot do anything but fight the demon.
	\item The vessel takes all related tests with a bonus of +20.
	\item The demon cannot flee once the process has started.
\end{itemize}

\section{Binding}
\label{possession:binding}
Lastly a demon can be ritually bound to the target.
The binding is strong and very hard to undo.
\paragraph{Advantages:}
It is much easier to control and much harder to break than the other options.
It also reduces the corruption on the body when manifesting powers.
\paragraph{Disadvantages:}
The ritual leader and the vessel may not be the same person.
The reduced demonic influence also greatly reduces its benefits.
In addition the demon constantly works to break the seal - it hates being trapped.
\paragraph{The Path:}
To learn how to bind a demon to a target, refer to \ref{ritual:binding}.
\paragraph{Effects:}
The demon is bound tightly to the vessel.
This binding cannot be removed without destroying the vessel.
\\%
Becoming possessed this way grants various alterations:
\begin{itemize}
	\setlength\itemsep{-8mm} \vspace{-8mm}
	\item The vessel's Courage is no longer reduced by its Insight.
	\item The vessel's body becomes non-senescent.
	\item The vessel can perceive incorporeal entities.
	\item The vessel shows small cosmetic alterations, in line with the demon possessing him.
		These could be a Nullshade Entity's piercing eyes or a Memory Wraith's eerie hum.
	\item Since the binding is so strong, the demon cannot take over control.
		Instead the demon subtly controls its vessel's actions.
		\\%
		Once each session one successful test will fail instead, as chosen by the GM,
			much like he had the \emph{Bane: Cursed by Fortuna}.
		(This is in addition to the Bane.)
		\\%
		For an outside observer this may simply look like tics.
	\item The vessel leaks otherworldly stench.
		He has a naturally worse disposition with mortals.
\end{itemize}
In addition the vessel may invoke the demon's power.
Whenever he does so, he takes 2 damage and gains 2 corruption.
\\%
For the rest of the encounter,
\begin{itemize}
	\setlength\itemsep{-8mm} \vspace{-8mm}
	\item he may use one of the demon's characteristics instead of his own OR
	\item he gains access to one of the demon's special features.
\end{itemize}
