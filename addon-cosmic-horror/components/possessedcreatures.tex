%TODO: explain corruption
%only demons can possess
%horrors can be possessed
%Passive Bonuses: increased injury threshold by demon's power?
%Active Ability - Invoke: for one action, gain a bonus to any characteristic equal to the character's corruption; gain D10 corruption
%Penalty: Insomnia, hostility towards horrors
\section{Force}
%TODO: demon possesses the target against its will
A demon goes to forcefully possess a target against his will.
This is a grueling endeavor to behold,
and a scarringly painful experience for the host.
\paragraph{Advantages:}
None. Absolutely none.
\paragraph{Disadvantage:}
The target suffers corruption from the exposure to the demon.
He also has very little control, once the demon takes over.
\paragraph{The Path:}
When the target is within arm's reach, and either unaware or grappled by the demon, the demon may initiate a Possession attempt.
It immediately makes a Courage test and collects the DoS.
It may spend an action each turn to repeat this test and collect the DoS.
\par \vspace{-5mm}
During this time the target can use an action once each turn to resist the attacking demon
	(one of them if multiple are attacking).
He makes a Courage test at a bonus equal to his Insight (as opposed to at a penalty).
Every DoS reduces the demon's DoS, one by one.
Failing the test means the target suffers 1 Corruption and the collected DoS remain untouched.
\par \vspace{-5mm}
If the demon's DoS fall below 0 (or the hold is broken), the demon is repelled and cannot attempt another possession until the next sundown (and may completely be banished).
Notably whenever the vessel-to-be itself intends to attack the demon, he suffers from \emph{Shock}.
\par \vspace{-5mm}
If, however, the DoS reach more than twice the target's Toughness bonus, the target becomes possessed.
It also suffers 1 Corruption for every round the attack took.
\par \vspace{-5mm}
While possessed the character can still use an action to reduce the Demon's DoS and the Demon can still use its to increase the connection.
If the target reaches his limit of either Corruption or Insight, his soul is completely lost and his body turns into a mutant.
\paragraph{Effects:}
The target gains the following alterations:
\begin{itemize}
	\vspace{-10mm}
	\setlength\itemsep{-10mm}
	\item the DEMONIC, DEADLY MELEE and VOLATILE trait
	\item +6 Natural Armor
	\item To act at all, the character must take an Opposed Courage test against the Demon during Declaration.
	If he wins or there's a stalemate, he gets to act as normal.
	If he fails, the demon acts instead.
\end{itemize}

\section{Invitation}
%TODO: the target opens its mind and invites a demon
%advantage: large stat boosts, fast
%disadvantage: very risky
\paragraph{Advantages:}
\paragraph{Disadvantages:}
\paragraph{The Path:}
\paragraph{Effects:}

\section{Binding}
\label{possession:binding}
Lastly a demon can be ritually bound to the target.
The binding is strong and very hard to undo.
\paragraph{Advantages:}
It is much easier to control and much harder to break than the other options.
It also reduces the corruption on the body when manifesting powers.
\paragraph{Disadvantages:}
The ritual leader and the vessel may not be the same person.
The reduced demonic influence also greatly reduces its benefits.
In addition the demon constantly works to break the seal - it hates being trapped.
\paragraph{The Path:}
To learn how to bind a demon to a target, refer to \ref{ritual:binding}.
\paragraph{Effects:}
%TODO
