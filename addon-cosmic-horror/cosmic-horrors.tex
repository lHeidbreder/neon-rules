\documentclass[12pt,a4paper,openany]{book}
%Custom Environments
\newenvironment{exampleblock}[1][1]
{
	\par
	\vspace{-5mm}
	\hfill
	\begin{minipage}
		{\dimexpr\columnwidth-#1cm}
	\begin{mdframed}[
		backgroundcolor=Gray!65,
		rightline=false,
		topline=false
		]
}{
	\end{mdframed}
	\end{minipage}
	\par
}

%%shortened itemize
\newenvironment{sitemize}[1][10]
{
	\begin{itemize}
	\vspace{-#1mm}
	\setlength\itemsep{-#1mm}
}{
	\end{itemize}
}

%Custom Commands
\newcommand{\ul}[1]{\underline{\smash{#1}}}
\newcommand{\breakline}{\vspace{.5cm} \hrule width \columnwidth \relax}
\newcommand{\derivedvalue}[3]{
	\begin{samepage}
	\subsubsection{#1 \textsubscript{\textlangle#2\textrangle}}
	\hfill
	\begin{minipage}{\dimexpr\columnwidth-1cm}
		#3
	\end{minipage}
	\end{samepage}
	\par
}
\newcommand{\specialrule}[2]{
	\begin{minipage}{\columnwidth}
		\textbf{\ul{#1:}}\\
		#2
	\end{minipage}
    \par
}
\newcommand{\supply}[4]{
	\begin{minipage}{\columnwidth}
		\textbf{\ul{#1}}\\
		\textit{Price: #3}\\
		\textit{Weight: #4}\\
		#2
	\end{minipage}
    \par
}
\newcommand{\service}[4]{
	\begin{minipage}{\columnwidth}
		\textbf{\ul{#1}}\\
		#2\\
		\textit{Cost:} #3 per #4
	\end{minipage}
	\par
}
\newcommand{\statuseffect}[2]{
	\begin{minipage}{\columnwidth}
		\textbf{\ul{#1:}}\\
		#2\\
	\end{minipage}
	\vspace{5mm}
}
\newcommand{\skill}[4][Basic]{
	\begin{minipage}{\columnwidth}
		\textbf{\ul{#2}}\\
		Difficulty: #1\\
		Common Characteristic: #3\\
%		Description:\\
		\textit{#4}\\
	\end{minipage}\par
}
\newcommand{\melee}[2]{\skill{#1}{Melee Base}{#2}}
\newcommand{\ranged}[2]{\skill{#1}{Ranged Base}{#2}}
\newcommand{\ability}[4]{
	\begin{minipage}{\columnwidth}
		\textbf{\ul{#1}} (#2 XP)\\
		\textit{Prerequisites}: #3\\
		\textit{Effect}:\\
		#4\\
	\end{minipage}
}
\newcommand{\maneuver}[3]{
	\begin{minipage}{\columnwidth}
		\textbf{\ul{MVR: #1}} (#2 XP)\\
		\textit{Prerequisites: #3\\}
		Effect:\\
		Allows the use of the maneuver \emph{#1} at normal penalties.\\
	\end{minipage}
}
\newcommand{\boon}[4][]{
	\begin{minipage}{\columnwidth}
		\label{boon::#2}
		\textbf{\ul{\smash{#2}}} \mbox{(#4 GP\ifthenelse{\isempty{#1}}{}{ }#1)}\\
		#3
	\end{minipage}
	\par
}
\newcommand{\bane}[4]{
	\boon[#4]{#1}{#2}{#3}
}
\newcommand{\rangedweapon}[9]{
	\vspace{2mm}
	\begin{minipage}{\columnwidth}
		\textbf{\ul{#1}}\\
		\textit{#2}\\
		\textbf{Weight}: #9\\
		\textbf{Price}: #8\\
		#3\\
		\textbf{Mag}: #5\\
		\textbf{Reload}: \mbox{#6 actions}\\
		\textbf{Range}: #4\\
		\ifthenelse{\isempty{#7}}{}{\textbf{Special Rules}: #7}
	\end{minipage}
	\par
}
\newcommand{\weaponmod}[5]{
	\begin{minipage}{\columnwidth}
		\textbf{\ul{#1}}\\
		\textit{#3}; \textit{#4}; \textit{#5}\\
		#2
	\end{minipage}
	\par
}
\newcommand{\meleecomponent}[6]{
	\begin{minipage}{\columnwidth}
		\textbf{\ul{#1:}}\\
		\textit{#2}\\
		\textbf{Weight}: #4; \textbf{Price}: #5\ifthenelse{\isempty{#6}}{}{; \textbf{Requirement}: #6}\\
		\textbf{Effect}: #3
	\end{minipage}
	\par
}
\newcommand{\ammo}[6]{
	\begin{minipage}{\columnwidth}
		\textbf{\ul{#1:}} \textit{#2}\\
		\textbf{Price}: #3; \textbf{Unit of sale}: #4\\
		\textbf{Weight/Bulk}: #5 \textit{\ifthenelse{\isempty{#6}}{each}{per #6}}
	\end{minipage}
	\par
}
\newcommand{\armor}[9]{
	\begin{minipage}{\columnwidth}
		\textbf{\ul{#1}} (covers: #6)\\
		\begin{tabular}{|r|r|r|r|}
			\hline
			Head & Chest & Arms & Legs\\
			\hline
			#2 & #3 & #4 & #5\\
			\hline
		\end{tabular}\par
		\vspace{2mm}
		\textit{Price:} #8; \textit{Weight:} #9\\
		#7
	\end{minipage}
	\par
}
\newcommand{\armormod}[6]{
	\begin{minipage}{\columnwidth}
		\textbf{\ul{#1}}\\
		#2\\
		\textit{Price:} #3; \textit{Weight:} #6\\
		\textit{Mod points:} #4\ifthenelse{\isequivalentto{N}{#5}}{}{; Requires power source}
	\end{minipage}
	\par
}
\newcommand{\pes}[6]{
	\begin{minipage}{\columnwidth}
		\textbf{\ul{#1}}\\
		Armor: #2; Threshold: #3\\
		\textit{Price:} #4; \textit{Weight:} #5
		\ifthenelse{\isempty{#6}}{}{\\} %conditional line break
		#6
	\end{minipage}
	\par
}
\newcommand{\implant}[7]{
	\begin{minipage}{\columnwidth}
		\textbf{#1}\\
		\textit{#2}\\
		\textit{Effect}: #3\\
		\textit{Price:} #6; \textit{Load:} #5\ifthenelse{\isequivalentto{-}{#4}}{}{; \textit{Slot:} #4}\\
		\textit{Available Mods:} #7
	\end{minipage}
	\par
}
\newcommand{\augmod}[3]{
	\begin{minipage}{\columnwidth}
		\textbf{#1}\\
		#2\\
		\textit{Cost:} #3
	\end{minipage}
	\par
}
\newcommand{\mod}[2]{\item \textbf{#1}: #2}
\newcommand{\psicomponent}[4]{\textbf{#1} & #2 & #3 & #4 \\}

%filler images
\newcommand{\filltopageendgraphics}[2][]{%
	\par
	\zsaveposy{top-\thepage}% Mark (baseline of) top of image
	\vfill
	\zsaveposy{bottom-\thepage}% Mark (baseline of) bottom of image
	\smash{\includegraphics[height=\dimexpr\zposy{top-\thepage}sp-\zposy{bottom-\thepage}sp\relax,#1]{#2}}%
	\par
}

%Base Building
\newcommand{\baselocation}[6]{
	\begin{minipage}{\columnwidth}
		\textbf{#1}\\
		Size: #3; Concealment: #4; Defense: #5\\
		Conditions: #6\\
		\textit{Cost:} #2
	\end{minipage}
	\par
}
\newcommand{\baseasset}[6]{
	\begin{minipage}{\columnwidth}
		\textbf{#1}\\
		\textit{#2}\\
		Size: #3\\
		Concealment: #4\\
		Defense: #5\\
		\textit{Cost:} #6
	\end{minipage}
	\par
}

%Narrative
\newcommand{\nrule}[3]{
	\begin{minipage}{\columnwidth}
		\section{#1}
		\textit{#2}\\
		\vspace{8mm}
		\begin{exampleblock}
			#3
		\end{exampleblock}
	\end{minipage}
	\par
}

%GM
\newcommand{\missiontype}[3]{\item \textbf{#1}: \textit{#2} #3}

\usepackage[utf8]{inputenc}
\usepackage[english]{babel}
\usepackage{textcomp}
\usepackage{xifthen}
\usepackage{tabularx}
\usepackage{tabto}
\usepackage{multicol}

\usepackage[bookmarks=true,colorlinks=true,linkcolor=cyan]{hyperref}

%Alignment
\usepackage[skip=10mm]{parskip}
\raggedbottom

% Title Image
\usepackage{wallpaper}
\def\coverimgpath{../art/\@title/cover}

\def\subtitle{Cosmic Horrors}

\begin{document}
	{\heading
\ThisCenterWallPaper{1}{\coverimgpath}
\maketitle}
{\hypersetup{hidelinks} \tableofcontents}

	
	\chapter{Introduction}
	Out there, in the void between the stars, there are things beyond comprehension, beyond reason. And the oldest and strongest emotion of mankind is fear, and the oldest and strongest kind of fear is fear of the unknown...
	%TODO: intro is very short, need to add a filler

	\chapter{Perils From Beyond}
	When the world beyond touches ours, mortals in the vicinity suffer - of three things specifically.
	\par
	\textbf{Insight} is the effect on the mind when someone tries to understand the dark beyond.
	It is usually temporary
		but so much out there will attempt to break your mind.
	\\
	\textbf{Corruption} is the effect on the body,
		setting in after extended exposure to the Dark
		or being watched by entities from beyond.
	\\
	\textbf{Powerful creatures} live beyond, attracted by strong emotion.
	%TODO: Horrors and Demons are not really distinguished in the bestiary
	First in line are the \emph{Demons} - wicked and malevolent beings from outside our perceived reality.
	\emph{Horrors}, on the other hand, only get their name due to their attraction to nightmares
		- concepts like malevolence do not apply to them.
	
	\section{Insight}
	\subsection{Gaining Insight}
	Characters gain Insight whenever they perceive horrors from beyond with any sense they were given naturally
		- i.e. hearing, seeing, smelling, tasting or touching.
	The feeling is generally unpleasant.
	\\
	Perceiving any creature with the DEMONIC tag will cause D5 Insight.
	Particularly powerful beings or groups may cause D10.
	\subsection{Reducing Insight}
	Characters may reduce their Insight by resting,
		secluded from such anomalies.
	It then reduces by 6 every day.
	\subsection{Effects of Insight}
	Every point of Insight reduces a character's Courage by 1.
	When Courage reaches 0 due to Insight,
		the character goes catatonic for at least a week and gains D10 corruption.
	After said week his Insight reduces back to 0,
		which often goes hand in hand with shallow memory loss,
		as the body sheds the remains of that experience.
	
	\section{Corruption}
	\subsection{Gaining Corruption}
	Corruption is gained when the dark powers try to enter a mortal body.
	\vspace{-8mm}
	\begin{itemize}
		\setlength\itemsep{-8mm}
		\item In areas with a lot of presence characters gain up to 1 corruption per hour.
		\item Participating in or crashing rituals, which ask for some outer entity's favor, will cause D5 corruption.
	\end{itemize}
	\subsection{Reducing Corruption}
	Corruption is basically permanent.
	It reduces by 1 per 2 years without contact to the Eldritch.
	Considering that high corruption attracts them, this is very hard to do.
	\subsection{Effects of Corruption}
	A character can accrue up to 100 corruption, at which point he will transform
		- sometimes into an entity from Beyond
		but much more likely into
		a puddle of goo on the floor or a pile of ash.\\
	While that transformation may be sudden and violent,
		the point of ascension is not reached subtly.
	His body will slowly mutate as corruption accumulates.
	Every full 10 points of corruption will cause a mutation (see Chapter \ref{ch:mutation}).
	
	\chapter{Mutation}
	\label{ch:mutation}
	Every now and then a character may be forced to mutate.
	Roll a D100, divide by 2 and consult the table below.
	The result may be modified by a tenth of the character's insight at the time of mutation.
	\begin{exampleblock}
		Most mutations are clearly visible and make social interactions much, much harder.
	\end{exampleblock}
	\begin{multicols}{2}
		\begin{enumerate}
			\setlength\itemsep{-10mm}
			\luaimport{components/mutations.csv}{mutation-tableentry.tpl}{muttable}
		\end{enumerate}
	\end{multicols}
	
	\section*{Mutations}
	\vspace{4mm}
	\begin{multicols}{2}
		\newcounter{mutations} %fixes hyperref links
		\luaimport{components/mutations.csv}{mutation.tpl}{mutations}
	\end{multicols}
	
	\chapter{Rituals}
	\label{ch:rituals}
	Rituals take a long time, are expensive and a lot of work to set up. They are however the prime way to ask the Great Ones for favor and protection in exchange for a tribute.
	\section{The Tribute}
	Great Ones respond well to unique things and strong emotions. The more personal to the participants this tribute is, the more likely the Great One will respond well - but nothing is guaranteed.
	\par
	\begin{center}
	\begin{tabular}{lr}
		\textbf{Live sacrifice} & \\
		Your lover or first born & +50 \\
		Your nemesis or rival & +30 \\
		A close family member or best friend & +20 \\
		A member of your species & +0 \\
		An animal & -10 \\
		\hline
		\textbf{Inanimate Objects} & \\
		A piece of fine craftsmanship & -10 \\
		Rare materials, valuables & -30 \\
		Dirt, literal dirt & -80
	\end{tabular}
	\end{center}
	\section{The Request}
	Their power is boundless and they offer many gifts. Almost everything may be requested but very few requests are actually fulfilled.
	\\
	The more selfish and direct the request, the higher the chances generally.
	Also, the more convoluted and needlessly complex and outright strange, the higher the chances generally.
	\\
	For many requests the leading character needs a very firm understanding of the infinite void and its inhabitants. 
	Some creatures may be very willing to fulfill even the strangest or most mundane requests
		- but if one does not know, who they are asking,
		the ritual may be doomed from the start.
	\par
	Examples of common requests include:
	\begin{sitemize}
		\item Learn about the Great Beyond, its inhabitants, its powers.
		\item Permanently meld minds with a willing participant to allow telepathic communication.
		\item Ask to mutate.
		\item Subjugate an unwilling participant.
		\item Summon a creature into real space or become possessed.
	\end{sitemize}
	\section{The Ritual}
	When preparation is done, the ritual then takes a few hours and a Philosophy test. The test may be modified by the tribute, potentially by the request and the leading character's Insight (one-to-one).
	\par
	During the performance of the ritual the participants will slowly unravel a small portion of the great Beyond, gaining them 2D10 Insight.\\
	If the ritual is successful, all participants are touched by untethered blasphemy, gaining D5 corruption.
	\section{The Outcome}
	A "successful" ritual does not necessarily mean the desired outcome. Instead the GM secretly rolls a D5 and consults the table below.\\
	For every 3 full DoS the ritual succeeded with, one option may be excluded by the leading character. If an excluded option is rolled, the leading character may choose the outcome instead.
	\par
	\vspace{-5mm}
	\begin{enumerate}
		\raggedright
		\setlength\itemsep{-8mm}
		\item The ritual succeeds without further incident.
		\item The ritual fizzles, nothing happens. Participants' corruption gain is reduced by 3, to a minimum of 0.
		\item The ritual succeeds. A demon attempts possession of the ritual leader.
		\item The ritual succeeds. A demon is summoned and manifests.
		\item The ritual succeeds. A random participant mutates (see p.\pageref{ch:mutation}), immediately and violently.
		The resulting mutation may not be adjusted using Insight.
		The leading character may have a higher chance to be chosen.
	\end{enumerate}
	\section{The Reward}
	If the ritual succeeds, the creature who answers will provide something close to the desired outcome - functionally the same.
	Depending on the specific creature, the effect may take a different shape however.
	\section{Examples}
	%\subsection*{name}
%\subsubsection*{Preparation}
%%what tribute? roughly how expensive?
%%tests? which? what penalty?
%\subsubsection*{Rites}
%%penalty on the test
%%duration
%\subsubsection*{Effects}
%%positive outcome
%%potential side effects
%
\subsection*{Summoning}
\label{ritual:summoning}
\subsubsection*{Preparation}
A summoning requires a piece of bait -
in this context that is an object of significant emotional weight to the participants.
Examples include a religious artifact, a piece of personal artwork, or a wedding ring.
\\%
An item that the participants are barely willing to part with may add a bonus up to +30.
\subsubsection*{Rites}
Calling for a creature from beyond - and waiting for an answer - takes a long time.
The ritual may take anywhere from two to six hours (D5+1, if in question).
\\%
After the time is up, the leader makes a test at a -20 penalty.
\subsubsection*{Effects}
A random demon (or group of demons) is summoned.
Their combined combat rating is equal to half the ritual's DoS.
\\%
If the ritual leader called a specific demon he is aware of
and that demon's combat rating is equal or less than the ritual's DoS,
that demon is summoned instead.

\subsection*{Binding}
\label{ritual:binding}
\subsubsection*{Preparation}
The vessel (a human or a weapon) needs to be fixed in place and a demon needs to be running free in the vicinity.
(See e.g. Summoning, p.\pageref{ritual:summoning})
\subsubsection*{Rites}
The ritual leader makes repeated Courage-based Philosophy tests
- as if he tried to possess the demon.
(See \ref{possession:force})
\\%
Whenever one of these opposed tests is successful,
the demon is forced to step closer to the vessel at the next opportunity.
\\%
Once the binding ritual starts, the demon will notice.
It becomes furious, attacking anyone on sight with a preference for the leader and any participants.
It will generally ignore the vessel, as it is of no threat.
\subsubsection*{Effects}
For the possession of creatures refer to \ref{possession:binding} (p. \pageref{possession:binding}).
For the possession of weapons refer to \ref{sec:possessedweapon} (p. \pageref{sec:possessedweapon}) instead.

\subsection*{Gift}
\subsubsection*{Preparation}
The vessel is anesthetized and presented to the void.
\subsubsection*{Rites}
Symbols are painted, burned or engraved on the vessel's natural body parts.
\\%
The process takes around two hours.
At the end the leader makes the Philosophy test.
\subsubsection*{Effects}
The target is blessed with a new mutation.
The random result may be adjusted not just by the vessel's
	but by the leader's Insight as well.
\\%
When determining the result,
	an additional mutation may not be prevented.

\subsection*{Intervention}
\subsubsection*{Preparation}
The leader etches or burns the request into his body.
Every letter costs 1 HP in a supernatural toll.
Depending on the length of the request,
another member may be required to cut the leader instead,
lest the pain may stop him from finishing this work.
\\%
The request may not be split between participants
	or left on another one's body but the ritual leader's.
It does however also not need to be coherently, humanly readable
	- the letters only need to appear somewhere.
\subsubsection*{Rites}
The participants ask the void beyond to fulfill their request.
\\%
The Philosophy test is at a penalty depending on the type of request.
It may range from -50 to +50
	- the more chaotic and strange the request,
	the easier it \emph{usually} is to ask.
\\%
\subsubsection*{Effects}
\emph{Since this is somewhat of a catchall ritual, the effect descriptions are vague...}
\par \vspace{-8mm}
On a successful test the request is fulfilled, at least partially.
However, there are caveats, namely how these responses are usually twisted.
\\%
A request for information is answered vaguely or through self-fulfilling prophecies.
A request for money may also awake sudden curiosity in the authorities or criminals.
A request for more physical strength may cause permanent pain,
 as the body begins to destroy itself.
\\%
The possibilities are endless, however -
Considering the request has to be very short,
any imprecision will lead to undesired side effects,
either immediately or some time down the line.

	
	\chapter{Possession}
	%TODO: explain corruption
%only demons can possess
%horrors can be possessed
%Passive Bonuses: increased injury threshold by demon's power?
%Active Ability - Invoke: for one action, gain a bonus to any characteristic equal to the character's corruption; gain D10 corruption
%Penalty: Insomnia, hostility towards horrors
\section{Force}
%TODO: demon possesses the target against its will
A demon goes to forcefully possess a target against his will.
This is a grueling endeavor to behold,
and a scarringly painful experience for the host.
\paragraph{Advantages:}
None. Absolutely none.
\paragraph{Disadvantage:}
The target suffers corruption from the exposure to the demon.
He also has very little control, once the demon takes over.
\paragraph{The Path:}
When the target is within arm's reach, and either unaware or grappled by the demon, the demon may initiate a Possession attempt.
It immediately makes a Courage test and collects the DoS.
It may spend an action each turn to repeat this test and collect the DoS.
\par \vspace{-5mm}
During this time the target can use an action once each turn to resist the attacking demon
	(one of them if multiple are attacking).
He makes a Courage test at a bonus equal to his Insight (as opposed to at a penalty).
Every DoS reduces the demon's DoS, one by one.
Failing the test means the target suffers 1 Corruption and the collected DoS remain untouched.
\par \vspace{-5mm}
If the demon's DoS fall below 0 (or the hold is broken), the demon is repelled and cannot attempt another possession until the next sundown (and may completely be banished).
Notably whenever the vessel-to-be itself intends to attack the demon, he suffers from \emph{Shock}.
\par \vspace{-5mm}
If, however, the DoS reach more than twice the target's Toughness bonus, the target becomes possessed.
It also suffers 1 Corruption for every round the attack took.
\par \vspace{-5mm}
While possessed the character can still use an action to reduce the Demon's DoS and the Demon can still use its to increase the connection.
If the target reaches his limit of either Corruption or Insight, his soul is completely lost and his body turns into a mutant.
\paragraph{Effects:}
The target gains the following alterations:
\begin{itemize}
	\vspace{-10mm}
	\setlength\itemsep{-10mm}
	\item the DEMONIC, DEADLY MELEE and VOLATILE trait
	\item +6 Natural Armor
	\item To act at all, the character must take an Opposed Courage test against the Demon during Declaration.
	If he wins or there's a stalemate, he gets to act as normal.
	If he fails, the demon acts instead.
\end{itemize}

\section{Invitation}
%TODO: the target opens its mind and invites a demon
%advantage: large stat boosts, fast
%disadvantage: very risky
\paragraph{Advantages:}
\paragraph{Disadvantages:}
\paragraph{The Path:}
\paragraph{Effects:}

\section{Binding}
\label{possession:binding}
Lastly a demon can be ritually bound to the target.
The binding is strong and very hard to undo.
\paragraph{Advantages:}
It is much easier to control and much harder to break than the other options.
It also reduces the corruption on the body when manifesting powers.
\paragraph{Disadvantages:}
The ritual leader and the vessel may not be the same person.
The reduced demonic influence also greatly reduces its benefits.
In addition the demon constantly works to break the seal - it hates being trapped.
\paragraph{The Path:}
To learn how to bind a demon to a target, refer to \ref{ritual:binding}.
\paragraph{Effects:}
%TODO

	\section{Possessed Weapons}
\label{sec:possessedweapon}
\subsection*{Binding}
Unlike living creatures an inanimate object always has to have a demon bound to it ritually
- as Demons are not interested in objects and objects cannot invite Demons.
\par
First a demon needs to running free, e.g. from Summoning.
The free demon is then distracted or incapacitated for 5 minutes while the object is prepared to house the spirit.
\par
Once the object is ready, the ritual leader must win a contest of wills
	(an opposed Courage-based philosophy test)
	against the demon, forcing it into the object and sealing it away.
\\%
The ritual leader may repeat this test as often as he likes and collect the difference in DoS.
Once he is satisfied, the total is noted down as the \emph{Binding Strength}.
If the total ever falls below 0, however, the leader is stunned for D5 rounds and the demon is frenzied.

\subsection*{Distortions}
When a demon possesses a weapon, it gains unfathomable new abilities and its outward appearance is drastically warped.
It can no longer be disassembled, becomes very resistant or outright immune to most damage, regenerates like a living creature and requires no maintenance.
\par
In addition draw a number of tarot cards equal to twice the summoning's DoS.
Then remove an amount of cards equal to the Binding Strength.\\
Go through all leftover cards and apply the results in \ref{sec:traits}.
\\%
Apply all of the remaining effects.
\begin{exampleblock}
	\itshape
	Ideally the players know only the meaning of each tarot card
	(e.g. \emph{Youth} for a straight Fool),
	not the the associated weapon trait.
\end{exampleblock}

\subsection{Traits}
\label{sec:traits}
\luaimport{components/weapontraits.csv}{arcanum.tpl}{possessedweapontraits}

	
	\chapter{Additional Content}
	\section{Boons}
\boon{Blessed Birth}{The character starts with a random \hyperref[ch:mutation]{mutation} and 3 corruption.}{6}
\section{Banes}
\bane{Mark of Sacrifice}
	{The character bears the mark of sacrifice somewhere on his body.
		Otherworldly creatures (at least all those with the DEMONIC trait)
		react aggressively towards him
		and he takes a -30 penalty to hide from them.
		At some point in time the mark will glow and a Harvest (p. \pageref{anomaly:harvest}) will begin,
		coming for his body and soul.
	}
	{7}
	{}
\bane{Watched}
	{Otherworldly entities are aware of the character.
		He always has at least 2 Insight.
		May be taken multiple times to increase the amount (2, 4, 6...).}
	{3}
	{each}
\section{Abilities}
\ability{Steeled}
	{400}
	{Courage 40}
	{Whenever the character would gain corruption,
		he makes a Con test at a penalty
		equal to 3 times the corruption.
		On a success the corruption gain is reduced by DoS, to a minimum of 0.}
\section{Bestiary}
\subsection{Traits}
\begin{tabularx}{\columnwidth}{l|X}
	Name & Effect \\ \hline
	DEMONIC & Other non-DEMONIC creatures suffer D5-1 Insight upon witnessing this creature. 
	This includes seeing, touching, hearing and smelling the creature, among others. \\
	\hline
	INCORPOREAL & The creature cannot interact with the physical world and vice versa.
\end{tabularx}

\subsection{Creatures}
\luaimport{components/bestiary.csv}{beast.tpl}{components/horrorbestiary}

	
	\chapter{Psionics Patch}
	When using both this and the Psionics Addon, you may want to use the following addendum to glue them together: 
	Insight increases the psionic target value one-to-one. However, the cost now not only deals damage to the caster but will also cause the same amount of corruption.\\
	By GM discretion some casters may use this new and some may use the old way of casting.
	
	\chapter{Credits}
	\section{Images}
While most images were, again, generated using Stable Diffusion,
	the images of the Tarot cards used were taken from \url{https://pixabay.com/}, created by \emph{giftedMG}.

\end{document}
